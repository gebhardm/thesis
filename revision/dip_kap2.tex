\chapter{Von CW-Komplexen zu orientierten Matroiden}

Nach der Bereitstellung der wesentlichen Fundamentalia soll es in den beiden
nun folgenden Kapiteln um die Möglichkeiten gehen, zu einem allgemeinen
d-dimensionalen CW-Komplex orientierte Matroide zu finden, mittels derer
eventuell zugehörige Koordinaten gefunden werden können. Diese sollen dann
das Bild des vorgelegten CW-Komplexes unter einer ho\-möo\-mor\-phen
Abbildung als durchdringungsfreien CW-Komplex (je zwei offene Zellen haben
leeren Schnitt) im Euklidischen Raum beschreiben.

In diesem Kapitel soll dabei die grundlegende Vorgehensweise dargestellt
werden, wie zunächst solche bezüglich Symmetrieeigenschaften des Komplexes
"`verträglichen"' orientierten Matroide bestimmt werden können. Dazu gilt
nämlich folgende Aussage.
\begin{quote}
{\sf
Existiert zu einem vorgelegten CW-Komplex $\C$ kein bezüglich vorgegebener
Symmetrieeigenschaften verträgliches orientiertes Matroid, so ist $\C$ nicht
mit diesen Symmetrien in den Euklidischen Raum einbettbar.
}
\end{quote}
Würde nämlich doch eine solche Einbettung existieren, so gehört zu dieser
eine Punktkonfiguration, die eine Matrix und somit ein orientiertes Matroid
induzieren würde, welches zu $\C$ "`verträglich"' wäre.

Im abschließenden dritten Kapitel sollen die bereitgestellten Mittel dazu
dienen, einige Beispiele zu untersuchen, wobei es als Abschluß um symmetrische
orientierte Matroide zur Dyckschen Karte gehen soll. Auch sollen weitere
Fragestellungen angegeben werden, die sich in diesem Zusammenhang stellen
(vgl. etwa \cite{Bo:86} und \cite{Bo:91}).

\section{Erste Definition der Verträglichkeit}

Zunächst soll allerdings eine Verträglichkeitsdefinition angegeben werden,
die eine allgemeine Brücke zwischen CW-Komplexen und orientierten Matroiden
schlägt.\\
Im Grundlagenteil wurde dazu als wichtiges Ergebnis aus der Theorie der
orientierten Matroide aufgeführt, wie sich unter Anwendung des topologischen
Repräsentationssatzes (Pseudo-)Sphärenarrangements und orientierte Matroide
verbinden lassen. Ausgehend von diesem Satz soll es nun gelingen, dies auch auf
allgemeine CW-Komplexen zu beziehen.\\
Betrachtet man als Ausgangspunkt ein beliebiges (Pseudo-)Sphärensystem auf
der $\S^d$, so ist dort durch das Arrangement eine Zerlegung in Zellen gegeben,
die wegen der Homöomorphieeigenschaften von Sphäre und $\R^d$ auch als
CW-Komplex interpretiert werden kann (vgl. Seite \pageref{cell}). Die maximalen
Zellen dieses Komplexes sind jene d-dimensionalen Gebiete auf der $\S^d$, deren
Rand durch Schnitte und dadurch induzierte Segmente der
(d$-$1)-(Pseudo-)Sphären, die gerade den niederdimensionalen Zellen
entsprechen, gegeben ist. Aufgrund seiner Entstehung sei dieser Komplex als
durch ein (Pseudo-)Sphärensystem induzierter CW-Komplex \idx{induzierter
CW-Komplex} bezeichnet. Handelt es sich bei einem Pseudosphärensystem um die
Darstellung eines orientierten Matroids und ist dieses sogar realisierbar (wenn
das Pseudosphärenarrangement zu einem Sphärenarrangement mit gleicher
Schnittstruktur homöomorph ist), so entspricht das zugehörige Sphärensystem
gerade der Realisierung des durch das Pseudosphärensystem induzierten
CW-Komplexes.\\
Betrachtet man das dem entstandenen Sphärensystem entsprechende
Hyperebenenarrangement, so ist mit dessen zugehörigen Normalenvektoren eine
Koordinatenmatrix gegeben, die ein Chirotop induziert, das ein zum
Sphärensystem äquivalentes orientiertes Matroid beschreibt. Entsprechend der
Definition der Polarität von Seite \pageref{polar} hat man mit diesen Normalen
eine Punktkonfiguration eines zum Ausgangssphärensystem polaren Komplexes. Das
Zusammenspiel der Realisierung eines durch ein Sphärensystem induzierten
Komplexes einerseits und des durch die extrahierte Matrix gegebenen Chirotops
andererseits ist über die Äquivalenz der Darstellungen orientierter Matroide
zu deuten. Aufgrund der verschiedenen Zugangsweisen soll klar unterschieden
werden, ob bei der Untersuchung eines CW-Komplexes von einem
Pseudosphärenarrangement oder einem Chirotop ausgegangen wird. Zunächst wollen
wir so nur die Darstellung eines durch ein Sphärensystem induzierten
CW-Komplexes betrachten. Wie das zugehörige Chirotop in Zusammenhang mit der
Struktur des Komplexes steht, kann dabei als gesonderte Fragestellung
aufgefaßt werden.

Aus obigen Grundüberlegungen ergibt sich nun die entscheidende erste
Definition, die die Verbindung zwischen CW-Komplexen und orientierten Matroiden
charakterisiert:
\begin{quote}
{\bf Ein orientiertes Matroid $\cal M$ soll verträglich zu einem beliebigen
CW-Komplex $\cal C$ heißen, wenn $\cal C$ als Unterkomplex in einem durch ein
$\cal M$ darstellendes Pseudosphärenarrangement induzierten CW-Komplex
enthalten ist.}
\end{quote}
Ist dieses Pseudosphärenarrangement zu einem Sphärensystem homöomorph, so ist
auch eine Realisierung des CW-Komplexes als entsprechende Teilstruktur des
Sphärensystems gegeben.\\
{\scsi
Hier könnte man das Pseudo- und das Sphärensystem auch als homotop bezeichnen,
wenn sich die beiden stetig unter Beibehaltung aller Schnitteigenschaften
ineinander überführen lassen, was man auch als Streckung des
Pseudosphärensystems bezeichnet.
}

\begin{figure}[htb]
$$
\beginpicture
\unitlength0.6cm
\setlinear
\setcoordinatesystem units <0.6cm,0.6cm>
\setplotarea x from -6 to 6, y from -3 to 3
\put {\circle*{0.15}} [Bl] at -5.5 -1.5
\put {\circle*{0.15}} [Bl] at -5.5 0
\put {\circle*{0.15}} [Bl] at -5 -1
\put {\circle*{0.15}} [Bl] at -4.5 0.5
\put {\circle*{0.15}} [Bl] at -4 -0.5
\put {\circle*{0.15}} [Bl] at -4 -1.5
\put {\circle*{0.15}} [Bl] at -3.5 0
\put {\circle*{0.15}} [Bl] at -3 -1
\plot -5.5 -1.5 -5.5 0 -4.5 0.5 -3.5 0 -3 -1 -4 -1.5 -5.5 -1.5 /
\plot -5.5 -1.5 -5 -1 -4 -0.5 -3.5 0 /
\plot -5.5 0 -5 -1 -4 -1.5 /
\plot -4.5 0.5 -4 -0.5 -3 -1 /
\put {\small CW-Komplex $\mapsto$ orientiertes Matroid
      $\mapsto$ Matrix} [Bl] at -6 -2.5
\circulararc 360 degrees from 1 0 center at 0 0
\ellipticalarc axes ratio 3:1 180 degrees from 1 0 center at 0 0
\startrotation by 0.5 0.866
\ellipticalarc axes ratio 3:1 180 degrees from 1 0 center at 0 0
\startrotation by -0.5 0.866
\ellipticalarc axes ratio 3:1 180 degrees from 1 0 center at 0 0
\stoprotation
\put {$\left(\begin{array}{cc}
             \cdot & \cdot \\
             \cdot & \cdot \\
             \cdot & \cdot
             \end{array}\right)$} [Bl] at 4 -0.5
\endpicture
$$
\caption{Der Weg von einem CW-Komplex zu einer möglichen Realisierung}
\label{quest}
\end{figure}

Um von beliebig vorgelegten CW-Komplexen entscheiden zu können, ob diese in
durch Sphä\-rensysteme induzierten CW-Komplexen als Unterkomplexe vorkommen,
muß zu\-nächst untersucht werden, wie solche Unterkomplexe aussehen können.
Dazu sei ein kleines Beispiel betrachtet, welches zeigen soll, wie sich diese
Unterkomplex\-eigenschaft äußert und welche Schwierigkeiten bei der
Betrachtung solcher Pseudosphärensysteme auftreten können.

Als erstes Beispiel betrachten wir dazu einen 1-Komplex, der aus sechs 0-Zellen
und sechs 1-Zellen bestehe, die ein topologisches Sechseck beschreiben (siehe
dazu Abbildung \ref{hexagon}). Der 1-Komplex entspricht dem \ul{Rand}komplex
einer 2-Zelle, so wie auch im allgemeinen Fall d-CW-Komplexe als (Teil des)
Randkomplex(es) einer (d+1)-Zelle gedeutet werden sollen. Dies geschieht in
Verallgemeinerung der Betrachtung von Randstrukturen von Mannigfaltigkeiten
und impliziert damit, in welcher Dimension ein zugehöriges beziehungsweise
verträgliches Sphärensystem zu suchen sein soll.\\
Da wir zum einen Durchdringungsfreiheit garantieren wollen, respektive die
Eigenschaft zu erhalten ist, daß alle offenen Zellen nach Abbildung in den
euklidischen Raum disjunkt sind, sowie zum anderen eine affin lineare
Realisierung angestrebt wird, ist im allgemeinen Fall die kleinste Dimension,
in der mit einer Realisierung zu rechnen ist, um eins höher als die maximale
Dimension der auftretenden Zellen des vorgelegten Komplexes.
Betrachtet man als einfachen Fall etwa ein (d+1)-Simplex, so stellt dessen Rand
gerade einen d-Komplex dar, der eine (d+1)-Zelle umschreibt.
Zusammenfassend soll also ein d-Komplex immer als (Teil der) Berandung
\ul{einer} (d+1)-Zelle aufgefaßt werden, womit als Ausgangssituation die
Realisierbarkeit eines d-dimensionalen CW-Komplexes immer im $\R^{d+1}$ zu
untersuchen ist.\\
Übertragen auf die Sphären bedeutet dies, daß der vorgelegte d-Komplex
die Berandung einer (d+1)-Zelle beschreibt, die auf einer $\S^{d+1}$ liege,
im Rand der (d+2)-Kugel. Die Pseudosphären, die den vorgelegten Komplex
implizieren sollen sind stetige Verformungen der $\S^d$, womit sich als
allgemeine Situation die Betrachtung orientierter Matroide im Rang d+2 ergibt,
entsprechend $\S^{d+1}=\partial B^{d+2}\subset\R^{d+2}$.\\
Nun aber zu unserem Sechseckbeispiel.

\begin{figure}[htb]
$$
\beginpicture
\unitlength0.6cm
\setlinear
\setcoordinatesystem units <0.6cm,0.6cm>
\setplotarea x from -6 to 6, y from -3 to 3.5
\put{ \beginpicture
   \setsolid \thicklines
   \setquadratic
   \plot 0.5 0.5 1.5 0.8 2.5 0.0 /
   \plot 2.5 0.0 2.5 1.0 3.5 1.5 /
   \plot 3.5 1.5 2.8 2.0 3.0 3.0 /
   \plot 3.0 3.0 2.3 2.7 1.5 3.5 /
   \plot 1.5 3.5 1.2 2.6 0.5 2.5 /
   \plot 0.5 2.5 1.1 1.5 0.5 0.5 /
   \setlinear \thinlines
   \put {\circle*{0.15}} [Bl] at 0.5 0.5
   \put {\circle*{0.15}} [Bl] at 2.5 0.0
   \put {\circle*{0.15}} [Bl] at 3.5 1.5
   \put {\circle*{0.15}} [Bl] at 3.0 3.0
   \put {\circle*{0.15}} [Bl] at 1.5 3.5
   \put {\circle*{0.15}} [Bl] at 0.5 2.5
   \endpicture } at 0 0
\put{ \beginpicture
      \setsolid \setlinear
        \ellipticalarc axes ratio 3:1 360 degrees from 2.5 0 center at 0 0
        \startrotation by 0 -1 about 0 0
        \ellipticalarc axes ratio 3:1 360 degrees from 2.5 0 center at 0 0
        \stoprotation
        \plot 0 2.5 0 -2.5 /
        \setdashes<1mm>
        \circulararc 360 degrees from 2.5 0 center at 0 0
        \put {\circle*{0.15}} [Bl] at 0 0.85
        \put {\circle*{0.15}} [Bl] at 0 -0.85
        \put {\circle*{0.15}} [Bl] at 0.8 0.8
        \put {\circle*{0.15}} [Bl] at -0.8 0.8
        \put {\circle*{0.15}} [Bl] at 0.8 -0.8
        \put {\circle*{0.15}} [Bl] at -0.8 -0.8
      \endpicture } at 5 0
\put{ \beginpicture
        \setsolid \setlinear
        \ellipticalarc axes ratio 3:1 360 degrees from 2.5 0 center at 0 0
        \startrotation by 0.5 0.866 about 0 0
        \ellipticalarc axes ratio 3:1 360 degrees from 2.5 0 center at 0 0
        \stoprotation
        \startrotation by -0.5 0.866 about 0 0
        \ellipticalarc axes ratio 3:1 360 degrees from 2.5 0 center at 0 0
        \stoprotation
        \setdashes<1mm>
        \circulararc 360 degrees from 2.5 0 center at 0 0
        \put {\circle*{0.15}} [Bl] at 0.95 0
        \put {\circle*{0.15}} [Bl] at -0.95 0
        \put {\circle*{0.15}} [Bl] at 0.475 0.823
        \put {\circle*{0.15}} [Bl] at -0.475 0.823
        \put {\circle*{0.15}} [Bl] at 0.475 -0.823
        \put {\circle*{0.15}} [Bl] at -0.475 -0.823
      \endpicture } at -5 0
\endpicture
$$
\caption{Einbettung eines topologischen Sechsecks}
\label{hexagon}
\end{figure}

Wie schon Abbildung \ref{hexagon} einer Darstellung des topologischen Sechsecks
zeigt, gibt es neben der anschaulichen Realisierung als ebenes Sechseck (die
"`übliche"') in einem Sphärensystem noch eine weitere Möglichkeit, ein
Sechseck als Unterkomplex zu finden. Dieser zweite Fall zeichnet sich sogar
dadurch aus, daß die Anzahl der auftretenden Sphären minimal ist (fünf statt
der sechs in der "`üblichen"' Darstellung). Zur Charakterisierung dieser
beiden Möglichkeiten kann zum einen ein beliebiger d-dimensionaler CW-Komplexes
als Randkomplex eines einzigen Topes (Gebietes auf der $\S^{d+1}$) auftreten
oder aber auch der Randstruktur einer Vereinigung von (benachbarten) Topes
entsprechen. Im Falle mehrerer beteiligter Topes ist die durch den vorgelegten
Komplex umschriebene (d+1)-Zelle das Innere der Vereinigung der Abschlüsse der
auftretenden Regionen.

Wird von einer weiteren Realisierungsmöglichkeit des Sechsecks (einer
nichtkonvexen) zu einem zugehörigen Sphärenarrangement übergegangen, so
zeigt sich, daß die beiden zuerst aufgeführten Fälle die im Sinne der
Unterkomplexeigenschaft einzigen Möglichkeiten sind, den Komplex in
einem Sphärensystem darzustellen. Läßt man nämlich zu, daß Segmente einer
(Pseudo-)Sphäre auch mehrmals zur Beschreibung vorgelegter Zellen Verwendung
finden, das heißt wenn nicht benachbarte Teile einer (Pseudo-)Sphäre
verschiedenen Zellen des gegebenen CW-Komplexes entsprechen sollen (vgl.
Abb.\ref{linehex}), kann hier die Unterkomplexeigenschaft verletzt werden.

\begin{figure}[htb]
$$
\beginpicture
\unitlength0.6cm
\setlinear
\setcoordinatesystem units <0.6cm,0.6cm>
\setplotarea x from -6 to 6, y from -3 to 3
\put{ \beginpicture
      \setsolid \setlinear
      \plot 0 1 -2.5 0 -1 0 0 -1 1 0 2.5 0 0 1 /
      \put {\circle*{0.15}} [Bl] at 0 1
      \put {\circle*{0.15}} [Bl] at -2.5 0
      \put {\circle*{0.15}} [Bl] at -1 0
      \put {\circle*{0.15}} [Bl] at 0 -1
      \put {\circle*{0.15}} [Bl] at 1 0
      \put {\circle*{0.15}} [Bl] at 2.5 0
      \endpicture } at -3 0
\put{ \beginpicture
        \setsolid \setlinear
        \ellipticalarc axes ratio 3:1 -180 degrees from 2.5 0 center at 0 0
        \startrotation by 0.819 0.574 about 0 0
        \ellipticalarc axes ratio 3:1 -180 degrees from 2.5 0 center at 0 0
        \stoprotation
        \startrotation by 0.766 -0.643 about 0 0
        \ellipticalarc axes ratio 3:1 -180 degrees from 2.5 0 center at 0 0
        \stoprotation
        \startrotation by 0.643 0.766 about 0 0
        \ellipticalarc axes ratio 3:1 180 degrees from 2.5 0 center at 0 0
        \stoprotation
        \startrotation by 0.5 -0.866 about 0 0
        \ellipticalarc axes ratio 3:1 --180 degrees from 2.5 0 center at 0 0
        \stoprotation
        \setdashes<1mm>
        \circulararc 360 degrees from 2.5 0 center at 0 0
        \put {\circle*{0.15}} [Bl] at 0.11 1.29
        \put {\circle*{0.15}} [Bl] at 1.24 -0.71
        \put {\circle*{0.15}} [Bl] at -1.45 -0.68
        \put {\circle*{0.15}} [Bl] at -0.3 -0.82
        \put {\circle*{0.15}} [Bl] at 0.27 -0.83
        \put {\circle*{0.15}} [Bl] at -0.05 -1
        \put {\circle{0.25}} [Bl] at -1.12 -0.09
        \put {\circle{0.25}} [Bl] at  1.05 -0.23
      \endpicture } at 3 0
\endpicture
$$
\caption{Ein "`verbotenes"' Sphärensystem}
\label{linehex}
\end{figure}

Was bei der polygonalen Darstellung des Sechsecks plausibel erscheint,
widerspricht der Definition der induzierten Zellen im Sphärenarrangement, nach
der jeder Schnitt von (Pseudo-)Sphären immer einer niederdimensionalen Zelle
entspricht. Dies ist in diesem Fall nicht gewährleistet. Vielmehr würde es
sich hier um die Realisierung eines topologischen Achtecks im Rand der
Vereinigung von vier Topes handeln. Im Sechseck sind die beiden zusätzlichen
0-Zellen (nicht ausgefüllte Kreise in Abb. \ref{linehex}) in der Realisierung
nicht vorhanden. Solch eine Situation soll grundsätzlich verboten sein, denn
sind $S_1$ und $S_2$ zwei benachbarte Zellen des induzierten Komplexes
($\ol{S_1}\cap\ol{S_2}\neq\emptyset$), so ist zwar int($\ol{S_1}\cup\ol{S_2}$)
nach Definition eine Zelle, sie ist aber kein Element des durch das
Sphärensystem induzierten Komplexes und verletzt somit die
Unterkomplexeigenschaft. Der betrachtete Unterkomplex des induzierten
CW-Komplexes hätte zwar den selben zugrundeliegenden Raum, aber mehr
Zellen als die Realisierung des vorgelegten CW-Komplexes, was nicht erlaubt ist.

Mit diesem ersten Beispiel läßt sich die Definition von oben nun weiter
präzisieren.
\begin{quote}
{\bf Ein orientiertes Matroid $\cal M$ soll verträglich zu einem beliebigen
CW-Komplex $\cal C$ heißen, wenn $\cal C$ dem Randkomplex des Abschlusses
einer Vereinigung von Topes eines $\cal M$ darstellenden
Pseudosphärenarrangements entspricht.}
\end{quote}

Zur Vertiefung des gerade Beschriebenen und als Ausgangsbasis für die
Untersuchung allgemeiner zweidimensionaler CW-Komplexe, zu der im dritten
Kapitel weiter Beispiele behandelt werden sollen, möge als zweites
Beispiel ein 2-Komplex betrachtet werden, der den Rand eines Torus beschreibt
(vgl. dazu auch \cite{Dau:89} und \cite{BoWi:87}). Der vorgelegte CW-Komplex
sei dazu nach Abbildung \ref{torus1} gegeben.

\begin{figure}[htb]
$$
\input torus
\settorus
$$
\caption{Torus}
\label{torus2}
\end{figure}

Nach der Beschreibung aller Eigenschaften des Torus, die sich mit den bisher
eingeführten Begriffen angeben lassen, soll ein Algorithmus entwickelt werden,
der die Vorgehensweise im generellen Fall kombinatorischer Komplexe
unterstützen soll.

\section{Von einem CW-Komplex zu einem Algorithmus}

Es soll nun ein Torus im $\R^3$ betrachtet werden, dessen kombinatorische
Struktur als zweidimensionaler CW-Komplex nach Abbildung \ref{torus1}
durch Kantenidentifikation gegeben sei. Entgegen der Betrachtung der
Einbettung triangulierter Tori (vgl. etwa \cite{Dau:89} und \cite{BoEg:91}),
sei hier gerade der Nichtsimplizialität allgemeiner CW-Komplexe Rechnung
getragen und eine Zerlegung in Vierecke betrachtet.

\begin{figure}[htb]
$$
\beginpicture
\unitlength0.6cm
\setlinear
\setcoordinatesystem units <0.6cm,0.6cm>
\setplotarea x from -3.5 to 3.5, y from -2 to 2
\plot -3 1.5 3 1.5 3 -1.5 -3 -1.5 -3 1.5 /
\plot -1 1.5 -1 -1.5 /
\plot  1 1.5  1 -1.5 /
\plot -3 0.5 3 0.5 /
\plot -3 -0.5 3 -0.5 /
\put {1} [br] at -3.1 1.6  \put {1} [bl] at  3.1 1.6  \put {1} [tl] at  3.1 -1.6
\put {1} [tr] at -3.1 -1.6 \put {2} [br] at  -1 1.6   \put {2} [tr] at  -1 -1.6
\put {3} [bl] at  1 1.6    \put {3} [tl] at  1 -1.6   \put {4} [br] at -3.1 0.5
\put {4} [bl] at  3.1 0.5  \put {5} [br] at -1.1 0.6  \put {6} [bl] at  1.1 0.6
\put {7} [tr] at -3.1 -0.5 \put {7} [tl] at  3.1 -0.5 \put {8} [tr] at -1.1 -0.6
\put {9} [tl] at  1.1 -0.6
\endpicture
$$
\caption{CW-Komplex "`Rand eines Torus"'}
\label{torus1}
\end{figure}

Der vorgelegte Komplex besitzt neun 0-Zellen, 18 1-Zellen und neun 2-Zellen,
nach der Eulerformel also Geschlecht Eins. Da von der Wahl der Zellen her schon
auf Realisierungen geschlossen werden kann (so wie eine mögliche in Abbildung
\ref{torus3} zu sehen ist), wäre es nun interessant herauszufinden, wie sich
dieser Komplex als Unterkomplex in ein Sphärensystem einfügt.

Was allerdings bei der Betrachtung etwa einer Pyramide oder eines anderen
konvexen 3-Polytops mit einer geringen Zellenanzahl noch anschaulich
verständlich erscheint (indem man sich 2-Sphären so "`ineinandergesteckt"'
denkt, daß pro 2-Zelle des Komplexes eine Sphäre diese durch einen Auschnitt
repräsentiert), gestaltet sich schon hier, insbesondere bei noch größerer
Zellenanzahl und einem Nicht-Konvexität induzierenden topologischen Geschlecht,
schon schwieriger. Mangels Übersichtlichkeit in der möglichen Darstellung als
Sphärensystem soll deshalb eine Alternative zur Suche nach orientierten
Matroiden angewandt werden, die unter Berücksichtigung der Symmetriegruppe des
vorgelegten CW-Komplexes zugehörige Chirotope bestimmen helfen soll, was eine
zweite Verträglichkeitsdefinition liefert.

\begin{figure}[htb]
$$
\input tritor
\settritor
$$
\caption{Eine mögliche Realisierung des Torus}
\label{torus3}
\end{figure}

Da Chirotope durch (formale) Matrizen bestimmt werden und dadurch implizit eine
eventuelle Punktkonfigurationen betrachtet wird, ist es wenig sinnvoll, eine
Verträglichkeitsdefinition von Chirotopen zu beliebigen CW-Komplexen anzugeben.
Vielmehr soll der Fall simplizialer Komplexe dahingehend erweitert werden, daß
nun mehr als (r+1) 0-Zellen/Punkte pro r-Seite zugelassen seien, in der Sprache
der (n$\times$r)-Matrizen also auch singuläre (r$\times$r)-Untermatrizen
auftreten können. Im Fall von 2-Komplexen fordert dies von der Gestalt der
vorgelegten CW-Komplexe, daß alle maximalen Zellen gleicher Dimension seien und
der Rand des Abschlusses einer jeden 2-Zelle einem Polygonzug,
im höherdimensionalen Fall einem Polytop, entspreche, der die 0- und 1-Zellen,
sowie deren Inzidenzen induziert. In Aigners Buch zur
Kombinatorik \cite{Aig:76} heißen solche 2-dimensionalen CW-Komplexe
Landkarten, was mit der Definition der Karten aus dem Abschnitt über
Symmetriebegriffe gleichzusetzen ist. Ausgehend von den 0-Zellen ergibt sich
mittels deren Sterne wieder der Komplex, was das Vorhandensein der nötigen n
Punkte für Chirotope, zusammen mit dem Rang r über die Dimension der maximalen
Zellen + 2 sicherstellt.\\
Nun ist aber ein Chirotop noch nicht verträglich mit einem CW-Komplex nur
aufgrund der Tatsache, daß die Anzahl der Punkte und der Rang übereinstimmen.
Als weitere Kopplung dient die Eigenschaft in gewissem Sinne gleiche
Symmetrieeigenschaften zu haben. Dazu sei daran erinnert, daß die Realisierung
eines kombinatorischen Komplexes eine Untergruppe der Symmetriegruppe von diesem
als eigene Symmetriegruppe besitzt. Die Realisierung weist hierbei den 0-Zellen
aber gerade jene Koordinaten zu, die in eine Matrix geschrieben, ein
zugehöriges Chirotop liefern. Damit kommt man nun zu folgender zweiter
Verträglichkeitsdefinition von CW-Komplexen und orientierten Matroiden.

\begin{quote}
{\bf Ein Chirotop $\chi$ soll verträglich zu einem polyedrisch begrenzten
CW-Komplex $\cal C$ heißen, wenn Punktanzahl und Rang übereinstimmen und
die Automorphismengruppe ${\cal A}(\chi)$ eine Untergruppe der Symmetriegruppe
${\cal A}({\cal C})$ des CW-Komplexes ist.}
\end{quote}

Nun soll für das Torusbeispiel zunächst die Symmetriegruppe des vorgelegten
Komplexes bestimmt werden. Dies kann mittels eines kurzen Programmes geschehen,
dessen Aufbau sich wie folgt gliedert.

\subsection{Bestimmung der Symmetriegruppe}

Vorgelegt sei eine Liste, die angibt, welche 0-Zellen im Rand des Abschlusses
jeder 2-Zelle eines vorgelegten 2-CW-Komplexes liegen. Die 2-Zellen seien dabei
durch geschlossene Polygonzüge begrenzt, die durch die 0-Zellen und sie
verbindende 1-Zellen in den Rändern der Abschlüsse der 2-Zellen gegeben
sind. Dies geschieht entsprechend obiger Definition und ist gerade im Hinblick
auf die Untersuchung von Karten von Mannigfaltigkeiten sehr zweckmäßig.\\
Zusätzlich sei auf den Rändern der 2-Zellen willkürlich eine Orientierung in
Form eines Durchlaufsinnes eingeführt. Diese soll dazu dienen, auf die
tatsächlich vorhandenen 1-Zellen, also Kanten, zurückschliessen zu können.
{\scsi
Entgegen simplizialer Komplexe ist eine 0-Zelle/Ecke im Rand einer 2-Zelle/Seite
im allgemeinen nicht mit allen anderen Ecken der Seite durch eine 1-Zelle/Kante
verbunden, was bedeutet, daß die Valenz einer Ecke im allgemeinen kleiner oder
gleich der Gesamt\-eckenzahl $-1$ ist, was gesondert berücksichtigt werden
muß.
}

Die entsprechende Zellenliste sei als Datei folgender Form vorgelegt:
\begin{verbatim}
torus.gon

4
1254 2365 3146 4587 5698 6479 7821 8932 9713*
\end{verbatim}
{\scsi
Die Datei {\it torus.gon} gliedert sich wie folgt: {\bf 4} besagt, daß pro
2-Zelle vier 0-Zellen vorkommen, also Vierecke; die folgende Liste enthält die
0-Zellen im Rand der abgeschlossenen 2-Zellen entsprechend ihres Durchlaufsinns.
}

Beim Einlesen der Liste wird die maximale Anzahl der vorhandenen 0-Zellen
bestimmt und als {\it npkt} gespeichert. In einer
({\it npkt}$\times${\it npkt})-Binärmatrix (Adjazenzmatrix) wird dabei auch
gespeichert, welche 0-Zellen durch 1-Zellen "`verbunden"' sind, also welche
"`Kanten"' vorliegen. Anschließend kann die Bestimmung der Symmetriegruppe
beginnen. Dazu werden in einer rekursiven Routine alle {\it npkt}! Permutationen
der 0-Zellen durchlaufen und als Abbildungsvorschrift $\pi$ an die
Symmetrieprüfroutine übergeben (der Algorithmus zur Erzeugung der
Permutationen stammt aus dem Algorithmenbuch von R. Sedgewick \cite{Sed:91}).
In dieser wird getestet, ob $\pi$ jede 2-Zelle auf eine vorhandene
2-Zelle abbildet und ob in diesem Fall deren Rand respektive Kanten der
Orientierung entsprechend durchlaufen werden. Erfüllt $\pi$ dies für alle
vorgelegten 2-Zellen, so wird $\pi$ in die Symmetriegruppe übernommen, sonst
gestrichen und mit der nächsten Permutation fortgefahren.\\
Nachdem alle Permutationen durchlaufen sind, liegen jene Elemente der
Permutationsgruppe $S_{\it npkt}$ vor, die einen Automorphismus auf dem
vorgelegten Komplex beschreiben. Diese werden als Symmetriegruppe in einer
Datei gespeichert. Angemerkt sei, daß durch den Durchlauf aller Permutationen
für eine große Anzahl von 0-Zellen sehr viel Rechenzeit nötig ist.
{\scsi
(Für die acht 0-Zellen eines 3-Würfels etwa eine Sekunde, obiger Torus wird
in etwa 10 Sekunden abgearbeitet, die zwölf 0-Zellen der Dyckschen Karte
beanspruchen dann schon etwas über einer Stunde -- gemessen auf einem
66 MHz-i486-PC mit einem C-Programm unter Linux.)} Für obigen Torus ergeben
sich so folgende Symmetrien (in Zykelschreibweise):
\begin{verbatim}
No.1 : identity                     No.37 : (162435)(798)
No.2 : (47)(58)(69)                 No.38 : (195)(276)(384)
No.3 : (23)(56)(89)                 No.39 : (1675)(2398)
No.4 : (23)(47)(59)(68)             No.40 : (195)(236478)
No.5 : (24)(37)(68)                 No.41 : (1576)(2893)
No.6 : (2734)(5896)                 No.42 : (186)(254793)
No.7 : (2437)(5698)                 No.43 : (153426)(789)
No.8 : (27)(34)(59)                 No.44 : (186)(294)(375)
No.9 : (12)(45)(78)                 No.45 : (16)(25)(34)(79)
No.10 : (12)(48)(57)(69)            No.46 : (194376)(285)
No.11 : (132)(465)(798)             No.47 : (16)(29)(57)
No.12 : (132)(495768)               No.48 : (1926)(4785)
No.13 : (1452)(3768)                No.49 : (125697)(384)
No.14 : (179652)(348)               No.50 : (1287)(3594)
No.15 : (146982)(375)               No.51 : (1397)(2684)
No.16 : (1782)(3495)                No.52 : (136587)(294)
No.17 : (123)(456)(789)             No.53 : (147)(258)(369)
No.18 : (123)(486759)               No.54 : (17)(28)(39)
No.19 : (13)(46)(79)                No.55 : (147)(268359)
No.20 : (13)(49)(58)(67)            No.56 : (17)(29)(38)(56)
No.21 : (145893)(276)               No.57 : (1538)(4697)
No.22 : (1793)(2486)                No.58 : (18)(35)(49)
No.23 : (1463)(2759)                No.59 : (168)(239745)
No.24 : (178563)(249)               No.60 : (1948)(2365)
No.25 : (1254)(3867)                No.61 : (157248)(369)
No.26 : (128964)(357)               No.62 : (18)(27)(39)(45)
No.27 : (14)(25)(36)                No.63 : (168)(249)(357)
No.28 : (174)(285)(396)             No.64 : (192738)(465)
No.29 : (14)(26)(35)(89)            No.65 : (159)(287463)
No.30 : (174)(295386)               No.66 : (1849)(2563)
No.31 : (139854)(267)               No.67 : (1629)(4587)
No.32 : (1364)(2957)                No.68 : (19)(26)(48)
No.33 : (15)(38)(67)                No.69 : (159)(267)(348)
No.34 : (1835)(4796)                No.70 : (183729)(456)
No.35 : (15)(24)(36)(78)            No.71 : (167349)(258)
No.36 : (184275)(396)               No.72 : (19)(28)(37)(46)
\end{verbatim}
Der vorgelegte Torus hat also eine kombinatorische Symmetriegruppe der
Ordnung 72. Nun gilt es Chirotope zu finden, die zumindest eine Untergruppe
dieser Symmetriegruppe als eigene Symmetriegruppe besitzen. Diese können dann
im Falle ihrer Realisierbarkeit dazu eingesetzt werden, zu dem vorgelegten
Komplex Koordinaten zu finden.

Zunächst müssen dazu aus der vorgelegten Symmetriegruppe $\cal A$ Untergruppen
G extrahiert werden, die als Symmetriegruppen zugehöriger Chirotope in
Betracht kommen. Genauer soll die Anzahl der zu suchenden Chirotope auf
die Anzahl derer beschränkt werden, die die gewünschten Symmetrieeigenschaften
besitzen. Die betreffenden Untergruppen können wieder mittels eines Programms
erzeugt werden, welches ausgehend von den Symmetrien $\sigma\in {\cal A}$ mit
der höchsten Elementordnung ord($\sigma$) (es ist eine Realisierung mit
möglichst hoher Symmetrie angestrebt) deren zyklische Gruppen
($\{id,\sigma,\sigma^2,\ldots,\sigma^{ord(\sigma)-1}\}$) bestimmt und alle
erzeugten Elemente aus $\cal A$ streicht, so daß am Ende die verschiedenen
maximalen zyklischen Untergruppen der vorgelegten Symmetriegruppe aufgelistet
werden. Die Wahl solcher zyklischen Untergruppen begründet sich in der leichten
Erzeugbarkeit und dient der Tatsache, daß solche Gruppen wahrscheinlicher als
Symmetriegruppen von Chirotopen angenommen werden, als etwa größere
zusammengesetzte. Für den Torus sind dies (die Zahlen in den eckigen Klammern
geben die Nummern der Elemente der vorgelegten obigen Automorphismengruppe an):
\begin{verbatim}
No.1 : { id, [12], [17], [2], [11], [18] }
No.2 : { id, [14], [38], [47], [69], [49] }
No.3 : { id, [15], [63], [68], [44], [26] }
No.4 : { id, [21], [69], [58], [38], [31] }
No.5 : { id, [24], [44], [33], [63], [52] }
No.6 : { id, [30], [53], [3], [28], [55] }
No.7 : { id, [36], [53], [9], [28], [61] }
No.8 : { id, [37], [17], [27], [11], [43] }
No.9 : { id, [40], [69], [5], [38], [65] }
No.10 : { id, [42], [63], [8], [44], [59] }
No.11 : { id, [46], [53], [19], [28], [71] }
No.12 : { id, [64], [17], [54], [11], [70] }
No.13 : { id, [6], [4], [7] }
No.14 : { id, [13], [35], [25] }
No.15 : { id, [16], [62], [50] }
No.16 : { id, [22], [72], [51] }
No.17 : { id, [23], [45], [32] }
No.18 : { id, [34], [20], [57] }
No.19 : { id, [39], [56], [41] }
No.20 : { id, [48], [10], [67] }
No.21 : { id, [60], [29], [66] }
No.22 : { id }
\end{verbatim}

Mittels dieser Untergruppen kann man sich nun auf die Suche nach etwaig
zugehörigen Chirotopen begeben, was wie folgt geschehen soll.

\subsection{Erzeugung verträglicher Chirotope}

Mit der Anzahl $npkt$ der vorkommenden 0-Zellen und der Dimension $d$ der
maximalen Zellen des vorgelegten CW-Komplexes ist die Gestalt "`möglich
zugehöriger"', respektive verträglicher Chirotope festgelegt.\\
Sie sind darstellbar durch Listen mit insgesamt ${npkt \choose d+2}$
Vorzeichen aus $\{-,0,+\}$, im Fall des Torus also mit ${9 \choose 4}=126$
Elementen, die den Vorzeichen der formalen Brackets $[\lambda]$ mit
$\lambda\in\Lambda(npkt,d+2)$ entsprechen.\\
Da die Anzahl der Möglichkeiten diese ${npkt \choose d+2}$ Vorzeichen
aufzufüllen mit $3^{npkt \choose d+2}$ (im Fall des Torus
$3^{126}\geq 10^{60}$) recht groß ist, soll mittels zusätzlicher Bedingungen
diese Anzahl verringert werden.\\
Dazu kann zuerst einmal die Gestalt der vorgelegten 2-Zellen dienen. Da diese
in der Realisierung als Facetten im Rand der entstehenden 3-Zelle eben sein
sollen, können die 0-Zellen im Rand ihrer Abschlüsse, bei einer Anzahl
größer als drei, entsprechend als linear abhängige Punkte gedeutet werden.
Da die Vorzeichen zu (zunächst) formalen Determinanten gehören, können die
Vorzeichen jener Brackets zu 0 gesetzt werden, die eine Auswahl der Indizes der
0-Zellen des Randes ein und derselben abgeschlossenen 2-Zelle enthalten.\\
Für den Torus sind so die Vorzeichen zu den Brackets $[1245]$, $[1278]$,
$[1346]$, $[1379]$, $[2356]$, $[2389]$, $[4578]$, $[4679]$ und $[5689]$ zu 0
zu setzen, was die Anzahl der Auf\-füll\-mög\-lich\-kei\-ten von $3^{126}$ auf
immerhin schon $3^{117}< 10^{56}$ reduziert.\\
Mit den zu erfüllenden Symmetrien läßt sich diese Zahl noch weiter
verringern. Dies geschieht durch Kopplung der Brackets unter den Elementen
$\sigma$ einer Symmetriegruppe G, die entsteht, wenn man die Orbits der Brackets
$[\lambda]$, die Menge aller Bilder der $[\lambda]$ unter den $\sigma\in G$, als
Äquivalenzklassen betrachtet.\\
Für den vorgelegten Torus sei als zu erfüllende Symmetrieeigenschaft der
Realisierung im folgenden als Beispiel die Untergruppe G =
$\{ id, [12], [17], [2], [11], [18] \}$ vorgelegt. Mit dieser erhält man eine
Kopplung, durch die mit der Vorgabe von 23 Bracketvorzeichen für
Repräsentanten der Orbits alle übrigen Vorzeichen bestimmbar sind. In Tabelle
\ref{tab} sind in den Zeilen die 23 verschiedenen Bracketorbits aufgelistet,
von denen die Brackets der ersten Spalte, die entsprechend lexikographischer
Ordnung minimal sind, als Repäsentanten gewählt werden.
Die Vorzeichen vor den übrigen Brackets sind die jeweils durch die
lexikographische Ordnung nach der Permutation zu berücksichtigen
Vorzeichenwechsel.

\begin{table}[htb]
{\small
$$
\begin{array}{cccccc}
Identität                   & {123456789\choose 312978645} &
{123456789\choose 231564897} & {123456789\choose 123789456} &
{123456789\choose 312645978} & {123456789\choose 231897564}\\
        &         &         &         &         &        \\
+[1234] & +[1239] & +[1235] & +[1237] & +[1236] & +[1238]\\
+[1245] & +[1379] & +[2356] & +[1278] & +[1346] & +[2389]\\
+[1246] & +[1389] & -[2345] & +[1279] & +[1356] & -[2378]\\
+[1247] & +[1369] & +[2358] & -[1247] & -[1369] & -[2358]\\
+[1248] & +[1349] & +[2359] & -[1257] & -[1367] & -[2368]\\
+[1249] & +[1359] & +[2357] & -[1267] & -[1368] & -[2348]\\
+[1256] & -[1378] & -[2346] & +[1289] & -[1345] & -[2379]\\
+[1258] & +[1347] & +[2369] & -[1258] & -[1347] & -[2369]\\
+[1259] & +[1357] & +[2367] & -[1268] & -[1348] & -[2349]\\
+[1269] & +[1358] & +[2347] & -[1269] & -[1358] & -[2347]\\
+[1456] & +[3789] & +[2456] & +[1789] & +[3456] & +[2789]\\
+[1457] & -[3679] & +[2568] & +[1478] & -[3469] & +[2589]\\
+[1458] & -[3479] & +[2569] & +[1578] & -[3467] & +[2689]\\
+[1459] & -[3579] & +[2567] & +[1678] & -[3468] & +[2489]\\
+[1467] & -[3689] & -[2458] & +[1479] & -[3569] & -[2578]\\
+[1468] & -[3489] & -[2459] & +[1579] & -[3567] & -[2678]\\
+[1469] & -[3589] & -[2457] & +[1679] & -[3568] & -[2478]\\
+[1489] & +[3459] & -[2579] & +[1567] & +[3678] & -[2468]\\
+[1568] & +[3478] & -[2469] & +[1589] & +[3457] & -[2679]\\
+[1569] & +[3578] & -[2467] & +[1689] & +[3458] & -[2479]\\
+[4567] & -[6789] & +[4568] & -[4789] & +[4569] & -[5789]\\
+[4578] & +[4679] & +[5689] & +[4578] & +[4679] & +[5689]\\
+[4579] & +[5679] & -[5678] & +[4678] & +[4689] & -[4589]
\end{array}$$}
\caption{Orbits der Brackets unter einer gewählten Symmetrieuntergruppe}
\label{tab}
\end{table}

Da bereits bestimmt wurde, welche Brackets sicherlich zu 0 zu setzen sind,
sind für eine vollständige Vorzeichenliste noch 21 Vorzeichen vorzugeben.
Allerdings reduziert sich die Anzahl der zu untersuchenden Vorzeichenlisten
noch nicht auf $3^{21}$, da Symmetrien $\sigma$ Chirotope $\chi$ sowohl auf
$\chi$ selbst, als auch auf $-\chi$ abbilden können, was von vornherein
nicht abzusehen ist und zu einer Gesamtzahl von möglichen Listen von, im Fall
des Torus, $2^5\cdot 3^{21}< 3.5\cdot 10^{11}$ führt. Der Faktor $2^5$ rührt
daher, daß pro Symmetrie ungleich der Identität $\chi$ oder $-\chi$ vorliegen
kann, also pro Permutation aus G auf $\chi$ und $-\chi$ zu testen ist.
\label{symmtest} Selbst diese Zahl von Vorzeichenlisten ist mit fast 335
Milliarden noch ziemlich hoch, weshalb auch diese Zahl durch weitere
Überlegungen reduziert werden soll, was wie folgt geschieht.

\subsection{Reduktion auf verträgliche Matroide}

Dazu nutzen wir die Tatsache, daß die Vorzeichenlisten letztendlich Chirotopen
entsprechen sollen und daß mit jedem Chirotop ein diesem zugrunde liegendes
Matroid gegeben ist. Von der Matroidseite her betrachtet, muß also zunächst
ein solches vorliegen, um es zu einem Chirotop orientieren zu können.
Betrachtet man die Brackets als Basen eines Matroids, so reduzieren sich nach
der Definition, die das Erfülltsein der Graßmann-Plücker-Relationen über
GF(2) fordert, die Auffüllmöglichkeiten für die "`Bracketvorzeichen"' auf
$2^{21}$, entsprechend der Möglichkeit, für jede Bracket 0 oder 1 zu setzen.
Der nächste Schritt soll dementsprechend der Erzeugung bezüglich der
vorgelegten Symmetrien verträglicher Matroide gewidmet sein. Zur ersten
Vorauswahl werden dazu die unter der Symmetriegruppe in Äquivalenzklassen
aufgeteilten 3-summandigen Graßmann-Plücker-Relationen untersucht.

Zur Erinnerung lassen sich die 3-summandigen Graßmann-Plücker-Relationen
mittels zweier Mengen $A:=\{a_1,\ldots,a_{r-2}\}$ und $B:=\{b_1,\ldots,b_4\}$
paarweise verschiedener Elemente aus $E=\{1,\ldots,npkt\}$ im Rang r
darstellen als
$$\{A|B\}=
  \begin{array}{l}
    +[a_1,\ldots,a_{r-2},b_1,b_2]\cdot [a_1,\ldots,a_{r-2},b_3,b_4]\\
    -[a_1,\ldots,a_{r-2},b_1,b_3]\cdot [a_1,\ldots,a_{r-2},b_2,b_4]\\
    +[a_1,\ldots,a_{r-2},b_1,b_4]\cdot [a_1,\ldots,a_{r-2},b_2,b_3]
  \end{array}=0$$
was zu ${npkt \choose r-2}\cdot{npkt-r+2 \choose 4}$, im Falle des Torus also
${9\choose 2}\cdot{7\choose 4}=1260$ verschiedenen Gleichungen führt.
Betrachtet man diese Gleichungen über dem zweielementigen Körper GF(2), so
müssen äquivalente Gleichungen vom Typ $X + Y + Z + XYZ = 0$ erfüllt sein, in
denen die X, Y und Z den Beträgen der Vorzeichen obiger Produkte entsprechen.
Der Summand $XYZ$ als Produkt über alle in der Gleichung vorkommenden Brackets
muß bei einer Betrachtung über GF(2) hinzugefügt werden, um auch die
erlaubten Fälle mit gleichzeitig $X=1$, $Y=1$ und $Z=1$ abzudecken.

Da die zu entstehenden Matroide bezüglich der vorgelegten Symmetriegruppe G
verträglich sein sollen, braucht man, analog der Brackets, nur Repräsentanten
der Orbits der Graßmann-Plücker-Relationen unter G betrachten, was deren
Anzahl auf 215 reduziert. Ein Teil dieser ist in Tabelle \ref{orbtab}
aufgezeigt.

\begin{table}[htb]
{\small
$$
\begin{array}{ccccc}
\{12|3456\} & \{12|3457\} & \{12|3458\} & \{12|3459\} & \{12|3467\} \\
\{12|3468\} & \{12|3469\} & \{12|3489\} & \{12|3568\} & \{12|3569\} \\
\{12|4567\} & \{12|4568\} & \{12|4569\} & \{12|4578\} & \{12|4579\} \\
\{12|4589\} & \{12|4679\} & \{12|4689\} & \{12|5689\} & \{14|2356\} \\
\ldots      &             &             &             &             \\
\{48|1257\} & \{48|1259\} & \{48|1267\} & \{48|1269\} & \{48|1279\} \\
\{48|1356\} & \{48|1357\} & \{48|1359\} & \{48|1367\} & \{48|1369\} \\
\{48|1379\} & \{48|1567\} & \{48|1569\} & \{48|1579\} & \{48|1679\} \\
\{48|2356\} & \{48|2357\} & \{48|2359\} & \{48|2367\} & \{48|2369\} \\
\{48|2379\} & \{48|2567\} & \{48|2569\} & \{48|2579\} & \{48|2679\} \\
\{48|3567\} & \{48|3569\} & \{48|3579\} & \{48|3679\} & \{48|5679\} \\
\end{array}$$}
\caption{Repräsentanten der Orbits der GPR unter G}
\label{orbtab}
\end{table}

Zu beachten ist hierbei, daß eine Permutation auf den Elementen einer
k-sum\-man\-di\-gen Graßmann-Plücker-Relation das Vorzeichen des
repräsentierten Polynoms ändert. Dies geschieht für Permutationen
$\pi_1:A\to A$, $\pi_2:B\to B$ und $\pi_3:C\to C$ gemäß
$$\{\pi_1(A)|\pi_2(B)|\pi_3(C)\} =
\mbox{sgn}(\pi_1)\cdot\mbox{sgn}(\pi_2)\cdot\mbox{sgn}(\pi_3)\cdot\{A|B|C\}$$
Bei Betrachtungen der Relationen über GF(2) ist dies nicht zu berücksichtigen,
wohl aber im Falle orientierter Matroide über GF(3).

Setzt man in diese 215 Gleichungen wiederum ein, welche Brackets unter der
Symmetriegruppe G bezüglich ihrer Orbits gleich beziehungweise welche Null
sind und streicht alle redundanten Gleichungen (doppelte oder vom Typ
$X + X + 0 = 0$), so erhält man für den Torus letztendlich 204 Gleichungen
(darunter 80 zweisummandige, wenn ein Summand gleich Null war), mit denen nun
mittels Fallunterscheidungen alle zur Symmetriegruppe G gehörigen Matroide
mit 9 Punkten im Rang 4 bestimmt werden können. Tabelle \ref{gltab} zeigt
einen Ausschnitt aus der Liste dieser Gleichungen, mit den Nummern der
Orbitrepräsentanten in runden Klammern.

\begin{table}[htb]
{\small
$$
\begin{array}{lll}
(1)(3)+(1)(4)+(1)(6)+(1)(3)(4)(6) & = & 0 \\
(1)(3)+(1)(5)+(1)(9)+(1)(3)(5)(9) & = & 0 \\
(1)(3)+(1)(6)+(1)(10)+(1)(3)(6)(10) & = & 0 \\
(1)(3)+(1)(7) & = & 0 \\
(1)(4)+(1)(5) & = & 0 \\
... & & \\
(17)(21)+(17)(23) & = & 0 \\
(19)(21)+(19)(23) & = & 0 \\
(19)(21)+(20)(21) & = & 0 \\
(19)(21)+(20)(23) & = & 0 \\
(19)(23)+(20)(21) & = & 0 \\
(19)(23)+(20)(23) & = & 0 \\
(20)(21)+(20)(23) & = & 0 \\
(21)+(23) & = & 0 \\
\end{array}$$}
\caption{Gleichungen, die alle (9,4)-Matroide unter G erfüllen müssen}
\label{gltab}
\end{table}

Die in den Tabellen \ref{orbtab} und \ref{gltab} aufgeführten Beziehungen
der Brackets in den Graßmann-Plücker-Relationen dienen einem C-Programm
{\sc Sym2mat} (siehe dazu den Anhang) als Ausgangsbasis, alle möglichen,
bezüglich einer Symmetriegruppe G und Bracket-Null- beziehungsweise
Bracket-Eins-Setzungen verträglichen Matroide zu erzeugen. Dies geschieht
mittels eines rekursiven Verfahrens, welches vorhandene Abhängigkeiten
ausnutzt und im folgenden beschrieben wird.

Das Programm {\sc Sym2mat} erhält als Eingabe eine Automorphismengruppe G,
die gewünschte Punktanzahl $npkt$ und den Rang $rang$, sowie nach Berechnung
der Repräsentanten der Bracketorbits unter G gezielt zu Null oder Eins
gesetzte Brackets ("`gesetzt"' bezieht sich hier immer auf Untersuchungen
bezüglich GF(2)).\\
Aus diesen Informationen werden nun die (wie in Tabelle \ref{gltab}) unter G
reduzierten dreisummandigen Graßmann-Plücker-Relationen als zu erfüllendes
Gleichungssystem bereitgestellt. Dabei sind aus Gleichungen vom Typ
$$AB + CD + EF + ABCDEF = 0$$
zum Teil Gleichungen mit nur zwei Summanden (ein Bracketorbit gleich Null)
$$AB + CD = 0$$
oder durch Gleichsetzungen Gleichungen mit Quadraten
$$AA + CD + EE + AACDEE = 0$$
oder Gemische aus beiden Typen entstanden, wenn A, B, C, D, E und F
Repräsentanten von Bracketorbits bezeichnen (eine Begründung für das
Erscheinungsbild dieser Gleichungen findet sind in \cite{BoOlRi:91}).\\
Für das Auffüllen mit Nullen und Einsen sind nun zunächst die
zweisummandigen Gleichungen interessant, da aus ihnen unter Eins-Setzung
eines enthaltenen Bracketorbit\-repräsentanten eventuell Gleichheiten anderer
Orbits induziert werden.\\
Unter den vorhandenen Gleichungen werden zu Beginn solche gesucht, die von der
Gestalt $AA + BB = 0$ beziehungsweise $1A + 1B = 0$ und ihrer kommutativen
Äquivalente sind, da hieraus direkt $A = B$ abzuleiten ist. Diese Gleichheit
wird pro Rekursionsstufe in ein "`Bracket-Gleichheitsfeld"' eingetragen und A
als frei zu bestimmende Bracket für folgende Rekursionsschritte gespeichert.\\
Die aus allen vorhandenen Gleichungen resultierenden Gleichheiten werden nun
in das Gleichungssystem übernommen. Ein Sortierungsschritt eliminiert darauf
alle nach Gleichsetzung von Bracketorbitrepräsentanten redundanten (doppelte
oder triviale) Gleichungen, prüft auf direkte Schlüsse der Form
$1 + B = 0 \follows B = 1$ und $1 + 1 + B = 0\follows B = 0$, sowie Erfüllung
der Gleichungen und stellt so immer das aktuelle reduzierte Gleichungssystem
zur Verfügung. Eine Sortierung der Gleichungen zeichnet sich als
programmtechnisch sinnvoll aus und verfährt nach der lexikographischen Ordnung,
nach der für $AB + CD + EF$ immer gelte, daß $A\leq B$, $C\leq D$, $E\leq F$,
sowie $A\leq C$, $A\leq E$, $C\leq E$, wenn wiederum A, B, C, D, E und F
Repräsentanten von Bracketorbits bezeichnen. Zudem seien die Summanden, die
eine Null enthalten in der Gleichung immer zuletzt aufgeführt.\\
Haben sich nach diesem Verfahren keine freien Bracketorbits ergeben, so werden
die verbliebenen ungesetzten als frei angesehen. Im nächsten Schritt
werden in einer Schleife nacheinander die freien Bracketorbits abwechselnd auf
0 und 1 gesetzt und obige Prozedur wiederholt, woraus sich die Rekursion ergibt.

Folge dieser Vorgehensweise ist der Aufbau einer Baumstruktur von
Bracketorbitäquivalenzen, in der, da alle möglichen Abhängigkeiten und
Gleichheiten in den einzelnen Ästen berücksichtigt werden, alle möglichen
bezüglich der Symmetriegruppe G verträglichen Matroide erzeugt werden.

\begin{figure}[p]
\begin{center}
\btab{ll}
{\bf Routine} & {\sf Löse\_Gleichungssystem\_über\_GF(2)}\\
          & \\
Eingabe : & - Anzahl der Gleichungen \\
          & - Das Gleichungssystem \\
          & - Gesetzte Bracketorbits \\
          & - Abhängigkeiten der Bracketorbits untereinander \\
          & \\
Schritt 1 : & - Untersuche alle Gleichungen auf Gleichheiten \\
            & \hspace*{3ex} $\left.\begin{array}{l}
               AA+BB=0\\
               AA+1B=0\\
               AA+B1=0\\
               1A+BB=0\\
               1A+1B=0\\
               1A+B1=0\\
               A1+BB=0\\
               A1+1B=0\\
               A1+B1=0\end{array}\right\} \follows A = B,~A\mbox{ frei}$\\
            & - Minimiere Gleichheiten : $A=B,~B=C\follows A=C$\\
            & - Übertrage Gleichheiten in das Gleichungssystem\\
            & - Sortiere und reduziere Gleichungssystem\\
            & \hspace*{3ex} Überprüfe dabei Schlüsse auf $A=1,~A=0$\\
            & \hspace*{3ex} Überprüfe dabei, ob Gleichungen erfüllt sind\\
            & - Bei Erfüllung aller Gleichungen, schreibe mögliches Matroid\\
Schritt 2 : & - Ermittlung weiterer freier Bracketorbits aus\\
            & \hspace*{3ex} $\left.\begin{array}{l}
              AB+AC=0\\
              BA+CA=0\\
              BA+AC=0\end{array}\right\} \follows A\mbox{ frei}$\\
            & - Bis jetzt keine Freien, so Ungesetzte frei\\
Schritt 3 : & - Setzen der freien Bracketorbits auf 0 und 1\\
            & \hspace*{3ex} Gehe in einer Schleife alle Orbits durch\\
            & \hspace*{4ex} Setze freien Orbit auf 0 bzw. 1\\
            & \hspace*{4ex} {\sf Löse\_Gleichungssystem\_über\_GF(2)}\\
            & \hspace*{4ex} mit neuen Annahmen
\etab
\caption{Routine zur Bestimmung möglicher verträglicher Matroide}
\label{solvegpr}
\end{center}
\end{figure}

Die Ausgabe von {\sc Sym2mat} liefert so die dreisummandigen
Graßmann-Plücker-Relationen erfüllende Elementlisten, die entsprechend der
über GF(2) betrachteten Brackets jedem $[\lambda]$ mit $\lambda\in\Lambda(n,r)$
eine 0 oder 1 zuweist. Da das Erfülltsein der dreisummandigen
Graßmann-Plücker-Relationen nur eine notwendige Bedingung für das Vorliegen
eines Matroids darstellt, müßten nun in einem folgenden Schritt die
Elementlisten ebenfalls mit den übrigen k-summandigen
Graßmann-Plücker-Relationen über GF(2) getestet werden, damit letztendlich
nur Matroide vorliegen, die dann verträglich zur Symmetriegruppe G orientiert
werden sollen.
{\scsi
Hierzu haben Bokowski et al. gezeigt, daß über GF(2) die sogenannten
"`Odd-Polynomials"' zu den Graßmann-Plücker-Polynomen, die aufsummierten
ungeraden elementarsymmetrischen Funktionen der auftretenden Bracketprodukte,
darauf zu überprüfen sind, ob sie mit den vorgegebenen Betragswerten der
Elementeliste Null ergeben. Als Beispiel für die 4-summandigen
Graßmann-Plücker-Relationen sind so Gleichungen vom Typ
$A+B+C+D+ABC+ABD+ACD+BCD=0$ als Summe der ersten und dritten
elementarsymmetrischen Funktion zu testen, wenn A,B,C und D jeweils das
Produkt zweier Brackets darstellen.
}

Für das Torusbeispiel von oben liefert {\sc Sym2mat} 43 verschiedene
Elementlisten, von denen ein Teil in Tabelle \ref{tormat} dargestellt ist.
Dabei kann das triviale Matroid (erste Zeile, nur Nullen) von vornherein als
"`unerwünschtes"' gestrichen werden, da es sicher jede Symmetrie erfüllt,
aber keine Information über den vorgelegten Komplex liefert. Interessant sind
gerade solche Matroide, die lineare Unabhängigkeiten, eine "`Räumlichkeit"'
des Komplexes induzieren, dazu aber gleich mehr.

\begin{table}[htb]
\begin{center}
{\scriptsize\tt
\btab{c}
000000000000000000000000000000000000000000000000000000000000000\\
000000000000000000000000000000000000000000000000000000000000000\\
\hline
000000000000000000000000000000000000000000000000000000000000000\\
000000000000000000000000000000000000000000000000111011101111011\\
\hline
000000000000000000000000000000000000000000000000100000100000000\\
000000000000100010000000000000100000000001000000000000000000000\\
\hline
$\vdots$\\
\hline
111111011111111111011101111111111101111111111110011011011111101\\
111111101111010101111111101110011110111110111111111011101111011\\
\hline
111111011111111111011101111111111101111111111111111111111111101\\
111111101111111111111111111111111111111111111111000000000000000\\
\hline
111111011111111111011101111111111101111111111111111111111111101\\
111111101111111111111111111111111111111111111111111011101111011
\etab
}
\end{center}
\caption{Mögliche Matroid-Elementlisten zum Torusbeispiel}
\label{tormat}
\end{table}

{\sc Sym2mat} liefert ausgehend vom Test der nur dreisummandigen
Graßmann-Plücker-Polynome "`möglicherweise"' Matroide. Um nun sicherzugehen,
daß für die weitere Bearbeitung nur wirkliche Matroide vorliegen, müssen die
ausgegebenen Elementlisten darauf getestet werden, ob sie Matroide darstellen
oder nicht. Dazu können, wie angedeutet, die übrigen k-summandigen
($4\leq k\leq rang+1$) Graßmann-Plücker-Relationen in Form der
Odd-Polynomials getestet werden. Ebenso kann der Matroidbeweis aber auch auf
dem Nachweis basieren, daß die Elementlisten das Basisaxiom der Definition
eines Matroids erfüllen müssen. Nach \cite{Bj:93}, Seite 81, muß
so überprüft werden, ob die Menge
$${\cal B}=\{[\lambda_1,\ldots,\lambda_d]\neq 0~|~(\lambda_1,\ldots,\lambda_d)
\in\Lambda (n,d)\}$$
die Eigenschaft besitzt, daß für je zwei "`Basen"' $B_1$ und $B_2$ aus
$\cal B$ und alle $b_1\in B_1\backslash B_2$ ein $b'_i\in B_2\backslash B_1$
so existiert, daß $((B_1\backslash b_1)\cup b'_i)$ aus $\cal B$ stammt, was
nach Abbildung \ref{testmatroid} geschehen kann.

\begin{figure}[htb]
\begin{center}
\btab{ll}
{\bf Routine} & {\sf Teste\_auf\_Matroid}\\
          & \\
Eingabe : & - Elementliste $\ul{\cal M}$ des vermeindlichen Matroids \\
          & - Für alle $1\leq i\leq {npkt \choose rang}$ \\
          & \hspace*{2ex} Ist $matroid[i]\neq 0$ \\
          & \hspace*{4ex} Sei $B_1$ die Bracket zu $matroid[i]$ \\
          & \hspace*{4ex} Für alle $1\leq (j\neq i)\leq {npkt \choose rang}$ \\
          & \hspace*{6ex} Ist $matroid[j]\neq 0$ \\
          & \hspace*{8ex} Sei $B_2$ die Bracket zu $matroid[j]$ \\
          & \hspace*{8ex} $P = B_1\backslash B_2$ \\
          & \hspace*{8ex} $Q = B_2\backslash B_1$ \\
          & \hspace*{8ex} Für alle $p\in P$\\
          & \hspace*{10ex} Existiert in Q kein Element q mit \\
          & \hspace*{12ex} $(B_1\backslash p)\cup q\in{\cal B}$\\
          & \hspace*{10ex} so ist $\ul{\cal M}$ kein Matroid\\
Ausgabe : & - Matroid oder nicht
\etab
\caption{Routine zum Test auf die Matroideigenschaft einer Elementliste}
\label{testmatroid}
\end{center}
\end{figure}

Liegt die Elementliste des zu überprüfenden vermeindlichen Matroids als ein
Feld $matroid[i]$ mit $1\leq i\leq {n\choose d}$ und Einträgen 0 oder 1 vor,
etwa wie die Ausgabe von {\sc Sym2mat}, so läßt sich mit einer Routine
entsprechend Abbildung \ref{testmatroid} auf die Matroideigenschaft der Liste
testen. Dazu sind im schlechtesten Fall, wenn alle Basen überprüft werden
müßten, ${npkt\choose rang}\cdot\left({npkt\choose rang}-1\right)\cdot rang^2$
Operationen nötig. Dies ist zwar auf den ersten Blick wesentlich mehr, als die
Überprüfung der Graßmann-Plücker-Relationen ergeben würde, bedenkt man
aber, daß dazu alle k-summandigen ($4\leq k\leq rang+1$) rekursiv erzeugt und
in Bracketgleichungen (die Odd-Poly\-no\-mi\-als) umgewandelt werden
müssen, so ist der (Rechen-)\-Aufwand von der gleichen Größenordnung, so
daß der viel einfacher zu implementierende Basistest auch seine Berechtigung
besitzt.

Führt man diese Überprüfung durch, so ergeben sich aus den 43 vorgelegten
Listen für den Torus 25 nichttriviale Matroide, von denen ein Teil in Tabelle
\ref{torusmat} aufgelistet ist. Diese könnten nun als Eingabe für ein
Matroid-Orientierungsprogramm dienen, welches mit seiner Beschreibung das Ziel
dieser Arbeit darstellt und eine Möglichkeit liefert, zu einem CW-Komplex
über dessen Symmetrien verträgliche orientierte Matroide zu erzeugen.

\begin{table}[htb]
\begin{center}
{\scriptsize\tt
\btab{c}
000000000000000000000000000000000000000000000000000000000000000\\
000000000000000000000000000000000000000000000000111011101111011\\
\hline
000000000000000000000000000000000000011111111111111111100000000\\
000000000111111111111111111001111111111111111110000000000000000\\
\hline
000000000000000000000000000000000000011111111111111111100000000\\
000000000111111111111111111001111111111111111110111011101111011\\
\hline
$\vdots$\\
\hline
111111011111111111011101111111111101111111111110011011011111101\\
111111101111010101111111101110011110111110111111111011101111011\\
\hline
111111011111111111011101111111111101111111111111111111111111101\\
111111101111111111111111111111111111111111111111000000000000000\\
\hline
111111011111111111011101111111111101111111111111111111111111101\\
111111101111111111111111111111111111111111111111111011101111011\\
\etab
}
\end{center}
\caption{Die bezüglich G verträglichen Matroide zum Torus}
\label{torusmat}
\end{table}

Betrachtet man zuvor die Struktur der erzeugten Matroide, so stellt man fest,
daß unter diesen noch viele "`uninteressante"' zu finden sind. Dies sind jene,
die "`zuviele"' Nullen enthalten, entsprechend der Determinantendeutung der
Basen also eine "`flache"' Punktkonfiguration repräsentieren.\\
Um solche von vorn herein ausschließen zu können, kann man sich des
vorgelegten Komplexes bedienen und fordern, welche Brackets sicher nicht Null,
also Basen des Matroids sein sollen. Solche Basen ergeben sich aus den Brackets,
die sich aus den Indizes von Punkten benachbarter Zellen, die gefordert nicht
in einer Ebene liegen sollen, zusammensetzen lassen. Für den Torus ergibt diese
Forderung bei abgeschlossenen benachbarten Zellen $\mbox{conv}\{a,b,c,d\}$ und
$\mbox{conv}\{a,b,e,f\}$ mit den 0-Zellen-Indizes a bis f und der gemeinsamen
Kante $\ol{ab}$, folgende Brackets als Basen eines Matroids:
\begin{center}
{\small
[abce], [abcf], [abde], [abdf], [acde], [acdf],
[acef], [adef], [bcde], [bcdf], [bcef], [bdef]
}
\end{center}
Von den ${6\choose 4}$ Möglichkeiten, aus obigen Punkten eine Bracket zu
bilden also jene, die nicht die Zellen selbst und zwei gegenüberliegende
"`Außenkanten"' (hier etwa $\ol{cd}$ und $\ol{ef}$) beschreiben.

\begin{figure}[htb]
$$
\beginpicture
\unitlength0.6cm
\setlinear
\setcoordinatesystem units <0.6cm,0.6cm>
\setplotarea x from -2 to 2, y from -2 to 2
\plot -1.5 -1.5 -1.5 1.5 1.5 1.5 1.5 -1.5 -1.5 -1.5 /
\plot -1.5 0 1.5 0 /
\put {a} [Br] at -1.55 0
\put {b} [Bl] at  1.55 0
\put {c} [tr] at -1.55 -1.55
\put {d} [tl] at  1.55 -1.55
\put {e} [br] at -1.55  1.55
\put {f} [bl] at  1.55  1.55
\endpicture
$$
\caption{Benachbarte 2-Zellen}
\end{figure}

Eine gezielte Bestimmung der Nichtbasen und "`sicherlich"' Basen ergibt so eine
Grundstruktur für die verträglichen Matroide, die letztendlich über
{\sc Sym2mat} zu echten Rang (d+2) Matroiden führt. Ein Algorithmus für eine
solche Bestimmung läßt sich Abbildung \ref{setbases} entnehmen.

\begin{figure}[htb]
\begin{center}
\btab{ll}
{\bf Routine} & {\sf Setze\_sichere\_Basen}\\
          & \\
Eingabe : & CW-Komplex in Form der Zellenberandungen mit Rang r\\
Prozedur :& Für jedes Paar A, B von (r-2)-Zellen \\
          & \hspace*{2ex} erzeuge A$\cap$B \\
          & \hspace*{2ex} Ist $|A\cap B|\geq r-2$, also zumindest ein\\
          & \hspace*{2ex} (r-3)-Simplex im Rand, so erzeuge folgende Brackets \\
          & \hspace*{4ex} - für alle Auswahlen von (r-1) 0-Zellen aus A \\
          & \hspace*{6ex} die Brackets durch Ergänzung um einen Punkt aus B \\
          & \hspace*{4ex} - für alle Auswahlen von (r-1) 0-Zellen aus B \\
          & \hspace*{6ex} die Brackets durch Ergänzung um einen Punkt aus A \\
          & \hspace*{4ex} Sind die erzeugten Brackets nicht durch Auswahl von \\
          & \hspace*{4ex} Punkten einer (r-2)-Zelle entstanden und sind keine \\
          & \hspace*{4ex} Punkte doppelt vorhanden, so setze diese Brackets als \\
          & \hspace*{4ex} Basen in der Matroidliste auf den Wert 1 \\
          & Setze die Brackets zu Auswahlen von r Punkten einer (r-2)-Zelle \\
          & als Nichtbasis in der Matroidliste auf den Wert 0 \\
Ausgabe : & Liste, in der die Basen und Nichtbasen, sowie freien Brackets \\
          & verzeichnet sind.
\etab
\caption{Routine zur Bestimmung einer Grundstruktur für die Elementlisten}
\label{setbases}
\end{center}
\end{figure}

Damit ergibt sich als Grundstruktur für den Torus folgendes Bild.

{\small\tt
\begin{center}
1111110111?111????011101?11???1?1101111?1?111????1???1?111???0?\\
11?11110111????1???111?1??111???1?1?1??11??1?111111011101111011\\
\end{center}
}

womit von obigen 25 verträglichen Matroiden noch sechs übrigbleiben, die eine
Chance besitzen, zu verträglichen Rang 4 orientierten Matroiden des Torus
zu werden.

\begin{table}[htb]
\begin{center}
{\scriptsize\tt
\btab{c}
111111011101110001011101011010101101111010111110010001011110000\\
110111101110010100011111001110011010100110011111111011101111011\\
\hline
111111011111111111011101111111111101111111111110011011011111101\\
111111101111010101111111101110011110111110111111111011101111011\\
\hline
111111011111111110011101111101111101111111111001111111111101101\\
111111101111101111111101111111101111111111110111111011101111011\\
\hline
111111011111111110011101111101111101111111111111111111111101101\\
111111101111111111111111111111111111111111111111111011101111011\\
\hline
111111011111111111011101111111111101111111111001111111111111101\\
111111101111101111111101111111101111111111110111111011101111011\\
\hline
111111011111111111011101111111111101111111111111111111111111101\\
111111101111111111111111111111111111111111111111111011101111011\\
\etab
}
\end{center}
\caption{Die bezüglich G verträglichen Rang 4 Matroide zum Torus}
\label{torus4mat}
\end{table}

\clearpage
\subsection{Orientierung der erzeugten Matroide}

Im letzten Schritt sollen die erzeugten, bezüglich 0-Zellenanzahl
und Rang eines CW-Komplexes $\C$, sowie einer vorgelegten Symmetriegruppe G
verträglichen Matroide so orientiert werden, daß weiterhin die
Verträglichkeitseigenschaft bezüglich G gewährleistet ist.\\
Dazu werden wiederum zur Minimierung des Rechenaufwandes, die unter den
Symmetrien aus G reduzierten, zu erfüllenden dreisummandigen
Graßmann-Plücker-Relationen eingesetzt. Daß diese erfüllt sind, stellt für
die zu entstehenden orientierten Matroide nun eine notwendige und hinreichende
Bedingung dar, was sich aus dem Satz über die dreisummandigen Relationen von
Seite \pageref{dgpr} ergibt. Ziel ist die rekursive Erzeugung aller zu einem
vorgelegten Matroid bezüglich G verträglichen Orientierungen, sofern solche
überhaupt existieren.

Auf Seite \pageref{symmtest} war davon die Rede, daß die Anzahl der
Möglichkeiten ein orientiertes Matroid zu erzeugen davon abhängt, wie die
Elemente g$\in$G das entsprechende Chirotop $\chi$ abbilden, ob als Rotation
auf sich selbst oder als Reflexion auf sein negatives. Gerade bei der
Bestimmung der Orbits der Brackets unter G ist dies eine wichtige Information.
Erzwänge man nämlich eine bestimmte Variante (etwa alle id$\neq$g$\in$G seien
Rotationen), so kann es passieren, daß kein solches verträgliches orientiertes
Matroid existiert, da sich schon bei der Aufstellung der Bracketbahnen
Widersprüche ergeben.

Für den dreidimensionalen Würfel etwa (acht Punkte im Rang vier mit den
2-Zellen-Berandungen 1264, 2586, 5378, 3147, 4687 und 1253) bei Vorlage der
Symmetriegruppe $\{id, (23)(67), (24)(57), (34)(56), (234)(576), (243)(567)\}$
und dem zugrunde liegenden Matroid

\begin{center}{\small
$1011110111111101101111101110111111111111110111011111011011111110111101$
}\end{center}

ergibt sich für die Basis-Bracket [1234] folgender Orbit:
$$
\begin{array}{|ccc|c|c|c|c|}
\hline
id & \mapsto & (23)(67) & (24)(57) & (34)(56) & (234)(576) & (243)(567) \\
\hline
+[1234] & \mapsto & -[1234] & -[1234] & -[1234] & +[1234] & +[1234] \\
\hline
\end{array}
$$
Egal wie man sich entscheiden würde (alle id$\neq$g$\in$G Reflexionen
beziehungsweise Rotationen), der Orbit würde immer induzieren, daß
$[1234]=-[1234]$, also $[1234]=0$ gelten muß, was einen Widerspruch zum
zugrunde liegenden Matroid darstellt.\\
Hier ließe sich aus obigem Orbit allerdings direkt ablesen, daß es sich bei
den Permutationen (23)(67), (24)(57) und (34)(56) um Reflexionen, sowie bei
(234)(576) und (243)(567) um Rotationen handeln muß. Betrachtet man dazu
die Symmetriegruppe des Würfelchirotops (die identisch zur kombinatorischen
Symmetriegruppe ist), daß durch geeignete Koordinatenwahl (vgl. Abb.
\ref{cube}) entstanden ist, so entspricht dies auch der "`Wirklichkeit"'.
\vskip4mm

\centerline{\small\tt
+0+++-0--+++--0++0+----0+-+0++---+-++++---0++-0-++--0+-0++++---0---+0-
}\vskip4mm

Aus dieser Überlegung läßt sich nun folgender kleiner Hilfssatz ableiten, der
besagt:
\begin{quote}
Wann immer ein Element g einer für ein zu erzeugendes Chirotop geforderten
Symmetriegruppe G eine Bracket $[\lambda]\neq 0$ bis auf eine lexikographische
Sortierung $\sigma$ auf sich abbildet, so bildet in diesem Fall g das
Chirotop $\chi$ auf $\mbox{sgn}(\sigma)\chi$ ab.
\end{quote}
Der Beweis hierzu ist einfach in dem Sinne, daß g als geforderte Symmetrie
$\chi$ sicher auf $\chi$ selbst oder $-\chi$ abbildet. Wird eine Basis-Bracket
$[a_1\ldots a_r]$ des zugrunde liegenden Matroids nun, unter Vorschaltung einer
Sortierungspermutation (eine Permutation, die die $a_1$ bis $a_r$ der Größe
nach sortiert, also auf $\{1,\ldots,r\}$ wirkt), auf sich abgebildet, so ist
hierdurch, aufgrund der Symmetrieforderung, eindeutig festgelegt, ob nach g
$\chi$ oder $-\chi$ vorliegt, da sich die Symmetrie auf alle Basis-Brackets
bezieht.$\Box$

Für die übrigen Symmetrien, die keine Bracket bis auf ihr Vorzeichen
invariant lassen, bleibt nichts anderes übrig, als zunächst beide
Möglichkeiten in Betracht zu ziehen, wie sie auf ein $\chi$ wirken könnten
oder eine Fixierung vorzugeben. Schaltet man vor die Orientierungsversuche zu
einem Matroid einen Test nach obigem Hilfssatz vor, so kann sich hier der
Aufwand aber schon erheblich reduzieren. In bezug auf das Torusbeispiel ergibt
sich so, daß die Symmetrie $(47)(58)(69)$ die Bracket $[1247]$ auf $-[1247]$
abbildet und somit eine Reflexion der Chirotope $\chi$ zum Torus darstellt.
Damit bleiben $2^4$ Möglichkeiten für die anderen Elemente aus G, wie die
Bracketorbits bezüglich ihrer Gleichheiten zu deuten sind.

Ein erster Eindruck, wie ein Matroidorientierungsprogramm aussehen könnte,
ergibt sich nun wie folgt.
\begin{enumerate}
\item Lege eine Automorphismengruppe G und eine verträgliche Matroidliste
      für n Punkte im Rang r vor.
\item Bestimme die Bracketorbits unter G und prüfe auf Selbstabbildungen in
      obigem Sinne. Liegen solche vor, so merke den Symmetrietyp (Rot/Ref).
\item Bestimme die Orbits der dreisummandigen Graßmann-Plücker-Polynome
      unter G als Ausgangsebene für ein System zur Bestimmung der zugehörigen
      Chirotopvorzeichen.
\item Setze nacheinander alle Kombinationen für die unbestimmten Symmetrietypen
      in die Bracketorbits ein, so daß sich nun die gewünschten
      Bracketgleichheiten ergeben und
\begin{itemize}
\item Stelle das zugehörige "`Gleichungs"'system durch Einsetzen der
      Bracket\-orbitrepräsentanten (Vorzeichenwechsel berücksichtigen)
      in die GPP auf.
\item Ermittle die zugehörigen Chirotope durch Auffüllen und Erschließen von
      Vorzeichen für die Brackets in den GPP.
\end{itemize}
\item Gib alle ermittelten Vorzeichenlisten der Chirotope aus.
\end{enumerate}

Das Einsetzen und Erschließen von Vorzeichen in den GPP begründet sich
in der Deutung der Brackets als Determinanten von (r$\times$r)-Teilmatrizen
einer (n$\times$r)-Matrix, für die die Graßmann-Plücker-Relationen erfüllt
sind.\\
Stellt man das zu untersuchende Gleichungssystems auf, so zeigt sich noch eine
weitere Möglichkeit, auftretende Symmetrietypen als unsinnig zu erkennen.
Treten nämlich unter Berücksichtigung aller Sortierungsvorzeichen und
Nullsetzungen von Bracketorbits Gleichungen vom Typ
$(+[A])(+[B]) + (+[A])(+[B]) = 0$ auf, so kann die Wahl der Art der
Abbildungsvorschrift von $\chi\mapsto\pm\chi$ als nicht richtig angesehen
werden, da nichtsinguläre Matrizen A und B diese Gleichung nicht erfüllen
können.\\
Genauer ergibt sich für die dreisummandigen Gleichungen so, daß die
Vorzeichen der Determinanten derart gegeben sind, daß entweder genau ein
Summand (ein Bracketprodukt) positiv oder genau einer negativ ist. Ist ein
Summand 0, so sind die anderen beiden von entgegengesetztem Vorzeichen, denn nur
so ist zu erreichen, daß die Gleichung zu Null werden kann, wenn alle
Determinanten nichtnull sind. Für die Gleichungen
$$ A + (-B) + C = 0$$
ergeben sich so die Vorzeichen der Bracketprodukte A, $-$B und C zu
$$(+,-,+),~(+,-,-),~(+,+,-),~(-,-,+),~(-,+,+),~(-,+,-)$$
Solche Konstellationen gilt es nun im vorgelegten Gleichungssystem zu erzeugen.
Dazu bedienen wir uns zunächst wieder der Gleichungen, die nach dem Ersetzen
der Brackets durch ihre Repräsentanten unter G nur noch zwei Summanden
enthalten, die nichtnull sind, während der dritte durch eine Nichtbasis des
zugrunde liegenden Matroids verschwunden ist. Wieder kann bei Gleichungen
vom Typ AB + AC auf das Verhalten von B und C geschlossen werden, wenn A
gesetzt wurde.

Grundsätzlich ergeben sich aus den zweisummandigen Gleichungen folgende zu
be\-rücksichtigenden Möglichkeiten:
{\small
$$
\begin{array}{|l|l|l|l|}
\multicolumn{4}{l}{A=+ \mbox{ für }} \\
\hline
-(A+)~+~(BB) & -(+A)~+~(BB) & (A-)~+~(BB) & (-A)~+~(BB) \\
\hline
\multicolumn{4}{l}{B=+ \mbox{ für }} \\
\hline
-(AA)~+~(+B) & -(AA)~+~(B+) & (AA)~+~(-B) & (AA)~+~(B-) \\
\hline
\multicolumn{4}{l}{A=- \mbox{ für }} \\
\hline
-(A-)~+~(BB) & (A+)~+~(BB) & -(-A)~+~(BB) & (+A)~+~(BB) \\
\hline
\multicolumn{4}{l}{B=- \mbox{ für }} \\
\hline
-(AA)~+~(-B) & (AA)~+~(+B) & -(AA)~+~(B-) & (AA)~+~(B+) \\
\hline
\multicolumn{4}{l}{A=B \mbox{ für }} \\
\hline
 (+A)~+~(-B) &  (+A)~+~(B-) &  (-A)~+~(+B) &  (-A)~+~(B+) \\
 (A+)~+~(-B) &  (A+)~+~(B-) &  (A-)~+~(+B) &  (A-)~+~(B+) \\
-(+A)~+~(+B) & -(+A)~+~(B+) & -(-A)~+~(-B) & -(-A)~+~(B-) \\
-(A+)~+~(+B) & -(A+)~+~(B+) & -(A-)~+~(-B) & -(A-)~+~(B-) \\
\hline
\multicolumn{4}{l}{A=-B \mbox{ für }} \\
\hline
-(A+)~+~(B-) &  (A+)~+~(+B) & -(A-)~+~(B+) &  (A-)~+~(-B) \\
 (+A)~+~(+B) & -(+A)~+~(B-) & -(-A)~+~(B+) &  (-A)~+~(-B) \\
 (A+)~+~(B+) & -(A+)~+~(-B) & -(A-)~+~(+B) &  (A-)~+~(B-) \\
 (+A)~+~(B+) & -(+A)~+~(-B) & -(-A)~+~(+B) &  (-A)~+~(B-) \\
\hline
\multicolumn{4}{l}{A=B \mbox{ oder } A=-B \mbox{ für }} \\
\hline
-(AA)~+~(BB) & & & \\
\hline
\end{array}
$$}

Weiterhin lassen sich Vorzeichen aus dreisummandigen Gleichungen erschließen,
wenn genau ein Bracketvorzeichen unbestimmt ist. Gilt dann, daß das Vorzeichen
der beiden anderen Summanden gleich ist, so muß mit dem ungesetzten das
Vorzeichen des noch unbestimmten Summanden negativ dem der anderen sein.
Etwa gilt für $(++)~+~(--)~+~(-A)\fol (+,+,?)$, daß $A=+$ sein muß, damit
der letzte Summand negatives Vorzeichen erhält. Frei ist die Wahl von A, wenn
die bestimmten Summanden entgegengesetztes Vorzeichen besitzen.

Ist in einer Gleichung mehr als ein Vorzeichen unbestimmt, so können die
jeweils ungesetzen Bracketvorzeichen als Freiheitsgrade für die Vorzeichenwahl
angesehen werden. Ein Setzen dieser, sowie ein Erschließen weiterer Vorzeichen
nach obiger Auswahl, liefert bei rekursiver Bearbeitung des Gleichungssystems
wie bei der Matroiderzeugung eine Baumstruktur, in der alle verträglichen
orientierten Matroide mit der gewünschten Symmetrieeigenschaft erzeugt werden.

Mittels eines so aufgebauten Such- und Ersetzprogramms können nun die
zuvor bereitgestellten Matroide orientiert werden, was für das Torusbeispiel
für die vier orientierbaren Matroide zu den Vorzeichenlisten aus den Tabellen
\ref{torusom1} bis \ref{torusom6} führt. Hierbei ergibt sich zunächst, das
bei Vorgabe der Symmetriegruppe G mit dem erzeugenden Element
$\sigma = (132)(495768)$ folgende Symmetrietypen zuläßig sind:
$$\begin{array}{c|c|c|c|c|c}
\mbox{id} & \sigma & \sigma^2 & \sigma^3 & \sigma^4 & \sigma^5 \\
\hline
+ & + & + & - & - & - \\
+ & + & - & - & - & + \\
+ & - & + & - & + & - \\
+ & - & - & - & + & + \\
\end{array}$$
Hierbei bedeutet $+$ die Wirkung von $\sigma^i$ als Rotation und $-$ als
Reflektion auf die zu entstehenden $\chi$. Besonders wichtig ist, daß man
den gewählten Symmetrietyp beim Ermitteln der vollständigen Vorzeichenliste
wieder berücksichtigt, wenn man aus den Bracketorbitrepräsentanten die
übrigen Brackets rekonstruiert. Berücksichtigt man noch, daß der Symmetrietyp
von Quadraten von Permutationen eindeutig als Rotation festgelegt ist
($\mbox{typ}(\sigma)^{2k}=1$), so bleibt hier als einzig zuläßiger
Symmetrietyp
$$\begin{array}{c|c|c|c|c|c}
\mbox{id} & \sigma & \sigma^2 & \sigma^3 & \sigma^4 & \sigma^5 \\
\hline
+ & - & + & - & + & -
\end{array}$$
übrig. Was bei kleinen vorgelegten Gruppen, noch leicht von Hand zu lösen
ist, ist bei größeren Gruppen, deren Elemente zusammengesetzt sind, zwar
auch, aber mit ungleich höherem Aufwand zu bewerkstelligen, was so zur Zeit
noch nicht implementiert ist.

\begin{table}[htb]
Gewähltes Matroid:
\begin{center}
{\scriptsize\tt
\begin{tabular}{c}
111111011101110001011101011010101101111010111110010001011110000\\
110111101110010100011111001110011010100110011111111011101111011
\end{tabular}}\end{center}\vskip2mm

Vorgelegte Automorphismengruppe:
{\small
$$\{\mbox{id},(132)(495768),(123)(456)(789),(47)(58)(69),(132)(465)(798),
   (123)(486759)\}$$}\vskip2mm

Erzeugte Chirotope:
\begin{center}
{\scriptsize\tt\begin{tabular}{c}
+++---0+--0+--000+0---0+0++0-0+0++0-+--0+0++--+00+000-0---+0000\\
--0--++0+--00-0+000--+++00+-+00++0+0-00--00+--+----0++-0+---0--\\
\hline
+++---0-++0-++000-0+++0-0--0+0-0--0++--0+0++--+00+000-0-++-0000\\
++0++--0+--00-0+000--+++00+-+00++0+0-00--00+--+-+++0--+0-+++0++\\
\hline
---+++0+--0+--000+0---0+0++0-0+0++0--++0-0--++-00-000+0+--+0000\\
--0--++0-++00+0-000++---00-+-00--0-0+00++00-++-+---0++-0+---0--\\
\hline
---+++0-++0-++000-0+++0-0--0+0-0--0+-++0-0--++-00-000+0+++-0000\\
++0++--0-++00+0-000++---00-+-00--0-0+00++00-++-++++0--+0-+++0++\\
\end{tabular}}
\end{center}
\caption{\label{torusom1} Chirotope zum ersten Torusmatroid}
\end{table}

\begin{table}[htb]
\begin{center}
{\scriptsize\tt
\begin{tabular}{c}
111111011111111110011101111101111101111111111111111111111101101\\
111111101111111111111111111111111111111111111111111011101111011\\
\hline\hline
+++---0+---+-----00---0+++++0+++++0-+---++++--++++--+-----0--0-\\
-----++0+------+++---++++++-+++++++--+-----+--+----0++-0+---0--\\
\hline
+++---0-+++-+++++00+++0-----0-----0++---++++--++++--+---++0++0+\\
+++++--0+------+++---++++++-+++++++--+-----+--+-+++0--+0-+++0++\\
\hline
---+++0+---+-----00---0+++++0+++++0--+++----++----++-+++--0--0-\\
-----++0-++++++---+++------+-------++-+++++-++-+---0++-0+---0--\\
\hline
---+++0-+++-+++++00+++0-----0-----0+-+++----++----++-+++++0++0+\\
+++++--0-++++++---+++------+-------++-+++++-++-++++0--+0-+++0++\\
\end{tabular}}
\end{center}
\caption{\label{torusom4} Chirotope zum vierten Torusmatroid}
\end{table}

\begin{table}[htb]
\begin{center}
{\scriptsize\tt
\begin{tabular}{c}
111111011111111111011101111111111101111111111001111111111111101\\
111111101111101111111101111111101111111111110111111011101111011\\
\hline\hline
+++---0+--++--++++0---0+-++---+-++0-+--++-++-00--+++--+---+++0+\\
--+--++0+--++0++--+--+0+--+-+--0+-++--+--+++0-+-+++0--+0-+++0++\\
\hline
+++---0-++--++----0+++0-+--+++-+--0++--++-++-00--+++--+-++---0-\\
++-++--0+--++0++--+--+0+--+-+--0+-++--+--+++0-+----0++-0+---0--\\
\hline
---+++0+--++--++++0---0+-++---+-++0--++--+--+00++---++-+--+++0+\\
--+--++0-++--0--++-++-0-++-+-++0-+--++-++---0+-++++0--+0-+++0++\\
\hline
---+++0-++--++----0+++0-+--+++-+--0+-++--+--+00++---++-+++---0-\\
++-++--0-++--0--++-++-0-++-+-++0-+--++-++---0+-+---0++-0+---0--\\
\end{tabular}}
\end{center}
\caption{\label{torusom5} Chirotope zum fünften Torusmatroid}
\end{table}

\begin{table}[htb]
\begin{center}
{\scriptsize\tt
\begin{tabular}{c}
111111011111111111011101111111111101111111111111111111111111101\\
111111101111111111111111111111111111111111111111111011101111011\\
\hline\hline
+++---0+++++++++++0---0---+------+0-++++----++----++-++---+++0+\\
+++++++0+++++++---+++-------+------++-+++++-++--+++0--+0-+++0++\\
\hline
+++---0+---+-----+0---0+++++-+++++0-+---++++--++++--+-----+--0-\\
-----++0+------+++---++++++-+++++++--+-----+--+----0++-0+---0--\\
\hline
+++---0+---+------0---0+++++++++++0-+---++++--++++--+--------0-\\
-----++0+------+++---++++++-+++++++--+-----+--+----0++-0+---0--\\
\hline
+++---0+--++--++++0---0+-++---+-++0-+--++-++-+---+++--+---+++0+\\
--+--++0+--+++++--+--+-+--+-+---+-++--+--++++-+-+++0--+0-+++0++\\
\hline
+++---0+--++--++++0---0+-++---+-++0-+--++-++--+--+++--+---+++0+\\
--+--++0+--++-++--+--+++--+-+--++-++--+--+++--+-+++0--+0-+++0++\\
\hline
+++---0+--++--++++0---0+-++---+-++0-+--++-++--+--+++--+---+++0+\\
--+--++0+--++-++--+--+++--+-+--++-++--+--+++--+----0++-0+---0--\\
\hline
+++---0-+++-++++++0+++0-----------0++---++++--++++--+---+++++0+\\
+++++--0+------+++---++++++-+++++++--+-----+--+-+++0--+0-+++0++\\
\hline
+++---0-+++-+++++-0+++0-----+-----0++---++++--++++--+---++-++0+\\
+++++--0+------+++---++++++-+++++++--+-----+--+-+++0--+0-+++0++\\
\hline
+++---0-++--++----0+++0-+--+++-+--0++--++-++-+---+++--+-++---0-\\
++-++--0+--+++++--+--+-+--+-+---+-++--+--++++-+----0++-0+---0--\\
\hline
+++---0-++--++----0+++0-+--+++-+--0++--++-++--+--+++--+-++---0-\\
++-++--0+--++-++--+--+++--+-+--++-++--+--+++--+-+++0--+0-+++0++\\
\hline
+++---0-++--++----0+++0-+--+++-+--0++--++-++--+--+++--+-++---0-\\
++-++--0+--++-++--+--+++--+-+--++-++--+--+++--+----0++-0+---0--\\
\hline
+++---0-----------0+++0+++-++++++-0+++++----++----++-++-++---0-\\
-------0+++++++---+++-------+------++-+++++-++-----0++-0+---0--\\
\hline
---+++0+++++++++++0---0---+------+0-----++++--++++--+--+--+++0+\\
+++++++0-------+++---+++++++-++++++--+-----+--+++++0--+0-+++0++\\
\hline
---+++0+--++--++++0---0+-++---+-++0--++--+--++-++---++-+--+++0+\\
--+--++0-++--+--++-++---++-+-++--+--++-++---++-++++0--+0-+++0++\\
\hline
---+++0+--++--++++0---0+-++---+-++0--++--+--++-++---++-+--+++0+\\
--+--++0-++--+--++-++---++-+-++--+--++-++---++-+---0++-0+---0--\\
\hline
---+++0+--++--++++0---0+-++---+-++0--++--+--+-+++---++-+--+++0+\\
--+--++0-++-----++-++-+-++-+-+++-+--++-++----+-++++0--+0-+++0++\\
\hline
---+++0+---+-----+0---0+++++-+++++0--+++----++----++-+++--+--0-\\
-----++0-++++++---+++------+-------++-+++++-++-+---0++-0+---0--\\
\hline
---+++0+---+------0---0+++++++++++0--+++----++----++-+++-----0-\\
-----++0-++++++---+++------+-------++-+++++-++-+---0++-0+---0--\\
\hline
---+++0-++--++----0+++0-+--+++-+--0+-++--+--++-++---++-+++---0-\\
++-++--0-++--+--++-++---++-+-++--+--++-++---++-++++0--+0-+++0++\\
\hline
---+++0-++--++----0+++0-+--+++-+--0+-++--+--++-++---++-+++---0-\\
++-++--0-++--+--++-++---++-+-++--+--++-++---++-+---0++-0+---0--\\
\hline
---+++0-++--++----0+++0-+--+++-+--0+-++--+--+-+++---++-+++---0-\\
++-++--0-++-----++-++-+-++-+-+++-+--++-++----+-+---0++-0+---0--\\
\hline
---+++0-+++-++++++0+++0-----------0+-+++----++----++-++++++++0+\\
+++++--0-++++++---+++------+-------++-+++++-++-++++0--+0-+++0++\\
\hline
---+++0-+++-+++++-0+++0-----+-----0+-+++----++----++-+++++-++0+\\
+++++--0-++++++---+++------+-------++-+++++-++-++++0--+0-+++0++\\
\hline
---+++0-----------0+++0+++-++++++-0+----++++--++++--+--+++---0-\\
-------0-------+++---+++++++-++++++--+-----+--++---0++-0+---0--\\
\end{tabular}}
\end{center}
\caption{\label{torusom6} Chirotope zum sechsten Torusmatroid}
\end{table}

\clearpage
\section{Zusammenfassung der Vorgehensweise}

Da nun alle Schritte abgearbeitet wurden, die zu einem vorgelegten CW-Komplex
(angegebenen Aufbaus) über Symmetrie(unter)gruppen verträgliche orientierte
Matroide in Form von Chirotopen erzeugen, ist das Ziel dieser Arbeit erreicht.

Zusammenfassend kann die "`verträgliche Chirotoperzeugung"' nun wie folgt
beschrieben werden:

\begin{itemize}
\item Gegeben sei ein CW-Komplex $\C$, mit (im Fall eines 2-Komplexes)
      polygonal berandeten 2-Zellen.
\item Bestimme zu $\C$ dessen kombinatorische Symmetriegruppe Aut($\C$).
\item Bestimme eine Grundstruktur für die verträglichen Matroide.
\item Wähle aus Aut($\C$) eine "`geeignete"' Untergruppe aus und lege
      diese als feste (Untergruppe einer) Symmetriegruppe G verträglicher
      orientierter Matroide vor.
\item Erzeuge rekursiv bezüglich G verträgliche Matroide $\ul{\cal M}$
      mit der gewünschten Grundstruktur.
\item Versuche jedes so entstandene verträgliche Matroid $\ul{\cal M}$ zu
      orientieren, so daß $\ul{\cal M}$ zu bezüglich G verträglichen
      orientierten Matroiden $\cal M$ in Form von Chirotopen $\chi$ wird.
\item Nach der Bearbeitung liegen nun zu $\C$ im Existenzfall alle
      bezüglich $G\leq \mbox{Aut}(\C)$ verträglichen Chirotope $\chi$ vor,
      deren zugrunde liegendes Matroid die gewünschte Grundstruktur besitzt
      und eine Symmetriegruppe aufweisen, die zumindest G enthält.
\end{itemize}

In dem nun anschließenden letzten Kapitel sollen dazu noch einige Beispiele
bearbeitet werden.
