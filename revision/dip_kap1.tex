\chapter{Grundlagen}

\section{Grundbegriffe der Topologie}

Um Komplexe allgemein und CW-Komplexe im besonderen definieren zu können,
ist es nötig, einige grundlegende Definitionen und Eigenschaften
topologischer Räume bereitzustellen, die nicht nur hier, sondern in vielen
anderen Bereichen der Mathematik zum "`Grundvokabular"' gehören.

\subsection{Topologische Räume}

Ist X eine beliebige Menge, so kann man auf dieser die Potenzmenge
${\cal P}(X)$, die Menge aller Teilmengen von X, betrachten. Wählt man aus
${\cal P}(X)$ eine Teilmenge $\T$ aus, deren Elemente die drei Bedingungen

\bcent
\btab{ll}
1. & $\emptyset\in\T$ und $X\in\T$\\
2. & $T_1,T_2\in\T~\fol~T_1\cap T_2\in\T$\\
3. & $T_i\in\T$ für alle i aus einer Indexmenge I
     $\fol~\Cup\limits_{i\in I} T_i~\in~\T$\\
\etab
\ecent

erfüllen, so nennt man $\T$ eine {\bf Topologie} auf X\idx{Topologie auf X}
und Elemente $T\in\T$ {\bf offene Mengen}\idx{offene Mengen} des {\bf
topologischen Raumes}\idx{topologischer Raum} (X,$\T$), des Paares X zusammen
mit der Topologie $\T$. Die Komplemente offener Mengen nennt man {\bf
abgeschlossen}\idx{abgeschlossen}.\\
{\scsi
Ebenso könnte man eine Topologie $\T$ auf X auch mittels abgeschlossener Mengen
definieren. In diesem Fall wäre dann das Enthaltensein endlicher Vereinigungen
und beliebiger Durchschnitte abgeschlossener Mengen aus $\T$ zu fordern,
was der Komplementbildung obiger Bedingungen entspricht.
}\\
Ist Y eine Teilmenge eines topologischen Raumes (X,$\T$), so wird diese
selbst zu einem topologischen Raum, wenn man auf ihr die Topologie von X
induziert, das heißt als offene Mengen der Topologie $\T|Y$ die Mengen
$\{T\cap Y~|~T\in\T\}$ wählt. $\T|Y$ nennt man dann die von X auf Y
{\bf induzierte Topologie}, die oft auch als {\bf Spurtopologie} auf
Y\idx{Spurtopologie}\idx{induzierte Topologie} bezeichnet wird.

Als Beispiele für Topologien können etwa die durch Metriken
d metrischer Räume (X,d) induzierten Topologien angeführt werden.\\
{\scsi
Zur Erinnerung hier die Definition einer Metrik und eines metrischen Raumes.
Sei dazu X eine Menge. Eine Funktion $d:X\times X\to\R$ heißt Metrik, wenn\\
1. d(x,y) = d(y,x) $\geq$ 0\\
2. d(x,y) = 0 $\iff$ x = y\\
3. d(x,z) $\leq$ d(x,y) + d(y,z),\\
gilt. Das Paar (X,d) heißt dann ein metrischer Raum. Betrachtet man etwa
$(\R,\|\cdot\|_2)$, so stellen metrische Räume oft verwendete Spezialfälle
topologischer Räume dar.
}\\
Die offenen Mengen dieser durch Metriken induzierten Topologien sind gegeben
durch
$$\T_d=\{U\subseteq X:(\forall~p\in U) (\exists~\varepsilon > 0)
\mbox{ mit } \{y:d(y,p)<\varepsilon\}\subseteq U\}.$$
Sei $\|\cdot\|:V\to\R$ eine Abbildung eines Vektorraumes V über einem Körper
$\K$ in die reellen Zahlen. $\|\cdot\|$ heißt eine {\bf Norm}\idx{Norm} auf V,
wenn für alle $v,w\in V$ und $k\in\K$ $\|v\|$ dann und nur dann gleich 0 ist,
wenn $v=0$ gilt, $\|k\cdot v\|$ gleich $|k|\cdot\|v\|$ ist, sowie $\|v+w\|\leq
\|v\|+\|w\|$ gilt. (V,$\|\cdot\|$) heißt dann ein {\bf normierter Raum}.
\idx{normierter Raum} Normierte Räume sind metrische Räume, wenn man für
die d(x,y) die Norm der Differenz $\|x-y\|$ von x und y setzt.

Der $\R^n$ ist bezüglich der durch die Euklidische Norm $\|x\|_2 :=
\sqrt{\sum_{i=1}^n x_i^2}$ und der so induzierten Metrik gegebenen
offenen Mengen ein topologischer Raum, der auch als {\bf Euklidischer Raum}
$\E^n$ bezeichnet wird. Oft spricht man hier auch von der "`üblichen"'
Topologie auf $\R^n$. (Im folgenden sei $\R$ und $\E$ synonym verwendet, da immer
der Euklidische Raum gemeint sein wird.)

Spielt es keine Rolle, wie die Topologie eines topologischen Raumes
geartet ist, werden Aussagen über beliebige topologische Räume gemacht
oder ist es klar, welche Topologie (die "`übliche"') gemeint ist, so spricht
man oft der Einfachheit halber von einem topologischen Raum X, ohne
zusätzliche Angabe von $\T$.

Eine Abbildung $f:X\to Y$ zwischen topologischen Räumen X und Y heißt
{\bf stetig},\idx{stetig} wenn die Urbilder offener Mengen in Y offen in X sind.
Dies bedeutet, daß es zu jeder offenen Menge V in Y eine offene Menge U in X
so gibt, daß f(U) = V gilt.
Ist $f:X\to Y$ eine bijektive Abbildung zwischen X und Y, die selbst
und deren Umkehrabbildung $f^{-1}:Y\to X$ stetig sind, so heißt f ein
{\bf Homöomorphismus}\idx{Hom\öomorphismus}. Gibt es zwischen zwei
topologischen Räume X und Y solch einen Homöomorphismus, so heißen X und Y
{\bf homöomorph}\idx{hom\öomorph}.
Als wichtiges Beispiel sei angeführt, daß die n-dimensionale offene Vollkugel
(in einem normierten Raum) $\stackrel{\circ}{B^n} := \{x~:~\|x\|< 1\}$
homöomorph dem $\R^n$ ist (als Abbildung wähle man etwa
$\phi:\stackrel{\circ}{B^n}\to\R^n$ mit $\phi(x)=\frac{1}{1-\|x\|}x$). Weiter
ist die (n$-$1)-Sphäre $\S^{n-1}:=\{x~:~\|x\|=1\}$, der Rand der Vollkugel
$B^n$, von der man einen Punkt wegläßt, sie "`punktiert"', homöomorph dem
$\R^{n-1}$. (Man nutze die stereographische Projektion
$\phi:\S^{n-1}-\{e_n\}\to\R^{n-1}$ mit
$y=(\frac{x_1}{1-x_n},\ldots,\frac{x_{n-1}}{1-x_n})$).

Bestimmte Teilmengen topologischer Räume können nun durch folgende Begriffe
charakterisiert werden. Ist M eine beliebige Teilmenge eines
topologischen Raumes (X,$\T$), so heißt eine weitere Teilmenge $U\subset X$
eine {\bf Umgebung}\idx{Umgebung} von M, falls es eine offene Menge T in $\T$
gibt, so daß $M\subset T\subset U$ gilt.
Ein Punkt p aus M ist {\bf innerer Punkt}\idx{innerer Punkt} von M, falls M
Umgebung von  $\{$p$\}$ ist, p ist {\bf Berührungspunkt}\idx{Ber\ührungspunkt}
von M, falls jede Umgebung von p nichtleeren Schnitt mit M hat und p ist ein
{\bf Häufungspunkt}\idx{H\äufungspunkt}, falls p Berührungspunkt von
M$-\{$p$\}$ (M ohne den Punkt p) ist. Ein Punkt p heißt {\bf Randpunkt}
\idx{Randpunkt} von M, wenn er Berührungspunkt von M und X$-$M ist.
Aus der Definition einer Topologie mittels abgeschlossener Mengen
folgt, daß jede offene Menge $T\in\T$ in einer kleinsten abgeschlossenen
Menge, dem {\bf Abschluß} $\ol{T}$\idx{Abschlu\3} von T, enthalten ist. Dieser
Abschluß oder auch diese {\bf abgeschlossene Hülle} von T ist der Durchschnitt
aller abgeschlossenen Mengen, die T enthalten.

Eine Teilmenge eines topologischen Raumes soll {\bf kompakt} heißen, wenn sie
das Überdeckungskriterium von Heine-Borel erfüllt. Dieses besagt, daß
eine Menge K in einem topologischen Raum X genau dann kompakt\idx{kompakt}
ist, wenn jede offene Überdeckung (jede Vereinigung offener Mengen aus X, die
eine Obermenge von K ist) die Auswahl einer endlichen Teilüberdeckung
(also eine Überdeckung mit nur endlich vielen offenen Mengen) zuläßt.
{\bf Konvex} heißt eine Teilmenge K eines topologischen Vektorraumes (eines
Vektorraumes mit Topologie, bezüglich der Addition und Multiplikation mit
Skalaren stetig sind)\idx{konvex}, wenn für je zwei Punkte a und b aus K
die Verbindungsstrecke $\{x=\lambda a +(1-\lambda) b:\lambda\in [0,1]\}$
der beiden auch in K liegt. Eine nichtleere Teilmenge S einer Teilmenge K
eines Vektorraumes heißt {\bf Extremmenge} von K\idx{Extremmenge}
falls für alle $x,y\in K,~0<t<1$, und $(1-t)x+ty\in S$ die Punkte x und y aus
S stammen. Die {\bf Extrempunkte}\idx{Extrempunkt} von K sind die einpunktigen
Extremmengen von K (vgl.\cite{Ru:91}, Seite 74).
Eine kompakte, konvexe Teilmenge K des $\R^d$ heißt nun {\bf Polytop} oder
Polyedermenge, falls die Menge ihrer Extrempunkte endlich ist.
(vgl.\cite{Gr:67}, S.31)\idx{Polytop}

\subsection{Produkttopologie}

Betrachtet man eine Mengenfamilie $\{X_j:j\in J\}$, eine Menge, die zu jedem
Element j einer Indexmenge J eine Menge X$_j$ enthält, so ist das {\bf
kartesische Produkt}\idx{kartesisches Produkt} der X$_j$ die Menge aller
Auswahlfunktionen $f:J\to\Cup\limits_{j\in J} X_j$ mit $f(j)\in X_j$,
geschrieben als
$$
\prod\limits_{j\in J}X_j:=
\left\{f~|~f:J\to\Cup\limits_{j\in J} X_j,f(j)\in X_j\right\}.
$$
Für $\{f(j):j\in J\}$ schreibt man kurz $(x_j)_{j\in J}$, was dem Sachverhalt
der Auswahl eines Elements aus allen $X_j$ wohl gerechter wird.
Daß ein unendliches kartesisches Produkt nichtleer ist, ist nicht ohne weiteres
einsichtig und wird durch das Auswahlaxiom sichergestellt, welches besagt:
\begin{quote}
{\bf
Für jede Mengenfamilie $\{X_j:j\in J\}$ nichtleerer Mengen ist das Produkt
$\prod\limits_{j\in J}X_j$ nichtleer.
}
\end{quote}
Bei Betrachtung endlicher Produkte kommt dieses Axiom nicht zum tragen, ist
aber bei der Übertragung auf den allgemeinen Fall unendlicher Mengenfamilien
unerläßlich, weswegen es hier bei der Darstellung allgemeiner topologischer
Grundlagen auch nicht fehlen soll.

Wählt man die $X_j$ als topologische Räume, so ist mit den Topologien
$\T_j$ der $X_j$ auch auf dem Produkt eine Topologie, die
{\bf Produkttopologie}, \idx{Produkttopologie} gegeben. Setzt man
$$\B:=\left\{\prod\limits_{j\in J}U_j~:~U_j\in\T_j, U_j=X_j,\mbox{ bis auf
endlich viele Ausnahmen } j\in J\right\},$$
so ist die Produkttopologie gerade $\T := \left\{\Cup B~:~B\in\B\right\}.$
$\B$ heißt eine {\bf Basis}\idx{Basis einer Topologie} der Topologie, da jede
offene Menge als Vereinigung von Mengen aus $\B$ dargestellt werden kann. Die
Formulierung "`bis auf endlich viele Ausnahmen"' ist dabei eine Folge des
Auswahlaxioms.

\subsection{Teilweise geordnete Mengen und Verbände}

Da gerade vom Auswahlaxiom die Rede war, soll an dieser Stelle die Terminologie
geordneter Mengen dargestellt werden, da auf Komplexen eine natürliche Ordnung
existiert, diese Mengen aber auch sonst eine hilfreiche Ergänzung bieten.

Eine Relation "`$\leq$"' auf einer Menge X heißt {\bf Partialordnung} oder
einfach nur Ordnung, wenn sie transitiv ($x\leq y, y\leq z\fol x\leq z~\forall
x,y,z\in X$), reflexiv ($x\leq x~\forall x\in X$) und antisymmetrisch ($x\leq y,
y\leq x\fol x = y~\forall x,y\in X$) ist. Die Menge X heißt zusammen mit
$\leq$ eine {\bf teilweise geordnete Menge} oder {\bf Poset} (partial ordered
set) (X,$\leq$).\idx{Ordnung}\idx{Poset}\idx{teilweise geordnete Menge}
(X,$\leq$) heißt {\bf totalgeordnet},\idx{totalgeordnet} wenn für alle
x,y$\in$X allemal x$\leq$y oder y$\leq$x gilt. Jede totalgeordnete Teilmenge
einer teilweise geordneten Menge heißt {\bf Kette} oder Turm.\idx{Kette}
\idx{Turm} Ist jede Kette in (X,$\leq$) nach oben beschränkt, so heißt
(X,$\leq$) auch induktiv geordnet.
Hier kann nun ein zum Auswahlaxiom äquivalentes Lemma, das Lemma von Zorn
angeführt werden, das den gegenwärtigen Bezug zu den topologischen Räumen
darstellt.\idx{Lemma von Zorn}
\begin{quote}
{\bf Jede induktiv geordnete Menge besitzt maximale Elemente.}
\end{quote}
Besitzt eine teilweise geordnete Menge (X,$\leq$) ein eindeutiges minimales
Element, so sei dieses mit \^0 bezeichnet, analog ein eindeutiges maximales
Element mit \^1. Gibt es in einer teilweise geordneten Menge (X,$\leq$) diese
Elemente, so heißt (X,$\leq$) {\bf beschränkt}. Gilt für zwei Elemente x
und y aus einer beschränkten teilweise geordneten Menge X, daß x$<$y ist, und
existiert kein weiteres Element z in X, welches zwischen x und y liegt
(x$<$z$<$y), so bezeichnet man y als {\bf Decke}\idx{Decke} von x
beziehungsweise x als {\bf Kodecke} von y. Die Decken von \^0 heißen dann
{\bf Atome}\idx{Atome} und die Kodecken von \^1 {\bf Koatome} von X.
Besitzen alle maximalen Ketten $x_0<x_1<\ldots<x_l$ die gleiche Länge $\ell$,
so heißt (X,$\leq$) rein und $\ell$ ist die Länge
von X. In diesem Fall ist der Rang $\rho(x)$ eines x$\in$X die Länge der
geordneten Teilmenge $X_{\leq x}:=\{y\in X:y\leq x\}$. Wie diese Teilmengen
teilweise geordneter Mengen, so können auch {\bf Ordnungsintervalle}
\idx{Ordnungsintervalle} $[x,y]\subseteq X$ mit $x,y\in X$ als Mengen
$\{z\in X:x\leq z\leq y\}$ definiert werden, analog offene Ordnungsintervalle
$(x,y) := \{z\in X:x<z<y\}$ im üblichen Sinne.

Eine teilweise geordnete Menge (X,$\leq$) heißt {\bf Verband},\idx{Verband}
wenn zu allen Paaren x,y$\in$ X kleinste obere Schranken $x\vee y$ (Joins) und
größte untere Schranken $x\wedge y$ (Meets) existieren. In einem endlichen
Verband L existieren Joins und Meets für beliebige Teilmengen von L, deshalb
ist L beschränkt und es ist \^0 = $\wedge$L und \^1 = $\vee$L.
Kürzer kann man sagen, daß eine teilweise geordnete Menge L genau dann ein
Verband ist, wenn \^0 existiert und wenn für alle Paare x,y$\in$L die Joins
$x\vee y$ existieren. 

\subsection{Quotientenräume}

Nach diesem Abstecher nun zurück zu den topologischen Räumen. Hier ist
ein weiterer wichtiger Begriff, gerade im Zusammenhang mit Komplexen und
darauf induzierten Topologien, der des Quotientenraumes, welcher im folgenden
eingeführt werden soll.

Sind X und Y Mengen, so kann mittels eines Tricks die Summe oder {\bf disjunkte
Vereinigung}\idx{Summe von Mengen}\idx{disjunkte Vereinigung} der beiden
Mengen definiert werden, wobei X und Y als Teilmengen erhalten bleiben
(Bei der "`normalen"' Vereinigung ist ja $X\cup X = X$).
Diese Summe $X+Y$ wird so als Vereinigung $X\times \{0\} \cup Y\times \{1\}$
erklärt. Analog ist die {\bf Summe topologischer Räume} zu bilden,
wenn die offenen Mengen der Topologie der Summe zu
$\{U+V|U\in\cS,V\in\T\}$ gewählt werden, wobei $\cS$ die Topologie auf X und
$\T$ die Topologie auf Y bezeichnet.

Sei $\sim$ eine Äquivalenzrelation (reflexiv (x $\sim$ x), symmetrisch
(x $\sim$ y $\iff$ y $\sim$ x) und transitiv (x $\sim$ y, y $\sim$ z
$\fol$ x $\sim$ z)) auf einem topologischen Raum $(X,\T)$,
$\tilde{x}:=\{y\in X:x\sim y\}$ eine und $\tilde{X}:=X/_\sim =
\{\tilde{x}|x\in X\}$ die Menge aller $\sim-$Äquivalenzklassen. Die Abbildung
$\pi:X\to\tilde{X}$, die jedes Element x aus X auf seine Äquivalenzklasse
abbildet heißt {\bf Quotientenabbildung}\idx{Quotientenabbildung} des
topologischen Raumes (X,$\T$) in den {\bf Quotientenraum}
$(\tilde{X},\tilde{\T})$ \idx{Quotientenraum} von X modulo $\sim$ mit der
{\bf Quotiententopologie} \idx{Quotiententopologie} $\tilde{\T} :=
\{\tilde{\T}\subset\tilde{X}:\pi^{-1}(\tilde{\T})\in\T\}$.
Die Quotientenabbildung $\pi$ ist stetig und $\tilde{\T}$ ist die feinste
Topologie bezüglich der $\pi$ stetig ist.\\
{\scsi
Seien $\cS$ und $\T$ zwei Topologien auf X. $\cS$ heißt feiner als $\T$,
wenn sie mehr offene Mengen enthält, also $\cS\supseteq\T$ gilt. Analog
heißt $\T$ gröber als $\cS$.
}\\
Ein Beispiel hierfür ist im Abschnitt zu den Abbildungen von CW-Komplexen
angegeben. (siehe Abb.\ref{inzidenz})

Da allein mit Angabe einer Topologie zu einer Menge X noch nicht viel
über Eigenschaften der Punkte in X ausgesagt werden kann, fordert man
in Form von Axiomen zusätzliche "`Trennungseigenschaften"'.

In dieser Arbeit wird folgendes Axiom, das zweite Trennungsaxiom oder
auch {\bf Hausdorffbedingung}, völlig genügen, weshalb auf die Auflistung
der weiteren Axiome verzichtet werden soll (vgl. dazu etwa \cite{Os:92},
Seite 50).\idx{Hausdorff-Axiom}
\begin{quote}
(T$_2$) Zu je zwei Punkten x und y in X gibt es disjunkte Umgebungen.
\end{quote}
Ein topologischer Raum, der dieses Axiom erfüllt wird auch als
{\bf Hausdorff-Raum}\idx{Hausdorff-Raum} bezeichnet.

Nun aber zu den eigentlichen Objekten, um die es in dieser Arbeit gehen soll.

\section{CW-Komplexe}

Nach der doch recht abstrakten Einführung topologischer Räume, von
Produkttopologien und Quotientenräumen wollen wir uns nun einer konkreten
Klasse solcher Räume zuwenden, die veranschaulichen, was man mit oben
definierten Begriffen überhaupt anfangen kann. Um allerdings die CW-Komplexe
definieren zu können, werden noch einige zusätzliche Eigenschaften benötigt.

So versteht man unter einer {\bf Zerlegung}\idx{Zerlegung} eines topologischen
Raumes X eine Menge paarweise disjunkter offener Teilmengen von X (versehen mit
der Spurtopologie selbst topologische Räume), deren Vereinigung ganz X ergibt.
Zu jedem x$\in$X kann man also einen eindeutigen Teilraum in der Zerlegung
angeben. Eine {\bf n-Zelle}\idx{Zelle}\idx{n-Zelle} ist ein topologischer Raum,
der homöomorph dem $\R^n$ ist. Damit kann eine {\bf Zellenzerlegung}
\idx{Zellenzerlegung} als eine Zerlegung eines topologischen Raumes in
Teilräume definiert werden, die Zellen sind. Über die Homöomorphie der
d-Zellen zu $\R^d$ kann die {\bf Dimension eines zellenzerlegten Raumes} als die
maximale Dimension n der auftretenden Zellen definiert werden.

Die Vereinigung aller Zellen der Dimension $\leq$ d eines zellenzerlegten
Raumes heißt dabei {\bf d-Gerüst} oder {\bf d-Skelett} des betreffenden Raumes.
\idx{Ger\üst}\idx{Skelett}\\
Weiterhin sollen zwei Zellen {\bf benachbart} heißen,\idx{benachbart} wenn der
Durchschnitt ihrer Abschlüsse nichtleer ist.

Im Jahre 1949 führte J.H.C. Whitehead den Begriff des CW-Komplexes (C für
closure finite und W für weak topology) (vgl. \cite{Ja:90}, Seite 109ff.)
ein, der eine sehr flexible Struktur innerhalb der Topologie darstellt.
Ein {\bf CW-Komplex}\idx{CW-Komplex} ($X,\cal E$) ist ein Hausdorff-Raum X
zusammen mit einer Zellenzerlegung $\cal E$ von X, der folgende drei Axiome
erfüllt:
\bn
\item (Charakteristische Abbildungen): Zu jeder n-Zelle $e\in\cal E$ existiert
      eine stetige Abbildung $\Phi_e:B^n\to X$ der n-Kugel in X,
      die einen Homöomorphismus zwischen der offenen n-Kugel
      (dem $\R^n$) und e induziert und die die (n-1)-Sphäre $\S^{n-1}$
      in das (n-1)-Skelett von $\cal E$ abbildet.
\item (Hüllenendlichkeit): Jeder Punkt der abgeschlossenen Hülle $\ol{e}$
      einer jeden Zelle $e\in\cal E$ besitzt einen Umgebung, die nur endlich
      viele andere Zellen trifft.
\item (Schwache Topologie): Teilmengen $A\subset X$ sind genau dann
      abgeschlossen, wenn für alle $e\in\cal E$ auch die Mengen $A\cap\ol{e}$
      abgeschlossen sind.
\en

Da es in dieser Arbeit nur um endliche CW-Komplexe gehen soll, ist das
zweite Axiom trivialerweise immer erfüllt.

\begin{figure}[htb]
$$
\beginpicture
\unitlength1cm
\setlinear
\setcoordinatesystem units <1cm,1cm>
\setplotarea x from -1 to 6, y from -0.5 to 3.5
\setsolid \thicklines
\setquadratic
\plot 1.0 1.5 1.5 2.5 2.5 2.5 3.5 2.5 3.5 1.5 2.0 1.0 1.0 1.5 /
\plot 3.5 1.5 4.5 1.5 5.5 1.0 /
\plot 1.0 1.5 1.0 0.5 2.0 1.0 /
\plot 2.5 2.5 3.0 3.0 2.8 3.5 /
\setlinear \thinlines
\put {\circle*{0.15}} [Bl] at 1.0 1.5
\put {\circle*{0.15}} [Bl] at 2.0 1.0
\put {\circle*{0.15}} [Bl] at 3.5 1.5
\put {\circle*{0.15}} [Bl] at 5.5 1.0
\put {\circle*{0.15}} [Bl] at 2.5 2.5
\put {\circle*{0.15}} [Bl] at 2.8 3.5
\put {\scsi sechs 0-Zellen} [bl] at 4.0 2.5
\put {\scsi sieben 1-Zellen} [tl] at 2.0 0.0
\put {\scsi zwei 2-Zellen} [bl] at -1.0 2.5
\endpicture
$$
\caption{Beispiel eines CW-Komplexes im $\E^2$}
\label{CW-Komplex}
\end{figure}

Ein {\bf Unterkomplex} $(X',{\cal E}')$ eines CW-Komplexes (X,$\cal E$)
\idx{CW-Komplex!Unterkomplex eines $\sim$} ist nun die Vereinigung X$'$ aller
Zellen aus einer Teilmenge $\cal E'$ von Zellen aus $\cal E$ zusammen mit
$\cal E'$, wenn eine der folgenden drei äquivalenten Bedingungen erfüllt ist:

\btab{ll}
(1) & $(X',{\cal E}')$ ist ein CW-Komplex.\\
(2) & X$'$ ist abgeschlossen in X.\\
(3) & Der Abschluß $\ol{e}$ jeder Zelle e aus $\cal E'$ liegt in X$'$.
\etab

Zum Beweis von $(1)\follows (2)$ ist zu zeigen, daß für alle
$e\in{\cal E}$ die Mengen $\ol{e}\cap X'$ abgeschlossen in X sind.
Da $(X,{\cal E})$ hüllenendlich ist, bedeutet dies, die Abgeschlossenheit
von $\ol{e}\cap (\Cup_{e'\in {\cal E}'}e')=
\ol{e}\cap (e'_1\cup\ldots\cup e'_l)$ für $e'_i\in{\cal E}',~1\leq i\leq l$
und alle $e\in\cal E$ zu klären. Hilfreich dazu ist folgende Bemerkung
(vgl.\cite{Ja:90},S. 110).\\
Ist $\cal E$ eine Zerlegung eines Hausdorff-Raumes X, die das erste Axiom für
CW-Komplexe erfüllt, so ist $\ol{e}=\Phi_e(B^n)$ für jedes $e\in\cal E$
kompakt und der Zellenrand $\ol{e}\backslash e=\Phi_e(\S^{n-1})$ liegt im
(n-1)-Gerüst von X.\\
Zum Beweis dieser Bemerkung nutzt man, daß für stetige Abbildungen f und
Mengen M die Inklusion $f(\ol{M})\subset \ol{f(M)}$ gilt. Damit erhält man hier
$e\subset\Phi_e(B^n)\subseteq\ol{\Phi_e(\stackrel{\circ}{B^n}})=\ol{e}$.
$\Phi_e(B^n)$ ist als stetiges Bild eines Kompaktums in X abgeschlossen und nach
obiger Inklusion gleich $\ol{e}$. Nun aber zurück zum Äquivalenzbeweis. Da die
$\Phi_e$ von $(X',{\cal E}')$ auch charakteristisch bezüglich $(X,{\cal E})$
sind, liefert die Bemerkung simultan den Nachweis von $(1)\follows (3)$, da
$\ol{e}$ in $X$ Hülle von e in $X$, damit aber auch in $X'$ ist. Nutzt man
dies, so ist auch $\ol{e}\cap (e'_1\cup\ldots\cup e'_l)=\ol{e}\cap
(\ol{e}'_1\cup\ldots\cup \ol{e}'_l)$ abgeschlossen.
$(3)\follows (2)$ ist wegen der Betrachtung abgeschlossener Zellen in $X'$
unmittelbar einsichtig. Zum Nachweis von $(2),(3)\follows (1)$ sind die drei
CW-Komplex-Axiome zu überprüfen. Die $\Phi_e$ von $(X,{\cal E})$ sind wegen
${\cal E'}\subset {\cal E}$ auch für $(X',{\cal E'})$ verwendbar. Ebenso
läßt sich für die Hüllenendlichkeit argumentieren.
Da alle in $X'$ abgeschlossenen Mengen dies auch in $X$ sind, müssen für
den Nachweis des dritten Axioms nur $e\in{\cal E}\backslash \cal E'$ betrachtet
werden. Für ein $A\subset X'$ und ein $e\in{\cal E}\backslash{\cal E'}$ ist
dazu $A\cap\ol{e}$ zu betrachten. Zellen $e\in{\cal E}\backslash{\cal E'}$
tragen nichts zum Schnitt mit A bei. Wegen der Hüllenendlichkeit von X gibt
es aber eine endliche Anzahl von $e'_i\in{\cal E'}$, mit denen
$A\cap\ol{e}=A\cap (\Cup e'_i)\cap\ol{e}$ gilt. Nach oben ist
$A\cap\ol{e}=A\cap (\Cup \ol{e'_i})\cap\ol{e}$.
$A\cap\ol{e}=A\cap (\Cup \ol{e'_i})$ ist nach Voraussetzung abgeschlossen,
somit auch $A\cap\ol{e}$.\hfill $\Box$

Betrachtet man zweidimensionale CW-Komplexe, so kann man, wie gewohnt, die
auftretenden 0-Zellen als "`{\bf Ecken}"', die 1-Zellen als "`{\bf Kanten}"'
und die 2-Zellen als "`{\bf Flächen}"' bezeichnen -- als Beispiel sei etwa
der Randkomplex eines Würfels angeführt, der aus sechs "`Flächen"', zwölf
"`Kanten"' und acht "`Ecken"' besteht.

\subsection{Simpliziale Komplexe}

Als eine spezielle Art von CW-Komplexen können die simplizialen Komplexe
aufgefaßt werden, die aufgrund ihrer einfacher zu beschreibenden Struktur
in der Vergangenheit oftmals eher eingesetzt wurden, als die
allgemeinen (nichtsimplizialen) CW-Komplexe. Diese seien hier, obwohl sie nicht
direkter Gegenstand dieser Arbeit sein sollen, der Vollständigkeit halber,
sowie als zusätzliche Beispielklasse aufgeführt.

Um Simplizes definieren zu können, muß gesagt werden, wann ein Punkt
x aus einer Teilmenge X des $\E^d$ {\bf linear abhängig} heißen soll, nämlich
dann, wenn Punkte $x_1,\ldots,x_r\in X$ und Skalare $\lambda_1,\ldots,\lambda_r$
so existieren, daß $x=\lambda_1x_1+\ldots+\lambda_rx_r$ gilt. Ist
$\lambda_1+\ldots+\lambda_r=1$, so heißt x {\bf affin abhängig}, sind zudem
alle $\lambda_i\geq 0$, so {\bf konvex abhängig}. Eine Teilmenge
$X\subseteq\E^d$ heißt {\bf affin abhängig}, falls für $x_1,\ldots,x_r\in X$
und Skalare $\lambda_1,\ldots,\lambda_r$ mit mindestens einem $\lambda_i\neq 0$
eine Relation von der Form $\lambda_1x_1+\ldots+\lambda_rx_r=0$ und
$\lambda_1+\ldots+\lambda_r=0$ existiert. Ansonsten heißt X affin unabhängig.
(Alle obigen Ausführungen gelten auch für allgemeine Vektorräume, wie
so oft wird aber der Anschaulichkeit wegen der $\R^d$ verwendet.)

Ein {\bf m-dimensionales Simplex}\idx{m-Simplex}\idx{Simplex} (kurz m-Sim\-plex)
im $\R^n$ ist die {\bf konvexe Hülle} von m+1 affin unabhängigen Punkten, m+1
Punkten in allgemeiner Lage, also für $p_0,p_1,\ldots,p_m\in\R^n$, $p_i$
affin unabhängig mit $1\leq i\leq m$
$$
   \Delta(p_0,p_1,\ldots,p_m) =
   \left\{ p\in\R^n~:~p = \sum\limits_{i=0}^m \lambda_i p_i, \lambda_i\geq 0,
   \sum\limits_{i=0}^m \lambda_i = 1\right\}.
$$
Die konvexe Hülle von m+1 Einheitsvektoren im $\R^n$ (vgl. Abb.\ref{simplex})
heißt {\bf Standard-m-Simplex},\idx{Standard-m-Simplex} die $p_i$ heißen
{\bf Eckpunkte} (vertices) des Simplex.

\begin{figure}[htb]
$$
\beginpicture
\unitlength1cm
\setlinear
\setcoordinatesystem units <0.6cm,0.6cm>
\setplotarea x from -2 to 2, y from -2 to 2
\put{ \beginpicture
\setsolid
\put {\circle*{0.1}} [Bl] at 0 0
\put {$p_0$} [tr] at 0 0
\endpicture } at -6 0
\put{ \beginpicture
\setsolid
\plot -0.75 -0.75 0.75 0.75 /
\put {\circle*{0.1}} [Bl] at -0.75 -0.75
\put {$p_0$} [tr] at -0.75 -0.75
\put {\circle*{0.1}} [Bl] at 0.75 0.75
\put {$p_1$} [bl] at 0.75 0.75
\endpicture } at -2 0
\put{ \beginpicture
\setsolid
\plot -1 -1 1 -1 0 0.732 -1 -1 /
\put {\circle*{0.1}} [Bl] at -1 -1
\put {$p_0$} [tr] at -1 -1
\put {\circle*{0.1}} [Bl] at 1 -1
\put {$p_1$} [tl] at 1 -1
\put {\circle*{0.1}} [Bl] at 0 0.732
\put {$p_2$} [bl] at 0 0.732
\endpicture } at 2 0
\put{ \beginpicture
\setsolid
\plot -0.884 -0.257 1.085 -0.376 /
\plot 0 1.023 1.085 -0.376 /
\plot -0.884 -0.257 0 1.023 /
\plot -0.2 -0.9 -0.884 -0.257 /
\plot 1.085 -0.376 -0.2 -0.9 /
\setdashes <1mm>
\plot -0.2 -0.9 0 1.023 /
\put {\circle*{0.1}} [Bl] at -0.884 -0.257
\put {$p_0$} [tr] at -0.884 -0.257
\put {\circle*{0.1}} [Bl] at 1.085 -0.376
\put {$p_1$} [bl] at 1.085 -0.376
\put {\circle*{0.1}} [Bl] at 0 1.023
\put {$p_2$} [bl] at 0 1.023
\put {\circle*{0.1}} [Bl] at -0.2 -0.9
\put {$p_3$} [tl] at -0.2 -0.9
\endpicture } at 6 0
\endpicture
$$
\caption{Einige Simplizes}
\label{simplex}
\end{figure}

Die konvexe Hülle einer Auswahl von r paarweise verschiedenen Eckpunkten
eines m-Simplex S heißt {\bf r-dimensionales Teilsimplex}\idx{Teilsimplex} oder
{\bf r-Seite}\idx{Simplex!Seite eines $\sim$} von S. Der {\bf Rand} $\partial S$
eines m-Simplex S ist die Vereinigung aller (m-1)-dimensionalen Teilsimplizes
von S. Das {\bf offene Simplex}\idx{offenes Simplex} $\stackrel{\circ}{S}$ ist
dann S ohne dessen Rand, also $S\backslash\partial S$.

Es sei angemerkt, daß jedes m-Simplex S homöomorph zur m-Vollkugel $B^m$
und der Rand von S homöomorph zur (m-1)-Sphäre ist. Da die offene Vollkugel
$\stackrel{\circ}{B^n}$ homöomorph zu $\R^n$ ist, ist das offene n-Simplex 
eine n-Zelle nach obiger Definition.

Ein {\bf simplizialer Komplex}\idx{simplizialer Komplex} $\cal C$ im $\R^n$ ist
nun eine Menge von Simplizes, die folgende Bedingungen erfüllt:

\btab{ll}
1. & $\cal C$ enthält mit jedem Simplex auch dessen sämtliche Teilsimplizes.\\
2. & Der Durchschnitt zweier Simplizes aus $\cal C$ ist leer oder\\
   & gemeinsames Teilsimplex der beiden.\\
3. & Enthält $\cal C$ unendlich viele Simplizes, so ist $\cal C$ lokal\\
   & endlich, das heißt jeder Punkt in einem Simplex aus $\cal C$ besitzt\\
   & eine Umgebung, die nur endlich viele Simplizes aus $\cal C$ schneidet.
\etab

(Wiederum steht hier der $\R^n$ beziehungsweise $\E^n$ als Synonym für einen
beliebigen Vektorraum über einem Körper $\K$, in dem o.a. Konzepte ähnlich
definierbar sind.)

Betrachtet man die offenen Simplizes eines simplizialen Komplexes, so
erfüllen diese auch die Axiome eines CW-Komplexes. Folglich ist der {\bf einem
simplizialen Komplex} $\cal C$ {\bf zugrundeliegende topologische Raum}
\idx{simplizialer Komplex!zugrundeliegender Raum} oder {\bf Trägerraum}
\idx{simplizialer Komplex!Tr\ägerraum} gegeben durch
$$|{\cal C}| := \bigcup\limits_{S\in{\cal C}} S \subset \R^n,$$
analog der Zellenzerlegung des einem CW-Komplex zugrundeliegenden
Hausdorff-Raumes.

Verschiedene Komplexe können so den gleichen zugrundeliegenden Teilraum des
$\R^n$ besitzen. Simpliziale Komplexe werden auch als "`Polyeder"', gemäß
ihrer Darstellung im Anschauungsraum, bezeichnet.\\
Der {\bf Randkomplex} eines simplizialen k-Komplexes $\cal C$ ist der Komplex
der aus allen Simplizes besteht, deren Dimension k-1 nicht übersteigt, wenn
die maximale Dimension der Simplizes aus $\cal C$ k ist, also das
(k-1)-Skelett.\idx{simplizialer Komplex!Randkomplex}\\
{\bf Unterkomplexe} sind allgemein Teilmengen des Komplexes, die
selbst wieder Komplexe sind.\idx{simplizialer Komplex!Unterkomplex}

Ist $\cal C$ ein simplizialer Komplex und C $\subset\cal C$, so ist der
{\bf Stern}\idx{simplizialer Komplex!Stern} st(C;$\cal C$) von C der kleinste
Unterkomplex von Elementen aus $\cal C$, der C enthält. Der {\bf Antistern}
\idx{simplizialer Komplex!Antistern} ast(C;$\cal C$) von C ist der
Unterkomplex von $\cal C$ aller Elemente, deren Schnitt mit C leer ist. Damit
kann der {\bf Link}\idx{simplizialer Komplex!Link}
link(C;$\cal C$) als der Durchschnitt von Stern und Antistern von C definiert
werden. In gleicher Weise läßt sich dies auch für die Abschlüsse der Zellen
in CW-Komplexen definieren.

\subsection{Abbildungen von CW-Komplexen}

Eine stetige Abbildung f zwischen zwei simplizialen Komplexen $\cal C$
und ${\cal D}$ heißt {\bf simplizial},\idx{simpliziale Abbildung} wenn
Simplizes aus $\cal C$ affin auf Simplizes aus ${\cal D}$ abgebildet werden.
Analog kann eine Abbildung $f:(X,{\cal E})\to (Y,{\cal F})$
zwischen CW-Komplexen als {\bf zellulär}\idx{zellul\äre Abbildung} bezeichnet
werden, wenn sie surjektiv ist und Zellen auf Zellen abbildet.\\
Eine simpliziale oder zelluläre Abbildung f heißt {\bf nicht-degenerierend},
\idx{nicht-degenerierend} falls f die Dimension einer jeden Zelle erhält,
wenn also die Dimension von S und f(S) für jede Zelle S aus $\cal C$
übereinstimmt.\\
Eine simpliziale Abbildung $f:{\cal C} \to {\cal C}$ eines simplizialen
Komplexes $\cal C$ auf sich, heißt {\bf Symmetrie},\idx{Symmetrie!simplizialer
Komplexe}\label{symm} falls sie bijektiv ist und die Dimension aller Simplizes
erhält. Somit sind die Automorphismen von $\cal C$ gerade dessen Symmetrien.
Analog ist eine Symmetrie eines CW-Komplexes wieder eine bijektive zelluläre
Abbildung eines CW-Komplexes auf sich, die die Dimension aller Zellen erhält.

Um sich den Trägerraum eines Komplexes genauer anschauen zu können und
diesen zu beschreiben, ist es sinnvoll hier den Begriff der
{\bf Mannigfaltigkeit}\idx{Mannigfaltigkeit} der Dimension d einzuführen.
Dabei handelt es sich um einen Hausdorff-Raum mit abzählbarer Basis
(eine Basis der Topologie bestehe aus abzählbar vielen offenen Mengen),
in dem jeder Punkt eine offene Umgebung besitzt, die zu einer offenen Teilmenge
des $\R^d$ homöomorph ist. Insbesondere bezeichnet man einen Hausdorff-Raum,
in dem jeder Punkt eine zu einer Kreisscheibe homöomorphe Umgebung besitzt
als {\bf Fläche}\idx{Fl\äche}. In bezug auf simpliziale Komplexe ist eine
{\bf 2-Pseudo-Mannigfaltigkeit}\idx{Pseudo-Mannigfaltigkeit} ein endlicher
geschlossener simplizialer 2-Komplex, das heißt ein Komplex, in dem jede Kante
(jedes 1-Simplex) an genau zwei Dreiecken (2-Simplizes) beteiligt ist.

{\scsi
Das Bild eines simplizialen 2-Komplexes unter einer stetigen Abbildung
$f:{\cal C}\to \R^3$ muß nicht durchdringungsfrei sein. Im Falle von
Triangulierungen der Kleinschen Flasche etwa (vgl. die Dissertation von Cervone,
\cite{Ce:93}) ist die beschriebene (nicht orientierbare) Mannigfaltigkeit nur
mit Selbstdurchdringung im $\R^3$ darstellbar, was aus dem 2-Komplex
nicht ohne weiteres ersichtlich ist. Bijektive Abbildungen f, deren Bild
durchdringungsfrei ist, heißen Einbettung von $\cal C$ in den $\R^3$.
Ist f nur lokal bijektiv, also in einer Umgebung jedes Punktes, so heißt f
Immersion.
}

Da wir uns im dritten Kapitel mit regulären Karten beschäftigen wollen,
sei hier noch die Definition dieser angegeben. Zerlegt man eine geschlossene
reelle 2-Man\-nig\-fal\-tig\-keit in $f_2$ einfach zusammenhängende
(wegzusammenhängend und jede Schleife nullhomotop, das heißt zu einem Punkt
zusammenziehbar), nicht-überlappende Gebiete (Seiten), deren Durchschnitte
$f_1$ Kanten bilden, die sich in $f_0$ Ecken schneiden, so nennt man diese
Unterteilung eine {\bf Karte} der 2-Mannigfaltigkeit. {\bf Regulär} heißt die
Karte, wenn ihre Automorphismengruppe Flaggen-transitiv ist, aber dazu später
mehr (vgl. Symmetriebegriffe Seite \pageref{flag}, eine kurze Beschreibung
findet sich in \cite{BoWi:87}).

Wie oben erwähnt, sind die Zellen von CW-Komplexen und die Simplizes
simplizialer Komplexe selbst topologische Räume mit der durch den Trägerraum
induzierten Topologie. Speziell kennt man aber umgekehrt bei einem
simplizialen Komplex auch dessen Trägerraum schon dann bis auf Homöomorphie,
wenn die Anzahl der wesentlichen Simplizes (solche, die nicht schon als
Seiten größerer Simplizes vorkommen), sowie deren Inzidenzen, bekannt ist.
Dazu beachte man, daß der Trägerraum X eines simplizialen n-Komplexes als
disjunkte Vereinigung der d-Simplizes $S_d$ des Kom\-plex\-es dargestellt werden
kann.
$$X=(S_0+\ldots+S_0)+\ldots+(S_n+\ldots+S_n)$$
Indiziert man die 0-Simplizes mit positiven ganzen Zahlen, den sogenannte
Simplexzahlen,\idx{Simplexzahl} so ist damit eine Äquivalenzrelation $\sim$ :=
"`zu identifizierende 0-Simplizes"' gegeben, die mit den Simplizes den
Quotientenraum X$/_\sim$ beschreibt und so einem Homöomorphismus in X
bereitstellt. In Abbildung \ref{inzidenz} entsteht so aus acht disjunkten
Teilräumen des $\E^2$, den Dreiecken, mittels Eckenidentifikation der
Quotientenraum "`Rand eines Oktaeder"' und damit als Trägerraum eine
2-Mannigfaltigkeit. Wieder gilt eine Analogie zu zweidimensionalen CW-Komplexen,
deren 2-Zellen durch Polygonzüge (Ecken und Kanten) begrenzt sind.

\begin{figure}[htb]
$$
\beginpicture
\unitlength1cm
\setlinear
\setcoordinatesystem units <0.6cm,0.6cm>
\setplotarea x from -2.5 to 2.5, y from -2.5 to 2.5
\put{ \beginpicture
\setsolid
\plot -0.909 0.188 -1.083 -0.157 /
\plot -1.083 -0.157 0 1.5 /
\plot 0 1.5 -0.909 0.188 /
\plot -1.083 -0.157 0.909 -0.188 /
\plot 0.909 -0.188 0 1.5 /
\plot 0.909 -0.188 1.083 0.157 /
\plot 1.083 0.157 0 1.5 /
\plot 0 -1.5 -1.083 -0.157 /
\plot 0 -1.5 0.909 -0.188 /
\setdashes <1mm>
\plot 1.083 0.157 -0.909 0.188 /
\plot -0.909 0.188 0 -1.5 /
\plot 0 -1.5 1.083 0.157 /
\put {\scsi 1} [Br] at -1.2 0.2
\put {\scsi 2} [Br] at -1.4 -0.2
\put {\scsi 3} [Bl] at 1.2 -0.2
\put {\scsi 4} [Bl] at 1.3 0.2
\put {\scsi 5} [bl] at 0 1.6
\put {\scsi 6} [tl] at 0 -1.7
\endpicture } at 3 0
\put{ \beginpicture
\setsolid
\plot -1 0 -1.5 -0.866 -1 -1.732 -0.5 -0.866 0.5 -0.866 1 0 2 0 1.5 0.866 0.5 0.866 0 0 -1 0 /
\plot -1 0 -0.5 -0.866 /
\plot 0 0 0.5 -0.866 /
\plot 0.5 0.866 1 0 /
\plot 1.5 0.866 1 0 /
\plot -1.5 -0.866 -0.5 -0.866 0 0 1 0 /
\put {\scsi 5} [br] at -1 0
\put {\scsi 1} [Br] at -1.5 -0.866
\put {\scsi 6} [tl] at -1 -1.732
\put {\scsi 2} [tl] at -0.5 -0.9
\put {\scsi 6} [tl] at 0.5 -0.866
\put {\scsi 4} [tl] at 1 -0.05
\put {\scsi 6} [tl] at 2 0
\put {\scsi 1} [bl] at 1.5 0.9
\put {\scsi 5} [br] at 0.5 0.9
\put {\scsi 3} [br] at 0 0.05
\endpicture } at -3 0
\endpicture
$$
\caption{Der Quotientenraum "`Oktaeder"'}
\label{inzidenz}
\end{figure}

Unter Ausnutzung der Nummerierung der 0-Simplizes eines simplizialen Komplexes
(bei Betrachtung des Quotientenraumes) kann jede Symmetrie
\idx{Symmetrie!eines simplizialen Komplexes} als Element der Permutationsgruppe
$S_n$ dargestellt werden, wenn n die Anzahl der verschiedenen 0-Simplizes
beziehungsweise Eckpunkte bezeichnet.

\subsection{Inzidenzen und Anheftungen}

Die Betrachtung von allgemeinen CW-Komplexen hat gegenüber simplizialen
Komplexen den Vorteil eine größere Vielfalt bei der Beschreibung von
Zellenberandungen zu haben. Auch besteht etwa ein zu \S$^2$ homöomorpher
simplizialer Komplex aus mindestens 14 Simplizes (zum Aufbau des Randes eines
3-Simplex braucht man vier 0-Simplizes, sechs 1-Simplizes und vier 2-Simplizes),
während zu einem homöomorphen allgemeinen CW-Komplex ganze zwei Zellen
benötigt werden (eine 2-Zelle und eine 0-Zelle). Zudem sind CW-Komplexe als
ein Axiomensystem erfüllende Hausdorff-Räume nicht von vornherein auf
Vektorräume beschränkt, wie dies bei den simplizialen Komplexen der Fall ist.
Gerade in bezug auf die orientierten Matroide kann für diese so eine
natürliche Topologie angegeben werden, die unabhängig von simplizialen
Eigenschaften ist.\\
Natürlich haben CW-Komplexe nicht nur Vorteile. Aufgrund ihrer größeren
Allgemeinheit ist eine algebraische Beschreibung nicht so elegant
möglich, wie es bei den simplizialen Komplexen mittels der Simplexzahlen
geschehen kann. Was hier bei den simplizialen Komplexen die Inzidenzangaben
sind, sind in gewisser Hinsicht die {\bf Anheftungsabbildungen} bei den
CW-Komplexen (X,$\cal E$). Dabei handelt es sich um
\idx{CW-Komplex!Anheftungsabbildung} stetige Abbildungen
$\varphi:\S^{n-1}\to X^{n-1}$ der (n-1)-Sphäre in das (n-1)-Skelett von X, so
daß der Quotientenraum $X\cup_{\varphi} B^n$, nach
dem x und $\varphi$(x) für äquivalent erklärt werden, wieder ein CW-Komplex,
jetzt mit einer n-Zelle mehr ist. Der Zellenrand der neuen Zelle ist
$\varphi(\S^{n-1})\subset X^{n-1}$, ein stetiges Bild der (n-1)-Sphäre.
(Das Anheften kann man sich bildlich so vorstellen, daß zwei Luftballons
zusammengeklebt werden, wobei man die Berührungspunkte der beiden Ballons
identifiziert.) Mittels Anheftungsabbildungen können so alle Gerüste
von CW-Komplexen erzeugt werden. Genau dies ist eine der Schwierigkeiten beim
Umgang mit CW-Komplexen, denn eine Algebraisierung der Anheftungsabbildungen
(im Gegensatz zu den genauen Inzidenzvorschriften der simplizialen Komplexe) ist
erst durch Einsatz der Homologietheorie zu erreichen. Deren Darstellung würde
an dieser Stelle allerdings etwas zu weit führen. Als einführende Lektüre sei
hier stellvertretend das Topologiebuch von Ossa (vgl.\cite{Os:92}, Seite 159ff.)
genannt.

Als eine der Wurzeln der Homologietheorie und als fundamentales Ergebnis bei
der Untersuchung von Polyedern darf allerdings der Begriff der
Euler-Charakteristik\idx{Euler-Charakteristik} und der Eulersche Polyedersatz
\idx{Eulerscher Polyedersatz} nicht fehlen.\\
Mit $f_k(\cal C)$ für $0\leq k\leq d$ seien dabei die Anzahlen der
k-dimensionalen Elemente eines d-Komplexes $\cal C$ (entweder eines simplizialen
oder CW-Komplexes) bezeichnet. Die Folge $f({\cal C})=(f_0,f_1,\ldots,f_d)$
wird {\bf f-Vektor}\idx{f-Vektor} von $\cal C$ genannt.\\
Über den f-Vektor läßt sich die {\bf Euler-Charakteristik} $\chi(\cal C)$
eines d-Komplexes als
$$ \chi({\cal C}) = \sum\limits_{k=0}^{d-1} (-1)^k f_k({\cal C})=2-2g, $$
definieren. Der {\bf Eulersche Polyedersatz} besagt dazu
\begin{satz}
Die Euler-Charakteristik ist eine Homöomorphieinvariante.
\end{satz}
Dies bedeutet, daß homöomorphe d-Komplexe die gleiche Eulercharakteristik
besitzen. Die Konstante g beschreibt dabei das {\bf Geschlecht} des
\idx{Geschlecht} topologischen Gebildes, das heißt die Anzahl der Henkel
beziehungsweise Löcher, die das Objekt hat. (Hier tiefer einzusteigen, würde
in die Homotopietheorie, die Theorie der stetigen Verformbarkeit, führen, zu
der wieder der Verweis auf \cite{Os:92}, Seite 70ff., gegeben sei.)
Für Polyeder P ohne Henkel im $\R^3$, das heißt Polyeder vom Geschlecht
Null, gilt nach oben also
$\chi(P)=\#\mbox{Ecken}-\#\mbox{Kanten}+\#\mbox{Flächen}=2$.
Für endliche CW-Komplexe kann die Euler-Charakteristik als
Wechselsumme ihrer Zellenanzahlen in den einzelnen Dimensionen leicht
berechnet werden. So ist etwa $\chi(\S^n) = 1 + (-1)^n$ und
$\chi(\S^1\times\S^1) = 1 - 2 + 1 = 0$.

\section{Orientierte Matroide}

Die Autoren Björner, Las Vergnas, Sturmfels, White und Ziegler beginnen ihr
Buch über orientierte Matroide (vgl. \cite{Bj:93}) mit dem Satz
\begin{quote}{\sf
"`{\it Oriented matroids} can be thought of as a combinatorial abstraction
of point configurations over the reals, of real hyperplane arrangements, of
convex polytopes, and of directed graphs."'
}\end{quote}
Eine Verbindung zwischen CW-Komplexen und orientierten Matroiden zu suchen
erscheint nach diesem Satz nicht schwierig, da mit reellen Hyperebenen schon
etwas in der Art der benötigten n-Zellen gegeben ist und mit den spezielleren
simplizialen Komplexen sowohl konvexe Polytope, als auch Punktkonfigurationen
erfaßt werden können.

Nun fußt die Theorie der orientierten Matroide in einer Vielzahl
unterschiedlicher Axiomensysteme, die untereinander äquivalent (Die Beweise
der Äquivalenz sind nach \cite{Bj:93} alles andere als trivial) doch, wie
in dem einführenden Satz angedeutet, die unterschiedlichsten Ausgangsebenen
beschreiben. In \cite{Bj:93} werden vier fundamentale Axiomensysteme
\idx{orientiertes Matroid!Axiomensysteme} hervorgehoben:

\btab{ll}
(1) & Kreisaxiome aus der Motivation gerichteter Graphen,\\
(2) & Orthogonalitätsaxiome orthogonaler Paare reeller Vektorunterräume,\\
(3) & Chirotope von Punktkonfigurationen und konvexen Polytopen, sowie\\
(4) & Vektoraxiome reeller Hyperebenenarrangements.
\etab

{\scsi
Der Begriff Chirotop ist eine Abwandlung des Begriffs Chiralität nach
Dreiding und Dress. In der organischen Chemie bezeichnet Chiralität
(Händigkeit) eine Dissymmetrie im räumlichen Aufbau chemischer
Verbindungen und stellt eine notwendige und hinreichende Voraussetzung für das
Auftreten optischer Aktivität dar. (vgl. Literatur zur organischen Chemie,
etwa Flörke/Wolff, Kursthemen Chemie, Organische Chemie und Biochemie, Dümmler
Verlag, Bonn 1984)
}

In \cite{Bj:93} werden orientierte Matroide über alle vier Axiomensysteme
studiert und mit Beispielen untermauert. Da hier auch mittels orientierter
Matroide argumentiert werden soll, sei zunächst eine Einführung gegeben,
was orientierte Matroide überhaupt darstellen und wie sie, für diese Arbeit
nützlich, eingesetzt werden können.

Wie auch in anderen Publikationen, so möchte ich auch hier zunächst anhand
der Chiro\-top\-axiome orientierte Matroide einführen, um dann mit den
"`Vektoraxiomen für reelle Hyperebenenarrangements"', die schon in Richtung
dessen gehen, was für allgemeine CW-Komplexe benötigt wird, das
Einsatzgebiet für diese Arbeit abzustecken.

\subsection{Orientierte Matroide von Punktkonfigurationen}

Orientierte Matroide beziehen sich immer auf eine endliche Menge E,
die der Einfachheit halber als eine geordnete Indexmenge $\{1,2,\ldots,n\}$
beschrieben sein soll. Die Elemente von E können als Indizes von Punkten im
reellen euklidischen Raum, von Hyperebenen oder auch abstrakt, ohne direkten
Bezug zu etwas "`Realisiertem"' aufgefaßt werden.

Zunächst seien die Elemente von E als Indizes von n Punkten
$p_1,\ldots,p_n$ im $\R^{r-1}$ in allgemeiner Lage aufzufassen oder einfacher
als die Punkte selbst. Verwendet man homogene Koordinaten {(jeder affine Punkt
$p_i$ im $\R^{r-1}$ mit den Koordinaten $(p_i^1,p_i^2,\ldots,p_i^{r-1})$ wird
anschaulich als Vektor $(1,p_i)$ im $\R^r$ aufgefaßt)}, so definieren die n
Punkte eine (n$\times$r)-Matrix über $\R$.
$$\left(\begin{array}{cccc}
        1 & p_1^1 & \ldots & p_1^{r-1} \\
        1 & p_2^1 & \ldots & p_2^{r-1} \\
        \vdots & \vdots & \ddots & \vdots \\
        1 & p_n^1 & \ldots & p_n^{r-1} \end{array}\right)$$

{\scsi
Motivation des Einsatzes homogener Koordinaten ist die, daß Punkte des
affinen euklidischen (r-1)-Raumes durch Einbettung in den projektiven r-Raum
$\P^r$ bezüglich ihrer Lage besser beschrieben werden können. Stichwort
ist hierbei die Kompaktifizierung des euklidischen Raumes, in dem Sinne,
daß nun unendlich ferne Punkte ebenfalls erfaßt werden können. Man
vergleiche dies mit der stereographischen Projektion der 2-Sphäre ohne den
Nordpol N auf $\R^2$, die mittels des Zusatzes $N \mapsto \{\infty\}$ einen
Homöomorphismus zwischen dem (nun kompakten) $\R^2\cup\{\infty\}$ und der
kompakten $S^2$ darstellt.
}

Eine Auswahl $(\lambda_1,\ldots,\lambda_r)$, mit $\lambda_i\in E$, von r
paarweise verschiedenen Punkten aus E bildet ein r-Simplex, dessen
Orientierung mittels des Vorzeichens der (r$\times$r)-Unterdeterminante
sign(det$(\lambda_1,\ldots,\lambda_r)$) oben definierter Matrix aus den
$\lambda_i$-ten Zeilen, analog dem Umlaufsinn eines Dreiecks (vgl.
Abb.\ref{orient}), beschrieben werden kann (vgl. \cite{BoEg:91}).
Genauer beschreibt die Determinante
$\det(\lambda_1,\ldots,\lambda_k,\ldots,\lambda_r)$ die Seite der
orientierten Hyperebene
$\mbox{aff}\{\lambda_1,\ldots,\lambda_{k-1},\lambda_{k+1},\ldots,\lambda_r\}$,
auf der der Punkt $\lambda_k$, für alle k aus $\{1,\ldots,r\}$, liegt.

\begin{figure}[hbt]
$$
\beginpicture
\unitlength1cm
\setlinear
\setcoordinatesystem units <0.6cm,0.6cm>
\setplotarea x from 0 to 5, y from -1 to 3
\put{ \beginpicture
\setsolid
\plot 0 0 2 0 0 2 0 0 /
\put {\circle*{0.1}} [Bl] at 0 0
\put {\circle*{0.1}} [Bl] at 0 2
\put {\circle*{0.1}} [Bl] at 2 0
\put {\circle{0.5}} [Bl] at 0.6 0.6
\put {\vector(0,1){0.15}} [Bl] at 1 0.5
\put {\scsi (0,0)} [tr] at 0 0
\put {\scsi (1,0)} [tl] at 2 0
\put {\scsi (0,1)} [br] at 0 2
\put {$\left|\begin{array}{ccc} 1 & 0 & 0 \\
                                1 & 1 & 0 \\
                                1 & 0 & 1
             \end{array}\right|= +1$} [Bl] at 3 1
\endpicture } at -5.5 0
\put{ \beginpicture
\setsolid
\plot 0 0 2 0 0 2 0 0 /
\put {\circle*{0.1}} [Bl] at 0 0
\put {\circle*{0.1}} [Bl] at 0 2
\put {\circle*{0.1}} [Bl] at 2 0
\put {\circle{0.5}} [Bl] at 0.6 0.6
\put {\vector(0,-1){0.15}} [Bl] at 1 0.7
\put {\scsi (0,0)} [tr] at 0 0
\put {\scsi (1,0)} [tl] at 2 0
\put {\scsi (0,1)} [br] at 0 2
\put {$\left|\begin{array}{ccc} 1 & 0 & 0 \\
                                1 & 0 & 1 \\
                                1 & 1 & 0
             \end{array}\right|= -1$} [Bl] at 3 1
\endpicture } at 5.5 0
\endpicture
$$
\caption{Orientierung eines Dreiecks}
\label{orient}
\end{figure}

Als {\bf orientiertes Matroid zur Punktmenge E} wird nun die Information
bezeichnet, die sich aus den Vorzeichen der Determinanten zu allen
r-elementigen Untermengen $\{\lambda_1,\ldots,\lambda_r\}$ von E ergibt.
Bezeichnet $\Lambda (n,r)$ die Menge aller geordneten r-Tupel
\idx{geordnete r-Tupel} von n Elementen, das heißt
$$\Lambda (n,r) := \left\{(\lambda_1,\ldots,\lambda_r)~|~1\leq\lambda_1
<\ldots<\lambda_r \leq n,\lambda_i\in\{1,\ldots,n\},1\leq i\leq r \right\},$$
so kann folgende Definition gegeben werden:
\bcent
\fbox{\parbox{14.2cm}{
  Eine Abbildung $\chi:\Lambda (n,r)\to\{-1,0,+1\}$ oder deren eindeutige
  alternierende Erweiterung $\chi:\{1,\ldots,n\}^r\to\{-1,0,+1\}$ heißt
  {\bf orientiertes Matroid} vom Rang r mit n Punkten, wenn für alle
  $\lambda\in\Lambda (n,r+1)$ und für alle $\mu\in\lambda (n,r-1)$ die Menge
  $$\left\{(-1)^i\cdot\chi (\lambda_1,\ldots,\lambda_{i-1},\lambda_{i+1},
  \ldots,\lambda_{r+1})\cdot\chi (\mu_1,\ldots,\mu_{r-1},\lambda_i)~|~i\in\{1,
  \ldots,r+1\}\right\}$$
  entweder $\{-1,+1\}$ enthält oder gleich $\{0\}$ ist.
}}\ecent

Dabei heißt $\chi:E^r\to\{-1,0,+1\}$ {\bf alternierend},\idx{alternierend} wenn
$$\chi(x_{\sigma_1},x_{\sigma_2},\ldots,x_{\sigma_r}) =
\mbox{sign}(\sigma)\chi(x_1,x_2,\ldots,x_r)$$
für alle $x_i~(1\leq i\leq r)$ aus E und jede Permutation $\sigma$ aus der
Menge S$_E$ aller Permutationen der Elemente aus E gilt. Das entstehende
orientierte Matroid heißt {\bf simplizial},
\idx{orientiertes Matroid!simpliziales $\sim$} wenn die Abbildung $\chi$
E$^r$ in $\{-1,+1\}$ abbildet, das heißt, wenn $\chi(\lambda )\neq 0$ für
alle $\lambda\in\Lambda (n,r)$ ist (vgl. \cite{BoEg:91}). Es heißt
{\bf affin} oder {\bf azyklisch},\idx{orientiertes Matroid!affines $\sim$}
\idx{orientiertes Matroid!azyklisches $\sim$} wenn für alle
$\lambda\in\Lambda(n,r+1)$ die Menge
$$\left\{(-1)^i\cdot\chi (\lambda_1,\ldots,\lambda_{i-1},\lambda_{i+1},
\ldots,\lambda_{r+1})~|~i\in\{1,\ldots,r+1\}\right\}$$
entweder $\{-1,+1\}$ enthält oder gleich $\{0\}$ ist. Abbildung \ref{cube}
zeigt ein orientiertes Matroid zum dreisimensionalen Würfel. Die Schreibweise
von $\chi(\Lambda)$ ist so zu verstehen, daß zeilenweise alle Vorzeichen zu
den kanonisch geordneten (elementweise "`$\leq$"') Tupeln aus $\Lambda$
aufgeführt sind.

\begin{figure}[htb]
$$
\beginpicture
\unitlength1cm
\setlinear
\setcoordinatesystem units <0.6cm,0.6cm>
\setplotarea x from -3 to 5, y from -2 to 2
\setsolid \thicklines
\put {\beginpicture
\setsolid
\plot 0.597 -1.378 -1.281 -1.144 /
\plot 0.597 -1.378 0.597 0.501 /
\plot 0.597 0.501 -1.281 0.735 /
\plot -1.281 0.735 -1.281 -1.144 /
\plot 1.281 -0.735 0.597 -1.378 /
\plot 1.281 -0.735 1.281 1.144 /
\plot 1.281 1.144 0.597 0.501 /
\plot -0.597 1.378 1.281 1.144 /
\plot -1.281 0.735 -0.597 1.378 /
\setdashes <1mm>
\plot -0.597 -0.501 1.281 -0.735 /
\plot -1.281 -1.144 -0.597 -0.501 /
\plot -0.597 -0.501 -0.597 1.378 /
\endpicture} at -2.5 0
\put {\scsi$\left(\begin{array}{cccc}
             1 & 0 & 0 & 0 \\
             1 & 1 & 0 & 0 \\
             1 & 0 & 1 & 0 \\
             1 & 0 & 0 & 1 \\
             1 & 1 & 1 & 0 \\
             1 & 1 & 0 & 1 \\
             1 & 0 & 1 & 1 \\
             1 & 1 & 1 & 1
      \end{array}\right)$} [Bl] at 0 0
\put {\scsi$\chi(\Lambda)=\left.\begin{array}{cccccccccccccccc}
          \{&+&0&+&+&+&-&0&-&-&+&+&+&-&-&\\
            &0&+&+&0&+&-&-&-&-&0&+&-&+&0&\\
            &+&+&-&-&-&+&-&+&+&+&+&-&-&-&\\
            &0&+&+&-&0&-&+&+&-&-&0&+&-&0&\\
            &+&+&+&+&-&-&-&0&-&-&-&+&0&-&\}
            \end{array}\right.$} [Bl] at 4.8 0
\endpicture
$$
\caption{Ein Würfel mit seinem Chirotop}
\label{cube}
\end{figure}

Der {\bf Satz von Radon} besagt, daß für jede endliche Punktmenge X im $\E^r$
mit einer Punkteanzahl $\geq r+2$ eine Zerlegung von X in disjunkte Teilmengen
X$_1$ und X$_2$ existiert, für die
$\mbox{conv X}_1~\cap\mbox{conv X}_2~\neq~\emptyset$ gilt. Eine solche
Zerlegung heißt {\bf Radonpartition}.\idx{Radonpartition}
Die Radonpartitionen oben definierter Punktmenge E liefern die sogenannten
{\bf Kreise} (eine Bezeichnung, die von der Motivation über gerichtete
Graphen her stammt) des orientierten Matroids. Diese sind für
$\mu\in\Lambda(n,r+1)$ und $1\leq i\leq n$ gegeben durch
\idx{orientiertes Matroid!Kreis eines $\sim$}
$$ C_\mu(i) := \left\{\begin{array}{ll}
   (-1)^j\chi(\mu_1,\ldots,\mu_{j-1},\mu_{j+1},\ldots,\mu_{r+1}) &
   \mbox{für } i=\mu_j\\
   0 & \mbox{sonst} \end{array} \right.$$
Die Menge ${\cal K}(\chi):=\{\pm C_\mu|\mu\in \Lambda (n,r+1)\}$ beschreibt
damit alle Kreise des orientierten Matroids $\chi$.

Ist analog $\lambda\in\Lambda (n,d-1)$, so heißt
$C^*_\lambda(i) := \chi(\lambda_1,\ldots,\lambda_{r-1},i)$ für
$1\leq i\leq n$ {\bf Kokreis}\idx{orientiertes Matroid!Kokreis eines $\sim$}
von $\chi$\label{kokreis} und
${\cal K}^*(\chi) := \{\pm C^*_\lambda|\lambda\in\Lambda (n,r-1)\}$
ist die Menge aller Kokreise.

Ist $\chi$ durch eine konkrete Punktkonfiguration gegeben, so entsprechen
die Kokreise den Hyperebenen, die durch die Punkte mit $C^*_\lambda(i)=0$
gegeben sind. Am Beispiel des Chirotops zum Würfel (Abb.\ref{cube}) ergibt sich
etwa ein Kreis $C_\mu$ mit $\mu=(2,4,5,7,8)$ als $(0-0++0-0)$.\\
Der Kokreis $C^*_\lambda$ mit $\lambda=(1,3,7)$ hat die Darstellung
$(0+00++0+)$, woraus sich ablesen läßt, daß es sich um eine Stützhyperebene
handelt, da alle Punkte auf einer Seite der durch die Punkte 1,3 und 7
induzierten Hyperebene H (angezeigt durch vier +, sowie den Punkt 4 auf H) liegen.

Setzt man Mittel der linearen Algebra ein und untersucht die Eigenschaften
der verwendeten Determinanten genauer, so kann obige Situation auch
unabhängig von einer konkreten Punktkonfiguration beziehungsweise einer
konkreten Matrix als Definition für abstrakt aufzufassende {\bf Chirotope}
\idx{Chirotop} vom Rang r auf E dienen. Dies führt(e) zu den folgenden
"`Chirotopaxiomen"'\idx{Chirotopaxiome}

\bcent
\fbox{\parbox{13cm}{
\btab{ll}
(B0) & $\chi$ ist nicht identisch mit der Nullabbildung\\
(B1) & $\chi$ ist alternierend\\
(B2) & Für alle $x_1,x_2,\ldots,x_r,y_1,y_2,\ldots,y_r,$ aus E mit\\
     & $\chi(y_i,x_2,x_3,\ldots,x_r)\cdot\chi(y_1,y_2,\ldots,y_{i-1},x_i,
       y_{i+1},y_{i+2},\ldots,y_r) \geq 0$\\
     & bei $i=1,2,\ldots,r$ gilt
       $\chi(x_1,x_2,\ldots,x_r)\cdot\chi(y_1,y_2,\ldots,y_r) \geq 0$
\etab}}
\ecent

Die Äquivalenz von Chirotopen und orientierten Matroiden wurde 1982 durch
Lawrence mit folgendem Satz gesichert.

\begin{satz}
Sei $r\in\N$ und E sei eine Menge. Eine Abbildung
$\chi:E^r\to\{-1,0,+1\}$ ist genau dann äquivalent zu einem orientierten
Matroid vom Rang r auf E, wenn sie ein Chirotop ist.
\end{satz}

Der Beweis ist nachzulesen in \cite{Bj:93}, Seite 128ff. Wie oben angedeutet,
fordert Axiom B2 die Erfüllung abstrakter Graßmann-Plücker-Relationen,
\idx{Gra\3\-mann-Pl\ücker-Relation} die in ihrer expliziten Form mit
Determinanten für alle $x_1,x_2,\ldots,x_r,y_1,\ldots,y_r \in \R^r$ die
Identität
$$\det(x_1,x_2,\ldots,x_r) \cdot \det(y_1,y_2,\ldots,y_r)$$
$$= \sum\limits_{i=1}^r (-1)^{i-1} \det(y_i,x_2,\ldots,x_r) \cdot
    \det(x_1,y_1,\ldots,y_{i-1},y_{i+1},\ldots,y_r)$$
liefert (vgl.\cite{Na:72}, Stichwort "`Graßmannsche Mannigfaltigkeit"').

Mittels formaler Brackets (formaler Determinanten) lassen sich die allgemeinen
{\bf k-summandigen Graßmann-Plücker-Relationen} für 3$\leq$k$\leq$r+1, mit
\idx{Gra\3\-mann-Pl\ücker-Relation!k-summandige}
Mengen paarweise verschiedener Elemente $A=\{a_1,\ldots,a_{d-k+1}\}$,
$B=\{b_1,\ldots,b_{k-2}\}$ und C=$\{c_1,\ldots,c_k\}$ aus E schreiben als

{\small
$$\{A|B|C\}=$$
$$\sum\limits_{i=1}^k(-1)^{i+1}\cdot
[a_1,\ldots,a_{d-k+1},c_1,\ldots,c_{i-1},c_{i+1},\ldots,c_k]\cdot
[a_1,\ldots,a_{d-k+1},b_1,\ldots,b_{k-2},c_i]=0$$}

Unabhängig von gegebenen Punkten kann so ein orientiertes Matroid mittels
einer Vorzeichenliste definiert werden, die zu den
formalen Brackets $[\lambda_1,\ldots,\lambda_r]$ gehört und mit der die
abstrakten Graß\-mann-Plücker-Relationen B2 gelten.

Die oben angeführten Graßmann-Plücker-Relationen liefern also eine weitere
Charakterisierung für orientierte Matroide. Bemerkenswert ist, daß der
folgende Satz (siehe \cite{Bj:93}, Seite 138) über dreisummandige
Graßmann-Plücker-Relationen
\idx{Gra\3\-mann-Pl\ücker-Relation!dreisummandige $\sim$en}
gilt.

Bevor wir allerdings zur Formulierung des Satzes kommen, seien an dieser Stelle
zunächst die Definitionen der verwendeten Begriffe Matroid und Basen eines
Matroids eingefügt (vgl. dazu auch Aigner, Kombinatorik Bd.II (\cite{Aig:76}),
Seite 17ff., sowie \cite{Schu:92}, Seite 30).\label{matroid}

Der {\bf Steinitzsche Austauschsatz}\idx{Steinitz, Austauschsatz von} besagt,
daß wenn in einem endlichdimensionalen Vektorraum V ein Vektor v nicht linear
abhängig von einer unabhängigen Menge von Vektoren $\{u_1,\ldots,u_n\}$, aber
abhängig von $\{u_1,\ldots,u_n,w\}$ ist, so ist der Vektor w linear abhängig
von $\{u_1,\ldots,u_n,v\}$.\\
Als Verallgemeinerung hiervon wird mit E=$\{1,\ldots,n\}$ und einer Teilmenge
$\B$ der Potenzmenge von E das geordnete Paar (E,$\B$) als ein {\bf Matroid}
\idx{Matroid} bezeichnet, wenn die leere Menge \O\ in $\B$ liegt, mit jeder
Menge $B\in\B$ auch deren Teilmengen in $\B$ liegen, sowie für alle $B_1,
B_2\in\B$ und $x\in B_1\backslash B_2$ ein y aus $B_2\backslash B_1$ existiert,
so daß $(B_1\backslash\{x\})\cup\{y\}$ Element von $\B$ ist. Die Mengen aus
$\B$ heißen {\bf unabhängig}, maximale Mengen in $\B$ werden aufgrund ihrer
gemeinsamen Mächtigkeit {\bf Basen des Matroids} genannt. Diese Mächtigkeit
wird auch als {\bf Rang des Matroids} bezeichnet. Ist E zusammen mit einer
(wie oben definierten) Abbildung $\chi$ ein orientiertes Matroid, so kann man
alle (formalen) Brackets betrachten, die unter $\chi$ ungleich Null sind. Die
Menge der zugehörigen r-Tupel aus $\Lambda(n,r)$ all dieser Brackets nennt man
supp($\chi$). Dieser {\bf Träger} supp($\chi$) bildet die Menge der Basen eines
Matroids, das dem {\bf orientierten Matroid} (E,$\chi$) {\bf zugrundeliegende
Matroid}. In bezug auf die zu erfüllenden Graßmann-Plücker-Relationen ist ein
Matroid die Einschränkung eines über dem Körper GF(3) definierten
orientierten Matroids auf GF(2), was sich auch aus dem folgenden ableiten
läßt.
\begin{satz}\label{dgpr}
Eine Abbildung $\chi:E^r\to\{-1,0,+1\}$ (in der bisherigen Notation) ist genau
dann ein Chirotop, wenn folgende zwei Bedingungen erfüllt sind:

\btab{ll}
(B1$'$) & $\chi$ ist alternierend, und die Menge der r-Untermengen
          $\{x_1,\ldots,x_r\}$ aus E\\
        & mit $\chi(x_1,\ldots,x_r)\neq 0$ ist die Menge der Basen eines
          Matroids vom \\
        & Rang r auf E.\\
(B2$''$) & Für alle $x_1,\ldots,x_r,y_1,y_2\in E$ gilt, falls\\
         & $\chi(y_1,x_2,\ldots,x_r)\cdot\chi(x_1,y_2,x_3,\ldots,x_r)\geq 0$
           und\\
         & $\chi(y_2,x_2,\ldots,x_r)\cdot\chi(y_1,x_1,x_3,\ldots,x_r)\geq 0$
           erfüllt sind, auch\\
         & $\chi(x_1,x_2,\ldots,x_r)\cdot\chi(y_1,y_2,x_3,\ldots,x_r)\geq 0$
\etab
\end{satz}

Die dreisummandigen Graßmann-Plücker-Relationen können im Rang r für
disjunkte Teilmengen $A=\{a_1,\ldots,a_{d-2}\}$ und $B=\{b_1,\ldots,b_4\}$
von E mittels (formaler) Brackets geschrieben werden als
$$\{a_1,\ldots,a_{d-2}|b_1,\ldots,b_4\}:=
  \begin{array}{l}
    +[a_1,\ldots,a_{d-2},b_1,b_2]\cdot [a_1,\ldots,a_{d-2},b_3,b_4]\\
    -[a_1,\ldots,a_{d-2},b_1,b_3]\cdot [a_1,\ldots,a_{d-2},b_2,b_4]\\
    +[a_1,\ldots,a_{d-2},b_1,b_4]\cdot [a_1,\ldots,a_{d-2},b_2,b_3]
  \end{array}=0$$
Im Falle von $\chi:E^d\to\{-1,+1\}$ läßt sich so in einem Programm testen,
ob bei Vorgabe von $\chi$ ein Chirotop und damit ein orientiertes Matroid
vorliegt. Allgemein muß hier zusätzlich überprüft werden, ob durch eine
vorgelegte Vorzeichenliste die nach B1$'$ geforderten Basen eines Matroids
gegeben sind --- als Alternative kann man hier auch die k-summandigen
Graßmann-Plücker-Relationen von B2 nachrechnen.

Wie man sich überlegen kann, ist aus einer Punktkonfiguration über die
zugehörige Punktmatrix für n $\geq$ d Punkte immer ein orientiertes Matroid
durch ein Chirotop gegeben, da hierbei die k-summandigen
Graßmann-Plücker-Relationen immer erfüllt sind (nachweisbar ist dies mittels
des Laplaceschen Entwicklungssatzes für Determinanten). Umgekehrt kann
zu diesem auch wieder eine (zumindest isomorphe) Punktkonfiguration angeben
werden, da man ja weiß, daß eine solche existiert. Im allgemeinen Fall kann
allerdings von einem Chirotop nicht auf eine zugehörige Punktkonfiguration
geschlossen werden. Dieses Problem, zu einem orientierten Matroid eine
entsprechende Punktmatrix zu finden, bezeichnet man als Suche nach einer
Realisierung.\label{real}

Als eine {\bf Realisierung}\idx{Realisierung} eines orientierten Matroids
${\cal M}=(E,\chi)$ vom Rang r über einer totalgeordneten n-elementigen Menge
E bezeichnet man eine Abbildung $\Phi$, die E in den $\R^r$ abbildet, so daß
$\chi(e_1,e_2,\ldots,e_r)=\mbox{sign det}(\Phi(e_1),\Phi(e_2),\ldots,\Phi(e_r))$
für alle $e_i\in E$ gilt, wenn $\chi:E^r\to\{-1,0,+1\}$ das zugehörige
Chirotop bezeichnet. Existiert solch ein $\Phi$, so heißt $\cal M$
realisierbar.

{\scsi
Eine wichtige Eigenschaft im Zusammenhang mit der Realisierung von
orientierten Matroiden ist, daß diese immer (zumindest) lokal realisierbar
sind (in diesem Sinne sind sie lokal in den Euklidischen Raum einbettbar). Dazu
wird in \cite{Bj:93} (Korollar 3.6.3, Seite 140) gezeigt, daß jedes azyklische
Rang r orientierte Matroid als abstrakte (r-1)-dimensionale affine
Punktkonfiguration angesehen werden kann, von der jede (r+2)-punktige
Unterkonfiguration koordinatisierbar ist. Bokowski und Richter-Gebert haben
zu diesem Themenbereich interessante Arbeiten beigesteuert.
}

Die Suche nach Realisierungen beliebiger orientierter Matroide ist als eine
wichtige Fragestellung in \cite{Bj:93} angesprochen und Gegenstand aktueller
Arbeiten, wie zum Beispiel auch \cite{Schu:92} und \cite{Dau:89}, in denen von
simplizialen orientierten Matroiden ausgegangen wird. Hier soll nun das Problem
der Realisierbarkeit von allgemeinen CW-Komplexen, im Sinne durchdringungsfreier
Darstellung im euklidischen Raum, auf die Problematik bei orientierten Matroiden
übertragen werden, wobei nun auch der nichtsimpliziale Fall zuzulassen ist.
Gibt es nämlich zu einem beliebigen CW-Komplex $\cal C$ ein realisierbares
orientiertes Matroid $(E,\chi)$, das gewisse Eigenschaften von $\cal C$
bewahrt, so sind durch $\Phi$ Punkte im $\R^r$ gegeben, mit denen $\cal C$ ohne
Selbstdurchdringungen dargestellt werden kann. Solch ein $(E,\chi)$ heißt eine
{\bf Matroideinbettung} von $\cal C$.\idx{Matroideinbettung} Zunächst muß
hierzu aber eine Verbindung zwischen CW-Komplexen und orientierten Matroiden
angegeben werden, die gerade diese "`gewissen Eigenschaften"' beschreibt.\\
Da wir es bei CW-Komplexen mit zum $\R^n$ homöomorphen Hausdorff-Räumen,
zutun haben, soll nun eine weitere Repräsentationsmöglichkeit orientierter
Matroide angegeben werden, die (Pseudo-)Hyperebenen nutzt, welche dazu dienen
sollen, oben geforderte Verbindung zwischen den beiden betrachteten
Objektklassen zu schaffen.

\subsection{Orientierte Matroide von Hy\-per\-ebe\-nen- und
            Pseu\-do\-hy\-per\-ebe\-nen\-arrange\-ments}

Zur Motivation betrachte man wieder eine endliche Menge $E=\{1,2,\ldots,n\}$
und ein zentrales (den Nullpunkt enthaltendes) Arrangement orientierter
Hyperebenen im $\R^r$ (Beispiel Abb.\ref{hyper}), das gegeben sei durch
$${\cal A}=(H_e=\{x\in \R^r:<x,a_e>=0\})_{e\in E}$$ mit
einer Familie $(a_e)_{e\in E}$ von Normalenvektoren ungleich dem Nullvektor
und dem üblichen Skalarprodukt $<x,y>:=\sum_{i=1}^rx_i\cdot y_i$ für x und
y aus $\R^r$ (beziehungsweise einem Vektorraum über $\K$).
Um Orientierungen unterscheiden zu können, seien die durch die Hyperebenen
gegebenen Halbräume in Richtung der Normalen als positiv ausgezeichnet.
Zu allen $x\in\R^r$ sei $\sigma(x)=(\sigma_e(x)=\mbox{sign}<x,a_e>)_{e\in E}$
ein Vorzeichenvektor mit $|E|=n$ Komponenten der Gestalt $+$, $-$ oder $0$,
als Abkürzung für $+1$, $-1$ und $0$ entsprechend der Signumfunktion,
der die Lage von x bezüglich jeder Hyperebene $H_e$ beschreibt.
Die Abbildung $\sigma$ bildet dabei den $\R^r$ in $\{+,-,0\}^E$, die Menge
aller Vorzeichenvektoren mit $|E|$ Komponenten, ab.
Jene Punkte des $\R^r$, die gleiche Vorzeichenvektoren besitzen, bilden
Zellen der Zerlegung, die durch $\cal A$ auf $\R^r$ induziert wird. Diese Zellen
sind konvexe offene Teilmengen linearer Teilräume des $\R^r$, und heißen
{\bf Topes}\idx{Tope} beziehungsweise {\bf Regionen}.\label{cell}
Mit $\sigma$ ist so eine Äquivalenzrelation gegeben, denn es gilt für
Punkte $x,y\in\R^r$ die Gleichheit $\sigma(x)=\sigma(y)$ genau dann, wenn
x und y aus der gleichen Zelle der Zerlegung des $\R^r$ durch $\cal A$ stammen.
Anstelle aller $x\in \R^r$ brauchen so nur deren Äquivalenzklassen
bezüglich $\sigma$ betrachtet werden, denn die Vorzeichenvektoren im Bild des
$\R^r$ unter $\sigma$ induzieren genau die Zellen, was eine kombinatorische
Beschreibung von $\cal A$ durch $\sigma$ darstellt.

\begin{figure}[htb]
$$
\beginpicture
\unitlength0.6cm
\setlinear
\setcoordinatesystem units <0.6cm,0.6cm>
\setplotarea x from 0 to 14, y from 0 to 9.5
\setsolid
\plot 5 8 5 5 4 4 4 7 6 9 6 8.35 /
\plot 4 4.65 3.5 4.5 3.5 7.5 6.5 8.5 6.5 7.5 /
\plot 3.5 7.5 3.5 8.5 6.5 7.5 6.5 4.5 6 4.65 /
\plot 4 8.35 4 9 6 7 6 4.2 5 5 /
\plot 6.5 6.85 7 7 8.5 5 2.5 3 1 5 3.5 5.8 /
\plot 3.5 3.35 3.5 2 4 2.2 /
\plot 4 3.5 4 1.5 5 2.5 5 3.85 /
\plot 5 2.5 6 1.5 6 4.15 /
\plot 6 2.2 6.5 2 6.5 4.3 /
\put {\scsi H$_1$} [Bl] at 2 4.5
\put {\scsi H$_2$} [Bl] at 4.3 5.7
\put {\scsi H$_3$} [Bl] at 2.8 7
\put {\scsi H$_4$} [Bl] at 2.8 8
\put {\scsi H$_5$} [Bl] at 3.5 9.2
\put {\scsi $+$} [Bl] at 2.5 3.5 \put {\scsi $-$} [Bl] at 2.5 2.7
\put {\scsi $+$} [Bl] at 4.5 7
\put {\scsi $-$} [Bl] at 5.25 7
\put {\scsi $+$} [Bl] at 3.75 7.25
\put {\scsi $+$} [Bl] at 3.75 8
\put {\scsi Regionen} [Bl] at 9 9
\put {\scsi $++++-$, $+-++-$,} [Bl] at 9 8
\put {\scsi $+--+-$, $+----$,} [Bl] at 9 7.5
\put {\scsi $+---+$, $++--+$,} [Bl] at 9 7
\put {\scsi $+++-+$, $+++++$,} [Bl] at 9 6.5
\put {\scsi $-+++-$, $--++-$,} [Bl] at 9 6
\put {\scsi $---+-$, $-----$,} [Bl] at 9 5.5
\put {\scsi $----+$, $-+--+$,} [Bl] at 9 5
\put {\scsi $-++-+$, $-++++$} [Bl] at 9 4.5
\endpicture
$$
\caption{Fünf Hyperebenen im $\R^3$}
\label{hyper}
\end{figure}

Nun betrachte man zu einer endlichen Menge E allgemeiner zunächst beliebige,
das heißt nicht über eine Abbildung $\sigma$ an Hyperebenen gekoppelte,
Vorzeichenvektoren\idx{Vorzeichenvektor} $X,Y\in\{+,-,0\}^E$.
Der {\bf Träger}\idx{Vorzeichenvektor!Tr\äger} eines solchen Vektors X ist
$\ul{X}=\{e\in E:X_e\neq 0\}$, seine {\bf Nullmenge}
\idx{Vorzeichenvektor!Nullmenge} $z(X)=X^0=E\backslash\ul{X}=\{e\in E:X_e=0\}$,
die Menge seiner positiven Elemente $X^+=\{e\in E:X_e=+\}$ und analog die Menge
seiner negativen Elemente $X^-=\{e\in E:X_e=-\}$. Der Vektor, der für alle
$e\in E$ aus 0 besteht, heiße analog dem $\R^d$ Nullvektor und sei mit 0
bezeichnet. Zu $X$ sei $-X$ gegeben durch die Vorzeichenumkehrung von X, also
$(-X_e)=-$ für $X_e=+$, $(-X_e)=+$ für $X_e=-$ und $(-X_e)=0$ für $X_e=0$.
Weiter sei eine {\bf Zusammensetzung}\idx{Vorzeichenvektor!Zusammensetzung}
$X\circ Y$ von X und Y dadurch gegeben, daß $(X\circ Y)_e=X_e$ für
$X_e\neq 0$ und $Y_e$ sonst gelte. Die {\bf Trennungsmenge}
\idx{Vorzeichenvektor!Trennungsmenge} von X und Y sei
$S(X,Y)=\{e\in E:X_e=-Y_e\neq 0\}$.

Mit diesen Vorgaben heißt eine Menge ${\cal L}\subseteq\{+,-,0\}^E$ {\bf Menge
der Kovektoren eines orientierten Matroids}, wenn folgende Axiome erfüllt
sind, die dementsprechend als {\bf Kovektoraxiome}\idx{Kovektoraxiome}
bezeichnet werden.

\bcent
\fbox{\parbox{14cm}{
\btab{ll}
(L0) & ${\cal L}$ enthält den Nullvektor 0.\\
(L1) & Mit $X\in\cal L$ ist auch $-X$ in ${\cal L}$ enthalten.\\
(L2) & Sind X und Y aus $\cal L$, so auch deren Zusammensetzung $X\circ Y$\\
(L3) & Zu X und Y aus $\cal L$ und e aus S(X,Y) existiert ein Z in
       ${\cal L}$ mit Z$_e=0$\\
     & und Z$_f=(X\circ Y)_f=(Y\circ X)_f$ für alle f $\notin$ S(X,Y).
\etab}}
\ecent

Um die Begriffswahl der Vorzeichenvektoren als "`\ul{Ko}vektoren eines
orientierten Matroids"' verstehen zu können, müssen an dieser Stelle die
Begriffe Polarität und Dualität eingeführt werden, wie sie im gegenwärtigen
Zusammenhang zu verstehen sein sollen. Dazu ist ein kurzer Abstecher in die
Welt der topologischen Vektorräume nötig, der uns die Definition der
gewünschten Begriffe liefert.

Allgemein werden zwei topologische Vektorräume F und G über einem gemeinsamen
Körper $\K$ als ein {\bf duales Paar}\idx{duales Paar} (F,G) bezeichnet, wenn
es zwischen ihnen eine Bilinearform $f:F\times G\to\K$ gibt, für die zum einem
für jedes feste $0\neq x_0\in F$ ein $y\in G$ existiert, so daß
$f(x_0,y)\neq 0$ gilt, und zum anderen analog zu jedem festen $0\neq y_0\in G$
ein $x\in F$ gewählt werden kann, mit $f(x,y_0)\neq 0$. Ist nun (F,G) solch
ein duales Paar, so ist die {\bf Polare}\idx{Polare} $A^*$ einer Menge
$A\subseteq F$ definiert als
$$ A^*=\{y\in G~:~|f(x,y)|\leq 1,~\forall x\in A\}.$$

Nach Grünbaum (vgl. \cite{Gr:67}, Seite 46ff.) heißen zwei d-dimensionale
Polytope $P$ und $P^*$ dual\idx{dual} wenn zwischen ihnen eine bijektive
Abbildung $\Psi$ der r-Seiten ($1\leq r\leq d$) von $P$ auf die Seiten von
$P^*$ existiert, die Inklusionen umkehrt. Dies bedeutet, daß für zwei Seiten
$F_1$ und $F_2$ von $P$ mit $F_1\subset F_2$ unter $\Psi$ gilt, daß
$\Psi(F_1)\supset\Psi(F_2)$ ist. In der Theorie der orientierten Matroide
wird dieser Sachverhalt als {\bf Polarität}\idx{Polarit\ät} bezeichnet.\\
Da hier Vektorräume über $\R$ betrachtet werden, ist eine Bilinearform f
durch das übliche Skalarprodukt auf $\R^d$ gegeben. Damit ist die zu einem
Polytop $P$ gehörige polare Menge $P^*$ gerade
$\{y\in\R^d~|~<x,y>\leq 1,~\forall x\in P\}$.
Bildet man die konvexe Hülle der Extrempunkte von $P^*$, so ist diese ebenfalls
ein Polytop, das als das zu P polare bezeichnet wird und entsprechend seiner
Herkunft ebenfalls die Bezeichnung $"`P^*"'$ erhält. Nach obiger Definition ist
$P^*$ auch dual zu $P$, da sich eine inklusionsumkehrende Abbildung der
Seiten von $P$ auf die Seiten von $P^*$ angeben läßt. Anschaulicher ist dieser
Sachverhalt darzustellen, wenn man die Situation betrachtet, daß der einen
Punkt P im $\R^d$ beschreibende Ortsvektor $x_P$ zum einen den Punkt als solchen
bezeichnet, aber auch als Normale einer zu $x_P$ orthogonalen (Hyper-)Ebene mit
der Hesseschen Normalform $<x_P,x>=1$ aufgefaßt werden kann (vgl. obige
Definition der Hyperebenenarrangements). Betrachtet man nun einerseits die
konvexe Hülle einer endlichen Anzahl von Punkten $x_P$ und andererseits die
durch die (Hyper-)Ebenen mit den Normalen $x_P$ eingeschlossene Zelle
(gerade der Durchschnitt der Halbräume, die durch $<x_P,x>\leq 1$ gegeben sind,
so ergibt sich gerade die oben eingeführte Polarität zwischen einer
Punktkonfiguration und der entsprechenden polaren (Hyper-)Ebenenkonfiguration.
\label{polar} Eindrucksvolle Beispiele sind die Polaritäten der platonischen
Körper im $\R^3$ (das selbstpolare Tetraeder, die polaren Paare Würfel und
Oktaeder, sowie Dodekaeder und Ikosaeder).
Speziell entsprechen unter Polarität im $\R^3$ also Punkte Flächen, Kanten
Kanten und Flächen Punkten. Verallgemeinert auf höhere Dimensionen
bedeutet dies gerade, daß im $\R^d$ (d-i)-Seiten von $P$ im polaren Fall zu
i-Seiten von $P^*$ werden (was gleichzeitig Dualität induziert).\\
Bei orientierten Matroiden sei mit Dualität\idx{Dualit\ät} die Situation
bezeichnet, wenn man, wie Bokowski in seinem Beitrag zum "`Handbook of Convex
Geometry"' (\cite{Bo:93}) angibt, etwa die Kreise eines orientierten Matroids
$\cal M$ als Kokreise eines anderen orientierten Matroids $\cal M^*$ ansieht,
welches dann das zu $\cal M$ Duale darstellt (vgl. Kreis-Definitionen von
Seite \pageref{kokreis}).\\
Wenn also von Vektoren und Kovektoren orientierter Matroide die Rede ist, so
spiegelt dies gerade die dualen Versionen wider. (Da die Dualität orientierter
Matroide in dieser Arbeit nicht weiter vertieft wird, möchte ich auf
\cite{Bj:93},S.115ff.,S.157ff. und \cite{Schu:92}, Seite 40ff. verweisen.)

Der Einfachheit halber sei im folgenden eine Menge ${\cal L}$, die die
Kovektoraxiome erfüllt beziehungsweise ein Paar $(E,{\cal L})$ als
orientiertes Matroid bezeichnet. Dieses habe den Rang r, entsprechend der
Dimension des Raumes, in dem die Hyperebenen liegen und n$=|E|$ Elemente.

Eine Teilordnung "`$\leq$"' auf $\{+,-,0\}$ mit $0<+$, $0<-$ und bezüglich
der $+$ und $-$ nicht vergleichbar sind, induziert auf $\{+,-,0\}^E$ über
den komponentenweisen Vergleich der Vorzeichenvektoren eine Produktteilordnung.
(Es gilt $y\leq x$, wenn für alle $e\in E$ die Komponente $y_e\in\{0,x_e\}$
ist.) Ist ${\cal L}$ ein orientiertes Matroid nach obiger Definition, so wird
dieses, wenn man es mit einem abstrakten \^1-Element vereinigt und mit
eben definierter Teilordnung betrachtet zu einem Verband, dem (großen)
Seitenverband $\hat{\cal L}={\cal L}\cup \{\hat{1}\}$\idx{Seitenverband} von
$\cal L$. Bezüglich "`$\leq$"' maximale Elemente in $\cal L$ heißen
{\bf Topes} oder auch Regionen\idx{Topes!eines orientierten Matroids} von
$\cal L$ und entsprechen gerade den im realisierbaren Fall von Hyperebenen
"`ausgeschnittenen"' Teilräumen. (vgl. Abb.\ref{hyper})

Bei der Untersuchung orientierter Matroide sind auch Teilstrukturen dieser
interessant, die aus Operationen auf einem orientierten Matroid entstehen. So
ist eine {\bf Restriktion}\idx{Restriktion} eines Vorzeichenvektors
$X\in\{+,-,0\}^E$ auf eine Teilmenge F$\subset$E der Vorzeichenvektor
$X|_F\in\{+,-,0\}^F$, definiert durch $(X|_F)_e=X_e$ für alle e$\in$F. Ist
${\cal L}\subseteq\{+,-,0\}^E$
die Menge der Kovektoren eines orientierten Matroids $\cal M$ auf E und ist
A$\subseteq$E, so ist die Menge der Kovektoren der {\bf Deletion}\idx{Deletion}
${\cal M}\backslash A$ die Menge ${\cal L}\backslash A
:=\{X|_{E\backslash A}:X\in{\cal L}\}\subseteq\{+,-,0\}^{E\backslash A}$.
Die Menge der Kovektoren der {\bf Kontraktion}\idx{Kontraktion} ${\cal M}/A$
ist die Menge ${\cal L}/ A:=\{X|_{E\backslash A}:X\in{\cal L}\mbox{ und }
A\subseteq X^0\}\subseteq\{+,-,0\}^{E\backslash A}$.
Unter einer {\bf Reorientierung}\idx{Reorientierung} $_{-A}{\cal M}$ versteht
man die Menge $_{-A}{\cal L}:=\{_{-A}X:X\in{\cal L}\}$, wobei $_{-A}X$ definiert
ist durch $(_{-A}X)^+=(X^+\backslash A)\cup (X^-\cap A), (_{-A}X)^0=X^0$ und
$(_{-A}X)^-=(X^-\backslash A)\cup (X^+\cap A)$, also einer lokalen
Vorzeichenumkehrung.

Gehen wir nochmals zurück zum Ausgangspunkt eines zentralen
Hyperebenenarrangements $\cal A$. Da Mengen von Hyperebenen
H$_e=\{x\in\R^r:<x,a_e>=0\}$ betrachtet werden, kann man äquivalent statt der
H$_e$ auch deren Einschränkung auf die Einheitssphäre $\S^{r-1}\subset\R^r$
heranziehen (Großkreise auf der $\S^2$ oder im allgemeinen Fall (r-2)-Sphären
auf der $\S^{r-1}$), ohne daß an der Repräsentation der Schnitteigenschaften
etwas verändert wird. (Dies ist möglich, da durch die "`zentralen"'
Hyperebenen schon der lineare respektive projektive Fall betrachtet wird,
eine Kompaktifizierung der Hyperebenen also nur formal vollzogen werden muß,
eben durch den Übergang zu einem Sphärensystem.)
Die durch die $\{x\in\R^r:<x,a_e>=0,\|x\|=1\}$ induzierte Zerlegung der
$\S^{r-1}$ läßt sich nun analog den Hyperebenen orientieren, indem man jene
Hemisphären als positiv annimmt, die in Richtung der Normalen $(a_e)_{e\in E}$
der H$_e$ liegen. Werden die Schnitteigenschaften der (r-2)-Sphären
untereinander nicht verändert, so können diese sogar mittels stetiger
Abbildungen (in \cite{Bj:93} werden diese als "`zahm"' bezeichnet, vgl. S. 225)
deformiert werden, wobei die Information des zugehörigen orientierten Matroids
erhalten bleibt. So gelangt man zur Definition von Pseudosphären. Dabei wird
eine Teilmenge $S\subset\S^{r-1}$ als {\bf Pseudosphäre}\idx{Pseudosph\äre}
bezeichnet, wenn ein Homöomorphismus $h:\S^{r-1}\to\S^{r-1}$ existiert, so
daß $S=h(S^{r-2})$ gilt. Hierbei ist $S^{r-2}=\{x\in\S^{r-1}:x_r=0\}$ der
Äquator der $\S^{r-1}$ (vgl. Abb.\ref{pseudo}). Ein {\bf
Pseudosphärenarrangement} in $\S^r$ ist dann als ein endliches Mengensystem
${\cal A}=(S_e)_{e\in E}$ definiert, wenn (vgl. \cite{Bj:93}, S.227) zum einen
$S_A=\Cap_{e\in A} S_e$ für alle $A\subseteq E$ eine Sphäre darstellt und zum
anderen für $S_A\not\subseteq S_e$ mit $A\subseteq E,e\in E$, sowie $S^+_e$ und
$S^-_e$ den beiden Hemisphären von $S_e$, $S_A\cap S_e$ eine Pseudosphäre in
$S_A$ mit den Seiten $S_A\cap S^+_e$ und $S_A\cap S^-_e$ ist.

Ein Ergebnis von Folkman und Lawrence aus dem Jahre 1978 liefert nun den
Zusammenhang zwischen Pseudosphärenarrangements und orientierten Matroiden.
\begin{satz}
Sei ${\cal A}=(S_e)_{e\in E}$ ein orientiertes Pseudosphärenarrangement in
der $\S^r$ und $\sigma:\S^r\to\{+,-,0\}^E$ die Abbildung, die definiert
sei über
$$\sigma(x)_e=\left\{\begin{array}{ll}
                        +, & \mbox{ für } x\in S^+_e\\
                        -, & \mbox{ für } x\in S^-_e\\
                        0, & \mbox{ für } x\in S_e\end{array}\right.$$
Dann ist ${\cal L(A)} := \{\sigma(x):x\in\S^r\}\cup\{0\}\subseteq\{+,-,0\}^E$
die Menge der Kovektoren eines orientierten Matroids. Ist dim($S_e$)=k, so
ist der Rang von $\cal L(A)$ gleich r-k. Ist $\cal A$ essentiell, das heißt
$S_E=\Cap_{e\in E} S_e$ ist leer, so ist der Rang von $\cal L(A)$ gleich r+1.
\end{satz}

Mit dem Beweis der Umkehrung schließt sich hier der Topologische
Repräsentationssatz nach Folkman und Lawrence für orientierte Matroide an
(vgl \cite{Bj:93}, Seite 233). Dieser besagt
\idx{Topologischer Repr\äsentationssatz}
\begin{satz}
{\bf Topologischer Repräsentationssatz:} Ist ${\cal L}\subseteq\{+,-,0\}^E$, so
ist $\cal L$ genau dann die Menge der Kovektoren eines schleifenfreien
orientierten Matroids vom Rang r+1, wenn ein orientiertes
Pseudosphärenarrangement $\cal A$ auf der $\S^{r+1+k}$ existiert, für das
$k=\mbox{dim}(\Cap_{e\in E}\S_e)$ gilt und bezüglich dem $\cal L$ gleich
$\cal L(A)$ ist.
\end{satz}

Schleifenfrei bedeutet, daß die Vorzeichenstruktur jedes Topes keine Null
enthält. Als Folgerung hieraus läßt sich zeigen, daß schleifenfreie
orientierte Matroide vom Rang r+1 (bis auf Reorientierung und Isomorphie)
bijektiv zu essentiellen orientierten Arrangements von Pseudosphären auf
$\S^r$ (bis auf topologische Äquivalenz) korrespondieren.

Dieses herausragende Ergebnis liefert also die Äquivalenz von
Pseudosphärenarrangements und orientierten Matroiden respektive der gewählten
Axiomatik. Hier ist es wichtig, daß von \ul{Pseu\-do}\-sphären die Rede ist,
da ein topologisches Arrangement (im Sinne informationserhaltender stetiger
Verformung) nicht homöomorph zu einem Sphärensystem sein muß.
Ist dies aber der Fall, so handelt es sich um die Darstellung eines
realisierbaren orientierten Matroids, analog der Definition der Realisierung,
wie sie im Abschnitt über die Chirotope angegeben wurde (Seite \pageref{real}).

Zu bemerken ist, daß der reelle projektive Raum $\P^d$ der Dimension d aus
der $\S^d$ durch Identifikation aller antipodischen (diametral
gegenüberliegenden) Punkte hervorgeht. Ist $\pi(x)=\{x,-x\}:\S^d\to \P^d$,
so werden in natürlicher Weise die Nullpunkt-symmetrischen Teilmengen
der d-Sphäre mit allgemeinen Teilmengen des projektiven Raumes
identifiziert. Hierüber können oben eingeführte Pseudosphären auch als
Pseudohyperebenen in $\P^d$ aufgefaßt werden, womit auch eine Klassifikation
orientierter Matroide im projektiven Raum mittels Pseudohyperebenen gegeben ist.

\begin{figure}[htb]
$$
\beginpicture
\unitlength0.6cm
\setlinear
\setcoordinatesystem units <0.6cm,0.6cm>
\setplotarea x from -3 to 3, y from -3 to 3
\put {\beginpicture
  \setsolid
  \circulararc 180 degrees from 2 0 center at 0 0
  \ellipticalarc axes ratio 4:1 180 degrees from -2 0 center at 0 0
  \setdashes<1.5mm>
  \ellipticalarc axes ratio 4:1 -180 degrees from -2 0 center at 0 0
  \setsolid
  \plot -1.73 1 -3 1 -4 -1 3 -1 4 1 1.73 1 /
  \circulararc 120 degrees from -1.73 -1 center at 0 0
  \put {\vector(0,1){1.5}} [Bl] at 0 0
  \put {\vector(1,0){0.2}} [Bl] at 0 -0.5
  \put {\scsi $\S^{d-1}$} [Bl] at 1.5 2
  \put {\scsi $\S^{d-2}_a$} [Bl] at 1.5 -0.8
  \put {\scsi a} [Bl] at 0.3 1.5
  \put {\scsi $H_a$} [Bl] at 2.5 0.5
  \setquadratic
  \plot -2 0 -1.3 0.3 -0.7 0.1 0.2 -0.1 1.2 0.15 1.6 0.2 2 0 /
  \setdashes<1mm>
  \plot 2 0 1.3 -0.3 0.7 -0.1 -0.2 0.1 -1.2 -0.15 -1.6 -0.2 -2 0 /
\endpicture} at -4 1
\put {\beginpicture
  \setsolid \setlinear
  \circulararc 360 degrees from 2 0 center at 0 0
  \setquadratic
  \plot -1.732 -1 -0.5 -1 0.5 -0.2 1 0.5 1.732 1 /
  \plot 0 -2 0.35 -1 0 0 -0.5 1 0 2 /
  \plot 1.732 -1 1 0 0.3 0.3 -1 0.5 -1.732 1 /
  \setdashes<1mm>
  \plot 1.732 1 0.5 1 -0.5 0.2 -1 -0.5 -1.732 -1 /
  \plot 0 2 -0.35 1 0 0 0.5 -1 0 -2 /
  \plot -1.732 1 -1 0 -0.3 -0.3 1 -0.5 1.732 -1 /
  \setdots<1mm> \setlinear
  \plot 0.5 1.95 0.5 3 -0.5 2 -0.5 -3 0.5 -2 0.5 -1.95 /
  \plot -2 0.4 -3 1 -2 2 3 -1 2 -2 1.35 -1.55 /
  \plot -2 -0.4 -3 -1 -2 -2 3 1 2 2 1.35 1.55 /
\endpicture} at 4 0
\endpicture
$$
\caption{Von einer Hyperebene zu einer Pseudosphäre}
\label{pseudo}
\end{figure}

Zusammenfassend kann man sagen, daß sich orientierte Matroide als
Verallgemeinerung aus den verschiedensten Bereichen der Mathematik motivieren
lassen. Dabei führt in verschiedenen Fällen die allgemeinste Struktur
bezüglich einer zu erfüllenden Eigenschaft unabhängig von der gewählten
Ausgangssituation zu diesem Konzept. Der Name "`Matroid"' stammt dabei von
Whitney aus dem Jahre 1935 und leitet sich von einer Klasse fundamentaler
Beispiele solcher Objekte ab, die aus Matrizen hervorgehen. (Als
Übersichtsartikel sei hierzu \cite{Bo:93} genannt.)

\section{Symmetriebegriffe}

Ist ein allgemeiner CW-Komplex realisierbar, so ist es erstrebenswert,
möglichst viele Symmetrieeigenschaften von diesem in die Realisierung
"`hinüberzuretten"'. Was man hierunter versteht, soll im folgenden
beschrieben werden.

Wenn man im $\E^d$ von {\bf Symmetrien}\idx{Symmetrie} spricht, so meint man
damit die Automorphismen von $\E^d$, die Abstände und Orthogonalität
invariant lassen. Dies sind gerade die orthogonalen Transformationen.
Die Menge aller orthogonalen Transformationen ist bezüglich der
Hintereinanderausführung eine Gruppe, die {\bf orthogonale Gruppe}
${\cal O}(d)$.\idx{orthogonale Gruppe} Elemente aus ${\cal O}(d)$ lassen sich
als orthogonale (d$\times$d)-Matrizen über $\R$ darstellen und stellen damit
eine Untergruppe der allgemeinen linearen Gruppe $\mbox{GL}_d(\R)$, der Gruppe
aller (d$\times$d)-Matrizen über $\R$, dar.
$${\cal O}(d) = \{ A\in\mbox{GL}_d(\R)~|~A^T A = I_d\}$$
{\scsi
Eine Matrix heißt orthogonal, wenn sie die kanonischen Einheitsvektoren auf
eine Orthonormalbasis abbildet, das heißt, wenn ihre Spaltenvektoren
auf Länge Eins normiert sind, sowie das Skalarprodukt von je zwei
verschiedenen Spaltenvektoren Null ergibt. Als Folgerung hieraus ist die
Determinante einer orthogonalen Matrix immer $\pm 1$. Wie mit linearer Algebra
gezeigt werden kann, haben orthogonale Matrizen die Eigenschaft, daß ihre
Inversen gerade ihre Transponierten sind, was zur Charakterisierung
der orthogonalen Gruppe dient.\\
Zur Erinnerung ist eine Gruppe eine Menge G zusammen mit einer Abbildung
$*:G\times G\to G$, der Gruppenoperation, die folgendes Axiomensystem erfüllt.
So ist G bezüglich $*$ abgeschlossen, besitzt ein neutrales Element e
($g*e=e*g$ für alle $g\in G$) und zu jedem $g\in G$ existiert ein inverses
Element $g^{-1}$ ($g*g^{-1}=e=g^{-1}*g$). Als Literatur zur Linearen Algebra
sei Herrn Artmanns Buch \cite{Art:89}, sowie zur Einführung in die
Algebra Serge Langs Buch \cite{La:79} empfohlen.
}\\
Damit wird die Matrizenmultiplikation zur Gruppenoperation.
Matrizen aus ${\cal O}(d)$ mit Determinante -1 heißen {\bf Spiegelungen} oder
auch {\bf Reflektionen},\idx{Reflektion}\idx{Spiegelung} Matrizen mit
Determinante +1 werden als {\bf Drehungen}\idx{Drehung} oder {\bf Rotationen}
\idx{Rotation} bezeichnet. Die Drehungen bilden dabei eine Untergruppe der
${\cal O}(d)$, die als spezielle orthogonale Gruppe ${\cal SO}(d)$ bezeichnet
wird.\idx{spezielle orthogonale Gruppe}\\
{\scsi
Die Untergruppe ${\cal SO}(d)$ ist gesondert ausgezeichnet, da es sich bei
ihr um die Komponente des Eins-Elements (das heißt Neutralelement) von
${\cal O}(d)$ handelt, wenn diese in ihre zwei Zusammenhangskomponenten
zerlegt wird. (vgl. \cite{Os:92}, Seite 130ff.)}

Einen geometrischen d-Komplex $\cal C$, das heißt die metrische Realisierung
eines kombinatorisch vorgelegten allgemeinen CW- oder simplizialen d-Komplexes,
bezeichnet man als {\bf symmetrisch},\idx{kombinatorischer Komplex}
\idx{geometrischer Komplex}\idx{geometrischer Komplex!symmetrischer $\sim$}
wenn eine Untergruppe G der ${\cal O}(d)$ existiert, deren sämtliche
Elemente $g\in G$ Automorphismen auf $\cal C$ sind, das heißt für die
sämtlich $g({\cal C}) = {\cal C}$ gilt. Die $g\in G$ heißen Symmetrien
von $\cal C$, G selbst heißt, gemäß der Gruppenstruktur, die Symmetriegruppe
\idx{Symmetriegruppe} von $\cal C$.\idx{Symmetriegruppe!$\sim$ geometrischer
Komplexe}
Die Anzahl der Elemente einer Symmetriegruppe bezeichnet man als deren
(Symmetrie-){\bf Ordnung}. Dies ist nicht mit der Ordnung einer Symmetrie, das
heißt eines Elements einer Symmetriegruppe, zu verwechseln, die angibt, wie oft
die betreffende Symmetrie angewendet werden muß, um wieder die Identität zu
erreichen. Mit "`hoher Symmetrie"' bezeichnet man dabei die Situation, daß
die Anzahl der Elemente der betreffenden Symmetriegruppe, gemessen an
der Anzahl der permutierten Ecken eines Komplexes, groß ist
\idx{Symmetrieordnung} (etwa auch die Situation, daß die Symmetriegruppe des
geometrischen eine "`große"' Untergruppe der Symmetriegruppe des kombinatorische
Komplexes ist).

Betrachtet man, wie auf Seite \pageref{symm}, die Inzidenzstrukturen
simplizialer Komplexe, die durch die Simplexzahlen impliziert werden, so
heißt die sich durch Umnummerierung (Permutation) der Ecken (Menge der
0-Simplizes) ergebende Automorphismengruppe des Komplexes, deren Elemente
nicht-degenerierend auf den Simplizes des Komplexes wirken, Symmetriegruppe
des simplizialen Komplexes.\idx{Symmetriegruppe!$\sim$ simplizialer Komplexe}
Mittels der Simplexzahlen kann jede solche Symmetrie, wie schon erwähnt,
als Element der Permutationsgruppe $S_n$, mit der Anzahl n der auftretenden
verschiedenen 0-Simplizes, beschrieben werden, also als Permutation auf der
Eckenmenge $E=\{1,\ldots,n\}$. Auch hier heißt die Gesamtheit der Symmetrien
des Komplexes entsprechend ihrer Gruppenstruktur Symmetriegruppe des
simplizialen Komplexes.

Da simpliziale Komplexe einen Spezialfall von CW-Komplexen darstellen, kann
man in einfacher Weise den Symmetriebegriff für simpliziale Komplexe
auf die CW-Komplexe erweitern. Wie im einen Fall bei nicht-degenerierenden,
bijektiven simplizialen Abbildungen von Symmetrien gesprochen wird, so können
dimensionserhaltende bijektive Selbstabbildungen von CW-Komplexen auch als
Symmetrien bezeichnet werden, wie dies schon im Abschnitt über CW-Komplexe
beschrieben wurde (in jedem Fall betrachtet man die Automorphismen des
Komplexes).

Für (zunächst) simpliziale d-Komplexe kann man den Begriff der Flagge
\idx{Flagge} für das $d+1$-Tupel (0-Simplex, 1-Simplex, $\ldots$, d-Simplex)
einführen, dessen Elemente paarweise inzidieren (Im Falle von 2-Kom\-plexen
ist dies ein Tripel (Ecke, Kante, Seite)). Automorphismen kombinatorischer
Komplexe\label{flag} erhalten solche Flaggen. Eine Gruppe G kombinatorischer
Automorphismen heißt nun {\bf Flaggen-transitiv},\idx{Flaggen-transitiv}
falls es zu jedem Paar von Flaggen des Komplexes einen Automorphismus aus G
gibt, der die erste auf die zweite Flagge abbildet. Ist die Symmetriegruppe
eines kombinatorischen Komplexes flaggentransitiv, so bezeichnet man den
betreffenden Komplex als {\bf regulär}.
\idx{kombinatorischer Komplex!$\sim$regul\ärer}
Bekannt ist diese Eigenschaft vor allem von den (regulären) Platonischen
Körpern, zu denen Analoga gesucht werden, indem man Möglichkeiten der
Realisierung regulärer (CW-)Komplexe untersucht, wie etwa die Dycksche Karte
(vgl. Kap.3).

Jede Symmetrie eines geometrischen Komplexes (eine als Matrix "`darstellbare"'
Symmetrie) induziert nun eine kombinatorische Symmetrie auf dem
zugrundeliegenden kombinatorischen Komplex. Folglich ist die Gruppe der
geometrischen Symmetrien eine Untergruppe der Gruppe der kombinatorischen
Symmetrien eines Komplexes.\\
In der Tat handelt es sich im allgemeinen Fall "`nur"' um eine Untergruppe, es
treten nämlich zumeist mehr kombinatorische als metrische Symmetrien auf (vgl.
die Betrachtung der Dyckschen Karte im dritten Kapitel). Die "`überzähligen"'
kombinatorischen Symmetrien bezeichnet man als versteckte Symmetrien
\idx{versteckte Symmetrien} des realisierbaren kombinatorischen Komplexes mit
dieser Eigenschaft. Im Falle konvexer 3-Polytope im $\E^3$ sind die metrische
und kombinatorische Symmetriegruppe stets gleich (isomorph), doch schon im
nicht-konvexen Fall können versteckte Symmetrien auftreten (vgl. \cite{Bo:91},
Seite 4). Als Beispiel für eine Symmetrieuntersuchung möge ein Torus dienen,
wie er in Kapitel 2, Seite \pageref{torus1} beschrieben ist.

Analog der bisherigen Definitionen werden Permutationen $\sigma\in S_n$ auf
der Grundmenge $E=\{1,\ldots,n\}$ eines orientierten Matroids $\chi$ als
Symmetrien von diesem bezeichnet, wenn es sich bei $\sigma$ um einen
Automorphismus handelt, der $\chi$ auf $\chi$ selbst (Drehung) oder
auf $-\chi$ (Spiegelung, Reflektion) abbildet. Die Gesamtheit aller
solchen Automorphismen eines orientierten Matroids $\chi$ heißt wiederum,
entsprechend ihrer Gruppenstruktur, Symmetriegruppe von $\chi$.
\idx{Symmetriegruppe!$\sim$ orientierter Matroide} Dieser Symmetriebegriff
läßt sich anschaulich an den Symmetrien der durch Pseudosphärensysteme
induzierten Komplexe illustrieren, aber auch an Vorzeichenwechseln des
Vorzeichenvektors eines orientierten Matroids festmachen.

Bei der Untersuchung der möglichen Symmetrien im Raum (hier ist der
dreidimensionale Euklidische Raum gemeint), die im letzten Jahrhundert
erfolgte und deren zentrales Ergebnis zum Ende des 19. Jahrhunderts die
Klassifizierung aller Raumsymmetrien lieferte, stellte sich heraus, daß bei
Raumsymmetrien von Gitterstrukturen, das heißt von periodischen Strukturen,
die den gesamten $\R^3$ ausfüllen, Symmetrieelemente nur mit den Ordnungen
1, 2, 3, 4 und 6 auftreten können. (vgl. etwa die Darstellung der
geschichtlichen Entwicklung der Symmetriekonzepte der Kristallographie von E.
Scholz \cite{Scho:89}). Dieses Ergebnis wurde in der Diplomarbeit von Ronald
Dauster (\cite{Dau:89}, Seite 30ff.) aufgegriffen und auf Realisierungen
orientierter Matroide übertragen. Danach sind affine Realisierungen
simplizialer orientierter Matroide nur mit oben angegebenen Ordnungen möglich,
was zumindest die Verifikation der Symmetrien bei vermeindlichen Realisierungen
unterstützt. Affine Realisierung bedeutet hier gerade, daß das orientierte
Matroid von einer Matrix stammt deren Zeilen explizite Punkte in homogenen
Koordinaten beschreiben, also alle in der selben Hyperebene des $\R^d$ liegen,
also zu $(1,p_i)$ normiert werden können und so einer affinen Punktmenge
im $\R^{d-1}$ entsprechen.

Da wir uns in dieser Arbeit aber mit allgemeinen CW-Komplexen beschäftigen
wollen, infolge dessen gerade eine Erweiterung auf nichtsimpliziale orientierte
Matroide anstreben, haben wir diese Einschränkung der Ordnungen der
Symmetrieelemente leider nicht zur Verfügung. Statt dessen stellt sich die
interessante Frage, welche Symmetrien bei der Realisierung beliebiger orientierter
Matroide auftreten können, ob Einschränkungen existieren oder ob es sogar eine
maximale auftretende Elementordnung gibt. Ein solches Ergebnis würde zu einem
Analogon der Klassifizierung der Raumsymmetrien führen und sicher einer
Bereicherung der Darstellung der Theorie der orientierten Matroide dienen.
