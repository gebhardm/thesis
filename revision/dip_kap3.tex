\chapter{Beispiele}

In diesem abschlie"senden dritten Kapitel sollen nun mittels des Schemas aus
Kapitel 2 einige CW-Komplexe bearbeitet und als Beispiele dargestellt werden.

\section{Die Papposkonfiguration in der Ebene}

Zu Beginn widmen wir uns als einem Rang 3 Beispiel dem Satz von Pappos, der
besagt\idx{Pappos, Satz von}:
\begin{quote}
Auf zwei Geraden $g_1$ und $g_2$ seien je drei Punkte $A_1,~B_1,~C_1$ und
$A_2,~B_2,~C_2$ gegeben, die vom Schnittpunkt $O$ der beiden Geraden
verschieden sind. Dann liegen die Schnittpunkte
$A=(B_1C_2\cap B_2C_1)$, $B=(A_1C_2\cap A_2C_1)$ und $C=(A_1B_2\cap A_2B_1)$
der kreuzweisen Verbindungen auf einer Geraden $g$.
\end{quote}
Die entsprechende Abbildung hierzu (Abb.\ref{pappos}) liefert nun einen
eindimensionalen CW-Komplex, dessen 0-Zellen mit den Ziffern 1 bis 9 indiziert
seien. 

\begin{figure}[htb]
$$
\beginpicture
\unitlength1cm
\setlinear
\setcoordinatesystem units <1cm,1cm>
\setplotarea x from 0 to 12, y from 0 to 6
\plot 1.5 1.5 6 4 /
\plot 1.5 1.5 10.5 4.5 /
\plot 1.5 1.5 8.5 0.5 /
\plot 1.5 3.5 10.5 4.5 /
\plot 1.5 3.5 8.5 0.5 /
\plot 1.5 3.5 5 1 /
\plot 5 1 10.5 4.5 /
\plot 6 4 8.5 0.5 /
\plot 3.1 2.38 7.15 2.38 /
\put {\circle*{0.1}} [Bl] at 1.5 1.5
\put {\circle*{0.1}} [Bl] at 5 1
\put {\circle*{0.1}} [Bl] at 8.5 0.5
\put {\circle*{0.1}} [Bl] at 1.5 3.5
\put {\circle*{0.1}} [Bl] at 6 4
\put {\circle*{0.1}} [Bl] at 10.5 4.5
\put {\circle*{0.1}} [Bl] at 3.075 2.38
\put {\circle*{0.1}} [Bl] at 4.125 2.38
\put {\circle*{0.1}} [Bl] at 7.175 2.38
\put {$A_1$} [br] at 1.5 3.55
\put {$B_1$} [br] at 6 4.05
\put {$C_1$} [br] at 10.5 4.55
\put {$A_2$} [tr] at 1.5 1.45
\put {$B_2$} [tr] at 5 0.95
\put {$C_2$} [bl] at 8.5 0.55
\put {$A$} [Bl] at 2.2 2.3
\put {$B$} [Bl] at 4.1 1.9
\put {$C$} [Bl] at 7.5 2.2
\endpicture
$$
\caption{CW-Komplex zum Satz von Pappos}
\label{pappos}
\end{figure}

Aus der Abbildung kann man diese Indizierung etwa nach $A_1\hat{=}1$,
$B_1\hat{=}2$, $C_1\hat{=}3$, $A_2\hat{=}4$, $B_2\hat{=}5$, $C_2\hat{=}6$,
$A\hat{=}7$, $B\hat{=}8$ und $C\hat{=}9$ anbringen. Zur Bestimmung der
Symmetriegruppe "uberlegt man sich, da"s in diesem Fall jedes
0-Zellen-Tripel einer Geraden auf ein anderes 0-Zellen-Tripel, welches
ebenfalls auf einer Geraden liegen soll, abzubilden ist.\\
Es sind also jene Permutationen aus der ${\cal S}_9$ gesucht, die
die Menge
$$\left\{\{1,2,3\},\{1,4,8\},\{1,5,9\},\{2,4,7\},\{2,6,9\},\{3,5,7\},
\{3,6,8\},\{4,5,6\},\{7,8,9\}\right\}$$
auf sich abbilden.

Testet man nun mit einem Programm alle 9! Permutationen der 0-Zellen, so
erh"alt man 108 Automorphismen, die die Automorphismengruppe des Papposkomplexes
darstellen. Diese sind in Tabelle \ref{papposaut} aufgezeigt.

\begin{table}%[htb]
{\footnotesize
$$
\begin{array}{lllll}
identity&(49)(58)(67)&(23)(45)(89)&(23)(48)(59)(67)&(24)(38)(59)\\
(298354)(67)&(285)(349)&(25)(39)(67)&(28)(34)(67)&(258)(394)\\
(245389)(67)&(29)(35)(48)&(12)(56)(78)&(12)(49)(57)(68)&(132)(456)(798)\\
(132)(486957)&(142)(387)(569)&(198642)(357)&(187562)(349)&(157862)(39)\\
(186572)(34)&(156872)(394)&(145792)(386)&(192)(356)(487)&(123)(465)(789)\\
(123)(475968)&(13)(46)(79)&(13)(47)(58)(69)&(146973)(285)&(196473)(25)\\
(183)(246)(597)&(154783)(296)&(189653)(247)&(153)(297)(468)&(147963)(28)\\
(197463)(258)&(124)(378)(596)&(129754)(368)&(14)(37)(69)&(197364)(58)\\
(14)(25)(36)(79)&(196374)(285)&(153624)(789)&(184)(296)(375)&(154)(237)(689)\\
(184)(236597)&(137964)(258)&(136974)(28)&(127865)(349)&(126875)(39)\\
(145)(273)(698)&(195)(264873)&(142635)(798)&(195)(274)(386)&(15)(26)(34)(78)\\
(186275)(394)&(15)(27)(68)&(187265)(49)&(135)(279)(486)&(138745)(269)\\
(176)(349)&(16)(39)(58)&(176)(245983)&(16)(295483)&(16)(24)(35)(89)\\
(176)(295384)&(16)(25)(34)&(176)(285)(394)&(176)(258)&(16)(28)(49)\\
(176)(235489)&(16)(238459)&(17)(34)(58)&(167)(394)&(17)(248953)\\
(167)(298453)&(167)(238954)&(17)(235984)&(167)(285)&(17)(25)(49)\\
(167)(258)(349)&(17)(28)(39)&(167)(248359)&(17)(29)(38)(45)&(127568)(34)\\
(126578)(394)&(148)(279563)&(198)(263)(457)&(138)(264)(579)&(135698)(274)\\
(18)(26)(57)&(156278)(49)&(157268)(349)&(18)(27)(39)(56)&(148)(269)(357)\\
(192738)(456)&(124689)(375)&(129)(365)(478)&(146379)(58)&(19)(36)(47)\\
(136479)(285)&(137469)(25)&(189)(236)(475)&(159)(237846)&(159)(247)(368)\\
(183729)(465)&(147369)(258)&(19)(28)(37)(46)&&
\end{array}$$}
\caption{Die Automorphismengruppe des Papposkomplexes}
\label{papposaut}
\end{table}

Die sich ergebenden verschiedenen zyklischen Untergruppen dieser
Automorphismengruppe sind, wenn obige Zykel der Reihe nach durchnummeriert
werden, aus Tabelle \ref{papposgrp} abzulesen.

\begin{table}%[htb]
{\footnotesize $$\begin{array}{ll}
\{id,[6],[7],[4],[10],[11]\} & \{id,[16],[25],[2],[15],[26]\}\\
\{id,[18],[44],[70],[95],[97]\} & \{id,[19],[68],[57],[81],[86]\}\\
\{id,[20],[69],[94],[79],[50]\} & \{id,[21],[79],[55],[69],[85]\}\\
\{id,[22],[81],[91],[68],[49]\} & \{id,[23],[105],[80],[54],[38]\}\\
\{id,[29],[81],[100],[68],[47]\} & \{id,[30],[74],[41],[61],[102]\}\\
\{id,[32],[95],[73],[44],[60]\} & \{id,[33],[54],[62],[105],[90]\}\\
\{id,[35],[61],[108],[74],[48]\} & \{id,[36],[68],[39],[81],[101]\}\\
\{id,[40],[61],[28],[74],[99]\} & \{id,[42],[81],[27],[68],[107]\}\\
\{id,[43],[15],[67],[25],[53]\} & \{id,[46],[95],[8],[44],[87]\}\\
\{id,[52],[105],[9],[54],[104]\} & \{id,[56],[81],[13],[68],[93]\}\\
\{id,[58],[69],[14],[79],[92]\} & \{id,[63],[81],[12],[68],[77]\}\\
\{id,[64],[10],[65],[7],[72]\} & \{id,[66],[81],[3],[68],[83]\}\\
\{id,[71],[81],[5],[68],[76]\} & \{id,[75],[7],[84],[10],[78]\}\\
\{id,[96],[25],[82],[15],[106]\} & \{id,[17],[37]\}\\
\{id,[24],[98]\} & \{id,[31],[89]\}\\
\{id,[34],[59]\} & \{id,[45],[51]\}\\
\{id,[88],[103]\} & \{id\}
\end{array}$$}
\caption{Die zyklischen Untergruppen zum Papposkomplex}
\label{papposgrp}
\end{table}

Nach der Bestimmung dieser Gruppen zum Papposkomplex kann nun begonnen werden
zugeh"orige Matroide zu suchen. Dazu betrachten wir zuerst die Gestalt des
vorgelegten CW-Komplexes, nach dem im Rang 3 die drei Punkte auf einer Geraden
sicher eine Nichtbasis des Matroids liefern. Damit stehen folgende Brackets
sicher f"ur die Determinanten singul"arer Matrizen und k"onnen als Nichtbasen
sicher zu Null gesetzt werden:
$$[123],[148],[159],[247],[269],[357],[368],[456]\mbox{ und }[789]$$
Aus den "`benachbarten 1-Zellen"' folgt nun eine Wahl f"ur Basen, die
ein vertr"agliches Matroid sicherlich besitzt, womit sich eine
Grundstruktur zu
{\tt $$
\begin{array}{c}
0111111111111111011110?111111111111101111?\\
11101111111?101110111101111111111111111110
\end{array}
$$}

ergibt. Mittels {\sc Sym2mat} erh"alt man hierzu bei Vorgabe der vollen
Symmetriegruppe Aut($\cal C$) die beiden Matroide
{\tt $$
\begin{array}{c}
011111111111111101111001111111111111011110\\
111011111110101110111101111111111111111110
\end{array}
$$}

und
{\tt $$
\begin{array}{c}
011111111111111101111011111111111111011111\\
111011111111101110111101111111111111111110
\end{array}
$$}

von denen das erste eine Symmetriegruppe der Ordnung 432 und das zweite eine
Symmetriegruppe der Ordnung 108 besitzt, die jeweils die komplette
Automorphismengruppe des Papposkomplexes enthalten beziehungsweise zu dieser
identisch sind.

Zur Symmetrieuntergruppe mit dem erzeugenden Element $\sigma=(298354)(67)$
erh"alt man zwei weitere, n"amlich
{\tt $$
\begin{array}{c}
011111111111111101111001111111111111011111\\
111011111111101110111101111111111111111110
\end{array}
$$}
sowie
{\tt $$
\begin{array}{c}
011111111111111101111011111111111111011110\\
111011111110101110111101111111111111111110
\end{array}
$$}

Bez"uglich der Symmetriegruppe $G=\{<(298354)(67)>\}$ sind diese Matroide
allerdings mittels des vorgestellten Algorithmus nicht orientierbar.

\clearpage
\section{Der dreidimensionale W"urfel}

Ein vielverwendetes Beispiel (vgl. \cite{Bj:93}) stellt der dreidimensionale
W"urfel dar, der auch hier behandelt werden soll. Gegeben als ein aus
Vierecken aufgebauter CW-Komplex, l"a"st er sich durch die Liste der Polygonz"uge
$$\left\{\{1264\},\{2586\},\{5378\},\{3147\},\{4687\},\{1253\}\right\}$$
beschreiben.

\begin{figure}[htb]
$$
\beginpicture
\unitlength1cm
\setlinear
\setcoordinatesystem units <1cm,1cm>
\setplotarea x from 0 to 7, y from 0 to 6
\plot 0.5 2 0.5 3.5 2 3.5 2 5 3.5 5 3.5 3.5 5 3.5 6.5 3.5 6.5 2 5 2 3.5 2
      3.5 0.5 2 0.5 2 2 0.5 2 /
\plot 2 2 2 3.5 3.5 3.5 3.5 2 2 2 /
\plot 5 3.5 5 2 /
\put {1} [tl] at 2.05 4.95 \put {3} [tr] at 3.45 4.95
\put {1} [tl] at 0.55 3.45 \put {2} [tr] at 1.95 3.45
\put {5} [tr] at 3.45 3.45 \put {3} [tr] at 4.95 3.45
\put {1} [tr] at 6.45 3.45 \put {4} [bl] at 0.55 2.05
\put {6} [br] at 1.95 2.05 \put {8} [br] at 3.45 2.05
\put {7} [br] at 4.95 2.05 \put {4} [br] at 6.45 2.05
\put {4} [bl] at 2.05 0.55 \put {7} [br] at 3.45 0.55
\endpicture
$$
\caption{CW-Komplex zum 3-W"urfel}
\label{wuerfel}
\end{figure}

Seine Symmetriegruppe hat die Ordnung 48 und setzt sich aus den
Permutationen aus Tabelle \ref{cubeaut} zusammen. Betrachet man nun
noch den Komplex zur Erzeugung einer Grundstruktur f"ur die vertr"aglichen
Matroide, so ist diese wie folgt gegeben:

{\small\tt
$$10111101111111?11011111?111?111?111111?111?111?11111011?11111110111101$$
}

Bez"uglich der Symmetriegruppe $\{<(147852)(36)>\}$ erh"alt man daraus folgende
vertr"aglichen Matroide
{\small\tt
$$\begin{array}{c}
1011110111111101101111101110111011111101110111011111011011111110111101\\[1mm]
1011110111111101101111101110111111111111110111011111011011111110111101\\[1mm]
1011110111111101101111111110111011111101111111011111011111111110111101\\[1mm]
1011110111111101101111111110111111111111111111011111011111111110111101\\[1mm]
1011110111111111101111101111111011111101110111111111011011111110111101\\[1mm]
1011110111111111101111101111111111111111110111111111011011111110111101\\[1mm]
1011110111111111101111111111111011111101111111111111011111111110111101\\[1mm]
1011110111111111101111111111111111111111111111111111011111111110111101
\end{array}
$$
}

\begin{table}%[htb]
{\footnotesize
$$
\begin{array}{lllll}
\mbox{identity}&(34)(56)&(23)(67)&(243)(567)&(234)(576)\\
(24)(57)&(12)(35)(46)(78)&(12)(36)(45)(78)&(1352)(4786)&(147852)(36)\\
(137862)(45)&(1462)(3785)&(1253)(4687)&(126873)(45)&(13)(25)(47)(68)\\
(1473)(2685)&(13)(27)(45)(68)&(146853)(27)&(125874)(36)&(1264)(3587)\\
(1374)(2586)&(14)(26)(37)(58)&(135864)(27)&(14)(27)(36)(58)&(15)(48)\\
(165)(348)&(15)(23)(48)(67)&(1675)(2483)&(1765)(2348)&(175)(248)\\
(156)(384)&(16)(38)&(1576)(2384)&(16)(24)(38)(57)&(176)(238)\\
(1756)(2438)&(1567)(2843)&(167)(283)&(157)(284)&(1657)(2834)\\
(17)(28)&(17)(28)(34)(56)&(18)(25)(36)(47)&(18)(264735)&(18)(253746)\\
(18)(26)(37)(45)&(18)(27)(35)(46)&(18)(27)(36)(45)
\end{array}$$}
\caption{Die Automorphismengruppe des 3-W"urfels}
\label{cubeaut}
\end{table}

Legt man wiederum $\{<(147852)(36)>\}$ als Symmetriegruppe f"ur vertr"agliche
orientierte Matroide vor, so erh"alt man nur f"ur eins der oben bestimmten Matroide
Vorzeichenlisten:

{\small\tt
$$\begin{tabular}{c}
+0+-+-0-++-+-+0-+0++-++0---0+--+---+-+--+-0-++0+-++-0++0-++--+-0--++0-\\[1mm]
-0-+-+0+--+-+-0+-0--+--0+++0-++-+++-+-++-+0+--0-+--+0--0+--++-+0++--0+\\[1mm]
-0++--0-+--++-0++0-++-+0++-0-+--+-+---+--+0---0++-++0+-0+-+-++-0-+-+0+\\[1mm]
+0--++0+-++--+0--0+--+-0--+0+-++-+-+++-++-0+++0--+--0-+0-+-+--+0+-+-0-
\end{tabular}
$$
}

Diese orientierten Matroiden besitzen die Grundstruktur des "`echten"' W"urfelmatroids
(siehe Abbildung \ref{cube}). Dabei ist die zweite Vorzeichenliste die negative der ersten,
sowie die vierte die negative der dritten, es liegen also nur zwei verschiedene
Orientierungen zum CW-Komplex "`W"urfel"' mit der gegebenen Symmetriegruppe vor. Interessant
ist hier zu erw"ahnen, da"s die erste Orientierungsvariante eine Symmetriegruppe
der Ordnung 12 besitzt, w"ahrend die zweite Variante die volle Symmetriegruppe des
W"urfelkomplexes der Ordnung 48 aufweist.

\clearpage
\section{Ein weiterer Torus}

Als n"achstes Beispiel sei ein weiterer Torus in Gestalt des CW-Komplexes
in Abbildung \ref{tor2} vorgelegt.

\begin{figure}[htb]
$$
\beginpicture
\unitlength0.75cm
\setlinear
\setcoordinatesystem units <0.75cm,0.75cm>
\setplotarea x from 0 to 7, y from 0 to 7
\setsolid
\plot 0.5 0.5 0.5 6.5 6.5 6.5 6.5 0.5 0.5 0.5 /
\plot 0.5 2.5 6.5 2.5 /
\plot 0.5 4.5 6.5 4.5 /
\plot 2.5 0.5 2.5 6.5 /
\plot 4.5 0.5 4.5 6.5 /
\plot 0.5 2.5 2.5 0.5 /
\plot 0.5 4.5 4.5 0.5 /
\plot 0.5 6.5 6.5 0.5 /
\plot 2.5 6.5 6.5 2.5 /
\plot 4.5 6.5 6.5 4.5 /
\put {1} [br] at 0.45 6.55 \put {2} [br] at 2.5 6.55
\put {3} [bl] at 4.5 6.55  \put {1} [bl] at 6.55 6.55
\put {4} [br] at 0.45 4.5  \put {5} [tr] at 2.45 4.45
\put {6} [tr] at 4.45 4.45 \put {4} [bl] at 6.55 4.5
\put {8} [tr] at 0.45 2.5  \put {9} [bl] at 2.55 2.55
\put {7} [bl] at 4.55 2.55 \put {8} [tl] at 6.55 2.5
\put {1} [tr] at 0.45 0.45 \put {2} [tr] at 2.5 0.45
\put {3} [tl] at 4.5 0.45  \put {1} [tl] at 6.55 0.45
\endpicture
$$
\caption{Ein weiterer Torus}
\label{tor2}
\end{figure}

Dieser CW-Komplex, der auch einen simplizialen Komplex darstellt, besitzt
eine Symmetriegruppe der Ordnung 108. Aus dieser legen wir die Untergruppe der
Ordnung 6 mit den Erzeugenden (123)(456)(789) und (16)(25)(34)(78)(9) als
Symmetriegruppe vor, bez"uglich der vertr"agliche orientierte Matroide erzeugt
werden sollen.

Aus den Zellenberandungen l"a"st sich f"ur vertr"agliche Matroide zun"achst
folgende Grundstruktur angeben,

\begin{center}
{\footnotesize\tt\begin{tabular}{c}
??????1????1?1????1?1111?????????11????1????????????1????1???1?\\
????1?11?????????11??????????????1????????1?????????111?1?11???
\end{tabular}}
\end{center}

mittels der sich mit {\sc Sym2mat} insgesamt 52 vertr"agliche Matroide
erzeugen lassen. Diese Matroide sind in den Tabellen \ref{tor6matA}
und \ref{tor6matB} aufgezeigt.

Zu diesen 52 Matroiden lassen sich nun mittels eines Orientierungsprogramms,
welches in der in Kapitel 2 vorgestellten Vorgehensweise fu"st, zu insgesamt
17 Matroiden Vorzeichenlisten erzeugen. Von diesen 17 sind 7 unzul"a"sig,
da sie zu mehr als einen Symmetrietyp Vorzeichenlisten liefern, also nicht
eindeutig bez"uglich einer Symmetriegruppe orientierbar sind.
Die 10 verbliebenen orientierbaren Matroide sind in Tabelle \ref{mattor}
aufgezeigt.

Stellvertretend f"ur alle erzeugten Orientierungen der vertr"aglichen Matroide
sind in Tabelle \ref{orimattor} die Orientierungen des f"unften Matroids
von Tabelle \ref{mattor} aufgezeigt. Hierbei wurden die erzeugten negativen
Listen zu bereits vorkommenden weggelassen.

Die Auswahl gerade dieses Matroids ist nicht willk"urlich, sondern entlehnt
sich der Kenntnis, da"s eine nichtsimpliziale Realisierung des Torus von
Abbildung \ref{tor2} existiert, die gerade ein solches zugrunde liegendes
Matroid besitzt. Mit dem Programm {\sc Chiman} zur Manipulation von Chirotopen
von Peter Schuchert l"a"st sich zu den erzeugten orientierten Matroiden der
Aufbau der induzierten konvexen H"ulle einer entsprechenden Punktkonfiguration
ermitteln. Dadurch ist bestimmt, welche formalen Punkte die konvexe H"ulle einer
entsprechenden Punktkonfiguration zum vorliegenden orientierten Matroid
aufspannen. F"ur die orientierten Matroide aus Tabelle \ref{orimattor} sind
dies f"ur das erste, zweite, vierte und f"unfte alle 9 Punkte, f"ur das dritte
und sechste nur die Punkte 1 bis 6. F"ur die "ubrigen orientierten Matroide
liefert {\sc Chiman} keine Ergebnisse.

Abbildung \ref{tor2real} zeigt das "au"sere Erscheinungsbild der Realisierung
des Torus in Form eines Oktaeder, in dem zwei gegen"uberliegende Dreiecke
die "Offnungen des Henkels bilden. Im Innern befinden sich die restlichen
drei Punkte, die auch die Spiegelsymmetrie induzieren.

\begin{figure}[htb]
$$
\beginpicture
\unitlength1cm
\setlinear
\setcoordinatesystem units <1cm,1cm>
\setplotarea x from 0 to 6, y from 0 to 6
\plot 3 5.5 1.5 4 0.5 2 3 0.5 4.5 2 5.5 4 3 5.5 /
\plot 3 5.5 0.5 2 4.5 2 3 5.5 /
\setdashes<1mm>
\plot 1.5 4 3 0.5 5.5 4 1.5 4 /
\setlinear
\setshadegrid span <1.5pt>
\vshade 0.5 2 2 <,z,,> 1.5 1.4 4 3 0.5 0.5 /
\vshade 3 5.5 5.5 <,z,,> 4.5 2 4.6 5.5 4 4 /
\setdashes<5mm>
\plot 0.5 1.4 6 4.8 /
\put {\circle*{0.1}} [Bl] at 3 5.5
\put {\circle*{0.1}} [Bl] at 1.5 4
\put {\circle*{0.1}} [Bl] at 5.5 4
\put {\circle*{0.1}} [Bl] at 0.5 2
\put {\circle*{0.1}} [Bl] at 4.5 2
\put {\circle*{0.1}} [Bl] at 3 0.5
\endpicture
$$
\caption{Realisierung des zweiten Torus}
\label{tor2real}
\end{figure}

\begin{table}[htb]
\begin{center}
{\scriptsize\tt
\begin{tabular}{c}
000000111011110011111111011110011111010110100010111111101110111\\
100111110110110111110000111000110111110111110000000111111011100\\[1mm]
000000111011110011111111011110011111010110101010111111111110111\\
100111110110110111110100111100110111110111110011000111111111111\\[1mm]
000000111101011101111111101011101111001101111101100011101111010\\
111011110101101111101111000001101100001101111110000111111011100\\[1mm]
000000111101011101111111101011101111001101111101100111111111010\\
111011110101101111101111010101101101001101111111000111111111111\\[1mm]
000000111111111111111111111111111111011111100011111111101111111\\
111111110111111111111000111001111111111111110000000111111011100\\[1mm]
000000111111111111111111111111111111011111110111111011111111111\\
111111110111111111111011101101111110111111111101000111111111111\\[1mm]
000000111111111111111111111111111111011111110111111111111111111\\
111111110111111111111011111101111111111111111101000111111111111\\[1mm]
000000111111111111111111111111111111011111111111100011101111111\\
111111110111111111111111000001111110001111111110000111111011100\\[1mm]
000000111111111111111111111111111111011111111111111011111111111\\
111111110111111111111111101101111110111111111111000111111111111\\[1mm]
000000111111111111111111111111111111011111111111111111101111111\\
111111110111111111111111111001111111111111111110000111111011100\\[1mm]
000000111111111111111111111111111111011111111111111111111111111\\
111111110111111111111111111101111111111111111111000111111111111\\[1mm]
111111101011010101101111000011011110100110101001100111110111011\\
100010111110101111100100010111100101000111110011111011101111011\\[1mm]
111111101011011101111111100011011111101110111101100011100111011\\
101011111110101111101111000011101100000111111110111111111011100\\[1mm]
111111101011011101111111100011011111101110111101100111110111011\\
101011111110101111101111010111101101000111111111111111111111111\\[1mm]
111111101011110101111111010011011111110110100001111111100111011\\
100111111110101111110000111011100111110111110000111111111011100\\[1mm]
111111101011110101111111010011011111110110101001111111110111011\\
100111111110101111110100111111100111110111110011111111111111111\\[1mm]
111111101011111101101111110011011110111110111101111111110111011\\
101110111110101111111111111111101111110111111111111011101111011\\[1mm]
111111101011111101111111110011011111111110110101111011110111011\\
101111111110101111111011101111101110110111111101111111111111111\\[1mm]
111111101011111101111111110011011111111110110101111111110111011\\
101111111110101111111011111111101111110111111101111111111111111\\[1mm]
111111101011111101111111110011011111111110111101111011110111011\\
101111111110101111111111101111101110110111111111111111111111111\\[1mm]
111111101011111101111111110011011111111110111101111111100111011\\
101111111110101111111111111011101111110111111110111111111011100\\[1mm]
111111101011111101111111110011011111111110111101111111110111011\\
101111111110101111111111111111101111110111111111111111111111111\\[1mm]
111111100101111010101111110100100110111101010110011010010100110\\
011110111001010001111011101110011110111000101101111011101111011\\[1mm]
111111101111111111101111110111111110111111111111111111110111111\\
111110111111111111111111111111111111111111111111111011101111011\\[1mm]
111111101111111111111111110111111111111111100011111111100111111\\
111111111111111111111000111011111111111111110000111111111011100\\[1mm]
111111101111111111111111110111111111111111110111111011110111111\\
111111111111111111111011101111111110111111111101111111111111111
\end{tabular}}
\end{center}
\caption{\label{tor6matA} Matroide zum zweiten Torusbeispiel Teil A}
\end{table}

\begin{table}[htb]
\begin{center}
{\scriptsize\tt
\begin{tabular}{c}
111111101111111111111111110111111111111111110111111111110111111\\
111111111111111111111011111111111111111111111101111111111111111\\[1mm]
111111101111111111111111110111111111111111111111100011100111111\\
111111111111111111111111000011111110001111111110111111111011100\\[1mm]
111111101111111111111111110111111111111111111111111011110111111\\
111111111111111111111111101111111110111111111111111111111111111\\[1mm]
111111101111111111111111110111111111111111111111111111100111111\\
111111111111111111111111111011111111111111111110111111111011100\\[1mm]
111111101111111111111111110111111111111111111111111111110111111\\
111111111111111111111111111111111111111111111111111111111111111\\[1mm]
111111110111111110111111111101110111111111011111011110011101111\\
011111111011011001111111111111011111111010101111111111111111111\\[1mm]
111111111011110111111111011111011111110110100011111111101111111\\
100111111110111111110000111011110111110111110000111111111011100\\[1mm]
111111111011110111111111011111011111110110101011111111111111111\\
100111111110111111110100111111110111110111110011111111111111111\\[1mm]
111111111101111111111111111111101111111101111111100011101111110\\
111111111101111111111111000011111110001101111110111111111011100\\[1mm]
111111111101111111111111111111101111111101111111111011111111110\\
111111111101111111111111101111111110111101111111111111111111111\\[1mm]
111111111101111111111111111111101111111101111111111111101111110\\
111111111101111111111111111011111111111101111110111111111011100\\[1mm]
111111111101111111111111111111101111111101111111111111111111110\\
111111111101111111111111111111111111111101111111111111111111111\\[1mm]
111111111111011101111111101011111111101111111101100011101111011\\
111011111111101111101111000011101100001111111110111111111011100\\[1mm]
111111111111011101111111101011111111101111111101100111111111011\\
111011111111101111101111010111101101001111111111111111111111111\\[1mm]
111111111111111011111111111110111111111111100010111111101110111\\
111111111111110111111000111010111111111111110000111111111011100\\[1mm]
111111111111111011111111111110111111111111110110111111111110111\\
111111111111110111111011111110111111111111111101111111111111111\\[1mm]
111111111111111011111111111110111111111111111110111111101110111\\
111111111111110111111111111010111111111111111110111111111011100\\[1mm]
111111111111111011111111111110111111111111111110111111111110111\\
111111111111110111111111111110111111111111111111111111111111111\\[1mm]
111111111111111111101111111111111110111111111111111111111111111\\
111110111111111111111111111111111111111111111111111011101111011\\[1mm]
111111111111111111111111111111111111111111100011111111101111111\\
111111111111111111111000111011111111111111110000111111111011100\\[1mm]
111111111111111111111111111111111111111111110111111011111111111\\
111111111111111111111011101111111110111111111101111111111111111\\[1mm]
111111111111111111111111111111111111111111110111111111111111111\\
111111111111111111111011111111111111111111111101111111111111111\\[1mm]
111111111111111111111111111111111111111111111111100011101111111\\
111111111111111111111111000011111110001111111110111111111011100\\[1mm]
111111111111111111111111111111111111111111111111111011111111111\\
111111111111111111111111101111111110111111111111111111111111111\\[1mm]
111111111111111111111111111111111111111111111111111111101111111\\
111111111111111111111111111011111111111111111110111111111011100\\[1mm]
111111111111111111111111111111111111111111111111111111111111111\\
111111111111111111111111111111111111111111111111111111111111111
\end{tabular}}
\end{center}
\caption{\label{tor6matB} Matroide zum zweiten Torusbeispiel Teil B}
\end{table}

\begin{table}%[htb]
\begin{center}
{\scriptsize\tt\begin{tabular}{c}
111111101011011101111111100011011111101110111101100111110111011\\
101011111110101111101111010111101101000111111111111111111111111\\[1mm]
111111101011110101111111010011011111110110101001111111110111011\\
100111111110101111110100111111100111110111110011111111111111111\\[1mm]
111111101011111101111111110011011111111110110101111111110111011\\
101111111110101111111011111111101111110111111101111111111111111\\[1mm]
111111101011111101111111110011011111111110111101111011110111011\\
101111111110101111111111101111101110110111111111111111111111111\\[1mm]
111111101011111101111111110011011111111110111101111111110111011\\
101111111110101111111111111111101111110111111111111111111111111\\[1mm]
111111101111111111101111110111111110111111111111111111110111111\\
111110111111111111111111111111111111111111111111111011101111011\\[1mm]
111111101111111111111111110111111111111111111111111111110111111\\
111111111111111111111111111111111111111111111111111111111111111\\[1mm]
111111111101111111111111111111101111111101111111111111111111110\\
111111111101111111111111111111111111111101111111111111111111111\\[1mm]
111111111111111011111111111110111111111111111110111111111110111\\
111111111111110111111111111110111111111111111111111111111111111\\[1mm]
111111111111111111111111111111111111111111111111111111111111111\\
111111111111111111111111111111111111111111111111111111111111111
\end{tabular}}
\end{center}
\caption{\label{mattor}Die orientierbaren Matroide zum zweiten Torus}
\end{table}

\begin{table}%[htb]
\begin{center}
{\scriptsize\tt\begin{tabular}{c}
+++++++0+0--+--+0----++++-00+-0+-+--+-++-0++++0-++-++-++0+-+0+-\\
+0-+-++-+-+0-0++-++-+--++-+++-+0--++-+0-++-++--+---++-+++--++--\\[1mm]
+++++++0+0--+--+0----++++-00+-0+-+--+-++-0++-+0-++-++-++0+-+0+-\\
+0-+-++-+-+0-0++-++-++-++-+++-+0--++-+0-++-++-++---++-+++--++--\\[1mm]
+++++++0+0-----+0----+++++00+-0+-+--++++-0++++0-+-+++-++0+-+0+-\\
+0---++-+-+0-0++-++++--+---++-+0---++-0-++-++--+---++-+++--++--\\[1mm]
+++++++0-0+---+-0+++++++-+00-+0-+-++++-++0-+-+0+-+-++-++0++-0++\\
-0+----+++-0+0-+-+++-+-++-++++-0+--+-+0+-+-++-++-----+----+----\\[1mm]
+++++++0-0+---+-0+++++++-+00-+0-+-++++-++0-+-+0+-+--+-++0++-0++\\
-0+----+++-0+0-+-+++-+-+++++++-0+----+0+-+-++-++-----+----+----\\[1mm]
+++++++0-0+-----0+++++++++00-+0-+-+++++++0----0+-+--+-++0++-0++\\
-0-----+++-0+0-+-++++++-++++++-0-----+0+-+-+-+++-----+----+----\\[1mm]
+++---+0+0--++++0-+++++---00+-0+--+++----0+---0-++--+-+-0+-+0+-\\
+0+++--++-+0-0++-+---++-+++-+-+0+++--+0-++-+-++-+++--+---++--++\\[1mm]
+++---+0+0--++-+0-+++++-+-00+-0+--+++-+--0++-+0-++-++-+-0+-+0+-\\
+0-++--++-+0-0++-+--++-++-+-+-+0-+++-+0-++-++-+-+++--+---++--++\\[1mm]
+++---+0+0--++-+0-+++++-+-00+-0+--+++-+--0++-+0-++--+-+-0+-+0+-\\
+0-++--++-+0-0++-+--++-++++-+-+0-++--+0-++-++-+-+++--+---++--++\\[1mm]
+++---+0-0+-+++-0+---++---00-+0-++--+---+0-+++0+--+++-+-0++-0++\\
-0+++++-++-0+0-+-+-----+----++-0+++++-0+-+-++---+++++-++++-++++\\[1mm]
+++---+0-0+--++-0+---++--+00-+0-++--++--+0-+++0+-+-++-+-0++-0++\\
-0+-+++-++-0+0-+-+-+---++-+-++-0++-+-+0+-+-++---+++++-++++-++++\\[1mm]
+++---+0-0+--++-0+---++--+00-+0-++--++--+0-+-+0+-+-++-+-0++-0++\\
-0+-+++-++-0+0-+-+-+-+-++-+-++-0++-+-+0+-+-++-+-+++++-++++-++++
\end{tabular}}
\end{center}
\caption{\label{orimattor}Orientierungen f"ur ein Matroid zum zweiten Torus}
\end{table}

\clearpage
\section{Das Dodekaeder}

Als weiteren Vertreter f"ur die nichtsimplizialen Platonischen K"orper sei
nun ein Dodekaeder als Beispiel-Komplex gew"ahlt. Mit seinen 20 Punkten stellt
dieser Komplex schon eine Herausforderung an die Rechenzeit der implementierten
Programme dar, was gerade bei der Bestimmung der Symmetriegruppe und der
Erzeugung von Orientierungen von Matroiden als Behinderung ins Auge f"allt.

\begin{figure}[htb]
$$
\beginpicture
\unitlength0.75cm
\setlinear
\setcoordinatesystem units <0.75cm,0.75cm>
\setplotarea x from -3 to 3, y from -3 to 3
\setsolid
\plot -0.173 2.407 1.312 1.267 /
\plot 1.312 1.267 2.578 0.905 /
\plot 2.578 0.905 1.874 1.821 /
\plot 1.874 1.821 0.173 2.749 /
\plot 0.173 2.749 -0.173 2.407 /
\plot -1.312 1.918 0.173 2.749 /
\plot -1.312 1.918 -2.578 1.063 /
\plot -2.578 1.063 -1.874 1.365 /
\plot -1.874 1.365 -0.173 2.407 /
\plot -1.874 1.365 -1.439 -0.417 /
\plot -1.439 -0.417 0.53 -0.477 /
\plot 0.53 -0.477 1.312 1.267 /
\plot -0.173 -2.749 -1.874 -1.821 /
\plot 1.312 -1.918 -0.173 -2.749 /
\plot 1.312 -1.918 0.53 -0.477 /
\plot -1.439 -0.417 -1.874 -1.821 /
\plot 2.578 -1.063 1.312 -1.918 /
\plot 2.578 -1.063 2.578 0.905 /
\plot -2.578 -0.905 -2.578 1.063 /
\plot -1.874 -1.821 -2.578 -0.905 /
\setdashes <1mm>
\plot 1.874 1.821 1.439 0.417 /
\plot 1.439 0.417 -0.53 0.477 /
\plot -0.53 0.477 -1.312 1.918 /
\plot -0.173 -2.749 0.173 -2.407 /
\plot 0.173 -2.407 1.874 -1.365 /
\plot 1.874 -1.365 2.578 -1.063 /
\plot 0.173 -2.407 -1.312 -1.267 /
\plot -1.312 -1.267 -0.53 0.477 /
\plot 1.439 0.417 1.874 -1.365 /
\plot -2.578 -0.905 -1.312 -1.267 /
\endpicture
$$
\caption{Ein Dodekaeder}
\label{dodeka}
\end{figure}

Von Vorteil ist allerdings, da"s die Symmetriegruppe als die Duale der
Ikosaedergruppe der Ordnung 120 bekannt ist. Trotzdem ist bei ${20\choose 4}
=4845$ Elementen f"ur die Matroide mit erheblichem Aufwand zu rechnen.
Deshalb soll zun"achst die Grundstruktur der Matroide nach der Erscheinung des
vorgelegten Komplexes angegeben werden. Diese ist in Abbildung \ref{dodegrs}
zu sehen.

\begin{figure}[htb]
\begin{center}{\scriptsize\tt\begin{tabular}{l}
001111111?111????01111111?111????1111111?111????????????????????????????\\
???1111????????111????????00111???110111???11111???11???????1???????????\\
?????????01111111?111????1111111?111????????????11?????????????????11???\\
???????1??????????11????????1?????????11????????????1?????1?????????????\\
?1111111?111????11??????11????1????????????11??????????1??????????11????\\
????1?????????????????????????????1??????????????1111??????????111??????\\
????0011?????111011?????11111?????1111??????????????????????????????????\\
??????????1????????????11??????????1????????????????????????????????????\\
??????????????1??????????????11???????11?1???????11?????????????????????\\
????????????????????????????????1??011?????11111?????1111???????????????\\
??????????????????????????1??11?????1111??????11????????????????????????\\
?????????1?10111???11111???11???????????????????????????1111???111?????1\\
?????1??????????????11?????11????1????????????????????1?????????????????\\
?????????????????????????????????01111111?111????1111111?111?????????1?1\\
11????????????????????????????????????????11?11?????1?11?????10011??111?\\
????011??111??1??????????1111111?111????11??????11????1?????????????????\\
??????????????????11????????1?????????11????????????1?????1?????????????\\
?11????????????1????????????11??????????1??????????11????????1?????????1\\
1????????????1????????????????????1?????????????????????????????????????\\
??????????????????11????????????1?????1?????????????????????????????????\\
???????????????????????????????????????????????????????????1??????????11\\
????????1????????????????????????????????????????????11??????111??????11\\
???????????????????????????????????10111???11111???11???????1???????????\\
????????1111???111?????11????1??????????????111????1011??111??1?????????\\
?111??111???1?????????11??11?????????1???????????????????1111111?111????\\
11???1110011??1??????11??????????11???????????????????????????????????11\\
1?????11????11????011??11????????11??????11????1????????????11??????????\\
1??????????????????????????????11????????????1?????1??????????????1?????\\
?1111????????11???????????????????????????????????111?????11????11????01\\
1??11??1????????????????????????????????????????????????????????????????\\
????1????11??1?????1??????????????????????????????????????????????????1?\\
????????????????????????????????????????????????????????????????????1???\\
??????11????????????1????????????????????111????11?????11????1??????????\\
?????11????1011??111??1??????????111??111???1?????????11??11?????????11?\\
???????1?????????0011????11111?011????11111?11????11111?1???????????????\\
???????????????????????????????????1??????????????011????11111?11????111\\
11?????????????????????????????????11?????11????11????011??11??1?????11?\\
???11111?????????11?????????????????????????????????????????1????11??1??\\
1??1???????11?????????????????????????????????????????1???????????1?????\\
?????????????????????????????????????????????????1??1???????????????????\\
????????????????????????????????????????????????????????????????????1???\\
??1??????????????111???11???1?????????11???1?????????11????????1????????\\
?011????11111?11????11111?1?????????????????????????????????????????????\\
?????1????11??1?????11????11111?1???????11??????????????????????????????\\
????????????????11??1??1??011?????11111?????111?????????????????????????\\
?????????????????1??11?????111???????11?????????????????????????????????\\
1?11??????11???????????????????????????????????1????????????????????????\\
???????????1???????1????????????????????????????????????????????????????\\
??????????????1?????????11????11111???????1111??????????????????????????\\
????111???111??111??001101111?1???????11????????????????????????????????\\
?????????1????11??1??1???????????????????????????????????????????????11?\\
?1??1???????????????????????????????????????????????????????????????????\\
???????????????????????1?????1??????????????111???11???11??11111111???11\\
??1?????11??1?????01111?111111?1??11001???????11???????11?????????????11\\
????????????111?11?0111??????111??????111????????????11????????????111?1\\
1?011???????11?????????????????????????????????1?1??????????????????????\\
???????????1?1????????????????????????????111???11???11??11111111???11??\\
1?????11??1?????01111?111111?????111??????111????????????11?????????????\\
11?11?0111??????11?????????????????????????????????1?1??????????????????\\
???????????????1?1???????????????????????????????????????????111111?????\\
????????????????????1??1??1111111??1100??????11?????11????11??????????11\\
11011????111????11????11??????????111101?????1?????1??????????????1?????\\
1?????????111111??????????????1??????????1??1??1111111???111????11????11\\
??????????111101?????11????1??????????????1?????1?????????111111????????\\
??????1????????????????11110111110111110????1???1??1?1111111011??11??111\\
111111??11??1??1?111???????????????111111111011??11??111111111??11??1??1\\
?111???????????????1111001110111111111011111111111111100000111111???111?\\
??1111111???111111110               
\end{tabular}}\end{center}
\caption{\label{dodegrs} Die Grundstruktur zu Dodekaeder-Matroiden}
\end{figure}

Betrachtet man sich diese Grundstruktur, so erscheint es wenig sinnvoll,
dieses Beispiel testen zu wollen, da hierzu derzeitige Computer eine
Rechenzeit ben"otigen w"urden, die den zeitlichen Rahmen dieser Diplomarbeit
sprengen w"urde.

\clearpage
\section{Die Dycksche Karte}

In \cite{Bo:91} stellte Bokowski eine Realisierung der Dyckschen Karte
vor, deren zugeh"origer CW-Komplex in Abbildung \ref{dyck} zu sehen ist.

\begin{figure}[htb]
$$
\beginpicture
\unitlength1cm
\setlinear
\setcoordinatesystem units <1cm,1cm>
\setplotarea x from -6 to 6, y from -6 to 6
\plot 0 0 3 0 3.939 0.695 4.078 1.902 4.096 2.868 /
\plot 3 0 2.121 2.121 3.939 0.695 /
\plot 2.121 2.121 4.078 1.902 /
\plot 2.121 2.121 4.096 2.868 /
\plot 0 0 2.121 2.121 2.294 3.277 1.539 4.229 0.868 4.924 /
\plot 2.121 2.121 0 3 2.294 3.277 /
\plot 0 3 1.539 4.229 /
\plot 0 3 0.868 4.924 /
\plot 0 0 0 3 -0.695 3.939 -1.902 4.078 -2.868 4.096 /
\plot 0 3 -2.121 2.121 -0.695 3.939 /
\plot -2.121 2.121 -1.902 4.078 /
\plot -2.121 2.121 -2.868 4.096 /
\plot 0 0 -2.121 2.121 -3.277 2.294 -4.229 1.539 -4.924 0.868 /
\plot -2.121 2.121 -3 0 -3.277 2.294 /
\plot -3 0 -4.229 1.539 /
\plot -3 0 -4.924 0.868 /
\plot 0 0 -3 0 -3.939 -0.695 -4.078 -1.902 -4.096 -2.868 /
\plot -3 0 -2.121 -2.121 -3.939 -0.695 /
\plot -2.121 -2.121 -4.078 -1.902 /
\plot -2.121 -2.121 -4.096 -2.868 /
\plot 0 0 -2.121 -2.121 -2.294 -3.277 -1.539 -4.229 -0.868 -4.924 /
\plot -2.121 -2.121 0 -3 -2.294 -3.277 /
\plot 0 -3 -1.539 -4.229 /
\plot 0 -3 -0.868 -4.924 /
\plot 0 0 0 -3 0.695 -3.939 1.902 -4.078 2.868 -4.096 /
\plot 0 -3 2.121 -2.121 0.695 -3.939 /
\plot 2.121 -2.121 1.902 -4.078 /
\plot 2.121 -2.121 2.868 -4.096 /
\plot 0 0 2.121 -2.121 3.277 -2.294 4.229 -1.539 4.924 -0.868 /
\plot 2.121 -2.121 3 0 3.277 -2.294 /
\plot 3 0 4.229 -1.539 /
\plot 3 0 4.924 -0.868 /
\put {\scsi 12} [bl] at 0.3 0.1
\put {\scsi 1} [tl] at 2 2
\put {\scsi 2} [tr] at 2.7 -0.1
\put {\scsi 3} [bl] at 2 -2
\put {\scsi 4} [bl] at 0.1 -2.7
\put {\scsi 5} [br] at -2 -2
\put {\scsi 6} [bl] at -2.7 0.1
\put {\scsi 7} [tr] at -2 2
\put {\scsi 8} [tr] at -0.1 2.7
\put {\scsi 9} [bl] at 4 0.7
\put {\scsi 4} [bl] at 4.1 2
\put {\scsi 11} [bl] at 4.1 2.9
\put {\scsi 10} [bl] at 2.3 3.3
\put {\scsi 3} [bl] at 1.6 4.3
\put {\scsi 11} [bl] at 0.9 5
\put {\scsi 9} [br] at -0.7 4
\put {\scsi 2} [br] at -2 4.1
\put {\scsi 11} [br] at -2.9 4.1
\put {\scsi 10} [br] at -3.3 2.3
\put {\scsi 1} [br] at -4.3 1.6
\put {\scsi 11} [br] at -5 0.9
\put {\scsi 9} [tr] at -4 -0.7
\put {\scsi 8} [tr] at -4.1 -2
\put {\scsi 11} [tr] at -4.1 -2.9
\put {\scsi 10} [tr] at -2.3 -3.3
\put {\scsi 7} [tr] at -1.6 -4.3
\put {\scsi 11} [tr] at -0.9 -5
\put {\scsi 9} [tl] at 0.7 -4
\put {\scsi 6} [tl] at 2 -4.1
\put {\scsi 11} [tl] at 2.9 -4.1
\put {\scsi 10} [tl] at 3.3 -2.3
\put {\scsi 5} [tl] at 4.3 -1.6
\put {\scsi 11} [tl] at 5 -0.9
\endpicture
$$
\caption{Die Dycksche Karte}
\label{dyck}
\end{figure}

Als Analogon zu den Platonischen K"orpern besitzt die Dycksche regul"are Karte,
die in der "ublichen Notation f"ur kombinatorische 2-Mannigfaltigkeiten mit
$\{3,8\}_6$ bezeichnet wird, eine flaggentransitive Automorphismengruppe.
Ihr Aufbau ist bestimmt durch zw"olf Ecken, die mit der Valenz 8 "uber 48 Kanten
32 Seiten in Form von Dreiecken bilden. "Uber die Eulerformel l"a"st sich hier
als Geschlecht 3 bestimmen.\\
Bei Vorlage einer 2-Mannigfaltigkeit als kombinatorischem Komplex, ist es
interessant zu erfahren, ob eine Einbettung dieser in den dreidimensionalen
euklidischen Raum existiert, die aus einer endlichen Menge ebener Polygone
besteht, deren Vereinigung frei von Selbstdurchdringungen ist und zu der
kombinatorischen Mannigfaltigkeit korrespondiert.\\
Im Jahre 1986 fand nun Bokowski eine Einbettung oben beschriebener
Karte (beschrieben in \cite{Bo:86} und \cite{HoWi:91}), die allerdings keine
Symmetrien aufweist. Mit Hilfe des vorgestellten Schemas soll nun nach
orientierten Matroiden gesucht werden, die eine m"ogliche symmetrische
Einbettung der Dyckschen Karte induzieren k"onnten.

"Uber den Komplex l"a"st sich zun"achst aussagen, da"s aufgrund des Aufbaus
aus Dreiecken keine offensichtlichen Nichtbasen f"ur vertr"agliche Matroide
existieren, aber anhand der benachbarten Zellen wohl eine (wenn auch nicht
allzu stark besetzte) Grundstruktur induziert werden kann, die in der
folgenden Form gegeben ist:

{\tt $$
\begin{array}{c}
????????1????1?????????????????1??????1??1???????1?????\\
?????????1???????1???????????????????1????1??????1?????\\
??????????????????????1???1?????1?????1??????????1?????\\
???????1????1??????????????1??????1?????1???????????1??\\
??????????????????1???1????1?????????????????1????1????\\
????????????????1??1??????????????1?????1??????????1???\\
???????????1??1?????????????????1???????????1??1???????\\
???1???1???????????????????1????????????????11?????1???\\
1??????????1???????????1????1?????????????1????????????
\end{array}
$$}

Als Symmetriegruppe f"ur den Komplex erh"alt man eine Gruppe der Ordnung 192,
deren Elemente in Tabelle \ref{dyckaut} aufgef"uhrt sind.

\begin{table}[p]
{\scsi $$
\begin{array}{lll}
\mbox{identity} & (37)(48)(9C)(AB) & (2864)(9CAB)\\
(24)(37)(68)(BC) & (26)(48)(9A)(BC) & (26)(37)(9B)(AC)\\
(2468)(9BAC) & (28)(37)(46)(9A) & (29)(4C)(6A)(8B)\\
(2C8A6B49)(37) & (2A)(4B)(69)(8C) & (2B896C4A)(37)\\
(2B)(49)(6C)(8A) & (2A4C698B)(37) & (2C)(4A)(6B)(89)\\
(294B6A8C)(37) & (12)(34)(56)(78)(9C)(AB) & (12)(38)(47)(56)\\
(18765432)(9A) & (14765832)(9BAC) & (1652)(3874)(9B)(AC)\\
(1652)(3478)(9A)(BC) & (14365872)(BC) & (18365472)(9CAB)\\
(1C8792)(3A65B4) & (192)(3B8)(47C)(5A6) & (1B87A2)(3965C4)\\
(1A2)(3C8)(47B)(596) & (1A43B2)(5987C6) & (1B2)(3A8)(479)(5C6)\\
(1943C2)(5A87B6) & (1C2)(398)(47A)(5B6) & (13)(48)(57)(9A)\\
(1753)(9BAC) & (13)(24)(57)(68)(9C)(AB) & (1753)(2864)\\
(13)(26)(57)(BC) & (1753)(26)(48)(9CAB) & (13)(28)(46)(57)(9B)(AC)\\
(1753)(2468)(9A)(BC) & (13)(2A69)(4C8B)(57) & (1753)(2B8A6C49)\\
(13)(296A)(4B8C)(57) & (1753)(2C896B4A) & (13)(2B)(4A)(57)(6C)(89)\\
(1753)(2A8C694B) & (13)(2C)(49)(57)(6B)(8A) & (1753)(298B6A4C)\\
(12385674)(9CAB) & (12785634)(BC) & (14)(23)(58)(67)\\
(1854)(2763)(9C)(AB) & (16785234)(9BAC) & (16385274)(9A)\\
(1854)(2367)(9A)(BC) & (14)(27)(36)(58)(9B)(AC) & (1C4)(239)(5B8)(67A)\\
(1927B4)(3C85A6) & (1B4)(23A)(5C8)(679) & (1A27C4)(3B8596)\\
(194)(23B)(5A8)(67C) & (1C63A4)(27985B) & (1A4)(23C)(598)(67B)\\
(1B6394)(27A85C) & (15)(48)(9B)(AC) & (15)(37)(9A)(BC)\\
(15)(24)(68)(9A) & (15)(2864)(37)(9BAC) & (15)(26)(9C)(AB)\\
(15)(26)(37)(48) & (15)(28)(46)(BC) & (15)(2468)(37)(9CAB)\\
(15)(2B4A6C89) & (15)(2A69)(37)(4B8C) & (15)(2C496B8A)\\
(15)(296A)(37)(4C8B) & (15)(298C6A4B) & (15)(2C6B)(37)(4A89)\\
(15)(2A8B694C) & (15)(2B6C)(37)(498A) & (1256)(3874)(9A)(BC)\\
(1256)(3478)(9B)(AC) & (14325876)(9CAB) & (18325476)(BC)\\
(16)(25)(34)(78) & (16)(25)(38)(47)(9C)(AB) & (18725436)(9BAC)\\
(14725836)(9A) & (1A6)(259)(3C4)(7B8) & (1B47A6)(25C839)\\
(196)(25A)(3B4)(7C8) & (1C4796)(25B83A) & (1C6)(25B)(394)(7A8)\\
(1983C6)(25A47B) & (1B6)(25C)(3A4)(798) & (1A83B6)(25947C)\\
(1357)(9CAB) & (17)(35)(48)(BC) & (1357)(2864)(9A)(BC)\\
(17)(24)(35)(68)(9B)(AC) & (1357)(26)(48)(9BAC) & (17)(26)(35)(9A)\\
(1357)(2468) & (17)(28)(35)(46)(9C)(AB) & (1357)(2C4A6B89)\\
(17)(29)(35)(4B)(6A)(8C) & (1357)(2B496C8A) & (17)(2A)(35)(4C)(69)(8B)\\
(1357)(294C6A8B) & (17)(2C6B)(35)(498A) & (1357)(2A4B698C)\\
(17)(2B6C)(35)(4A89) & (12345678)(9A) & (12745638)(9BAC)\\
(18)(23)(45)(67)(9B)(AC) & (1458)(2763)(9A)(BC) & (16745238)(BC)\\
(16345278)(9CAB) & (1458)(2367)(9C)(AB) & (18)(27)(36)(45)\\
(1A67C8)(23B459) & (1B8)(279)(3A6)(45C) & (1967B8)(23C45A)\\
(1C8)(27A)(396)(45B) & (1B2398)(45C67A) & (1A8)(27B)(3C6)(459)\\
(1C23A8)(45B679) & (198)(27C)(3B6)(45A) & (129)(38B)(4C7)(56A)\\
(12C349)(56B78A) & (19)(3B)(5A)(7C) & (1C3A5B79)(48)\\
(149)(2B3)(58A)(6C7) & (18B769)(2A54C3) & (1C7A5B39)(2864)\\
(19)(24)(3C)(5A)(68)(7B) & (169)(2A5)(34B)(78C) & (16C389)(2B74A5)\\
(1A59)(26)(3C7B)(48) & (1B7A5C39)(26) & (189)(2C7)(36B)(4A5)\\
(14B729)(36A58C) & (1B3A5C79)(2468) & (1A59)(28)(3B7C)(46)\\
(12B34A)(56C789) & (12A)(38C)(4B7)(569) & (1B395C7A)(48)\\
(1A)(3C)(59)(7B) & (18C76A)(2954B3) & (14A)(2C3)(589)(6B7)\\
(1A)(24)(3B)(59)(68)(7C) & (1B795C3A)(2864) & (16B38A)(2C7495)\\
(16A)(295)(34C)(78B) & (1C795B3A)(26) & (195A)(26)(3B7C)(48)\\
(14C72A)(36958B) & (18A)(2B7)(36C)(495) & (195A)(28)(3C7B)(46)\\
(1C395B7A)(2468) & (12A78B)(34C569) & (12B)(38A)(497)(56C)\\
(1A7C593B)(48) & (1B)(3A)(5C)(79) & (18932B)(4A76C5)\\
(14B)(2A3)(58C)(697) & (1C5B)(24)(3A79)(68) & (197C5A3B)(2864)\\
(16A74B)(2938C5) & (16B)(2C5)(34A)(789) & (193C5A7B)(26)\\
(1C5B)(26)(397A)(48) & (14936B)(2C58A7) & (18B)(297)(36A)(4C5)\\
(1B)(28)(39)(46)(5C)(7A) & (1A3C597B)(2468) & (12C)(389)(4A7)(56B)\\
(12978C)(34B56A) & (1C)(39)(5B)(7A) & (197B5A3C)(48)\\
(14C)(293)(58B)(6A7) & (18A32C)(4976B5) & (1A7B593C)(2864)\\
(1B5C)(24)(397A)(68) & (16C)(2B5)(349)(78A) & (16974C)(2A38B5)\\
(1B5C)(26)(3A79)(48) & (1A3B597C)(26) & (18C)(2A7)(369)(4B5)\\
(14A36C)(2B5897) & (193B5A7C)(2468) & (1C)(28)(3A)(46)(5B)(79)
\end{array}$$}
\caption{Die Symmetriegruppe der Dyckschen Karte}
\label{dyckaut}
\end{table}

Zur Erzeugung vertr"aglicher Matroide sei hieraus eine zyklische
Untergruppe der Ordnung sechs, etwa $\{<(1C8792)(3A65B4)>\}$
als Symmetriegruppe ausgew"ahlt. Zu dieser wurden nach 10 Stunden Rechenzeit
auf einem i486-66 bereits 1332 verschiedene vertr"agliche Matroide mit oben
angegebener Grundstruktur erzeugt, wobei noch nicht alle zugeh"origen
Matroide erzeugt wurden. Vier willk"urlich ausgew"ahlte von diesen sind in
Abbildung \ref{dyckmat} als Beispiele angegeben.

\begin{figure}%[htb]
{\tt $$\begin{array}{c}
0000011110000111100000000000000111111110111100000111100\\
0000000111100000111100000000000000011110111100000111100\\
0000000000000000000001111111101111011110000000111100000\\
0000111100011110000000000011110111100001111000000011110\\
0000000000000000111111110111100000000000000001111011110\\
0000000000000111100111100000000001111000111100000111100\\
0000000001111011110000000000001111000000011110111100000\\
1111001111000000000000000111100000000000011111111011110\\
1111000000011110000011110111100000000000011110000000000\\
\end{array}$$
$$\begin{array}{c}
0101111111011111101111100000001110111110111111011111111\\
1101011111110111111111110111101101111110101110111101111\\
1111011110110011010011101111101111110111111111111101111\\
1011111111111111111111001111000111101111111111111011111\\
1110010111111101111100110111111110111110111111111011111\\
0111100011111111111111100111011111111111111110101111110\\
1101101101111110111011100011111111011110011110111111111\\
1011101111111111111100001111101111100111011111111111111\\
1111101111011110111111111101111110001110111110111111001\\
\end{array}$$
$$\begin{array}{c}
0101111111011111111101111111111111111111111101011111111\\
1101011111111111111111110111111111111111111111101111111\\
1111111111111111110111111111111111111111111111111111111\\
1011111111111111111111111111111111111111111111111111111\\
1110010111111111111111111111111011111111111111111111111\\
1111110111111111111111111111111111111111111110101111110\\
1101111111111111111111010111111111111110011110111111111\\
1111101111111111111111111111111111110111011111111111111\\
1111101111111111111111111111111110001110011111111111101\\
\end{array}$$
$$\begin{array}{c}
0111011111110111111111111111101111111110111111110111111\\
1111111011111010111111110111111110111111111111111111111\\
1111110101111010111111111110111111111111011011111111111\\
1110111111111111111111011111111111111111111111111110111\\
1111001100011110111111111111010111111111111111111111111\\
1110111101111111111111111111111111111111111111111111111\\
0111111111111111111111111110110111111111111111111011111\\
1111111111111111111101111111111111111101011111111111111\\
1111101111111111111111111111110101011011111110111110111\\
\end{array}$$}
\caption{Vertr"agliche Matroide zur Dyckschen Karte}
\label{dyckmat}
\end{figure}

Versucht man nun der Reihe nach alle erzeugten Matroide zu orientieren,
so erh"alt man schon f"ur die ersten beiden mit {\sc Sym2mat} erzeugten
Matroide vertr"agliche Vorzeichenlisten, die in den Tabellen \ref{dyckori1}
bis \ref{dyckori2B} zu sehen sind.
Hier wurden wiederum nur die "'positiven"' Listen aufgef"uhrt, so da"s sich
im ersten Fall zwei vertr"agliche Chirotope und im zweiten Fall acht
verschiedene Vorzeichenlisten ergeben.

\begin{table}%[htb]
{\tt
\begin{center}
\begin{tabular}{c}
0000011110000111100000000000000111111110111100000111100\\
0000000111100000111100000000000000011110111100000111100\\
0000000000000000000001111111101111011110000000111100000\\
0000111100011110000000000011110111100001111000000011110\\
0000000000000000111111110111100000000000000001111011110\\
0000000000000111100111100000000001111000111100000111100\\
0000000001111011110000000000001111000000011110111100000\\
1111001111000000000000000111100000000000011111111011110\\
1111000000011110000011110111100000000000011110000000000\\[2mm]
00000++++0000----00000000000000++++----0----00000----00\\
0000000----00000----000000000000000++++0++++00000----00\\
000000000000000000000++++----0----0----0000000----00000\\
0000++++000++++00000000000++++0++++0000----0000000----0\\
0000000000000000----++++0++++0000000000000000++++0++++0\\
0000000000000----00++++0000000000----000++++00000++++00\\
000000000----0----000000000000----0000000----0----00000\\
++++00++++000000000000000++++000000000000++++----0----0\\
----0000000----00000++++0++++000000000000----0000000000\\[2mm]
00000-++-0000+--+00000000000000+--+-++-0+--+00000-++-00\\
0000000-++-00000+--+000000000000000+--+0-++-00000-++-00\\
000000000000000000000-++-+--+0-++-0-++-0000000+--+00000\\
0000+--+000+--+00000000000-++-0+--+0000-++-0000000+--+0\\
0000000000000000+--+-++-0+--+0000000000000000+--+0-++-0\\
0000000000000+--+00-++-0000000000+--+000-++-00000+--+00\\
000000000-++-0+--+000000000000-++-0000000+--+0-++-00000\\
-++-00+--+000000000000000+--+000000000000+--+-++-0+--+0\\
+--+0000000-++-00000-++-0+--+000000000000+--+0000000000\\
\end{tabular}
\end{center}
}
\caption{\label{dyckori1}Das erste orientierbare Matroid zur Dyckschen Karte}
\end{table}

\begin{table}%[htb]
{\footnotesize\tt
\begin{center}\begin{tabular}{c}
00000++++0000----000++++00----0++++-------+--0000----00\\
0000000----00000----000000000++++00----0++++-------+--0\\
0----00000----0000000++++-------+------000000---+--0000\\
0000++++000----00++++0----+++++++-++000----0000000----0\\
00000000000++++0----+++++++-++0----0000000000+++++++-++\\
0000000000000++++00----0++++-------+--00----00000----00\\
00000++++-------+------000000---+--000000++++0----+++++\\
++-++0----0000000000+++++++-++00000000000++++-------+--\\
----000000---+--0000+++++++-++0000000000---+--000000000\\[2mm]
00000++++0000----000++++00----0++++----------0000----00\\
0000000----00000----000000000++++00++++0++++----------0\\
0----00000----0000000++++--------------000000------0000\\
0000++++000++++00++++0----++++++++++000----0000000----0\\
00000000000++++0----++++++++++0----0000000000++++++++++\\
0000000000000----00++++0++++----------00++++00000++++00\\
00000++++--------------000000------000000----0----+++++\\
+++++0++++0000000000++++++++++00000000000++++----------\\
----000000------0000++++++++++0000000000------000000000\\[2mm]
00000++++0000----000----00++++0++++----+----+0000----00\\
0000000----00000----000000000----00++++0+++++++++----+0\\
0++++00000++++0000000++++----+----+----000000+----+0000\\
0000++++000++++00----0++++++++-++++-000----0000000----0\\
00000000000----0----++++-++++-0++++0000000000++++-++++-\\
0000000000000----00++++0----+++++----+00++++00000++++00\\
00000--------+----+++++000000+----+000000----0---------\\
++++-0++++0000000000-----++++-00000000000++++----+----+\\
----000000+----+0000++++-++++-0000000000+----+000000000\\[2mm]
00000+--+0000-++-000-++-00+--+0+--+-++-+--+-+0000+--+00\\
0000000+--+00000-++-000000000+--+00-++-0-++-+--+-++-+-0\\
0-++-00000+--+0000000-++--++--++-+--++-000000+--+-+0000\\
0000-++-000-++-00+--+0+--+-++-+--+-+000+--+0000000-++-0\\
00000000000+--+0-++--++-+--+-+0+--+0000000000+--+-++-+-\\
0000000000000-++-00+--+0+--+-++-+--+-+00+--+00000-++-00\\
00000+--++--++--+-++--+000000-++-+-000000-++-0+--++--+-\\
++-+-0-++-0000000000-++-+--+-+00000000000-++-+--++--+-+\\
-++-000000-++-+-0000+--++--+-+0000000000+--+-+000000000
\end{tabular}
\end{center}
}
\caption{\label{dyckori2A}Orientierungen eines Matroids zur Dyckschen Karte
         Teil A}
\end{table}

\begin{table}%[htb]
{\footnotesize\tt
\begin{center}\begin{tabular}{c}
00000-++-0000+--+000+--+00-++-0+--+-++--+--+-0000-++-00\\
0000000-++-00000+--+000000000-++-00+--+0-++--++-+-++-+0\\
0+--+00000-++-0000000-++-+--++-++-+-++-000000-+--+-0000\\
0000+--+000+--+00-++-0-++--++--+--+-000-++-0000000+--+0\\
00000000000-++-0+--+-++--+--+-0-++-0000000000+--++-++-+\\
0000000000000+--+00-++-0-++-+--+-+--+-00-++-00000+--+00\\
00000-++--++--+--+--++-000000+-++-+000000+--+0-++--++-+\\
-++-+0+--+0000000000+--+-+--+-00000000000+--+-++--+--+-\\
+--+000000+-++-+0000-++--+--+-0000000000-+--+-000000000\\[2mm]
00000-++-0000+--+000-++-00+--+0+--+-++-++-+++0000-++-00\\
0000000-++-00000+--+000000000+--+00-++-0-++-+--+--+---0\\
0-++-00000+--+0000000-++-+--+--+----++-000000++-+++0000\\
0000+--+000-++-00+--+0+--+-++-++-+++000-++-0000000+--+0\\
00000000000+--+0+--+-++-++-+++0+--+0000000000+--+--+---\\
0000000000000-++-00+--+0+--+-++-++-+++00+--+00000-++-00\\
00000+--+-++-++-++++--+000000--+---000000-++-0-++-+--+-\\
-+---0-++-0000000000-++-++-+++00000000000+--+-++-++-+++\\
+--+000000--+---0000-++-++-+++0000000000++-+++000000000\\[2mm]
00000-++-0000+--+000-++-00+--+0+--+-++-++--++0000-++-00\\
0000000-++-00000+--+000000000+--+00+--+0-++-+--+--++--0\\
0-++-00000+--+0000000-++-+--+--++---++-000000++--++0000\\
0000+--+000+--+00+--+0+--+-++-++--++000-++-0000000+--+0\\
00000000000+--+0+--+-++-++--++0+--+0000000000+--+--++--\\
0000000000000+--+00-++-0+--+-++-++--++00-++-00000+--+00\\
00000+--+-++-++--+++--+000000--++--000000+--+0-++-+--+-\\
-++--0+--+0000000000-++-++--++00000000000+--+-++-++--++\\
+--+000000--++--0000-++-++--++0000000000++--++000000000\\[2mm]
00000----0000++++000++++00----0++++-----+-++-0000++++00\\
0000000++++00000++++000000000++++00----0++++-----+-++-0\\
0----00000----0000000++++++++-+-++-----000000-+-++-0000\\
0000----000----00++++0----+++++-+--+000++++0000000++++0\\
00000000000++++0+++++++++-+--+0----0000000000+++++-+--+\\
0000000000000++++00----0++++-----+-++-00----00000----00\\
00000++++++++-+-++-----000000-+-++-000000++++0+++++++++\\
-+--+0----0000000000+++++-+--+00000000000----++++-+-++-\\
++++000000-+-++-0000----+-+--+0000000000-+-++-000000000
\end{tabular}
\end{center}
}
\caption{\label{dyckori2B}Weitere Orientierungen eines Matroids zur Dyckschen
         Karte Teil B}
\end{table}

Es gilt abzuwarten, wieviele orientierte Matroide zur Dyckschen Karte
insgesamt existieren, die bez"uglich der vorgelegten Symmetriegruppe
vertr"aglich sind. Schon aufgrund des Ergebnisses f"ur die ersten beiden
erzeugten Matroide kann man optimistisch sein, da"s eine symmetrische
Einbettung zumindest nicht ausgeschlossen ist. (U. Brehm gibt eine symmetrische
Einbettung der Dyckschen Karte mit einer Symmetrie der Ordnung sechs an, die
sicher auch hier zu finden ist.)

\clearpage
\section{Schlu"sgedanken}

Bei der Bearbeitung der Dyckschen Karte, oder beim Herangehen an einen
Komplex der Gr"o"senordnung eines Dodekaeder, zeigt sich, da"s sich unter
Ber"ucksichtigung von Symmetrieeigenschaften die aufzuwendende Rechenzeit
zur Bearbeitung eines solchen Komplexes um Potenzen verringern l"a"st.
Gerade bei der Bereitstellung einer Grundstruktur f"ur vertr"agliche Matroide
l"a"st sich hier mittels der Ber"ucksichtigung von Symmetrien mehr erreichen,
als zur Zeit implementiert ist. Nutzt man weitere m"ogliche Vereinfachungen und
Reduktionen aus, so kann hier weiter optimiert werden, gerade wenn es um
gr"o"sere Beispiele geht.
Kleine Beispiele lassen sich aber ohne weiteres mittels des vorgestellten
Vorgehens in vertretbarer Zeit interessanten Ergebnissen zuf"uhren.

Hat man vertr"agliche orientierte Matroide zu einem Komplex erzeugt,
stellt sich nun die Frage, wie man die erzeugten orientierten Matroide
realisiert, um letztendlich Einbettungen zu den vorgelegten Komplexen zu
bekommen.\\
In Anbetracht dessen, da"s die erforderliche Rechenzeit f"ur Komplexe mit mehr
als zehn 0-Zellen weiter verringert werden kann und der technische
Fortschritt stetig schnellere Prozessoren f"ur "`Rechenmaschinen"' liefert,
wird es in absehbarer Zeit auch m"oglich sein, etwa die duale
Mannigfaltigkeit zur Dyckschen Karte mit immerhin schon 32 0-Zellen und
Achtecken als 2-Zellenberandungen f"ur die Welt der orientierten Matroide zu
erschlie"sen und "uber diese die noch offene Frage der Einbettbarkeit zu
kl"aren.

\section{Die verwendeten Programme}

Die in dieser Arbeit vorgestellten Algorithmen sind als ANSI-C Programme
implementiert und sollten mit einem ANSI-C Compiler (etwa GCC) auf jeder
Rechnerplattform lauff"ahig sein.

Trotz sorgf"altiger Tests kann es passiert sein, das die Programme nicht
v"ollig fehlerfrei sind, gerade da bei der Speicherausnutzung zur h"oheren
Effizienz auf Pointer und dynamische Allokierung zur"uckgegriffen wurde.
Daher ist die Nutzung der erstellten Programme -- es handelt sich
schlie"slich um Mittel zum Zweck -- auf eigene Gefahr und ohne jegliche
Gew"ahr von Seiten des Autors. Sollten Fehler aufgetreten sein, so ist
der Autor f"ur jeden Hinweis dankbar. Der Quelltext zu den Programmen
ist, da auf keinerlei Urheberrechte geachtet wurde, im Rahmen weiterer
Forschung frei zu ver"andern und weiterzugeben -- ein Urheberrechtsanspruch
besteht von Seiten des Autors nicht.

Folgende Programme fanden f"ur diese Arbeit Verwendung:

\begin{verbatim}
VRZSYM.C      berechnet zu einer Vorzeichenliste (Chirotop) deren
              Symmetrien
              Format der Eingabedatei: Name.chi
              #n Anzahl der Punkte
              #d Rang
              ............... (n ueber d) Vorzeichen +,-,0 *

MATSYM.C      berechnet zu einer Matroidliste deren Symmetrien
              Format der Eingabedatei: Name.mat wie Name.chi,
              Liste besteht aus Eintraegen 0 und 1

GONSYM.C      berechnet zu einem Polygon-Komplex dessen
              Automorphismen
              Format der Eingabedatei: Name.gon
              #n Anzahl der Punkte pro 2-Zelle
              a_1a_2...a_n b_1b_2...b_n ...
              (Liste der Polygone mit Umlaufsinn)
              a_i kann 0..9A..Z sein

SUREBASE.C    berechnet zu einer Datei Name.gon eine Grundstruktur
              fuer zugehoerige Matroide

SYM2MAT.C     Erzeugt Matroide zu einer Automorphismendatei
              Name.aut : #n Anzahl der Punkte
                         #m Anzahl der folgenden Symmetrien
                         1  2  3  4  5  ...  n  (Identitaet)
                         s1 s2 s3 s4 s5 ...  sn (eine Permutation)
                         usw.

ORIMAT.C      Erzeugt orientierte Matroide zu einer Datei Name.mat
              mit einer Automorphismendatei Name.aut
\end{verbatim}
