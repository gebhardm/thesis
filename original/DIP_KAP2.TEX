\chapter{Von CW-Komplexen zu orientierten Matroiden}

Nach der Bereitstellung der wesentlichen Fundamentalia soll es in den beiden
nun folgenden Kapiteln um die M"oglichkeiten gehen, zu einem allgemeinen
d-dimensionalen CW-Komplex orientierte Matroide zu finden, mittels derer
eventuell zugeh"orige Koordinaten gefunden werden k"onnen. Diese sollen dann
das Bild des vorgelegten CW-Komplexes unter einer ho\-m"oo\-mor\-phen
Abbildung als durchdringungsfreien CW-Komplex (je zwei offene Zellen haben
leeren Schnitt) im Euklidischen Raum beschreiben.

In diesem Kapitel soll dabei die grundlegende Vorgehensweise dargestellt
werden, wie zun"achst solche bez"uglich Symmetrieeigenschaften des Komplexes
"`vertr"aglichen"' orientierten Matroide bestimmt werden k"onnen. Dazu gilt
n"amlich folgende Aussage.
\begin{quote}
{\sf
Existiert zu einem vorgelegten CW-Komplex $\C$ kein bez"uglich vorgegebener
Symmetrieeigenschaften vertr"agliches orientiertes Matroid, so ist $\C$ nicht
mit diesen Symmetrien in den Euklidischen Raum einbettbar.
}
\end{quote}
W"urde n"amlich doch eine solche Einbettung existieren, so geh"ort zu dieser
eine Punktkonfiguration, die eine Matrix und somit ein orientiertes Matroid
induzieren w"urde, welches zu $\C$ "`vertr"aglich"' w"are.

Im abschlie"senden dritten Kapitel sollen die bereitgestellten Mittel dazu
dienen, einige Beispiele zu untersuchen, wobei es als Abschlu"s um symmetrische
orientierte Matroide zur Dyckschen Karte gehen soll. Auch sollen weitere
Fragestellungen angegeben werden, die sich in diesem Zusammenhang stellen
(vgl. etwa \cite{Bo:86} und \cite{Bo:91}).

\section{Erste Definition der Vertr"aglichkeit}

Zun"achst soll allerdings eine Vertr"aglichkeitsdefinition angegeben werden,
die eine allgemeine Br"ucke zwischen CW-Komplexen und orientierten Matroiden
schl"agt.\\
Im Grundlagenteil wurde dazu als wichtiges Ergebnis aus der Theorie der
orientierten Matroide aufgef"uhrt, wie sich unter Anwendung des topologischen
Repr"asentationssatzes (Pseudo-)Sph"arenarrangements und orientierte Matroide
verbinden lassen. Ausgehend von diesem Satz soll es nun gelingen, dies auch auf
allgemeine CW-Komplexen zu beziehen.\\
Betrachtet man als Ausgangspunkt ein beliebiges (Pseudo-)Sph"arensystem auf
der $\S^d$, so ist dort durch das Arrangement eine Zerlegung in Zellen gegeben,
die wegen der Hom"oomorphieeigenschaften von Sph"are und $\R^d$ auch als
CW-Komplex interpretiert werden kann (vgl. Seite \pageref{cell}). Die maximalen
Zellen dieses Komplexes sind jene d-dimensionalen Gebiete auf der $\S^d$, deren
Rand durch Schnitte und dadurch induzierte Segmente der
(d$-$1)-(Pseudo-)Sph"aren, die gerade den niederdimensionalen Zellen
entsprechen, gegeben ist. Aufgrund seiner Entstehung sei dieser Komplex als
durch ein (Pseudo-)Sph"arensystem induzierter CW-Komplex \idx{induzierter
CW-Komplex} bezeichnet. Handelt es sich bei einem Pseudosph"arensystem um die
Darstellung eines orientierten Matroids und ist dieses sogar realisierbar (wenn
das Pseudosph"arenarrangement zu einem Sph"arenarrangement mit gleicher
Schnittstruktur hom"oomorph ist), so entspricht das zugeh"orige Sph"arensystem
gerade der Realisierung des durch das Pseudosph"arensystem induzierten
CW-Komplexes.\\
Betrachtet man das dem entstandenen Sph"arensystem entsprechende
Hyperebenenarrangement, so ist mit dessen zugeh"origen Normalenvektoren eine
Koordinatenmatrix gegeben, die ein Chirotop induziert, das ein zum
Sph"arensystem "aquivalentes orientiertes Matroid beschreibt. Entsprechend der
Definition der Polarit"at von Seite \pageref{polar} hat man mit diesen Normalen
eine Punktkonfiguration eines zum Ausgangssph"arensystem polaren Komplexes. Das
Zusammenspiel der Realisierung eines durch ein Sph"arensystem induzierten
Komplexes einerseits und des durch die extrahierte Matrix gegebenen Chirotops
andererseits ist "uber die "Aquivalenz der Darstellungen orientierter Matroide
zu deuten. Aufgrund der verschiedenen Zugangsweisen soll klar unterschieden
werden, ob bei der Untersuchung eines CW-Komplexes von einem
Pseudosph"arenarrangement oder einem Chirotop ausgegangen wird. Zun"achst wollen
wir so nur die Darstellung eines durch ein Sph"arensystem induzierten
CW-Komplexes betrachten. Wie das zugeh"orige Chirotop in Zusammenhang mit der
Struktur des Komplexes steht, kann dabei als gesonderte Fragestellung
aufgefa"st werden.

Aus obigen Grund"uberlegungen ergibt sich nun die entscheidende erste
Definition, die die Verbindung zwischen CW-Komplexen und orientierten Matroiden
charakterisiert:
\begin{quote}
{\bf Ein orientiertes Matroid $\cal M$ soll vertr"aglich zu einem beliebigen
CW-Komplex $\cal C$ hei"sen, wenn $\cal C$ als Unterkomplex in einem durch ein
$\cal M$ darstellendes Pseudosph"arenarrangement induzierten CW-Komplex
enthalten ist.}
\end{quote}
Ist dieses Pseudosph"arenarrangement zu einem Sph"arensystem hom"oomorph, so ist
auch eine Realisierung des CW-Komplexes als entsprechende Teilstruktur des
Sph"arensystems gegeben.\\
{\scsi
Hier k"onnte man das Pseudo- und das Sph"arensystem auch als homotop bezeichnen,
wenn sich die beiden stetig unter Beibehaltung aller Schnitteigenschaften
ineinander "uberf"uhren lassen, was man auch als Streckung des
Pseudosph"arensystems bezeichnet.
}

\begin{figure}[htb]
$$
\beginpicture
\unitlength0.6cm
\setlinear
\setcoordinatesystem units <0.6cm,0.6cm>
\setplotarea x from -6 to 6, y from -3 to 3
\put {\circle*{0.15}} [Bl] at -5.5 -1.5
\put {\circle*{0.15}} [Bl] at -5.5 0
\put {\circle*{0.15}} [Bl] at -5 -1
\put {\circle*{0.15}} [Bl] at -4.5 0.5
\put {\circle*{0.15}} [Bl] at -4 -0.5
\put {\circle*{0.15}} [Bl] at -4 -1.5
\put {\circle*{0.15}} [Bl] at -3.5 0
\put {\circle*{0.15}} [Bl] at -3 -1
\plot -5.5 -1.5 -5.5 0 -4.5 0.5 -3.5 0 -3 -1 -4 -1.5 -5.5 -1.5 /
\plot -5.5 -1.5 -5 -1 -4 -0.5 -3.5 0 /
\plot -5.5 0 -5 -1 -4 -1.5 /
\plot -4.5 0.5 -4 -0.5 -3 -1 /
\put {\small CW-Komplex $\mapsto$ orientiertes Matroid
      $\mapsto$ Matrix} [Bl] at -6 -2.5
\circulararc 360 degrees from 1 0 center at 0 0
\ellipticalarc axes ratio 3:1 180 degrees from 1 0 center at 0 0
\startrotation by 0.5 0.866
\ellipticalarc axes ratio 3:1 180 degrees from 1 0 center at 0 0
\startrotation by -0.5 0.866
\ellipticalarc axes ratio 3:1 180 degrees from 1 0 center at 0 0
\stoprotation
\put {$\left(\begin{array}{cc}
             \cdot & \cdot \\
             \cdot & \cdot \\
             \cdot & \cdot
             \end{array}\right)$} [Bl] at 4 -0.5
\endpicture
$$
\caption{Der Weg von einem CW-Komplex zu einer m"oglichen Realisierung}
\label{quest}
\end{figure}

Um von beliebig vorgelegten CW-Komplexen entscheiden zu k"onnen, ob diese in
durch Sph"a\-rensysteme induzierten CW-Komplexen als Unterkomplexe vorkommen,
mu"s zu\-n"achst untersucht werden, wie solche Unterkomplexe aussehen k"onnen.
Dazu sei ein kleines Beispiel betrachtet, welches zeigen soll, wie sich diese
Unterkomplex\-eigenschaft "au"sert und welche Schwierigkeiten bei der
Betrachtung solcher Pseudosph"arensysteme auftreten k"onnen.

Als erstes Beispiel betrachten wir dazu einen 1-Komplex, der aus sechs 0-Zellen
und sechs 1-Zellen bestehe, die ein topologisches Sechseck beschreiben (siehe
dazu Abbildung \ref{hexagon}). Der 1-Komplex entspricht dem \ul{Rand}komplex
einer 2-Zelle, so wie auch im allgemeinen Fall d-CW-Komplexe als (Teil des)
Randkomplex(es) einer (d+1)-Zelle gedeutet werden sollen. Dies geschieht in
Verallgemeinerung der Betrachtung von Randstrukturen von Mannigfaltigkeiten
und impliziert damit, in welcher Dimension ein zugeh"origes beziehungsweise
vertr"agliches Sph"arensystem zu suchen sein soll.\\
Da wir zum einen Durchdringungsfreiheit garantieren wollen, respektive die
Eigenschaft zu erhalten ist, da"s alle offenen Zellen nach Abbildung in den
euklidischen Raum disjunkt sind, sowie zum anderen eine affin lineare
Realisierung angestrebt wird, ist im allgemeinen Fall die kleinste Dimension,
in der mit einer Realisierung zu rechnen ist, um eins h"oher als die maximale
Dimension der auftretenden Zellen des vorgelegten Komplexes.
Betrachtet man als einfachen Fall etwa ein (d+1)-Simplex, so stellt dessen Rand
gerade einen d-Komplex dar, der eine (d+1)-Zelle umschreibt.
Zusammenfassend soll also ein d-Komplex immer als (Teil der) Berandung
\ul{einer} (d+1)-Zelle aufgefa"st werden, womit als Ausgangssituation die
Realisierbarkeit eines d-dimensionalen CW-Komplexes immer im $\R^{d+1}$ zu
untersuchen ist.\\
"Ubertragen auf die Sph"aren bedeutet dies, da"s der vorgelegte d-Komplex
die Berandung einer (d+1)-Zelle beschreibt, die auf einer $\S^{d+1}$ liege,
im Rand der (d+2)-Kugel. Die Pseudosph"aren, die den vorgelegten Komplex
implizieren sollen sind stetige Verformungen der $\S^d$, womit sich als
allgemeine Situation die Betrachtung orientierter Matroide im Rang d+2 ergibt,
entsprechend $\S^{d+1}=\partial B^{d+2}\subset\R^{d+2}$.\\
Nun aber zu unserem Sechseckbeispiel.

\begin{figure}[htb]
$$
\beginpicture
\unitlength0.6cm
\setlinear
\setcoordinatesystem units <0.6cm,0.6cm>
\setplotarea x from -6 to 6, y from -3 to 3.5
\put{ \beginpicture
   \setsolid \thicklines
   \setquadratic
   \plot 0.5 0.5 1.5 0.8 2.5 0.0 /
   \plot 2.5 0.0 2.5 1.0 3.5 1.5 /
   \plot 3.5 1.5 2.8 2.0 3.0 3.0 /
   \plot 3.0 3.0 2.3 2.7 1.5 3.5 /
   \plot 1.5 3.5 1.2 2.6 0.5 2.5 /
   \plot 0.5 2.5 1.1 1.5 0.5 0.5 /
   \setlinear \thinlines
   \put {\circle*{0.15}} [Bl] at 0.5 0.5
   \put {\circle*{0.15}} [Bl] at 2.5 0.0
   \put {\circle*{0.15}} [Bl] at 3.5 1.5
   \put {\circle*{0.15}} [Bl] at 3.0 3.0
   \put {\circle*{0.15}} [Bl] at 1.5 3.5
   \put {\circle*{0.15}} [Bl] at 0.5 2.5
   \endpicture } at 0 0
\put{ \beginpicture
      \setsolid \setlinear
        \ellipticalarc axes ratio 3:1 360 degrees from 2.5 0 center at 0 0
        \startrotation by 0 -1 about 0 0
        \ellipticalarc axes ratio 3:1 360 degrees from 2.5 0 center at 0 0
        \stoprotation
        \plot 0 2.5 0 -2.5 /
        \setdashes<1mm>
        \circulararc 360 degrees from 2.5 0 center at 0 0
        \put {\circle*{0.15}} [Bl] at 0 0.85
        \put {\circle*{0.15}} [Bl] at 0 -0.85
        \put {\circle*{0.15}} [Bl] at 0.8 0.8
        \put {\circle*{0.15}} [Bl] at -0.8 0.8
        \put {\circle*{0.15}} [Bl] at 0.8 -0.8
        \put {\circle*{0.15}} [Bl] at -0.8 -0.8
      \endpicture } at 5 0
\put{ \beginpicture
        \setsolid \setlinear
        \ellipticalarc axes ratio 3:1 360 degrees from 2.5 0 center at 0 0
        \startrotation by 0.5 0.866 about 0 0
        \ellipticalarc axes ratio 3:1 360 degrees from 2.5 0 center at 0 0
        \stoprotation
        \startrotation by -0.5 0.866 about 0 0
        \ellipticalarc axes ratio 3:1 360 degrees from 2.5 0 center at 0 0
        \stoprotation
        \setdashes<1mm>
        \circulararc 360 degrees from 2.5 0 center at 0 0
        \put {\circle*{0.15}} [Bl] at 0.95 0
        \put {\circle*{0.15}} [Bl] at -0.95 0
        \put {\circle*{0.15}} [Bl] at 0.475 0.823
        \put {\circle*{0.15}} [Bl] at -0.475 0.823
        \put {\circle*{0.15}} [Bl] at 0.475 -0.823
        \put {\circle*{0.15}} [Bl] at -0.475 -0.823
      \endpicture } at -5 0
\endpicture
$$
\caption{Einbettung eines topologischen Sechsecks}
\label{hexagon}
\end{figure}

Wie schon Abbildung \ref{hexagon} einer Darstellung des topologischen Sechsecks
zeigt, gibt es neben der anschaulichen Realisierung als ebenes Sechseck (die
"`"ubliche"') in einem Sph"arensystem noch eine weitere M"oglichkeit, ein
Sechseck als Unterkomplex zu finden. Dieser zweite Fall zeichnet sich sogar
dadurch aus, da"s die Anzahl der auftretenden Sph"aren minimal ist (f"unf statt
der sechs in der "`"ublichen"' Darstellung). Zur Charakterisierung dieser
beiden M"oglichkeiten kann zum einen ein beliebiger d-dimensionaler CW-Komplexes
als Randkomplex eines einzigen Topes (Gebietes auf der $\S^{d+1}$) auftreten
oder aber auch der Randstruktur einer Vereinigung von (benachbarten) Topes
entsprechen. Im Falle mehrerer beteiligter Topes ist die durch den vorgelegten
Komplex umschriebene (d+1)-Zelle das Innere der Vereinigung der Abschl"usse der
auftretenden Regionen.

Wird von einer weiteren Realisierungsm"oglichkeit des Sechsecks (einer
nichtkonvexen) zu einem zugeh"origen Sph"arenarrangement "ubergegangen, so
zeigt sich, da"s die beiden zuerst aufgef"uhrten F"alle die im Sinne der
Unterkomplexeigenschaft einzigen M"oglichkeiten sind, den Komplex in
einem Sph"arensystem darzustellen. L"a"st man n"amlich zu, da"s Segmente einer
(Pseudo-)Sph"are auch mehrmals zur Beschreibung vorgelegter Zellen Verwendung
finden, das hei"st wenn nicht benachbarte Teile einer (Pseudo-)Sph"are
verschiedenen Zellen des gegebenen CW-Komplexes entsprechen sollen (vgl.
Abb.\ref{linehex}), kann hier die Unterkomplexeigenschaft verletzt werden.

\begin{figure}[htb]
$$
\beginpicture
\unitlength0.6cm
\setlinear
\setcoordinatesystem units <0.6cm,0.6cm>
\setplotarea x from -6 to 6, y from -3 to 3
\put{ \beginpicture
      \setsolid \setlinear
      \plot 0 1 -2.5 0 -1 0 0 -1 1 0 2.5 0 0 1 /
      \put {\circle*{0.15}} [Bl] at 0 1
      \put {\circle*{0.15}} [Bl] at -2.5 0
      \put {\circle*{0.15}} [Bl] at -1 0
      \put {\circle*{0.15}} [Bl] at 0 -1
      \put {\circle*{0.15}} [Bl] at 1 0
      \put {\circle*{0.15}} [Bl] at 2.5 0
      \endpicture } at -3 0
\put{ \beginpicture
        \setsolid \setlinear
        \ellipticalarc axes ratio 3:1 -180 degrees from 2.5 0 center at 0 0
        \startrotation by 0.819 0.574 about 0 0
        \ellipticalarc axes ratio 3:1 -180 degrees from 2.5 0 center at 0 0
        \stoprotation
        \startrotation by 0.766 -0.643 about 0 0
        \ellipticalarc axes ratio 3:1 -180 degrees from 2.5 0 center at 0 0
        \stoprotation
        \startrotation by 0.643 0.766 about 0 0
        \ellipticalarc axes ratio 3:1 180 degrees from 2.5 0 center at 0 0
        \stoprotation
        \startrotation by 0.5 -0.866 about 0 0
        \ellipticalarc axes ratio 3:1 --180 degrees from 2.5 0 center at 0 0
        \stoprotation
        \setdashes<1mm>
        \circulararc 360 degrees from 2.5 0 center at 0 0
        \put {\circle*{0.15}} [Bl] at 0.11 1.29
        \put {\circle*{0.15}} [Bl] at 1.24 -0.71
        \put {\circle*{0.15}} [Bl] at -1.45 -0.68
        \put {\circle*{0.15}} [Bl] at -0.3 -0.82
        \put {\circle*{0.15}} [Bl] at 0.27 -0.83
        \put {\circle*{0.15}} [Bl] at -0.05 -1
        \put {\circle{0.25}} [Bl] at -1.12 -0.09
        \put {\circle{0.25}} [Bl] at  1.05 -0.23
      \endpicture } at 3 0
\endpicture
$$
\caption{Ein "`verbotenes"' Sph"arensystem}
\label{linehex}
\end{figure}

Was bei der polygonalen Darstellung des Sechsecks plausibel erscheint,
widerspricht der Definition der induzierten Zellen im Sph"arenarrangement, nach
der jeder Schnitt von (Pseudo-)Sph"aren immer einer niederdimensionalen Zelle
entspricht. Dies ist in diesem Fall nicht gew"ahrleistet. Vielmehr w"urde es
sich hier um die Realisierung eines topologischen Achtecks im Rand der
Vereinigung von vier Topes handeln. Im Sechseck sind die beiden zus"atzlichen
0-Zellen (nicht ausgef"ullte Kreise in Abb. \ref{linehex}) in der Realisierung
nicht vorhanden. Solch eine Situation soll grunds"atzlich verboten sein, denn
sind $S_1$ und $S_2$ zwei benachbarte Zellen des induzierten Komplexes
($\ol{S_1}\cap\ol{S_2}\neq\emptyset$), so ist zwar int($\ol{S_1}\cup\ol{S_2}$)
nach Definition eine Zelle, sie ist aber kein Element des durch das
Sph"arensystem induzierten Komplexes und verletzt somit die
Unterkomplexeigenschaft. Der betrachtete Unterkomplex des induzierten
CW-Komplexes h"atte zwar den selben zugrundeliegenden Raum, aber mehr
Zellen als die Realisierung des vorgelegten CW-Komplexes, was nicht erlaubt ist.

Mit diesem ersten Beispiel l"a"st sich die Definition von oben nun weiter
pr"azisieren.
\begin{quote}
{\bf Ein orientiertes Matroid $\cal M$ soll vertr"aglich zu einem beliebigen
CW-Komplex $\cal C$ hei"sen, wenn $\cal C$ dem Randkomplex des Abschlusses
einer Vereinigung von Topes eines $\cal M$ darstellenden
Pseudosph"arenarrangements entspricht.}
\end{quote}

Zur Vertiefung des gerade Beschriebenen und als Ausgangsbasis f"ur die
Untersuchung allgemeiner zweidimensionaler CW-Komplexe, zu der im dritten
Kapitel weiter Beispiele behandelt werden sollen, m"oge als zweites
Beispiel ein 2-Komplex betrachtet werden, der den Rand eines Torus beschreibt
(vgl. dazu auch \cite{Dau:89} und \cite{BoWi:87}). Der vorgelegte CW-Komplex
sei dazu nach Abbildung \ref{torus1} gegeben.

\begin{figure}[htb]
$$
\input torus
\settorus
$$
\caption{Torus}
\label{torus2}
\end{figure}

Nach der Beschreibung aller Eigenschaften des Torus, die sich mit den bisher
eingef"uhrten Begriffen angeben lassen, soll ein Algorithmus entwickelt werden,
der die Vorgehensweise im generellen Fall kombinatorischer Komplexe
unterst"utzen soll.

\section{Von einem CW-Komplex zu einem Algorithmus}

Es soll nun ein Torus im $\R^3$ betrachtet werden, dessen kombinatorische
Struktur als zweidimensionaler CW-Komplex nach Abbildung \ref{torus1}
durch Kantenidentifikation gegeben sei. Entgegen der Betrachtung der
Einbettung triangulierter Tori (vgl. etwa \cite{Dau:89} und \cite{BoEg:91}),
sei hier gerade der Nichtsimplizialit"at allgemeiner CW-Komplexe Rechnung
getragen und eine Zerlegung in Vierecke betrachtet.

\begin{figure}[htb]
$$
\beginpicture
\unitlength0.6cm
\setlinear
\setcoordinatesystem units <0.6cm,0.6cm>
\setplotarea x from -3.5 to 3.5, y from -2 to 2
\plot -3 1.5 3 1.5 3 -1.5 -3 -1.5 -3 1.5 /
\plot -1 1.5 -1 -1.5 /
\plot  1 1.5  1 -1.5 /
\plot -3 0.5 3 0.5 /
\plot -3 -0.5 3 -0.5 /
\put {1} [br] at -3.1 1.6  \put {1} [bl] at  3.1 1.6  \put {1} [tl] at  3.1 -1.6
\put {1} [tr] at -3.1 -1.6 \put {2} [br] at  -1 1.6   \put {2} [tr] at  -1 -1.6
\put {3} [bl] at  1 1.6    \put {3} [tl] at  1 -1.6   \put {4} [br] at -3.1 0.5
\put {4} [bl] at  3.1 0.5  \put {5} [br] at -1.1 0.6  \put {6} [bl] at  1.1 0.6
\put {7} [tr] at -3.1 -0.5 \put {7} [tl] at  3.1 -0.5 \put {8} [tr] at -1.1 -0.6
\put {9} [tl] at  1.1 -0.6
\endpicture
$$
\caption{CW-Komplex "`Rand eines Torus"'}
\label{torus1}
\end{figure}

Der vorgelegte Komplex besitzt neun 0-Zellen, 18 1-Zellen und neun 2-Zellen,
nach der Eulerformel also Geschlecht Eins. Da von der Wahl der Zellen her schon
auf Realisierungen geschlossen werden kann (so wie eine m"ogliche in Abbildung
\ref{torus3} zu sehen ist), w"are es nun interessant herauszufinden, wie sich
dieser Komplex als Unterkomplex in ein Sph"arensystem einf"ugt.

Was allerdings bei der Betrachtung etwa einer Pyramide oder eines anderen
konvexen 3-Polytops mit einer geringen Zellenanzahl noch anschaulich
verst"andlich erscheint (indem man sich 2-Sph"aren so "`ineinandergesteckt"'
denkt, da"s pro 2-Zelle des Komplexes eine Sph"are diese durch einen Auschnitt
repr"asentiert), gestaltet sich schon hier, insbesondere bei noch gr"o"serer
Zellenanzahl und einem Nicht-Konvexit"at induzierenden topologischen Geschlecht,
schon schwieriger. Mangels "Ubersichtlichkeit in der m"oglichen Darstellung als
Sph"arensystem soll deshalb eine Alternative zur Suche nach orientierten
Matroiden angewandt werden, die unter Ber"ucksichtigung der Symmetriegruppe des
vorgelegten CW-Komplexes zugeh"orige Chirotope bestimmen helfen soll, was eine
zweite Vertr"aglichkeitsdefinition liefert.

\begin{figure}[htb]
$$
\input tritor
\settritor
$$
\caption{Eine m"ogliche Realisierung des Torus}
\label{torus3}
\end{figure}

Da Chirotope durch (formale) Matrizen bestimmt werden und dadurch implizit eine
eventuelle Punktkonfigurationen betrachtet wird, ist es wenig sinnvoll, eine
Vertr"aglichkeitsdefinition von Chirotopen zu beliebigen CW-Komplexen anzugeben.
Vielmehr soll der Fall simplizialer Komplexe dahingehend erweitert werden, da"s
nun mehr als (r+1) 0-Zellen/Punkte pro r-Seite zugelassen seien, in der Sprache
der (n$\times$r)-Matrizen also auch singul"are (r$\times$r)-Untermatrizen
auftreten k"onnen. Im Fall von 2-Komplexen fordert dies von der Gestalt der
vorgelegten CW-Komplexe, da"s alle maximalen Zellen gleicher Dimension seien und
der Rand des Abschlusses einer jeden 2-Zelle einem Polygonzug,
im h"oherdimensionalen Fall einem Polytop, entspreche, der die 0- und 1-Zellen,
sowie deren Inzidenzen induziert. In Aigners Buch zur
Kombinatorik \cite{Aig:76} hei"sen solche 2-dimensionalen CW-Komplexe
Landkarten, was mit der Definition der Karten aus dem Abschnitt "uber
Symmetriebegriffe gleichzusetzen ist. Ausgehend von den 0-Zellen ergibt sich
mittels deren Sterne wieder der Komplex, was das Vorhandensein der n"otigen n
Punkte f"ur Chirotope, zusammen mit dem Rang r "uber die Dimension der maximalen
Zellen + 2 sicherstellt.\\
Nun ist aber ein Chirotop noch nicht vertr"aglich mit einem CW-Komplex nur
aufgrund der Tatsache, da"s die Anzahl der Punkte und der Rang "ubereinstimmen.
Als weitere Kopplung dient die Eigenschaft in gewissem Sinne gleiche
Symmetrieeigenschaften zu haben. Dazu sei daran erinnert, da"s die Realisierung
eines kombinatorischen Komplexes eine Untergruppe der Symmetriegruppe von diesem
als eigene Symmetriegruppe besitzt. Die Realisierung weist hierbei den 0-Zellen
aber gerade jene Koordinaten zu, die in eine Matrix geschrieben, ein
zugeh"origes Chirotop liefern. Damit kommt man nun zu folgender zweiter
Vertr"aglichkeitsdefinition von CW-Komplexen und orientierten Matroiden.

\begin{quote}
{\bf Ein Chirotop $\chi$ soll vertr"aglich zu einem polyedrisch begrenzten
CW-Komplex $\cal C$ hei"sen, wenn Punktanzahl und Rang "ubereinstimmen und
die Automorphismengruppe ${\cal A}(\chi)$ eine Untergruppe der Symmetriegruppe
${\cal A}({\cal C})$ des CW-Komplexes ist.}
\end{quote}

Nun soll f"ur das Torusbeispiel zun"achst die Symmetriegruppe des vorgelegten
Komplexes bestimmt werden. Dies kann mittels eines kurzen Programmes geschehen,
dessen Aufbau sich wie folgt gliedert.

\subsection{Bestimmung der Symmetriegruppe}

Vorgelegt sei eine Liste, die angibt, welche 0-Zellen im Rand des Abschlusses
jeder 2-Zelle eines vorgelegten 2-CW-Komplexes liegen. Die 2-Zellen seien dabei
durch geschlossene Polygonz"uge begrenzt, die durch die 0-Zellen und sie
verbindende 1-Zellen in den R"andern der Abschl"usse der 2-Zellen gegeben
sind. Dies geschieht entsprechend obiger Definition und ist gerade im Hinblick
auf die Untersuchung von Karten von Mannigfaltigkeiten sehr zweckm"a"sig.\\
Zus"atzlich sei auf den R"andern der 2-Zellen willk"urlich eine Orientierung in
Form eines Durchlaufsinnes eingef"uhrt. Diese soll dazu dienen, auf die
tats"achlich vorhandenen 1-Zellen, also Kanten, zur"uckschliessen zu k"onnen.
{\scsi
Entgegen simplizialer Komplexe ist eine 0-Zelle/Ecke im Rand einer 2-Zelle/Seite
im allgemeinen nicht mit allen anderen Ecken der Seite durch eine 1-Zelle/Kante
verbunden, was bedeutet, da"s die Valenz einer Ecke im allgemeinen kleiner oder
gleich der Gesamt\-eckenzahl $-1$ ist, was gesondert ber"ucksichtigt werden
mu"s.
}

Die entsprechende Zellenliste sei als Datei folgender Form vorgelegt:
\begin{verbatim}
torus.gon

4
1254 2365 3146 4587 5698 6479 7821 8932 9713*
\end{verbatim}
{\scsi
Die Datei {\it torus.gon} gliedert sich wie folgt: {\bf 4} besagt, da"s pro
2-Zelle vier 0-Zellen vorkommen, also Vierecke; die folgende Liste enth"alt die
0-Zellen im Rand der abgeschlossenen 2-Zellen entsprechend ihres Durchlaufsinns.
}

Beim Einlesen der Liste wird die maximale Anzahl der vorhandenen 0-Zellen
bestimmt und als {\it npkt} gespeichert. In einer
({\it npkt}$\times${\it npkt})-Bin"armatrix (Adjazenzmatrix) wird dabei auch
gespeichert, welche 0-Zellen durch 1-Zellen "`verbunden"' sind, also welche
"`Kanten"' vorliegen. Anschlie"send kann die Bestimmung der Symmetriegruppe
beginnen. Dazu werden in einer rekursiven Routine alle {\it npkt}! Permutationen
der 0-Zellen durchlaufen und als Abbildungsvorschrift $\pi$ an die
Symmetriepr"ufroutine "ubergeben (der Algorithmus zur Erzeugung der
Permutationen stammt aus dem Algorithmenbuch von R. Sedgewick \cite{Sed:91}).
In dieser wird getestet, ob $\pi$ jede 2-Zelle auf eine vorhandene
2-Zelle abbildet und ob in diesem Fall deren Rand respektive Kanten der
Orientierung entsprechend durchlaufen werden. Erf"ullt $\pi$ dies f"ur alle
vorgelegten 2-Zellen, so wird $\pi$ in die Symmetriegruppe "ubernommen, sonst
gestrichen und mit der n"achsten Permutation fortgefahren.\\
Nachdem alle Permutationen durchlaufen sind, liegen jene Elemente der
Permutationsgruppe $S_{\it npkt}$ vor, die einen Automorphismus auf dem
vorgelegten Komplex beschreiben. Diese werden als Symmetriegruppe in einer
Datei gespeichert. Angemerkt sei, da"s durch den Durchlauf aller Permutationen
f"ur eine gro"se Anzahl von 0-Zellen sehr viel Rechenzeit n"otig ist.
{\scsi
(F"ur die acht 0-Zellen eines 3-W"urfels etwa eine Sekunde, obiger Torus wird
in etwa 10 Sekunden abgearbeitet, die zw"olf 0-Zellen der Dyckschen Karte
beanspruchen dann schon etwas "uber einer Stunde -- gemessen auf einem
66 MHz-i486-PC mit einem C-Programm unter Linux.)} F"ur obigen Torus ergeben
sich so folgende Symmetrien (in Zykelschreibweise):
\begin{verbatim}
No.1 : identity                     No.37 : (162435)(798)
No.2 : (47)(58)(69)                 No.38 : (195)(276)(384)
No.3 : (23)(56)(89)                 No.39 : (1675)(2398)
No.4 : (23)(47)(59)(68)             No.40 : (195)(236478)
No.5 : (24)(37)(68)                 No.41 : (1576)(2893)
No.6 : (2734)(5896)                 No.42 : (186)(254793)
No.7 : (2437)(5698)                 No.43 : (153426)(789)
No.8 : (27)(34)(59)                 No.44 : (186)(294)(375)
No.9 : (12)(45)(78)                 No.45 : (16)(25)(34)(79)
No.10 : (12)(48)(57)(69)            No.46 : (194376)(285)
No.11 : (132)(465)(798)             No.47 : (16)(29)(57)
No.12 : (132)(495768)               No.48 : (1926)(4785)
No.13 : (1452)(3768)                No.49 : (125697)(384)
No.14 : (179652)(348)               No.50 : (1287)(3594)
No.15 : (146982)(375)               No.51 : (1397)(2684)
No.16 : (1782)(3495)                No.52 : (136587)(294)
No.17 : (123)(456)(789)             No.53 : (147)(258)(369)
No.18 : (123)(486759)               No.54 : (17)(28)(39)
No.19 : (13)(46)(79)                No.55 : (147)(268359)
No.20 : (13)(49)(58)(67)            No.56 : (17)(29)(38)(56)
No.21 : (145893)(276)               No.57 : (1538)(4697)
No.22 : (1793)(2486)                No.58 : (18)(35)(49)
No.23 : (1463)(2759)                No.59 : (168)(239745)
No.24 : (178563)(249)               No.60 : (1948)(2365)
No.25 : (1254)(3867)                No.61 : (157248)(369)
No.26 : (128964)(357)               No.62 : (18)(27)(39)(45)
No.27 : (14)(25)(36)                No.63 : (168)(249)(357)
No.28 : (174)(285)(396)             No.64 : (192738)(465)
No.29 : (14)(26)(35)(89)            No.65 : (159)(287463)
No.30 : (174)(295386)               No.66 : (1849)(2563)
No.31 : (139854)(267)               No.67 : (1629)(4587)
No.32 : (1364)(2957)                No.68 : (19)(26)(48)
No.33 : (15)(38)(67)                No.69 : (159)(267)(348)
No.34 : (1835)(4796)                No.70 : (183729)(456)
No.35 : (15)(24)(36)(78)            No.71 : (167349)(258)
No.36 : (184275)(396)               No.72 : (19)(28)(37)(46)
\end{verbatim}
Der vorgelegte Torus hat also eine kombinatorische Symmetriegruppe der
Ordnung 72. Nun gilt es Chirotope zu finden, die zumindest eine Untergruppe
dieser Symmetriegruppe als eigene Symmetriegruppe besitzen. Diese k"onnen dann
im Falle ihrer Realisierbarkeit dazu eingesetzt werden, zu dem vorgelegten
Komplex Koordinaten zu finden.

Zun"achst m"ussen dazu aus der vorgelegten Symmetriegruppe $\cal A$ Untergruppen
G extrahiert werden, die als Symmetriegruppen zugeh"origer Chirotope in
Betracht kommen. Genauer soll die Anzahl der zu suchenden Chirotope auf
die Anzahl derer beschr"ankt werden, die die gew"unschten Symmetrieeigenschaften
besitzen. Die betreffenden Untergruppen k"onnen wieder mittels eines Programms
erzeugt werden, welches ausgehend von den Symmetrien $\sigma\in {\cal A}$ mit
der h"ochsten Elementordnung ord($\sigma$) (es ist eine Realisierung mit
m"oglichst hoher Symmetrie angestrebt) deren zyklische Gruppen
($\{id,\sigma,\sigma^2,\ldots,\sigma^{ord(\sigma)-1}\}$) bestimmt und alle
erzeugten Elemente aus $\cal A$ streicht, so da"s am Ende die verschiedenen
maximalen zyklischen Untergruppen der vorgelegten Symmetriegruppe aufgelistet
werden. Die Wahl solcher zyklischen Untergruppen begr"undet sich in der leichten
Erzeugbarkeit und dient der Tatsache, da"s solche Gruppen wahrscheinlicher als
Symmetriegruppen von Chirotopen angenommen werden, als etwa gr"o"sere
zusammengesetzte. F"ur den Torus sind dies (die Zahlen in den eckigen Klammern
geben die Nummern der Elemente der vorgelegten obigen Automorphismengruppe an):
\begin{verbatim}
No.1 : { id, [12], [17], [2], [11], [18] }
No.2 : { id, [14], [38], [47], [69], [49] }
No.3 : { id, [15], [63], [68], [44], [26] }
No.4 : { id, [21], [69], [58], [38], [31] }
No.5 : { id, [24], [44], [33], [63], [52] }
No.6 : { id, [30], [53], [3], [28], [55] }
No.7 : { id, [36], [53], [9], [28], [61] }
No.8 : { id, [37], [17], [27], [11], [43] }
No.9 : { id, [40], [69], [5], [38], [65] }
No.10 : { id, [42], [63], [8], [44], [59] }
No.11 : { id, [46], [53], [19], [28], [71] }
No.12 : { id, [64], [17], [54], [11], [70] }
No.13 : { id, [6], [4], [7] }
No.14 : { id, [13], [35], [25] }
No.15 : { id, [16], [62], [50] }
No.16 : { id, [22], [72], [51] }
No.17 : { id, [23], [45], [32] }
No.18 : { id, [34], [20], [57] }
No.19 : { id, [39], [56], [41] }
No.20 : { id, [48], [10], [67] }
No.21 : { id, [60], [29], [66] }
No.22 : { id }
\end{verbatim}

Mittels dieser Untergruppen kann man sich nun auf die Suche nach etwaig
zugeh"origen Chirotopen begeben, was wie folgt geschehen soll.

\subsection{Erzeugung vertr"aglicher Chirotope}

Mit der Anzahl $npkt$ der vorkommenden 0-Zellen und der Dimension $d$ der
maximalen Zellen des vorgelegten CW-Komplexes ist die Gestalt "`m"oglich
zugeh"origer"', respektive vertr"aglicher Chirotope festgelegt.\\
Sie sind darstellbar durch Listen mit insgesamt ${npkt \choose d+2}$
Vorzeichen aus $\{-,0,+\}$, im Fall des Torus also mit ${9 \choose 4}=126$
Elementen, die den Vorzeichen der formalen Brackets $[\lambda]$ mit
$\lambda\in\Lambda(npkt,d+2)$ entsprechen.\\
Da die Anzahl der M"oglichkeiten diese ${npkt \choose d+2}$ Vorzeichen
aufzuf"ullen mit $3^{npkt \choose d+2}$ (im Fall des Torus
$3^{126}\geq 10^{60}$) recht gro"s ist, soll mittels zus"atzlicher Bedingungen
diese Anzahl verringert werden.\\
Dazu kann zuerst einmal die Gestalt der vorgelegten 2-Zellen dienen. Da diese
in der Realisierung als Facetten im Rand der entstehenden 3-Zelle eben sein
sollen, k"onnen die 0-Zellen im Rand ihrer Abschl"usse, bei einer Anzahl
gr"o"ser als drei, entsprechend als linear abh"angige Punkte gedeutet werden.
Da die Vorzeichen zu (zun"achst) formalen Determinanten geh"oren, k"onnen die
Vorzeichen jener Brackets zu 0 gesetzt werden, die eine Auswahl der Indizes der
0-Zellen des Randes ein und derselben abgeschlossenen 2-Zelle enthalten.\\
F"ur den Torus sind so die Vorzeichen zu den Brackets $[1245]$, $[1278]$,
$[1346]$, $[1379]$, $[2356]$, $[2389]$, $[4578]$, $[4679]$ und $[5689]$ zu 0
zu setzen, was die Anzahl der Auf\-f"ull\-m"og\-lich\-kei\-ten von $3^{126}$ auf
immerhin schon $3^{117}< 10^{56}$ reduziert.\\
Mit den zu erf"ullenden Symmetrien l"a"st sich diese Zahl noch weiter
verringern. Dies geschieht durch Kopplung der Brackets unter den Elementen
$\sigma$ einer Symmetriegruppe G, die entsteht, wenn man die Orbits der Brackets
$[\lambda]$, die Menge aller Bilder der $[\lambda]$ unter den $\sigma\in G$, als
"Aquivalenzklassen betrachtet.\\
F"ur den vorgelegten Torus sei als zu erf"ullende Symmetrieeigenschaft der
Realisierung im folgenden als Beispiel die Untergruppe G =
$\{ id, [12], [17], [2], [11], [18] \}$ vorgelegt. Mit dieser erh"alt man eine
Kopplung, durch die mit der Vorgabe von 23 Bracketvorzeichen f"ur
Repr"asentanten der Orbits alle "ubrigen Vorzeichen bestimmbar sind. In Tabelle
\ref{tab} sind in den Zeilen die 23 verschiedenen Bracketorbits aufgelistet,
von denen die Brackets der ersten Spalte, die entsprechend lexikographischer
Ordnung minimal sind, als Rep"asentanten gew"ahlt werden.
Die Vorzeichen vor den "ubrigen Brackets sind die jeweils durch die
lexikographische Ordnung nach der Permutation zu ber"ucksichtigen
Vorzeichenwechsel.

\begin{table}[htb]
{\small
$$
\begin{array}{cccccc}
Identit"at                   & {123456789\choose 312978645} &
{123456789\choose 231564897} & {123456789\choose 123789456} &
{123456789\choose 312645978} & {123456789\choose 231897564}\\
        &         &         &         &         &        \\
+[1234] & +[1239] & +[1235] & +[1237] & +[1236] & +[1238]\\
+[1245] & +[1379] & +[2356] & +[1278] & +[1346] & +[2389]\\
+[1246] & +[1389] & -[2345] & +[1279] & +[1356] & -[2378]\\
+[1247] & +[1369] & +[2358] & -[1247] & -[1369] & -[2358]\\
+[1248] & +[1349] & +[2359] & -[1257] & -[1367] & -[2368]\\
+[1249] & +[1359] & +[2357] & -[1267] & -[1368] & -[2348]\\
+[1256] & -[1378] & -[2346] & +[1289] & -[1345] & -[2379]\\
+[1258] & +[1347] & +[2369] & -[1258] & -[1347] & -[2369]\\
+[1259] & +[1357] & +[2367] & -[1268] & -[1348] & -[2349]\\
+[1269] & +[1358] & +[2347] & -[1269] & -[1358] & -[2347]\\
+[1456] & +[3789] & +[2456] & +[1789] & +[3456] & +[2789]\\
+[1457] & -[3679] & +[2568] & +[1478] & -[3469] & +[2589]\\
+[1458] & -[3479] & +[2569] & +[1578] & -[3467] & +[2689]\\
+[1459] & -[3579] & +[2567] & +[1678] & -[3468] & +[2489]\\
+[1467] & -[3689] & -[2458] & +[1479] & -[3569] & -[2578]\\
+[1468] & -[3489] & -[2459] & +[1579] & -[3567] & -[2678]\\
+[1469] & -[3589] & -[2457] & +[1679] & -[3568] & -[2478]\\
+[1489] & +[3459] & -[2579] & +[1567] & +[3678] & -[2468]\\
+[1568] & +[3478] & -[2469] & +[1589] & +[3457] & -[2679]\\
+[1569] & +[3578] & -[2467] & +[1689] & +[3458] & -[2479]\\
+[4567] & -[6789] & +[4568] & -[4789] & +[4569] & -[5789]\\
+[4578] & +[4679] & +[5689] & +[4578] & +[4679] & +[5689]\\
+[4579] & +[5679] & -[5678] & +[4678] & +[4689] & -[4589]
\end{array}$$}
\caption{Orbits der Brackets unter einer gew"ahlten Symmetrieuntergruppe}
\label{tab}
\end{table}

Da bereits bestimmt wurde, welche Brackets sicherlich zu 0 zu setzen sind,
sind f"ur eine vollst"andige Vorzeichenliste noch 21 Vorzeichen vorzugeben.
Allerdings reduziert sich die Anzahl der zu untersuchenden Vorzeichenlisten
noch nicht auf $3^{21}$, da Symmetrien $\sigma$ Chirotope $\chi$ sowohl auf
$\chi$ selbst, als auch auf $-\chi$ abbilden k"onnen, was von vornherein
nicht abzusehen ist und zu einer Gesamtzahl von m"oglichen Listen von, im Fall
des Torus, $2^5\cdot 3^{21}< 3.5\cdot 10^{11}$ f"uhrt. Der Faktor $2^5$ r"uhrt
daher, da"s pro Symmetrie ungleich der Identit"at $\chi$ oder $-\chi$ vorliegen
kann, also pro Permutation aus G auf $\chi$ und $-\chi$ zu testen ist.
\label{symmtest} Selbst diese Zahl von Vorzeichenlisten ist mit fast 335
Milliarden noch ziemlich hoch, weshalb auch diese Zahl durch weitere
"Uberlegungen reduziert werden soll, was wie folgt geschieht.

\subsection{Reduktion auf vertr"agliche Matroide}

Dazu nutzen wir die Tatsache, da"s die Vorzeichenlisten letztendlich Chirotopen
entsprechen sollen und da"s mit jedem Chirotop ein diesem zugrunde liegendes
Matroid gegeben ist. Von der Matroidseite her betrachtet, mu"s also zun"achst
ein solches vorliegen, um es zu einem Chirotop orientieren zu k"onnen.
Betrachtet man die Brackets als Basen eines Matroids, so reduzieren sich nach
der Definition, die das Erf"ulltsein der Gra"smann-Pl"ucker-Relationen "uber
GF(2) fordert, die Auff"ullm"oglichkeiten f"ur die "`Bracketvorzeichen"' auf
$2^{21}$, entsprechend der M"oglichkeit, f"ur jede Bracket 0 oder 1 zu setzen.
Der n"achste Schritt soll dementsprechend der Erzeugung bez"uglich der
vorgelegten Symmetrien vertr"aglicher Matroide gewidmet sein. Zur ersten
Vorauswahl werden dazu die unter der Symmetriegruppe in "Aquivalenzklassen
aufgeteilten 3-summandigen Gra"smann-Pl"ucker-Relationen untersucht.

Zur Erinnerung lassen sich die 3-summandigen Gra"smann-Pl"ucker-Relationen
mittels zweier Mengen $A:=\{a_1,\ldots,a_{r-2}\}$ und $B:=\{b_1,\ldots,b_4\}$
paarweise verschiedener Elemente aus $E=\{1,\ldots,npkt\}$ im Rang r
darstellen als
$$\{A|B\}=
  \begin{array}{l}
    +[a_1,\ldots,a_{r-2},b_1,b_2]\cdot [a_1,\ldots,a_{r-2},b_3,b_4]\\
    -[a_1,\ldots,a_{r-2},b_1,b_3]\cdot [a_1,\ldots,a_{r-2},b_2,b_4]\\
    +[a_1,\ldots,a_{r-2},b_1,b_4]\cdot [a_1,\ldots,a_{r-2},b_2,b_3]
  \end{array}=0$$
was zu ${npkt \choose r-2}\cdot{npkt-r+2 \choose 4}$, im Falle des Torus also
${9\choose 2}\cdot{7\choose 4}=1260$ verschiedenen Gleichungen f"uhrt.
Betrachtet man diese Gleichungen "uber dem zweielementigen K"orper GF(2), so
m"ussen "aquivalente Gleichungen vom Typ $X + Y + Z + XYZ = 0$ erf"ullt sein, in
denen die X, Y und Z den Betr"agen der Vorzeichen obiger Produkte entsprechen.
Der Summand $XYZ$ als Produkt "uber alle in der Gleichung vorkommenden Brackets
mu"s bei einer Betrachtung "uber GF(2) hinzugef"ugt werden, um auch die
erlaubten F"alle mit gleichzeitig $X=1$, $Y=1$ und $Z=1$ abzudecken.

Da die zu entstehenden Matroide bez"uglich der vorgelegten Symmetriegruppe G
vertr"aglich sein sollen, braucht man, analog der Brackets, nur Repr"asentanten
der Orbits der Gra"smann-Pl"ucker-Relationen unter G betrachten, was deren
Anzahl auf 215 reduziert. Ein Teil dieser ist in Tabelle \ref{orbtab}
aufgezeigt.

\begin{table}[htb]
{\small
$$
\begin{array}{ccccc}
\{12|3456\} & \{12|3457\} & \{12|3458\} & \{12|3459\} & \{12|3467\} \\
\{12|3468\} & \{12|3469\} & \{12|3489\} & \{12|3568\} & \{12|3569\} \\
\{12|4567\} & \{12|4568\} & \{12|4569\} & \{12|4578\} & \{12|4579\} \\
\{12|4589\} & \{12|4679\} & \{12|4689\} & \{12|5689\} & \{14|2356\} \\
\ldots      &             &             &             &             \\
\{48|1257\} & \{48|1259\} & \{48|1267\} & \{48|1269\} & \{48|1279\} \\
\{48|1356\} & \{48|1357\} & \{48|1359\} & \{48|1367\} & \{48|1369\} \\
\{48|1379\} & \{48|1567\} & \{48|1569\} & \{48|1579\} & \{48|1679\} \\
\{48|2356\} & \{48|2357\} & \{48|2359\} & \{48|2367\} & \{48|2369\} \\
\{48|2379\} & \{48|2567\} & \{48|2569\} & \{48|2579\} & \{48|2679\} \\
\{48|3567\} & \{48|3569\} & \{48|3579\} & \{48|3679\} & \{48|5679\} \\
\end{array}$$}
\caption{Repr"asentanten der Orbits der GPR unter G}
\label{orbtab}
\end{table}

Zu beachten ist hierbei, da"s eine Permutation auf den Elementen einer
k-sum\-man\-di\-gen Gra"smann-Pl"ucker-Relation das Vorzeichen des
repr"asentierten Polynoms "andert. Dies geschieht f"ur Permutationen
$\pi_1:A\to A$, $\pi_2:B\to B$ und $\pi_3:C\to C$ gem"a"s
$$\{\pi_1(A)|\pi_2(B)|\pi_3(C)\} =
\mbox{sgn}(\pi_1)\cdot\mbox{sgn}(\pi_2)\cdot\mbox{sgn}(\pi_3)\cdot\{A|B|C\}$$
Bei Betrachtungen der Relationen "uber GF(2) ist dies nicht zu ber"ucksichtigen,
wohl aber im Falle orientierter Matroide "uber GF(3).

Setzt man in diese 215 Gleichungen wiederum ein, welche Brackets unter der
Symmetriegruppe G bez"uglich ihrer Orbits gleich beziehungweise welche Null
sind und streicht alle redundanten Gleichungen (doppelte oder vom Typ
$X + X + 0 = 0$), so erh"alt man f"ur den Torus letztendlich 204 Gleichungen
(darunter 80 zweisummandige, wenn ein Summand gleich Null war), mit denen nun
mittels Fallunterscheidungen alle zur Symmetriegruppe G geh"origen Matroide
mit 9 Punkten im Rang 4 bestimmt werden k"onnen. Tabelle \ref{gltab} zeigt
einen Ausschnitt aus der Liste dieser Gleichungen, mit den Nummern der
Orbitrepr"asentanten in runden Klammern.

\begin{table}[htb]
{\small
$$
\begin{array}{lll}
(1)(3)+(1)(4)+(1)(6)+(1)(3)(4)(6) & = & 0 \\
(1)(3)+(1)(5)+(1)(9)+(1)(3)(5)(9) & = & 0 \\
(1)(3)+(1)(6)+(1)(10)+(1)(3)(6)(10) & = & 0 \\
(1)(3)+(1)(7) & = & 0 \\
(1)(4)+(1)(5) & = & 0 \\
... & & \\
(17)(21)+(17)(23) & = & 0 \\
(19)(21)+(19)(23) & = & 0 \\
(19)(21)+(20)(21) & = & 0 \\
(19)(21)+(20)(23) & = & 0 \\
(19)(23)+(20)(21) & = & 0 \\
(19)(23)+(20)(23) & = & 0 \\
(20)(21)+(20)(23) & = & 0 \\
(21)+(23) & = & 0 \\
\end{array}$$}
\caption{Gleichungen, die alle (9,4)-Matroide unter G erf"ullen m"ussen}
\label{gltab}
\end{table}

Die in den Tabellen \ref{orbtab} und \ref{gltab} aufgef"uhrten Beziehungen
der Brackets in den Gra"smann-Pl"ucker-Relationen dienen einem C-Programm
{\sc Sym2mat} (siehe dazu den Anhang) als Ausgangsbasis, alle m"oglichen,
bez"uglich einer Symmetriegruppe G und Bracket-Null- beziehungsweise
Bracket-Eins-Setzungen vertr"aglichen Matroide zu erzeugen. Dies geschieht
mittels eines rekursiven Verfahrens, welches vorhandene Abh"angigkeiten
ausnutzt und im folgenden beschrieben wird.

Das Programm {\sc Sym2mat} erh"alt als Eingabe eine Automorphismengruppe G,
die gew"unschte Punktanzahl $npkt$ und den Rang $rang$, sowie nach Berechnung
der Repr"asentanten der Bracketorbits unter G gezielt zu Null oder Eins
gesetzte Brackets ("`gesetzt"' bezieht sich hier immer auf Untersuchungen
bez"uglich GF(2)).\\
Aus diesen Informationen werden nun die (wie in Tabelle \ref{gltab}) unter G
reduzierten dreisummandigen Gra"smann-Pl"ucker-Relationen als zu erf"ullendes
Gleichungssystem bereitgestellt. Dabei sind aus Gleichungen vom Typ
$$AB + CD + EF + ABCDEF = 0$$
zum Teil Gleichungen mit nur zwei Summanden (ein Bracketorbit gleich Null)
$$AB + CD = 0$$
oder durch Gleichsetzungen Gleichungen mit Quadraten
$$AA + CD + EE + AACDEE = 0$$
oder Gemische aus beiden Typen entstanden, wenn A, B, C, D, E und F
Repr"asentanten von Bracketorbits bezeichnen (eine Begr"undung f"ur das
Erscheinungsbild dieser Gleichungen findet sind in \cite{BoOlRi:91}).\\
F"ur das Auff"ullen mit Nullen und Einsen sind nun zun"achst die
zweisummandigen Gleichungen interessant, da aus ihnen unter Eins-Setzung
eines enthaltenen Bracketorbit\-repr"asentanten eventuell Gleichheiten anderer
Orbits induziert werden.\\
Unter den vorhandenen Gleichungen werden zu Beginn solche gesucht, die von der
Gestalt $AA + BB = 0$ beziehungsweise $1A + 1B = 0$ und ihrer kommutativen
"Aquivalente sind, da hieraus direkt $A = B$ abzuleiten ist. Diese Gleichheit
wird pro Rekursionsstufe in ein "`Bracket-Gleichheitsfeld"' eingetragen und A
als frei zu bestimmende Bracket f"ur folgende Rekursionsschritte gespeichert.\\
Die aus allen vorhandenen Gleichungen resultierenden Gleichheiten werden nun
in das Gleichungssystem "ubernommen. Ein Sortierungsschritt eliminiert darauf
alle nach Gleichsetzung von Bracketorbitrepr"asentanten redundanten (doppelte
oder triviale) Gleichungen, pr"uft auf direkte Schl"usse der Form
$1 + B = 0 \follows B = 1$ und $1 + 1 + B = 0\follows B = 0$, sowie Erf"ullung
der Gleichungen und stellt so immer das aktuelle reduzierte Gleichungssystem
zur Verf"ugung. Eine Sortierung der Gleichungen zeichnet sich als
programmtechnisch sinnvoll aus und verf"ahrt nach der lexikographischen Ordnung,
nach der f"ur $AB + CD + EF$ immer gelte, da"s $A\leq B$, $C\leq D$, $E\leq F$,
sowie $A\leq C$, $A\leq E$, $C\leq E$, wenn wiederum A, B, C, D, E und F
Repr"asentanten von Bracketorbits bezeichnen. Zudem seien die Summanden, die
eine Null enthalten in der Gleichung immer zuletzt aufgef"uhrt.\\
Haben sich nach diesem Verfahren keine freien Bracketorbits ergeben, so werden
die verbliebenen ungesetzten als frei angesehen. Im n"achsten Schritt
werden in einer Schleife nacheinander die freien Bracketorbits abwechselnd auf
0 und 1 gesetzt und obige Prozedur wiederholt, woraus sich die Rekursion ergibt.

Folge dieser Vorgehensweise ist der Aufbau einer Baumstruktur von
Bracketorbit"aquivalenzen, in der, da alle m"oglichen Abh"angigkeiten und
Gleichheiten in den einzelnen "Asten ber"ucksichtigt werden, alle m"oglichen
bez"uglich der Symmetriegruppe G vertr"aglichen Matroide erzeugt werden.

\begin{figure}[p]
\begin{center}
\btab{ll}
{\bf Routine} & {\sf L"ose\_Gleichungssystem\_"uber\_GF(2)}\\
          & \\
Eingabe : & - Anzahl der Gleichungen \\
          & - Das Gleichungssystem \\
          & - Gesetzte Bracketorbits \\
          & - Abh"angigkeiten der Bracketorbits untereinander \\
          & \\
Schritt 1 : & - Untersuche alle Gleichungen auf Gleichheiten \\
            & \hspace*{3ex} $\left.\begin{array}{l}
               AA+BB=0\\
               AA+1B=0\\
               AA+B1=0\\
               1A+BB=0\\
               1A+1B=0\\
               1A+B1=0\\
               A1+BB=0\\
               A1+1B=0\\
               A1+B1=0\end{array}\right\} \follows A = B,~A\mbox{ frei}$\\
            & - Minimiere Gleichheiten : $A=B,~B=C\follows A=C$\\
            & - "Ubertrage Gleichheiten in das Gleichungssystem\\
            & - Sortiere und reduziere Gleichungssystem\\
            & \hspace*{3ex} "Uberpr"ufe dabei Schl"usse auf $A=1,~A=0$\\
            & \hspace*{3ex} "Uberpr"ufe dabei, ob Gleichungen erf"ullt sind\\
            & - Bei Erf"ullung aller Gleichungen, schreibe m"ogliches Matroid\\
Schritt 2 : & - Ermittlung weiterer freier Bracketorbits aus\\
            & \hspace*{3ex} $\left.\begin{array}{l}
              AB+AC=0\\
              BA+CA=0\\
              BA+AC=0\end{array}\right\} \follows A\mbox{ frei}$\\
            & - Bis jetzt keine Freien, so Ungesetzte frei\\
Schritt 3 : & - Setzen der freien Bracketorbits auf 0 und 1\\
            & \hspace*{3ex} Gehe in einer Schleife alle Orbits durch\\
            & \hspace*{4ex} Setze freien Orbit auf 0 bzw. 1\\
            & \hspace*{4ex} {\sf L"ose\_Gleichungssystem\_"uber\_GF(2)}\\
            & \hspace*{4ex} mit neuen Annahmen
\etab
\caption{Routine zur Bestimmung m"oglicher vertr"aglicher Matroide}
\label{solvegpr}
\end{center}
\end{figure}

Die Ausgabe von {\sc Sym2mat} liefert so die dreisummandigen
Gra"smann-Pl"ucker-Relationen erf"ullende Elementlisten, die entsprechend der
"uber GF(2) betrachteten Brackets jedem $[\lambda]$ mit $\lambda\in\Lambda(n,r)$
eine 0 oder 1 zuweist. Da das Erf"ulltsein der dreisummandigen
Gra"smann-Pl"ucker-Relationen nur eine notwendige Bedingung f"ur das Vorliegen
eines Matroids darstellt, m"u"sten nun in einem folgenden Schritt die
Elementlisten ebenfalls mit den "ubrigen k-summandigen
Gra"smann-Pl"ucker-Relationen "uber GF(2) getestet werden, damit letztendlich
nur Matroide vorliegen, die dann vertr"aglich zur Symmetriegruppe G orientiert
werden sollen.
{\scsi
Hierzu haben Bokowski et al. gezeigt, da"s "uber GF(2) die sogenannten
"`Odd-Polynomials"' zu den Gra"smann-Pl"ucker-Polynomen, die aufsummierten
ungeraden elementarsymmetrischen Funktionen der auftretenden Bracketprodukte,
darauf zu "uberpr"ufen sind, ob sie mit den vorgegebenen Betragswerten der
Elementeliste Null ergeben. Als Beispiel f"ur die 4-summandigen
Gra"smann-Pl"ucker-Relationen sind so Gleichungen vom Typ
$A+B+C+D+ABC+ABD+ACD+BCD=0$ als Summe der ersten und dritten
elementarsymmetrischen Funktion zu testen, wenn A,B,C und D jeweils das
Produkt zweier Brackets darstellen.
}

F"ur das Torusbeispiel von oben liefert {\sc Sym2mat} 43 verschiedene
Elementlisten, von denen ein Teil in Tabelle \ref{tormat} dargestellt ist.
Dabei kann das triviale Matroid (erste Zeile, nur Nullen) von vornherein als
"`unerw"unschtes"' gestrichen werden, da es sicher jede Symmetrie erf"ullt,
aber keine Information "uber den vorgelegten Komplex liefert. Interessant sind
gerade solche Matroide, die lineare Unabh"angigkeiten, eine "`R"aumlichkeit"'
des Komplexes induzieren, dazu aber gleich mehr.

\begin{table}[htb]
\begin{center}
{\scriptsize\tt
\btab{c}
000000000000000000000000000000000000000000000000000000000000000\\
000000000000000000000000000000000000000000000000000000000000000\\
\hline
000000000000000000000000000000000000000000000000000000000000000\\
000000000000000000000000000000000000000000000000111011101111011\\
\hline
000000000000000000000000000000000000000000000000100000100000000\\
000000000000100010000000000000100000000001000000000000000000000\\
\hline
$\vdots$\\
\hline
111111011111111111011101111111111101111111111110011011011111101\\
111111101111010101111111101110011110111110111111111011101111011\\
\hline
111111011111111111011101111111111101111111111111111111111111101\\
111111101111111111111111111111111111111111111111000000000000000\\
\hline
111111011111111111011101111111111101111111111111111111111111101\\
111111101111111111111111111111111111111111111111111011101111011
\etab
}
\end{center}
\caption{M"ogliche Matroid-Elementlisten zum Torusbeispiel}
\label{tormat}
\end{table}

{\sc Sym2mat} liefert ausgehend vom Test der nur dreisummandigen
Gra"smann-Pl"ucker-Polynome "`m"oglicherweise"' Matroide. Um nun sicherzugehen,
da"s f"ur die weitere Bearbeitung nur wirkliche Matroide vorliegen, m"ussen die
ausgegebenen Elementlisten darauf getestet werden, ob sie Matroide darstellen
oder nicht. Dazu k"onnen, wie angedeutet, die "ubrigen k-summandigen
($4\leq k\leq rang+1$) Gra"smann-Pl"ucker-Relationen in Form der
Odd-Polynomials getestet werden. Ebenso kann der Matroidbeweis aber auch auf
dem Nachweis basieren, da"s die Elementlisten das Basisaxiom der Definition
eines Matroids erf"ullen m"ussen. Nach \cite{Bj:93}, Seite 81, mu"s
so "uberpr"uft werden, ob die Menge
$${\cal B}=\{[\lambda_1,\ldots,\lambda_d]\neq 0~|~(\lambda_1,\ldots,\lambda_d)
\in\Lambda (n,d)\}$$
die Eigenschaft besitzt, da"s f"ur je zwei "`Basen"' $B_1$ und $B_2$ aus
$\cal B$ und alle $b_1\in B_1\backslash B_2$ ein $b'_i\in B_2\backslash B_1$
so existiert, da"s $((B_1\backslash b_1)\cup b'_i)$ aus $\cal B$ stammt, was
nach Abbildung \ref{testmatroid} geschehen kann.

\begin{figure}[htb]
\begin{center}
\btab{ll}
{\bf Routine} & {\sf Teste\_auf\_Matroid}\\
          & \\
Eingabe : & - Elementliste $\ul{\cal M}$ des vermeindlichen Matroids \\
          & - F"ur alle $1\leq i\leq {npkt \choose rang}$ \\
          & \hspace*{2ex} Ist $matroid[i]\neq 0$ \\
          & \hspace*{4ex} Sei $B_1$ die Bracket zu $matroid[i]$ \\
          & \hspace*{4ex} F"ur alle $1\leq (j\neq i)\leq {npkt \choose rang}$ \\
          & \hspace*{6ex} Ist $matroid[j]\neq 0$ \\
          & \hspace*{8ex} Sei $B_2$ die Bracket zu $matroid[j]$ \\
          & \hspace*{8ex} $P = B_1\backslash B_2$ \\
          & \hspace*{8ex} $Q = B_2\backslash B_1$ \\
          & \hspace*{8ex} F"ur alle $p\in P$\\
          & \hspace*{10ex} Existiert in Q kein Element q mit \\
          & \hspace*{12ex} $(B_1\backslash p)\cup q\in{\cal B}$\\
          & \hspace*{10ex} so ist $\ul{\cal M}$ kein Matroid\\
Ausgabe : & - Matroid oder nicht
\etab
\caption{Routine zum Test auf die Matroideigenschaft einer Elementliste}
\label{testmatroid}
\end{center}
\end{figure}

Liegt die Elementliste des zu "uberpr"ufenden vermeindlichen Matroids als ein
Feld $matroid[i]$ mit $1\leq i\leq {n\choose d}$ und Eintr"agen 0 oder 1 vor,
etwa wie die Ausgabe von {\sc Sym2mat}, so l"a"st sich mit einer Routine
entsprechend Abbildung \ref{testmatroid} auf die Matroideigenschaft der Liste
testen. Dazu sind im schlechtesten Fall, wenn alle Basen "uberpr"uft werden
m"u"sten, ${npkt\choose rang}\cdot\left({npkt\choose rang}-1\right)\cdot rang^2$
Operationen n"otig. Dies ist zwar auf den ersten Blick wesentlich mehr, als die
"Uberpr"ufung der Gra"smann-Pl"ucker-Relationen ergeben w"urde, bedenkt man
aber, da"s dazu alle k-summandigen ($4\leq k\leq rang+1$) rekursiv erzeugt und
in Bracketgleichungen (die Odd-Poly\-no\-mi\-als) umgewandelt werden
m"ussen, so ist der (Rechen-)\-Aufwand von der gleichen Gr"o"senordnung, so
da"s der viel einfacher zu implementierende Basistest auch seine Berechtigung
besitzt.

F"uhrt man diese "Uberpr"ufung durch, so ergeben sich aus den 43 vorgelegten
Listen f"ur den Torus 25 nichttriviale Matroide, von denen ein Teil in Tabelle
\ref{torusmat} aufgelistet ist. Diese k"onnten nun als Eingabe f"ur ein
Matroid-Orientierungsprogramm dienen, welches mit seiner Beschreibung das Ziel
dieser Arbeit darstellt und eine M"oglichkeit liefert, zu einem CW-Komplex
"uber dessen Symmetrien vertr"agliche orientierte Matroide zu erzeugen.

\begin{table}[htb]
\begin{center}
{\scriptsize\tt
\btab{c}
000000000000000000000000000000000000000000000000000000000000000\\
000000000000000000000000000000000000000000000000111011101111011\\
\hline
000000000000000000000000000000000000011111111111111111100000000\\
000000000111111111111111111001111111111111111110000000000000000\\
\hline
000000000000000000000000000000000000011111111111111111100000000\\
000000000111111111111111111001111111111111111110111011101111011\\
\hline
$\vdots$\\
\hline
111111011111111111011101111111111101111111111110011011011111101\\
111111101111010101111111101110011110111110111111111011101111011\\
\hline
111111011111111111011101111111111101111111111111111111111111101\\
111111101111111111111111111111111111111111111111000000000000000\\
\hline
111111011111111111011101111111111101111111111111111111111111101\\
111111101111111111111111111111111111111111111111111011101111011\\
\etab
}
\end{center}
\caption{Die bez"uglich G vertr"aglichen Matroide zum Torus}
\label{torusmat}
\end{table}

Betrachtet man zuvor die Struktur der erzeugten Matroide, so stellt man fest,
da"s unter diesen noch viele "`uninteressante"' zu finden sind. Dies sind jene,
die "`zuviele"' Nullen enthalten, entsprechend der Determinantendeutung der
Basen also eine "`flache"' Punktkonfiguration repr"asentieren.\\
Um solche von vorn herein ausschlie"sen zu k"onnen, kann man sich des
vorgelegten Komplexes bedienen und fordern, welche Brackets sicher nicht Null,
also Basen des Matroids sein sollen. Solche Basen ergeben sich aus den Brackets,
die sich aus den Indizes von Punkten benachbarter Zellen, die gefordert nicht
in einer Ebene liegen sollen, zusammensetzen lassen. F"ur den Torus ergibt diese
Forderung bei abgeschlossenen benachbarten Zellen $\mbox{conv}\{a,b,c,d\}$ und
$\mbox{conv}\{a,b,e,f\}$ mit den 0-Zellen-Indizes a bis f und der gemeinsamen
Kante $\ol{ab}$, folgende Brackets als Basen eines Matroids:
\begin{center}
{\small
[abce], [abcf], [abde], [abdf], [acde], [acdf],
[acef], [adef], [bcde], [bcdf], [bcef], [bdef]
}
\end{center}
Von den ${6\choose 4}$ M"oglichkeiten, aus obigen Punkten eine Bracket zu
bilden also jene, die nicht die Zellen selbst und zwei gegen"uberliegende
"`Au"senkanten"' (hier etwa $\ol{cd}$ und $\ol{ef}$) beschreiben.

\begin{figure}[htb]
$$
\beginpicture
\unitlength0.6cm
\setlinear
\setcoordinatesystem units <0.6cm,0.6cm>
\setplotarea x from -2 to 2, y from -2 to 2
\plot -1.5 -1.5 -1.5 1.5 1.5 1.5 1.5 -1.5 -1.5 -1.5 /
\plot -1.5 0 1.5 0 /
\put {a} [Br] at -1.55 0
\put {b} [Bl] at  1.55 0
\put {c} [tr] at -1.55 -1.55
\put {d} [tl] at  1.55 -1.55
\put {e} [br] at -1.55  1.55
\put {f} [bl] at  1.55  1.55
\endpicture
$$
\caption{Benachbarte 2-Zellen}
\end{figure}

Eine gezielte Bestimmung der Nichtbasen und "`sicherlich"' Basen ergibt so eine
Grundstruktur f"ur die vertr"aglichen Matroide, die letztendlich "uber
{\sc Sym2mat} zu echten Rang (d+2) Matroiden f"uhrt. Ein Algorithmus f"ur eine
solche Bestimmung l"a"st sich Abbildung \ref{setbases} entnehmen.

\begin{figure}[htb]
\begin{center}
\btab{ll}
{\bf Routine} & {\sf Setze\_sichere\_Basen}\\
          & \\
Eingabe : & CW-Komplex in Form der Zellenberandungen mit Rang r\\
Prozedur :& F"ur jedes Paar A, B von (r-2)-Zellen \\
          & \hspace*{2ex} erzeuge A$\cap$B \\
          & \hspace*{2ex} Ist $|A\cap B|\geq r-2$, also zumindest ein\\
          & \hspace*{2ex} (r-3)-Simplex im Rand, so erzeuge folgende Brackets \\
          & \hspace*{4ex} - f"ur alle Auswahlen von (r-1) 0-Zellen aus A \\
          & \hspace*{6ex} die Brackets durch Erg"anzung um einen Punkt aus B \\
          & \hspace*{4ex} - f"ur alle Auswahlen von (r-1) 0-Zellen aus B \\
          & \hspace*{6ex} die Brackets durch Erg"anzung um einen Punkt aus A \\
          & \hspace*{4ex} Sind die erzeugten Brackets nicht durch Auswahl von \\
          & \hspace*{4ex} Punkten einer (r-2)-Zelle entstanden und sind keine \\
          & \hspace*{4ex} Punkte doppelt vorhanden, so setze diese Brackets als \\
          & \hspace*{4ex} Basen in der Matroidliste auf den Wert 1 \\
          & Setze die Brackets zu Auswahlen von r Punkten einer (r-2)-Zelle \\
          & als Nichtbasis in der Matroidliste auf den Wert 0 \\
Ausgabe : & Liste, in der die Basen und Nichtbasen, sowie freien Brackets \\
          & verzeichnet sind.
\etab
\caption{Routine zur Bestimmung einer Grundstruktur f"ur die Elementlisten}
\label{setbases}
\end{center}
\end{figure}

Damit ergibt sich als Grundstruktur f"ur den Torus folgendes Bild.

{\small\tt
\begin{center}
1111110111?111????011101?11???1?1101111?1?111????1???1?111???0?\\
11?11110111????1???111?1??111???1?1?1??11??1?111111011101111011\\
\end{center}
}

womit von obigen 25 vertr"aglichen Matroiden noch sechs "ubrigbleiben, die eine
Chance besitzen, zu vertr"aglichen Rang 4 orientierten Matroiden des Torus
zu werden.

\begin{table}[htb]
\begin{center}
{\scriptsize\tt
\btab{c}
111111011101110001011101011010101101111010111110010001011110000\\
110111101110010100011111001110011010100110011111111011101111011\\
\hline
111111011111111111011101111111111101111111111110011011011111101\\
111111101111010101111111101110011110111110111111111011101111011\\
\hline
111111011111111110011101111101111101111111111001111111111101101\\
111111101111101111111101111111101111111111110111111011101111011\\
\hline
111111011111111110011101111101111101111111111111111111111101101\\
111111101111111111111111111111111111111111111111111011101111011\\
\hline
111111011111111111011101111111111101111111111001111111111111101\\
111111101111101111111101111111101111111111110111111011101111011\\
\hline
111111011111111111011101111111111101111111111111111111111111101\\
111111101111111111111111111111111111111111111111111011101111011\\
\etab
}
\end{center}
\caption{Die bez"uglich G vertr"aglichen Rang 4 Matroide zum Torus}
\label{torus4mat}
\end{table}

\clearpage
\subsection{Orientierung der erzeugten Matroide}

Im letzten Schritt sollen die erzeugten, bez"uglich 0-Zellenanzahl
und Rang eines CW-Komplexes $\C$, sowie einer vorgelegten Symmetriegruppe G
vertr"aglichen Matroide so orientiert werden, da"s weiterhin die
Vertr"aglichkeitseigenschaft bez"uglich G gew"ahrleistet ist.\\
Dazu werden wiederum zur Minimierung des Rechenaufwandes, die unter den
Symmetrien aus G reduzierten, zu erf"ullenden dreisummandigen
Gra"smann-Pl"ucker-Relationen eingesetzt. Da"s diese erf"ullt sind, stellt f"ur
die zu entstehenden orientierten Matroide nun eine notwendige und hinreichende
Bedingung dar, was sich aus dem Satz "uber die dreisummandigen Relationen von
Seite \pageref{dgpr} ergibt. Ziel ist die rekursive Erzeugung aller zu einem
vorgelegten Matroid bez"uglich G vertr"aglichen Orientierungen, sofern solche
"uberhaupt existieren.

Auf Seite \pageref{symmtest} war davon die Rede, da"s die Anzahl der
M"oglichkeiten ein orientiertes Matroid zu erzeugen davon abh"angt, wie die
Elemente g$\in$G das entsprechende Chirotop $\chi$ abbilden, ob als Rotation
auf sich selbst oder als Reflexion auf sein negatives. Gerade bei der
Bestimmung der Orbits der Brackets unter G ist dies eine wichtige Information.
Erzw"ange man n"amlich eine bestimmte Variante (etwa alle id$\neq$g$\in$G seien
Rotationen), so kann es passieren, da"s kein solches vertr"agliches orientiertes
Matroid existiert, da sich schon bei der Aufstellung der Bracketbahnen
Widerspr"uche ergeben.

F"ur den dreidimensionalen W"urfel etwa (acht Punkte im Rang vier mit den
2-Zellen-Berandungen 1264, 2586, 5378, 3147, 4687 und 1253) bei Vorlage der
Symmetriegruppe $\{id, (23)(67), (24)(57), (34)(56), (234)(576), (243)(567)\}$
und dem zugrunde liegenden Matroid

\begin{center}{\small
$1011110111111101101111101110111111111111110111011111011011111110111101$
}\end{center}

ergibt sich f"ur die Basis-Bracket [1234] folgender Orbit:
$$
\begin{array}{|ccc|c|c|c|c|}
\hline
id & \mapsto & (23)(67) & (24)(57) & (34)(56) & (234)(576) & (243)(567) \\
\hline
+[1234] & \mapsto & -[1234] & -[1234] & -[1234] & +[1234] & +[1234] \\
\hline
\end{array}
$$
Egal wie man sich entscheiden w"urde (alle id$\neq$g$\in$G Reflexionen
beziehungsweise Rotationen), der Orbit w"urde immer induzieren, da"s
$[1234]=-[1234]$, also $[1234]=0$ gelten mu"s, was einen Widerspruch zum
zugrunde liegenden Matroid darstellt.\\
Hier lie"se sich aus obigem Orbit allerdings direkt ablesen, da"s es sich bei
den Permutationen (23)(67), (24)(57) und (34)(56) um Reflexionen, sowie bei
(234)(576) und (243)(567) um Rotationen handeln mu"s. Betrachtet man dazu
die Symmetriegruppe des W"urfelchirotops (die identisch zur kombinatorischen
Symmetriegruppe ist), da"s durch geeignete Koordinatenwahl (vgl. Abb.
\ref{cube}) entstanden ist, so entspricht dies auch der "`Wirklichkeit"'.
\vskip4mm

\centerline{\small\tt
+0+++-0--+++--0++0+----0+-+0++---+-++++---0++-0-++--0+-0++++---0---+0-
}\vskip4mm

Aus dieser "Uberlegung l"a"st sich nun folgender kleiner Hilfssatz ableiten, der
besagt:
\begin{quote}
Wann immer ein Element g einer f"ur ein zu erzeugendes Chirotop geforderten
Symmetriegruppe G eine Bracket $[\lambda]\neq 0$ bis auf eine lexikographische
Sortierung $\sigma$ auf sich abbildet, so bildet in diesem Fall g das
Chirotop $\chi$ auf $\mbox{sgn}(\sigma)\chi$ ab.
\end{quote}
Der Beweis hierzu ist einfach in dem Sinne, da"s g als geforderte Symmetrie
$\chi$ sicher auf $\chi$ selbst oder $-\chi$ abbildet. Wird eine Basis-Bracket
$[a_1\ldots a_r]$ des zugrunde liegenden Matroids nun, unter Vorschaltung einer
Sortierungspermutation (eine Permutation, die die $a_1$ bis $a_r$ der Gr"o"se
nach sortiert, also auf $\{1,\ldots,r\}$ wirkt), auf sich abgebildet, so ist
hierdurch, aufgrund der Symmetrieforderung, eindeutig festgelegt, ob nach g
$\chi$ oder $-\chi$ vorliegt, da sich die Symmetrie auf alle Basis-Brackets
bezieht.$\Box$

F"ur die "ubrigen Symmetrien, die keine Bracket bis auf ihr Vorzeichen
invariant lassen, bleibt nichts anderes "ubrig, als zun"achst beide
M"oglichkeiten in Betracht zu ziehen, wie sie auf ein $\chi$ wirken k"onnten
oder eine Fixierung vorzugeben. Schaltet man vor die Orientierungsversuche zu
einem Matroid einen Test nach obigem Hilfssatz vor, so kann sich hier der
Aufwand aber schon erheblich reduzieren. In bezug auf das Torusbeispiel ergibt
sich so, da"s die Symmetrie $(47)(58)(69)$ die Bracket $[1247]$ auf $-[1247]$
abbildet und somit eine Reflexion der Chirotope $\chi$ zum Torus darstellt.
Damit bleiben $2^4$ M"oglichkeiten f"ur die anderen Elemente aus G, wie die
Bracketorbits bez"uglich ihrer Gleichheiten zu deuten sind.

Ein erster Eindruck, wie ein Matroidorientierungsprogramm aussehen k"onnte,
ergibt sich nun wie folgt.
\begin{enumerate}
\item Lege eine Automorphismengruppe G und eine vertr"agliche Matroidliste
      f"ur n Punkte im Rang r vor.
\item Bestimme die Bracketorbits unter G und pr"ufe auf Selbstabbildungen in
      obigem Sinne. Liegen solche vor, so merke den Symmetrietyp (Rot/Ref).
\item Bestimme die Orbits der dreisummandigen Gra"smann-Pl"ucker-Polynome
      unter G als Ausgangsebene f"ur ein System zur Bestimmung der zugeh"origen
      Chirotopvorzeichen.
\item Setze nacheinander alle Kombinationen f"ur die unbestimmten Symmetrietypen
      in die Bracketorbits ein, so da"s sich nun die gew"unschten
      Bracketgleichheiten ergeben und
\begin{itemize}
\item Stelle das zugeh"orige "`Gleichungs"'system durch Einsetzen der
      Bracket\-orbitrepr"asentanten (Vorzeichenwechsel ber"ucksichtigen)
      in die GPP auf.
\item Ermittle die zugeh"origen Chirotope durch Auff"ullen und Erschlie"sen von
      Vorzeichen f"ur die Brackets in den GPP.
\end{itemize}
\item Gib alle ermittelten Vorzeichenlisten der Chirotope aus.
\end{enumerate}

Das Einsetzen und Erschlie"sen von Vorzeichen in den GPP begr"undet sich
in der Deutung der Brackets als Determinanten von (r$\times$r)-Teilmatrizen
einer (n$\times$r)-Matrix, f"ur die die Gra"smann-Pl"ucker-Relationen erf"ullt
sind.\\
Stellt man das zu untersuchende Gleichungssystems auf, so zeigt sich noch eine
weitere M"oglichkeit, auftretende Symmetrietypen als unsinnig zu erkennen.
Treten n"amlich unter Ber"ucksichtigung aller Sortierungsvorzeichen und
Nullsetzungen von Bracketorbits Gleichungen vom Typ
$(+[A])(+[B]) + (+[A])(+[B]) = 0$ auf, so kann die Wahl der Art der
Abbildungsvorschrift von $\chi\mapsto\pm\chi$ als nicht richtig angesehen
werden, da nichtsingul"are Matrizen A und B diese Gleichung nicht erf"ullen
k"onnen.\\
Genauer ergibt sich f"ur die dreisummandigen Gleichungen so, da"s die
Vorzeichen der Determinanten derart gegeben sind, da"s entweder genau ein
Summand (ein Bracketprodukt) positiv oder genau einer negativ ist. Ist ein
Summand 0, so sind die anderen beiden von entgegengesetztem Vorzeichen, denn nur
so ist zu erreichen, da"s die Gleichung zu Null werden kann, wenn alle
Determinanten nichtnull sind. F"ur die Gleichungen
$$ A + (-B) + C = 0$$
ergeben sich so die Vorzeichen der Bracketprodukte A, $-$B und C zu
$$(+,-,+),~(+,-,-),~(+,+,-),~(-,-,+),~(-,+,+),~(-,+,-)$$
Solche Konstellationen gilt es nun im vorgelegten Gleichungssystem zu erzeugen.
Dazu bedienen wir uns zun"achst wieder der Gleichungen, die nach dem Ersetzen
der Brackets durch ihre Repr"asentanten unter G nur noch zwei Summanden
enthalten, die nichtnull sind, w"ahrend der dritte durch eine Nichtbasis des
zugrunde liegenden Matroids verschwunden ist. Wieder kann bei Gleichungen
vom Typ AB + AC auf das Verhalten von B und C geschlossen werden, wenn A
gesetzt wurde.

Grunds"atzlich ergeben sich aus den zweisummandigen Gleichungen folgende zu
be\-r"ucksichtigenden M"oglichkeiten:
{\small
$$
\begin{array}{|l|l|l|l|}
\multicolumn{4}{l}{A=+ \mbox{ f"ur }} \\
\hline
-(A+)~+~(BB) & -(+A)~+~(BB) & (A-)~+~(BB) & (-A)~+~(BB) \\
\hline
\multicolumn{4}{l}{B=+ \mbox{ f"ur }} \\
\hline
-(AA)~+~(+B) & -(AA)~+~(B+) & (AA)~+~(-B) & (AA)~+~(B-) \\
\hline
\multicolumn{4}{l}{A=- \mbox{ f"ur }} \\
\hline
-(A-)~+~(BB) & (A+)~+~(BB) & -(-A)~+~(BB) & (+A)~+~(BB) \\
\hline
\multicolumn{4}{l}{B=- \mbox{ f"ur }} \\
\hline
-(AA)~+~(-B) & (AA)~+~(+B) & -(AA)~+~(B-) & (AA)~+~(B+) \\
\hline
\multicolumn{4}{l}{A=B \mbox{ f"ur }} \\
\hline
 (+A)~+~(-B) &  (+A)~+~(B-) &  (-A)~+~(+B) &  (-A)~+~(B+) \\
 (A+)~+~(-B) &  (A+)~+~(B-) &  (A-)~+~(+B) &  (A-)~+~(B+) \\
-(+A)~+~(+B) & -(+A)~+~(B+) & -(-A)~+~(-B) & -(-A)~+~(B-) \\
-(A+)~+~(+B) & -(A+)~+~(B+) & -(A-)~+~(-B) & -(A-)~+~(B-) \\
\hline
\multicolumn{4}{l}{A=-B \mbox{ f"ur }} \\
\hline
-(A+)~+~(B-) &  (A+)~+~(+B) & -(A-)~+~(B+) &  (A-)~+~(-B) \\
 (+A)~+~(+B) & -(+A)~+~(B-) & -(-A)~+~(B+) &  (-A)~+~(-B) \\
 (A+)~+~(B+) & -(A+)~+~(-B) & -(A-)~+~(+B) &  (A-)~+~(B-) \\
 (+A)~+~(B+) & -(+A)~+~(-B) & -(-A)~+~(+B) &  (-A)~+~(B-) \\
\hline
\multicolumn{4}{l}{A=B \mbox{ oder } A=-B \mbox{ f"ur }} \\
\hline
-(AA)~+~(BB) & & & \\
\hline
\end{array}
$$}

Weiterhin lassen sich Vorzeichen aus dreisummandigen Gleichungen erschlie"sen,
wenn genau ein Bracketvorzeichen unbestimmt ist. Gilt dann, da"s das Vorzeichen
der beiden anderen Summanden gleich ist, so mu"s mit dem ungesetzten das
Vorzeichen des noch unbestimmten Summanden negativ dem der anderen sein.
Etwa gilt f"ur $(++)~+~(--)~+~(-A)\fol (+,+,?)$, da"s $A=+$ sein mu"s, damit
der letzte Summand negatives Vorzeichen erh"alt. Frei ist die Wahl von A, wenn
die bestimmten Summanden entgegengesetztes Vorzeichen besitzen.

Ist in einer Gleichung mehr als ein Vorzeichen unbestimmt, so k"onnen die
jeweils ungesetzen Bracketvorzeichen als Freiheitsgrade f"ur die Vorzeichenwahl
angesehen werden. Ein Setzen dieser, sowie ein Erschlie"sen weiterer Vorzeichen
nach obiger Auswahl, liefert bei rekursiver Bearbeitung des Gleichungssystems
wie bei der Matroiderzeugung eine Baumstruktur, in der alle vertr"aglichen
orientierten Matroide mit der gew"unschten Symmetrieeigenschaft erzeugt werden.

Mittels eines so aufgebauten Such- und Ersetzprogramms k"onnen nun die
zuvor bereitgestellten Matroide orientiert werden, was f"ur das Torusbeispiel
f"ur die vier orientierbaren Matroide zu den Vorzeichenlisten aus den Tabellen
\ref{torusom1} bis \ref{torusom6} f"uhrt. Hierbei ergibt sich zun"achst, das
bei Vorgabe der Symmetriegruppe G mit dem erzeugenden Element
$\sigma = (132)(495768)$ folgende Symmetrietypen zul"a"sig sind:
$$\begin{array}{c|c|c|c|c|c}
\mbox{id} & \sigma & \sigma^2 & \sigma^3 & \sigma^4 & \sigma^5 \\
\hline
+ & + & + & - & - & - \\
+ & + & - & - & - & + \\
+ & - & + & - & + & - \\
+ & - & - & - & + & + \\
\end{array}$$
Hierbei bedeutet $+$ die Wirkung von $\sigma^i$ als Rotation und $-$ als
Reflektion auf die zu entstehenden $\chi$. Besonders wichtig ist, da"s man
den gew"ahlten Symmetrietyp beim Ermitteln der vollst"andigen Vorzeichenliste
wieder ber"ucksichtigt, wenn man aus den Bracketorbitrepr"asentanten die
"ubrigen Brackets rekonstruiert. Ber"ucksichtigt man noch, da"s der Symmetrietyp
von Quadraten von Permutationen eindeutig als Rotation festgelegt ist
($\mbox{typ}(\sigma)^{2k}=1$), so bleibt hier als einzig zul"a"siger
Symmetrietyp
$$\begin{array}{c|c|c|c|c|c}
\mbox{id} & \sigma & \sigma^2 & \sigma^3 & \sigma^4 & \sigma^5 \\
\hline
+ & - & + & - & + & -
\end{array}$$
"ubrig. Was bei kleinen vorgelegten Gruppen, noch leicht von Hand zu l"osen
ist, ist bei gr"o"seren Gruppen, deren Elemente zusammengesetzt sind, zwar
auch, aber mit ungleich h"oherem Aufwand zu bewerkstelligen, was so zur Zeit
noch nicht implementiert ist.

\begin{table}[htb]
Gew"ahltes Matroid:
\begin{center}
{\scriptsize\tt
\begin{tabular}{c}
111111011101110001011101011010101101111010111110010001011110000\\
110111101110010100011111001110011010100110011111111011101111011
\end{tabular}}\end{center}\vskip2mm

Vorgelegte Automorphismengruppe:
{\small
$$\{\mbox{id},(132)(495768),(123)(456)(789),(47)(58)(69),(132)(465)(798),
   (123)(486759)\}$$}\vskip2mm

Erzeugte Chirotope:
\begin{center}
{\scriptsize\tt\begin{tabular}{c}
+++---0+--0+--000+0---0+0++0-0+0++0-+--0+0++--+00+000-0---+0000\\
--0--++0+--00-0+000--+++00+-+00++0+0-00--00+--+----0++-0+---0--\\
\hline
+++---0-++0-++000-0+++0-0--0+0-0--0++--0+0++--+00+000-0-++-0000\\
++0++--0+--00-0+000--+++00+-+00++0+0-00--00+--+-+++0--+0-+++0++\\
\hline
---+++0+--0+--000+0---0+0++0-0+0++0--++0-0--++-00-000+0+--+0000\\
--0--++0-++00+0-000++---00-+-00--0-0+00++00-++-+---0++-0+---0--\\
\hline
---+++0-++0-++000-0+++0-0--0+0-0--0+-++0-0--++-00-000+0+++-0000\\
++0++--0-++00+0-000++---00-+-00--0-0+00++00-++-++++0--+0-+++0++\\
\end{tabular}}
\end{center}
\caption{\label{torusom1} Chirotope zum ersten Torusmatroid}
\end{table}

\begin{table}[htb]
\begin{center}
{\scriptsize\tt
\begin{tabular}{c}
111111011111111110011101111101111101111111111111111111111101101\\
111111101111111111111111111111111111111111111111111011101111011\\
\hline\hline
+++---0+---+-----00---0+++++0+++++0-+---++++--++++--+-----0--0-\\
-----++0+------+++---++++++-+++++++--+-----+--+----0++-0+---0--\\
\hline
+++---0-+++-+++++00+++0-----0-----0++---++++--++++--+---++0++0+\\
+++++--0+------+++---++++++-+++++++--+-----+--+-+++0--+0-+++0++\\
\hline
---+++0+---+-----00---0+++++0+++++0--+++----++----++-+++--0--0-\\
-----++0-++++++---+++------+-------++-+++++-++-+---0++-0+---0--\\
\hline
---+++0-+++-+++++00+++0-----0-----0+-+++----++----++-+++++0++0+\\
+++++--0-++++++---+++------+-------++-+++++-++-++++0--+0-+++0++\\
\end{tabular}}
\end{center}
\caption{\label{torusom4} Chirotope zum vierten Torusmatroid}
\end{table}

\begin{table}[htb]
\begin{center}
{\scriptsize\tt
\begin{tabular}{c}
111111011111111111011101111111111101111111111001111111111111101\\
111111101111101111111101111111101111111111110111111011101111011\\
\hline\hline
+++---0+--++--++++0---0+-++---+-++0-+--++-++-00--+++--+---+++0+\\
--+--++0+--++0++--+--+0+--+-+--0+-++--+--+++0-+-+++0--+0-+++0++\\
\hline
+++---0-++--++----0+++0-+--+++-+--0++--++-++-00--+++--+-++---0-\\
++-++--0+--++0++--+--+0+--+-+--0+-++--+--+++0-+----0++-0+---0--\\
\hline
---+++0+--++--++++0---0+-++---+-++0--++--+--+00++---++-+--+++0+\\
--+--++0-++--0--++-++-0-++-+-++0-+--++-++---0+-++++0--+0-+++0++\\
\hline
---+++0-++--++----0+++0-+--+++-+--0+-++--+--+00++---++-+++---0-\\
++-++--0-++--0--++-++-0-++-+-++0-+--++-++---0+-+---0++-0+---0--\\
\end{tabular}}
\end{center}
\caption{\label{torusom5} Chirotope zum f"unften Torusmatroid}
\end{table}

\begin{table}[htb]
\begin{center}
{\scriptsize\tt
\begin{tabular}{c}
111111011111111111011101111111111101111111111111111111111111101\\
111111101111111111111111111111111111111111111111111011101111011\\
\hline\hline
+++---0+++++++++++0---0---+------+0-++++----++----++-++---+++0+\\
+++++++0+++++++---+++-------+------++-+++++-++--+++0--+0-+++0++\\
\hline
+++---0+---+-----+0---0+++++-+++++0-+---++++--++++--+-----+--0-\\
-----++0+------+++---++++++-+++++++--+-----+--+----0++-0+---0--\\
\hline
+++---0+---+------0---0+++++++++++0-+---++++--++++--+--------0-\\
-----++0+------+++---++++++-+++++++--+-----+--+----0++-0+---0--\\
\hline
+++---0+--++--++++0---0+-++---+-++0-+--++-++-+---+++--+---+++0+\\
--+--++0+--+++++--+--+-+--+-+---+-++--+--++++-+-+++0--+0-+++0++\\
\hline
+++---0+--++--++++0---0+-++---+-++0-+--++-++--+--+++--+---+++0+\\
--+--++0+--++-++--+--+++--+-+--++-++--+--+++--+-+++0--+0-+++0++\\
\hline
+++---0+--++--++++0---0+-++---+-++0-+--++-++--+--+++--+---+++0+\\
--+--++0+--++-++--+--+++--+-+--++-++--+--+++--+----0++-0+---0--\\
\hline
+++---0-+++-++++++0+++0-----------0++---++++--++++--+---+++++0+\\
+++++--0+------+++---++++++-+++++++--+-----+--+-+++0--+0-+++0++\\
\hline
+++---0-+++-+++++-0+++0-----+-----0++---++++--++++--+---++-++0+\\
+++++--0+------+++---++++++-+++++++--+-----+--+-+++0--+0-+++0++\\
\hline
+++---0-++--++----0+++0-+--+++-+--0++--++-++-+---+++--+-++---0-\\
++-++--0+--+++++--+--+-+--+-+---+-++--+--++++-+----0++-0+---0--\\
\hline
+++---0-++--++----0+++0-+--+++-+--0++--++-++--+--+++--+-++---0-\\
++-++--0+--++-++--+--+++--+-+--++-++--+--+++--+-+++0--+0-+++0++\\
\hline
+++---0-++--++----0+++0-+--+++-+--0++--++-++--+--+++--+-++---0-\\
++-++--0+--++-++--+--+++--+-+--++-++--+--+++--+----0++-0+---0--\\
\hline
+++---0-----------0+++0+++-++++++-0+++++----++----++-++-++---0-\\
-------0+++++++---+++-------+------++-+++++-++-----0++-0+---0--\\
\hline
---+++0+++++++++++0---0---+------+0-----++++--++++--+--+--+++0+\\
+++++++0-------+++---+++++++-++++++--+-----+--+++++0--+0-+++0++\\
\hline
---+++0+--++--++++0---0+-++---+-++0--++--+--++-++---++-+--+++0+\\
--+--++0-++--+--++-++---++-+-++--+--++-++---++-++++0--+0-+++0++\\
\hline
---+++0+--++--++++0---0+-++---+-++0--++--+--++-++---++-+--+++0+\\
--+--++0-++--+--++-++---++-+-++--+--++-++---++-+---0++-0+---0--\\
\hline
---+++0+--++--++++0---0+-++---+-++0--++--+--+-+++---++-+--+++0+\\
--+--++0-++-----++-++-+-++-+-+++-+--++-++----+-++++0--+0-+++0++\\
\hline
---+++0+---+-----+0---0+++++-+++++0--+++----++----++-+++--+--0-\\
-----++0-++++++---+++------+-------++-+++++-++-+---0++-0+---0--\\
\hline
---+++0+---+------0---0+++++++++++0--+++----++----++-+++-----0-\\
-----++0-++++++---+++------+-------++-+++++-++-+---0++-0+---0--\\
\hline
---+++0-++--++----0+++0-+--+++-+--0+-++--+--++-++---++-+++---0-\\
++-++--0-++--+--++-++---++-+-++--+--++-++---++-++++0--+0-+++0++\\
\hline
---+++0-++--++----0+++0-+--+++-+--0+-++--+--++-++---++-+++---0-\\
++-++--0-++--+--++-++---++-+-++--+--++-++---++-+---0++-0+---0--\\
\hline
---+++0-++--++----0+++0-+--+++-+--0+-++--+--+-+++---++-+++---0-\\
++-++--0-++-----++-++-+-++-+-+++-+--++-++----+-+---0++-0+---0--\\
\hline
---+++0-+++-++++++0+++0-----------0+-+++----++----++-++++++++0+\\
+++++--0-++++++---+++------+-------++-+++++-++-++++0--+0-+++0++\\
\hline
---+++0-+++-+++++-0+++0-----+-----0+-+++----++----++-+++++-++0+\\
+++++--0-++++++---+++------+-------++-+++++-++-++++0--+0-+++0++\\
\hline
---+++0-----------0+++0+++-++++++-0+----++++--++++--+--+++---0-\\
-------0-------+++---+++++++-++++++--+-----+--++---0++-0+---0--\\
\end{tabular}}
\end{center}
\caption{\label{torusom6} Chirotope zum sechsten Torusmatroid}
\end{table}

\clearpage
\section{Zusammenfassung der Vorgehensweise}

Da nun alle Schritte abgearbeitet wurden, die zu einem vorgelegten CW-Komplex
(angegebenen Aufbaus) "uber Symmetrie(unter)gruppen vertr"agliche orientierte
Matroide in Form von Chirotopen erzeugen, ist das Ziel dieser Arbeit erreicht.

Zusammenfassend kann die "`vertr"agliche Chirotoperzeugung"' nun wie folgt
beschrieben werden:

\begin{itemize}
\item Gegeben sei ein CW-Komplex $\C$, mit (im Fall eines 2-Komplexes)
      polygonal berandeten 2-Zellen.
\item Bestimme zu $\C$ dessen kombinatorische Symmetriegruppe Aut($\C$).
\item Bestimme eine Grundstruktur f"ur die vertr"aglichen Matroide.
\item W"ahle aus Aut($\C$) eine "`geeignete"' Untergruppe aus und lege
      diese als feste (Untergruppe einer) Symmetriegruppe G vertr"aglicher
      orientierter Matroide vor.
\item Erzeuge rekursiv bez"uglich G vertr"agliche Matroide $\ul{\cal M}$
      mit der gew"unschten Grundstruktur.
\item Versuche jedes so entstandene vertr"agliche Matroid $\ul{\cal M}$ zu
      orientieren, so da"s $\ul{\cal M}$ zu bez"uglich G vertr"aglichen
      orientierten Matroiden $\cal M$ in Form von Chirotopen $\chi$ wird.
\item Nach der Bearbeitung liegen nun zu $\C$ im Existenzfall alle
      bez"uglich $G\leq \mbox{Aut}(\C)$ vertr"aglichen Chirotope $\chi$ vor,
      deren zugrunde liegendes Matroid die gew"unschte Grundstruktur besitzt
      und eine Symmetriegruppe aufweisen, die zumindest G enth"alt.
\end{itemize}

In dem nun anschlie"senden letzten Kapitel sollen dazu noch einige Beispiele
bearbeitet werden.
