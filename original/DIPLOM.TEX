\documentstyle[12pt,german,makeidx]{bk}%{book}
\newfont{\athene}{athene scaled 1000}
\pagestyle{headings}

\topmargin-1.04cm
\textheight23.5cm
\textwidth 15cm
\oddsidemargin 0.206cm
\evensidemargin 0.206cm
\marginparwidth 2.5cm
%\renewcommand{\baselinestretch}{1.2}
\newcommand{\bn}{\begin{enumerate}}
\newcommand{\en}{\end{enumerate}}
\newcommand{\btab}{\begin{tabular}}
\newcommand{\etab}{\end{tabular}}
\newcommand{\bcent}{\begin{center}}
\newcommand{\ecent}{\end{center}}
\newcommand{\R}{\mbox{$\rm I\!R$}}
\newcommand{\N}{\mbox{$\rm I\!N$}}
\newcommand{\K}{\mbox{$\rm I\!K$}}
\newcommand{\E}{\mbox{$\rm I\!E$}}
\renewcommand{\P}{\mbox{$\rm I\!P$}}
\renewcommand{\S}{{\bf S}}
\newcommand{\Z}{\mbox{$\rm Z\!\!Z$}}
\newcommand{\B}{{\cal B}}
\newcommand{\C}{{\cal C}}
\newcommand{\cS}{{\cal S}}
\newcommand{\T}{{\cal T}}
\newcommand{\Top}{{\cal O}}
\newcommand{\fol}{\Rightarrow}
\newcommand{\Cup}{\bigcup}
\newcommand{\Cap}{\bigcap}
\newcommand{\ftnt}{\footnotesize}
\newcommand{\scsi}{\scriptsize}
\newcommand{\idx}{\index}
\newcommand{\follows}{\Longrightarrow}
\newcommand{\ol}{\overline}
\newcommand{\ul}{\underline}
\newtheorem{satz}{Satz}[section]
\newtheorem{lemma}{Lemma}[section]
\parindent0em
\parskip1ex
\input prepictex
\input pictex
\input postpictex
\makeindex

\begin{document}

\begin{titlepage}
\begin{center}
\vspace*{5cm}

{\LARGE\bf Ein Algorithmus zum Testen von CW-Komplexen auf euklidische
           Einbettbarkeit}

\vskip2cm

{\Large Diplomarbeit}

von

{\large\bf Markus Gebhard}

am

Fachbereich Mathematik

der\vskip1.5cm

\begin{minipage}{2cm}
{\athene A}
\end{minipage}

Technischen Hochschule Darmstadt
\vskip1cm

Vorgelegt bei

{\large Herrn Prof. Dr. J. Bokowski}
\vskip1.5cm

\today
\end{center}
\end{titlepage}
\newpage
\tableofcontents
\newpage
\thispagestyle{plain}
\section*{Einleitung}

In dieser Arbeit sind "Uberlegungen aufgezeigt, wie sich zu beliebig
vorgegebenen kombinatorischen Komplexen zugeh"orige orientierte Matroide
finden lassen, mittels derer gegebenenfalls die Einbettbarkeit der
gegebenen Komplexe in den euklidischen Raum entschieden werden kann.\\
Dazu wird nach Bereitstellung der topologischen Grundlagen von CW-Komplexen, die
Vertr"aglichkeit dieser mit orientierten Matroide untersucht. Als Zugang wird
dabei eine Kopplung von CW-Komplexen "uber deren kombinatorische
Symmetriegruppen mit der Struktur von Chirotopen erreicht.

Die Vorstellung eines Algorithmus soll es dann erm"oglichen, zu einem gegebenen
kombinatorischen Komplex "uber dessen Symmetrien zun"achst zu
"`symmetriever\-tr"ag\-lichen"' Matroiden und durch deren Orientierung zu
entsprechenden orientierten Matroiden zu gelangen, die m"oglicherweise
Matrizen und damit Punktkonfigurationen induzieren, die einen vorgelegten
Komplex im euklidischen Raum metrisch beschreibbar machen.

Entgegen den in fr"uheren Arbeiten oftmals untersuchten simplizialen Komplexen
soll dazu eine m"oglichst generelle Verbindung zwischen allgemeinen CW-Komplexen
und orientierten Matroiden in ihrer Form als Chirotope geschaffen werden.

Fettgedruckte W"orter bezeichnen Definitionen und wichtige Passagen, die
Einsch"ube in kleingedruckter Schrift sollen zus"atzliche Informationen
bereitstellen, die als bekannt vorausgesetzte Definitionen betreffen oder
aber auch sonst f"ur das Verst"andnis hilfreich sind.

\section*{Danksagung}

Als demjenigen, der mich in die Welt der orientierten Matroide eingef"uhrt
und auch sonst viele interessante Aspekte der angewandten Geometrie vorgestellt
hat, m"ochte ich Herrn Prof. Dr. J"urgen Bokowski meinen besonderen Dank
aussprechen. Auch seinen Mitarbeitern, Peter Schuchert und Torsten-Karl Strempel,
geb"uhrt mein Dank, da sie bei Fragen und Unklarheiten immer zu einer
Diskussion und L"osung beigetragen haben. Meinen Eltern danke ich, da"s sie
mir mein Studium erm"oglicht haben, meiner Freundin Sylke f"ur ihre Geduld,
wenn alles etwas hektisch wurde. Weiterhin danke ich Torsten Trothe f"ur die
Tips, die die Lauff"ahigkeit der entstandenen C-Programme wesentlich erh"ohten.
Gewidmet sei diese Arbeit meinem Mathematiklehrer Robert Morgenstern, der
durch seine Art ein Mathematikstudium erst lohnenswert erscheinen lie"s.

\chapter{Grundlagen}

\section{Grundbegriffe der Topologie}

Um Komplexe allgemein und CW-Komplexe im besonderen definieren zu können,
ist es nötig, einige grundlegende Definitionen und Eigenschaften
topologischer Räume bereitzustellen, die nicht nur hier, sondern in vielen
anderen Bereichen der Mathematik zum "`Grundvokabular"' gehören.

\subsection{Topologische Räume}

Ist X eine beliebige Menge, so kann man auf dieser die Potenzmenge
${\cal P}(X)$, die Menge aller Teilmengen von X, betrachten. Wählt man aus
${\cal P}(X)$ eine Teilmenge $\T$ aus, deren Elemente die drei Bedingungen

\bcent
\btab{ll}
1. & $\emptyset\in\T$ und $X\in\T$\\
2. & $T_1,T_2\in\T~\fol~T_1\cap T_2\in\T$\\
3. & $T_i\in\T$ für alle i aus einer Indexmenge I
     $\fol~\Cup\limits_{i\in I} T_i~\in~\T$\\
\etab
\ecent

erfüllen, so nennt man $\T$ eine {\bf Topologie} auf X\idx{Topologie auf X}
und Elemente $T\in\T$ {\bf offene Mengen}\idx{offene Mengen} des {\bf
topologischen Raumes}\idx{topologischer Raum} (X,$\T$), des Paares X zusammen
mit der Topologie $\T$. Die Komplemente offener Mengen nennt man {\bf
abgeschlossen}\idx{abgeschlossen}.\\
{\scsi
Ebenso könnte man eine Topologie $\T$ auf X auch mittels abgeschlossener Mengen
definieren. In diesem Fall wäre dann das Enthaltensein endlicher Vereinigungen
und beliebiger Durchschnitte abgeschlossener Mengen aus $\T$ zu fordern,
was der Komplementbildung obiger Bedingungen entspricht.
}\\
Ist Y eine Teilmenge eines topologischen Raumes (X,$\T$), so wird diese
selbst zu einem topologischen Raum, wenn man auf ihr die Topologie von X
induziert, das heißt als offene Mengen der Topologie $\T|Y$ die Mengen
$\{T\cap Y~|~T\in\T\}$ wählt. $\T|Y$ nennt man dann die von X auf Y
{\bf induzierte Topologie}, die oft auch als {\bf Spurtopologie} auf
Y\idx{Spurtopologie}\idx{induzierte Topologie} bezeichnet wird.

Als Beispiele für Topologien können etwa die durch Metriken
d metrischer Räume (X,d) induzierten Topologien angeführt werden.\\
{\scsi
Zur Erinnerung hier die Definition einer Metrik und eines metrischen Raumes.
Sei dazu X eine Menge. Eine Funktion $d:X\times X\to\R$ heißt Metrik, wenn\\
1. d(x,y) = d(y,x) $\geq$ 0\\
2. d(x,y) = 0 $\iff$ x = y\\
3. d(x,z) $\leq$ d(x,y) + d(y,z),\\
gilt. Das Paar (X,d) heißt dann ein metrischer Raum. Betrachtet man etwa
$(\R,\|\cdot\|_2)$, so stellen metrische Räume oft verwendete Spezialfälle
topologischer Räume dar.
}\\
Die offenen Mengen dieser durch Metriken induzierten Topologien sind gegeben
durch
$$\T_d=\{U\subseteq X:(\forall~p\in U) (\exists~\varepsilon > 0)
\mbox{ mit } \{y:d(y,p)<\varepsilon\}\subseteq U\}.$$
Sei $\|\cdot\|:V\to\R$ eine Abbildung eines Vektorraumes V über einem Körper
$\K$ in die reellen Zahlen. $\|\cdot\|$ heißt eine {\bf Norm}\idx{Norm} auf V,
wenn für alle $v,w\in V$ und $k\in\K$ $\|v\|$ dann und nur dann gleich 0 ist,
wenn $v=0$ gilt, $\|k\cdot v\|$ gleich $|k|\cdot\|v\|$ ist, sowie $\|v+w\|\leq
\|v\|+\|w\|$ gilt. (V,$\|\cdot\|$) heißt dann ein {\bf normierter Raum}.
\idx{normierter Raum} Normierte Räume sind metrische Räume, wenn man für
die d(x,y) die Norm der Differenz $\|x-y\|$ von x und y setzt.

Der $\R^n$ ist bezüglich der durch die Euklidische Norm $\|x\|_2 :=
\sqrt{\sum_{i=1}^n x_i^2}$ und der so induzierten Metrik gegebenen
offenen Mengen ein topologischer Raum, der auch als {\bf Euklidischer Raum}
$\E^n$ bezeichnet wird. Oft spricht man hier auch von der "`üblichen"'
Topologie auf $\R^n$. (Im folgenden sei $\R$ und $\E$ synonym verwendet, da immer
der Euklidische Raum gemeint sein wird.)

Spielt es keine Rolle, wie die Topologie eines topologischen Raumes
geartet ist, werden Aussagen über beliebige topologische Räume gemacht
oder ist es klar, welche Topologie (die "`übliche"') gemeint ist, so spricht
man oft der Einfachheit halber von einem topologischen Raum X, ohne
zusätzliche Angabe von $\T$.

Eine Abbildung $f:X\to Y$ zwischen topologischen Räumen X und Y heißt
{\bf stetig},\idx{stetig} wenn die Urbilder offener Mengen in Y offen in X sind.
Dies bedeutet, daß es zu jeder offenen Menge V in Y eine offene Menge U in X
so gibt, daß f(U) = V gilt.
Ist $f:X\to Y$ eine bijektive Abbildung zwischen X und Y, die selbst
und deren Umkehrabbildung $f^{-1}:Y\to X$ stetig sind, so heißt f ein
{\bf Homöomorphismus}\idx{Hom\öomorphismus}. Gibt es zwischen zwei
topologischen Räume X und Y solch einen Homöomorphismus, so heißen X und Y
{\bf homöomorph}\idx{hom\öomorph}.
Als wichtiges Beispiel sei angeführt, daß die n-dimensionale offene Vollkugel
(in einem normierten Raum) $\stackrel{\circ}{B^n} := \{x~:~\|x\|< 1\}$
homöomorph dem $\R^n$ ist (als Abbildung wähle man etwa
$\phi:\stackrel{\circ}{B^n}\to\R^n$ mit $\phi(x)=\frac{1}{1-\|x\|}x$). Weiter
ist die (n$-$1)-Sphäre $\S^{n-1}:=\{x~:~\|x\|=1\}$, der Rand der Vollkugel
$B^n$, von der man einen Punkt wegläßt, sie "`punktiert"', homöomorph dem
$\R^{n-1}$. (Man nutze die stereographische Projektion
$\phi:\S^{n-1}-\{e_n\}\to\R^{n-1}$ mit
$y=(\frac{x_1}{1-x_n},\ldots,\frac{x_{n-1}}{1-x_n})$).

Bestimmte Teilmengen topologischer Räume können nun durch folgende Begriffe
charakterisiert werden. Ist M eine beliebige Teilmenge eines
topologischen Raumes (X,$\T$), so heißt eine weitere Teilmenge $U\subset X$
eine {\bf Umgebung}\idx{Umgebung} von M, falls es eine offene Menge T in $\T$
gibt, so daß $M\subset T\subset U$ gilt.
Ein Punkt p aus M ist {\bf innerer Punkt}\idx{innerer Punkt} von M, falls M
Umgebung von  $\{$p$\}$ ist, p ist {\bf Berührungspunkt}\idx{Ber\ührungspunkt}
von M, falls jede Umgebung von p nichtleeren Schnitt mit M hat und p ist ein
{\bf Häufungspunkt}\idx{H\äufungspunkt}, falls p Berührungspunkt von
M$-\{$p$\}$ (M ohne den Punkt p) ist. Ein Punkt p heißt {\bf Randpunkt}
\idx{Randpunkt} von M, wenn er Berührungspunkt von M und X$-$M ist.
Aus der Definition einer Topologie mittels abgeschlossener Mengen
folgt, daß jede offene Menge $T\in\T$ in einer kleinsten abgeschlossenen
Menge, dem {\bf Abschluß} $\ol{T}$\idx{Abschlu\3} von T, enthalten ist. Dieser
Abschluß oder auch diese {\bf abgeschlossene Hülle} von T ist der Durchschnitt
aller abgeschlossenen Mengen, die T enthalten.

Eine Teilmenge eines topologischen Raumes soll {\bf kompakt} heißen, wenn sie
das Überdeckungskriterium von Heine-Borel erfüllt. Dieses besagt, daß
eine Menge K in einem topologischen Raum X genau dann kompakt\idx{kompakt}
ist, wenn jede offene Überdeckung (jede Vereinigung offener Mengen aus X, die
eine Obermenge von K ist) die Auswahl einer endlichen Teilüberdeckung
(also eine Überdeckung mit nur endlich vielen offenen Mengen) zuläßt.
{\bf Konvex} heißt eine Teilmenge K eines topologischen Vektorraumes (eines
Vektorraumes mit Topologie, bezüglich der Addition und Multiplikation mit
Skalaren stetig sind)\idx{konvex}, wenn für je zwei Punkte a und b aus K
die Verbindungsstrecke $\{x=\lambda a +(1-\lambda) b:\lambda\in [0,1]\}$
der beiden auch in K liegt. Eine nichtleere Teilmenge S einer Teilmenge K
eines Vektorraumes heißt {\bf Extremmenge} von K\idx{Extremmenge}
falls für alle $x,y\in K,~0<t<1$, und $(1-t)x+ty\in S$ die Punkte x und y aus
S stammen. Die {\bf Extrempunkte}\idx{Extrempunkt} von K sind die einpunktigen
Extremmengen von K (vgl.\cite{Ru:91}, Seite 74).
Eine kompakte, konvexe Teilmenge K des $\R^d$ heißt nun {\bf Polytop} oder
Polyedermenge, falls die Menge ihrer Extrempunkte endlich ist.
(vgl.\cite{Gr:67}, S.31)\idx{Polytop}

\subsection{Produkttopologie}

Betrachtet man eine Mengenfamilie $\{X_j:j\in J\}$, eine Menge, die zu jedem
Element j einer Indexmenge J eine Menge X$_j$ enthält, so ist das {\bf
kartesische Produkt}\idx{kartesisches Produkt} der X$_j$ die Menge aller
Auswahlfunktionen $f:J\to\Cup\limits_{j\in J} X_j$ mit $f(j)\in X_j$,
geschrieben als
$$
\prod\limits_{j\in J}X_j:=
\left\{f~|~f:J\to\Cup\limits_{j\in J} X_j,f(j)\in X_j\right\}.
$$
Für $\{f(j):j\in J\}$ schreibt man kurz $(x_j)_{j\in J}$, was dem Sachverhalt
der Auswahl eines Elements aus allen $X_j$ wohl gerechter wird.
Daß ein unendliches kartesisches Produkt nichtleer ist, ist nicht ohne weiteres
einsichtig und wird durch das Auswahlaxiom sichergestellt, welches besagt:
\begin{quote}
{\bf
Für jede Mengenfamilie $\{X_j:j\in J\}$ nichtleerer Mengen ist das Produkt
$\prod\limits_{j\in J}X_j$ nichtleer.
}
\end{quote}
Bei Betrachtung endlicher Produkte kommt dieses Axiom nicht zum tragen, ist
aber bei der Übertragung auf den allgemeinen Fall unendlicher Mengenfamilien
unerläßlich, weswegen es hier bei der Darstellung allgemeiner topologischer
Grundlagen auch nicht fehlen soll.

Wählt man die $X_j$ als topologische Räume, so ist mit den Topologien
$\T_j$ der $X_j$ auch auf dem Produkt eine Topologie, die
{\bf Produkttopologie}, \idx{Produkttopologie} gegeben. Setzt man
$$\B:=\left\{\prod\limits_{j\in J}U_j~:~U_j\in\T_j, U_j=X_j,\mbox{ bis auf
endlich viele Ausnahmen } j\in J\right\},$$
so ist die Produkttopologie gerade $\T := \left\{\Cup B~:~B\in\B\right\}.$
$\B$ heißt eine {\bf Basis}\idx{Basis einer Topologie} der Topologie, da jede
offene Menge als Vereinigung von Mengen aus $\B$ dargestellt werden kann. Die
Formulierung "`bis auf endlich viele Ausnahmen"' ist dabei eine Folge des
Auswahlaxioms.

\subsection{Teilweise geordnete Mengen und Verbände}

Da gerade vom Auswahlaxiom die Rede war, soll an dieser Stelle die Terminologie
geordneter Mengen dargestellt werden, da auf Komplexen eine natürliche Ordnung
existiert, diese Mengen aber auch sonst eine hilfreiche Ergänzung bieten.

Eine Relation "`$\leq$"' auf einer Menge X heißt {\bf Partialordnung} oder
einfach nur Ordnung, wenn sie transitiv ($x\leq y, y\leq z\fol x\leq z~\forall
x,y,z\in X$), reflexiv ($x\leq x~\forall x\in X$) und antisymmetrisch ($x\leq y,
y\leq x\fol x = y~\forall x,y\in X$) ist. Die Menge X heißt zusammen mit
$\leq$ eine {\bf teilweise geordnete Menge} oder {\bf Poset} (partial ordered
set) (X,$\leq$).\idx{Ordnung}\idx{Poset}\idx{teilweise geordnete Menge}
(X,$\leq$) heißt {\bf totalgeordnet},\idx{totalgeordnet} wenn für alle
x,y$\in$X allemal x$\leq$y oder y$\leq$x gilt. Jede totalgeordnete Teilmenge
einer teilweise geordneten Menge heißt {\bf Kette} oder Turm.\idx{Kette}
\idx{Turm} Ist jede Kette in (X,$\leq$) nach oben beschränkt, so heißt
(X,$\leq$) auch induktiv geordnet.
Hier kann nun ein zum Auswahlaxiom äquivalentes Lemma, das Lemma von Zorn
angeführt werden, das den gegenwärtigen Bezug zu den topologischen Räumen
darstellt.\idx{Lemma von Zorn}
\begin{quote}
{\bf Jede induktiv geordnete Menge besitzt maximale Elemente.}
\end{quote}
Besitzt eine teilweise geordnete Menge (X,$\leq$) ein eindeutiges minimales
Element, so sei dieses mit \^0 bezeichnet, analog ein eindeutiges maximales
Element mit \^1. Gibt es in einer teilweise geordneten Menge (X,$\leq$) diese
Elemente, so heißt (X,$\leq$) {\bf beschränkt}. Gilt für zwei Elemente x
und y aus einer beschränkten teilweise geordneten Menge X, daß x$<$y ist, und
existiert kein weiteres Element z in X, welches zwischen x und y liegt
(x$<$z$<$y), so bezeichnet man y als {\bf Decke}\idx{Decke} von x
beziehungsweise x als {\bf Kodecke} von y. Die Decken von \^0 heißen dann
{\bf Atome}\idx{Atome} und die Kodecken von \^1 {\bf Koatome} von X.
Besitzen alle maximalen Ketten $x_0<x_1<\ldots<x_l$ die gleiche Länge $\ell$,
so heißt (X,$\leq$) rein und $\ell$ ist die Länge
von X. In diesem Fall ist der Rang $\rho(x)$ eines x$\in$X die Länge der
geordneten Teilmenge $X_{\leq x}:=\{y\in X:y\leq x\}$. Wie diese Teilmengen
teilweise geordneter Mengen, so können auch {\bf Ordnungsintervalle}
\idx{Ordnungsintervalle} $[x,y]\subseteq X$ mit $x,y\in X$ als Mengen
$\{z\in X:x\leq z\leq y\}$ definiert werden, analog offene Ordnungsintervalle
$(x,y) := \{z\in X:x<z<y\}$ im üblichen Sinne.

Eine teilweise geordnete Menge (X,$\leq$) heißt {\bf Verband},\idx{Verband}
wenn zu allen Paaren x,y$\in$ X kleinste obere Schranken $x\vee y$ (Joins) und
größte untere Schranken $x\wedge y$ (Meets) existieren. In einem endlichen
Verband L existieren Joins und Meets für beliebige Teilmengen von L, deshalb
ist L beschränkt und es ist \^0 = $\wedge$L und \^1 = $\vee$L.
Kürzer kann man sagen, daß eine teilweise geordnete Menge L genau dann ein
Verband ist, wenn \^0 existiert und wenn für alle Paare x,y$\in$L die Joins
$x\vee y$ existieren. 

\subsection{Quotientenräume}

Nach diesem Abstecher nun zurück zu den topologischen Räumen. Hier ist
ein weiterer wichtiger Begriff, gerade im Zusammenhang mit Komplexen und
darauf induzierten Topologien, der des Quotientenraumes, welcher im folgenden
eingeführt werden soll.

Sind X und Y Mengen, so kann mittels eines Tricks die Summe oder {\bf disjunkte
Vereinigung}\idx{Summe von Mengen}\idx{disjunkte Vereinigung} der beiden
Mengen definiert werden, wobei X und Y als Teilmengen erhalten bleiben
(Bei der "`normalen"' Vereinigung ist ja $X\cup X = X$).
Diese Summe $X+Y$ wird so als Vereinigung $X\times \{0\} \cup Y\times \{1\}$
erklärt. Analog ist die {\bf Summe topologischer Räume} zu bilden,
wenn die offenen Mengen der Topologie der Summe zu
$\{U+V|U\in\cS,V\in\T\}$ gewählt werden, wobei $\cS$ die Topologie auf X und
$\T$ die Topologie auf Y bezeichnet.

Sei $\sim$ eine Äquivalenzrelation (reflexiv (x $\sim$ x), symmetrisch
(x $\sim$ y $\iff$ y $\sim$ x) und transitiv (x $\sim$ y, y $\sim$ z
$\fol$ x $\sim$ z)) auf einem topologischen Raum $(X,\T)$,
$\tilde{x}:=\{y\in X:x\sim y\}$ eine und $\tilde{X}:=X/_\sim =
\{\tilde{x}|x\in X\}$ die Menge aller $\sim-$Äquivalenzklassen. Die Abbildung
$\pi:X\to\tilde{X}$, die jedes Element x aus X auf seine Äquivalenzklasse
abbildet heißt {\bf Quotientenabbildung}\idx{Quotientenabbildung} des
topologischen Raumes (X,$\T$) in den {\bf Quotientenraum}
$(\tilde{X},\tilde{\T})$ \idx{Quotientenraum} von X modulo $\sim$ mit der
{\bf Quotiententopologie} \idx{Quotiententopologie} $\tilde{\T} :=
\{\tilde{\T}\subset\tilde{X}:\pi^{-1}(\tilde{\T})\in\T\}$.
Die Quotientenabbildung $\pi$ ist stetig und $\tilde{\T}$ ist die feinste
Topologie bezüglich der $\pi$ stetig ist.\\
{\scsi
Seien $\cS$ und $\T$ zwei Topologien auf X. $\cS$ heißt feiner als $\T$,
wenn sie mehr offene Mengen enthält, also $\cS\supseteq\T$ gilt. Analog
heißt $\T$ gröber als $\cS$.
}\\
Ein Beispiel hierfür ist im Abschnitt zu den Abbildungen von CW-Komplexen
angegeben. (siehe Abb.\ref{inzidenz})

Da allein mit Angabe einer Topologie zu einer Menge X noch nicht viel
über Eigenschaften der Punkte in X ausgesagt werden kann, fordert man
in Form von Axiomen zusätzliche "`Trennungseigenschaften"'.

In dieser Arbeit wird folgendes Axiom, das zweite Trennungsaxiom oder
auch {\bf Hausdorffbedingung}, völlig genügen, weshalb auf die Auflistung
der weiteren Axiome verzichtet werden soll (vgl. dazu etwa \cite{Os:92},
Seite 50).\idx{Hausdorff-Axiom}
\begin{quote}
(T$_2$) Zu je zwei Punkten x und y in X gibt es disjunkte Umgebungen.
\end{quote}
Ein topologischer Raum, der dieses Axiom erfüllt wird auch als
{\bf Hausdorff-Raum}\idx{Hausdorff-Raum} bezeichnet.

Nun aber zu den eigentlichen Objekten, um die es in dieser Arbeit gehen soll.

\section{CW-Komplexe}

Nach der doch recht abstrakten Einführung topologischer Räume, von
Produkttopologien und Quotientenräumen wollen wir uns nun einer konkreten
Klasse solcher Räume zuwenden, die veranschaulichen, was man mit oben
definierten Begriffen überhaupt anfangen kann. Um allerdings die CW-Komplexe
definieren zu können, werden noch einige zusätzliche Eigenschaften benötigt.

So versteht man unter einer {\bf Zerlegung}\idx{Zerlegung} eines topologischen
Raumes X eine Menge paarweise disjunkter offener Teilmengen von X (versehen mit
der Spurtopologie selbst topologische Räume), deren Vereinigung ganz X ergibt.
Zu jedem x$\in$X kann man also einen eindeutigen Teilraum in der Zerlegung
angeben. Eine {\bf n-Zelle}\idx{Zelle}\idx{n-Zelle} ist ein topologischer Raum,
der homöomorph dem $\R^n$ ist. Damit kann eine {\bf Zellenzerlegung}
\idx{Zellenzerlegung} als eine Zerlegung eines topologischen Raumes in
Teilräume definiert werden, die Zellen sind. Über die Homöomorphie der
d-Zellen zu $\R^d$ kann die {\bf Dimension eines zellenzerlegten Raumes} als die
maximale Dimension n der auftretenden Zellen definiert werden.

Die Vereinigung aller Zellen der Dimension $\leq$ d eines zellenzerlegten
Raumes heißt dabei {\bf d-Gerüst} oder {\bf d-Skelett} des betreffenden Raumes.
\idx{Ger\üst}\idx{Skelett}\\
Weiterhin sollen zwei Zellen {\bf benachbart} heißen,\idx{benachbart} wenn der
Durchschnitt ihrer Abschlüsse nichtleer ist.

Im Jahre 1949 führte J.H.C. Whitehead den Begriff des CW-Komplexes (C für
closure finite und W für weak topology) (vgl. \cite{Ja:90}, Seite 109ff.)
ein, der eine sehr flexible Struktur innerhalb der Topologie darstellt.
Ein {\bf CW-Komplex}\idx{CW-Komplex} ($X,\cal E$) ist ein Hausdorff-Raum X
zusammen mit einer Zellenzerlegung $\cal E$ von X, der folgende drei Axiome
erfüllt:
\bn
\item (Charakteristische Abbildungen): Zu jeder n-Zelle $e\in\cal E$ existiert
      eine stetige Abbildung $\Phi_e:B^n\to X$ der n-Kugel in X,
      die einen Homöomorphismus zwischen der offenen n-Kugel
      (dem $\R^n$) und e induziert und die die (n-1)-Sphäre $\S^{n-1}$
      in das (n-1)-Skelett von $\cal E$ abbildet.
\item (Hüllenendlichkeit): Jeder Punkt der abgeschlossenen Hülle $\ol{e}$
      einer jeden Zelle $e\in\cal E$ besitzt einen Umgebung, die nur endlich
      viele andere Zellen trifft.
\item (Schwache Topologie): Teilmengen $A\subset X$ sind genau dann
      abgeschlossen, wenn für alle $e\in\cal E$ auch die Mengen $A\cap\ol{e}$
      abgeschlossen sind.
\en

Da es in dieser Arbeit nur um endliche CW-Komplexe gehen soll, ist das
zweite Axiom trivialerweise immer erfüllt.

\begin{figure}[htb]
$$
\beginpicture
\unitlength1cm
\setlinear
\setcoordinatesystem units <1cm,1cm>
\setplotarea x from -1 to 6, y from -0.5 to 3.5
\setsolid \thicklines
\setquadratic
\plot 1.0 1.5 1.5 2.5 2.5 2.5 3.5 2.5 3.5 1.5 2.0 1.0 1.0 1.5 /
\plot 3.5 1.5 4.5 1.5 5.5 1.0 /
\plot 1.0 1.5 1.0 0.5 2.0 1.0 /
\plot 2.5 2.5 3.0 3.0 2.8 3.5 /
\setlinear \thinlines
\put {\circle*{0.15}} [Bl] at 1.0 1.5
\put {\circle*{0.15}} [Bl] at 2.0 1.0
\put {\circle*{0.15}} [Bl] at 3.5 1.5
\put {\circle*{0.15}} [Bl] at 5.5 1.0
\put {\circle*{0.15}} [Bl] at 2.5 2.5
\put {\circle*{0.15}} [Bl] at 2.8 3.5
\put {\scsi sechs 0-Zellen} [bl] at 4.0 2.5
\put {\scsi sieben 1-Zellen} [tl] at 2.0 0.0
\put {\scsi zwei 2-Zellen} [bl] at -1.0 2.5
\endpicture
$$
\caption{Beispiel eines CW-Komplexes im $\E^2$}
\label{CW-Komplex}
\end{figure}

Ein {\bf Unterkomplex} $(X',{\cal E}')$ eines CW-Komplexes (X,$\cal E$)
\idx{CW-Komplex!Unterkomplex eines $\sim$} ist nun die Vereinigung X$'$ aller
Zellen aus einer Teilmenge $\cal E'$ von Zellen aus $\cal E$ zusammen mit
$\cal E'$, wenn eine der folgenden drei äquivalenten Bedingungen erfüllt ist:

\btab{ll}
(1) & $(X',{\cal E}')$ ist ein CW-Komplex.\\
(2) & X$'$ ist abgeschlossen in X.\\
(3) & Der Abschluß $\ol{e}$ jeder Zelle e aus $\cal E'$ liegt in X$'$.
\etab

Zum Beweis von $(1)\follows (2)$ ist zu zeigen, daß für alle
$e\in{\cal E}$ die Mengen $\ol{e}\cap X'$ abgeschlossen in X sind.
Da $(X,{\cal E})$ hüllenendlich ist, bedeutet dies, die Abgeschlossenheit
von $\ol{e}\cap (\Cup_{e'\in {\cal E}'}e')=
\ol{e}\cap (e'_1\cup\ldots\cup e'_l)$ für $e'_i\in{\cal E}',~1\leq i\leq l$
und alle $e\in\cal E$ zu klären. Hilfreich dazu ist folgende Bemerkung
(vgl.\cite{Ja:90},S. 110).\\
Ist $\cal E$ eine Zerlegung eines Hausdorff-Raumes X, die das erste Axiom für
CW-Komplexe erfüllt, so ist $\ol{e}=\Phi_e(B^n)$ für jedes $e\in\cal E$
kompakt und der Zellenrand $\ol{e}\backslash e=\Phi_e(\S^{n-1})$ liegt im
(n-1)-Gerüst von X.\\
Zum Beweis dieser Bemerkung nutzt man, daß für stetige Abbildungen f und
Mengen M die Inklusion $f(\ol{M})\subset \ol{f(M)}$ gilt. Damit erhält man hier
$e\subset\Phi_e(B^n)\subseteq\ol{\Phi_e(\stackrel{\circ}{B^n}})=\ol{e}$.
$\Phi_e(B^n)$ ist als stetiges Bild eines Kompaktums in X abgeschlossen und nach
obiger Inklusion gleich $\ol{e}$. Nun aber zurück zum Äquivalenzbeweis. Da die
$\Phi_e$ von $(X',{\cal E}')$ auch charakteristisch bezüglich $(X,{\cal E})$
sind, liefert die Bemerkung simultan den Nachweis von $(1)\follows (3)$, da
$\ol{e}$ in $X$ Hülle von e in $X$, damit aber auch in $X'$ ist. Nutzt man
dies, so ist auch $\ol{e}\cap (e'_1\cup\ldots\cup e'_l)=\ol{e}\cap
(\ol{e}'_1\cup\ldots\cup \ol{e}'_l)$ abgeschlossen.
$(3)\follows (2)$ ist wegen der Betrachtung abgeschlossener Zellen in $X'$
unmittelbar einsichtig. Zum Nachweis von $(2),(3)\follows (1)$ sind die drei
CW-Komplex-Axiome zu überprüfen. Die $\Phi_e$ von $(X,{\cal E})$ sind wegen
${\cal E'}\subset {\cal E}$ auch für $(X',{\cal E'})$ verwendbar. Ebenso
läßt sich für die Hüllenendlichkeit argumentieren.
Da alle in $X'$ abgeschlossenen Mengen dies auch in $X$ sind, müssen für
den Nachweis des dritten Axioms nur $e\in{\cal E}\backslash \cal E'$ betrachtet
werden. Für ein $A\subset X'$ und ein $e\in{\cal E}\backslash{\cal E'}$ ist
dazu $A\cap\ol{e}$ zu betrachten. Zellen $e\in{\cal E}\backslash{\cal E'}$
tragen nichts zum Schnitt mit A bei. Wegen der Hüllenendlichkeit von X gibt
es aber eine endliche Anzahl von $e'_i\in{\cal E'}$, mit denen
$A\cap\ol{e}=A\cap (\Cup e'_i)\cap\ol{e}$ gilt. Nach oben ist
$A\cap\ol{e}=A\cap (\Cup \ol{e'_i})\cap\ol{e}$.
$A\cap\ol{e}=A\cap (\Cup \ol{e'_i})$ ist nach Voraussetzung abgeschlossen,
somit auch $A\cap\ol{e}$.\hfill $\Box$

Betrachtet man zweidimensionale CW-Komplexe, so kann man, wie gewohnt, die
auftretenden 0-Zellen als "`{\bf Ecken}"', die 1-Zellen als "`{\bf Kanten}"'
und die 2-Zellen als "`{\bf Flächen}"' bezeichnen -- als Beispiel sei etwa
der Randkomplex eines Würfels angeführt, der aus sechs "`Flächen"', zwölf
"`Kanten"' und acht "`Ecken"' besteht.

\subsection{Simpliziale Komplexe}

Als eine spezielle Art von CW-Komplexen können die simplizialen Komplexe
aufgefaßt werden, die aufgrund ihrer einfacher zu beschreibenden Struktur
in der Vergangenheit oftmals eher eingesetzt wurden, als die
allgemeinen (nichtsimplizialen) CW-Komplexe. Diese seien hier, obwohl sie nicht
direkter Gegenstand dieser Arbeit sein sollen, der Vollständigkeit halber,
sowie als zusätzliche Beispielklasse aufgeführt.

Um Simplizes definieren zu können, muß gesagt werden, wann ein Punkt
x aus einer Teilmenge X des $\E^d$ {\bf linear abhängig} heißen soll, nämlich
dann, wenn Punkte $x_1,\ldots,x_r\in X$ und Skalare $\lambda_1,\ldots,\lambda_r$
so existieren, daß $x=\lambda_1x_1+\ldots+\lambda_rx_r$ gilt. Ist
$\lambda_1+\ldots+\lambda_r=1$, so heißt x {\bf affin abhängig}, sind zudem
alle $\lambda_i\geq 0$, so {\bf konvex abhängig}. Eine Teilmenge
$X\subseteq\E^d$ heißt {\bf affin abhängig}, falls für $x_1,\ldots,x_r\in X$
und Skalare $\lambda_1,\ldots,\lambda_r$ mit mindestens einem $\lambda_i\neq 0$
eine Relation von der Form $\lambda_1x_1+\ldots+\lambda_rx_r=0$ und
$\lambda_1+\ldots+\lambda_r=0$ existiert. Ansonsten heißt X affin unabhängig.
(Alle obigen Ausführungen gelten auch für allgemeine Vektorräume, wie
so oft wird aber der Anschaulichkeit wegen der $\R^d$ verwendet.)

Ein {\bf m-dimensionales Simplex}\idx{m-Simplex}\idx{Simplex} (kurz m-Sim\-plex)
im $\R^n$ ist die {\bf konvexe Hülle} von m+1 affin unabhängigen Punkten, m+1
Punkten in allgemeiner Lage, also für $p_0,p_1,\ldots,p_m\in\R^n$, $p_i$
affin unabhängig mit $1\leq i\leq m$
$$
   \Delta(p_0,p_1,\ldots,p_m) =
   \left\{ p\in\R^n~:~p = \sum\limits_{i=0}^m \lambda_i p_i, \lambda_i\geq 0,
   \sum\limits_{i=0}^m \lambda_i = 1\right\}.
$$
Die konvexe Hülle von m+1 Einheitsvektoren im $\R^n$ (vgl. Abb.\ref{simplex})
heißt {\bf Standard-m-Simplex},\idx{Standard-m-Simplex} die $p_i$ heißen
{\bf Eckpunkte} (vertices) des Simplex.

\begin{figure}[htb]
$$
\beginpicture
\unitlength1cm
\setlinear
\setcoordinatesystem units <0.6cm,0.6cm>
\setplotarea x from -2 to 2, y from -2 to 2
\put{ \beginpicture
\setsolid
\put {\circle*{0.1}} [Bl] at 0 0
\put {$p_0$} [tr] at 0 0
\endpicture } at -6 0
\put{ \beginpicture
\setsolid
\plot -0.75 -0.75 0.75 0.75 /
\put {\circle*{0.1}} [Bl] at -0.75 -0.75
\put {$p_0$} [tr] at -0.75 -0.75
\put {\circle*{0.1}} [Bl] at 0.75 0.75
\put {$p_1$} [bl] at 0.75 0.75
\endpicture } at -2 0
\put{ \beginpicture
\setsolid
\plot -1 -1 1 -1 0 0.732 -1 -1 /
\put {\circle*{0.1}} [Bl] at -1 -1
\put {$p_0$} [tr] at -1 -1
\put {\circle*{0.1}} [Bl] at 1 -1
\put {$p_1$} [tl] at 1 -1
\put {\circle*{0.1}} [Bl] at 0 0.732
\put {$p_2$} [bl] at 0 0.732
\endpicture } at 2 0
\put{ \beginpicture
\setsolid
\plot -0.884 -0.257 1.085 -0.376 /
\plot 0 1.023 1.085 -0.376 /
\plot -0.884 -0.257 0 1.023 /
\plot -0.2 -0.9 -0.884 -0.257 /
\plot 1.085 -0.376 -0.2 -0.9 /
\setdashes <1mm>
\plot -0.2 -0.9 0 1.023 /
\put {\circle*{0.1}} [Bl] at -0.884 -0.257
\put {$p_0$} [tr] at -0.884 -0.257
\put {\circle*{0.1}} [Bl] at 1.085 -0.376
\put {$p_1$} [bl] at 1.085 -0.376
\put {\circle*{0.1}} [Bl] at 0 1.023
\put {$p_2$} [bl] at 0 1.023
\put {\circle*{0.1}} [Bl] at -0.2 -0.9
\put {$p_3$} [tl] at -0.2 -0.9
\endpicture } at 6 0
\endpicture
$$
\caption{Einige Simplizes}
\label{simplex}
\end{figure}

Die konvexe Hülle einer Auswahl von r paarweise verschiedenen Eckpunkten
eines m-Simplex S heißt {\bf r-dimensionales Teilsimplex}\idx{Teilsimplex} oder
{\bf r-Seite}\idx{Simplex!Seite eines $\sim$} von S. Der {\bf Rand} $\partial S$
eines m-Simplex S ist die Vereinigung aller (m-1)-dimensionalen Teilsimplizes
von S. Das {\bf offene Simplex}\idx{offenes Simplex} $\stackrel{\circ}{S}$ ist
dann S ohne dessen Rand, also $S\backslash\partial S$.

Es sei angemerkt, daß jedes m-Simplex S homöomorph zur m-Vollkugel $B^m$
und der Rand von S homöomorph zur (m-1)-Sphäre ist. Da die offene Vollkugel
$\stackrel{\circ}{B^n}$ homöomorph zu $\R^n$ ist, ist das offene n-Simplex 
eine n-Zelle nach obiger Definition.

Ein {\bf simplizialer Komplex}\idx{simplizialer Komplex} $\cal C$ im $\R^n$ ist
nun eine Menge von Simplizes, die folgende Bedingungen erfüllt:

\btab{ll}
1. & $\cal C$ enthält mit jedem Simplex auch dessen sämtliche Teilsimplizes.\\
2. & Der Durchschnitt zweier Simplizes aus $\cal C$ ist leer oder\\
   & gemeinsames Teilsimplex der beiden.\\
3. & Enthält $\cal C$ unendlich viele Simplizes, so ist $\cal C$ lokal\\
   & endlich, das heißt jeder Punkt in einem Simplex aus $\cal C$ besitzt\\
   & eine Umgebung, die nur endlich viele Simplizes aus $\cal C$ schneidet.
\etab

(Wiederum steht hier der $\R^n$ beziehungsweise $\E^n$ als Synonym für einen
beliebigen Vektorraum über einem Körper $\K$, in dem o.a. Konzepte ähnlich
definierbar sind.)

Betrachtet man die offenen Simplizes eines simplizialen Komplexes, so
erfüllen diese auch die Axiome eines CW-Komplexes. Folglich ist der {\bf einem
simplizialen Komplex} $\cal C$ {\bf zugrundeliegende topologische Raum}
\idx{simplizialer Komplex!zugrundeliegender Raum} oder {\bf Trägerraum}
\idx{simplizialer Komplex!Tr\ägerraum} gegeben durch
$$|{\cal C}| := \bigcup\limits_{S\in{\cal C}} S \subset \R^n,$$
analog der Zellenzerlegung des einem CW-Komplex zugrundeliegenden
Hausdorff-Raumes.

Verschiedene Komplexe können so den gleichen zugrundeliegenden Teilraum des
$\R^n$ besitzen. Simpliziale Komplexe werden auch als "`Polyeder"', gemäß
ihrer Darstellung im Anschauungsraum, bezeichnet.\\
Der {\bf Randkomplex} eines simplizialen k-Komplexes $\cal C$ ist der Komplex
der aus allen Simplizes besteht, deren Dimension k-1 nicht übersteigt, wenn
die maximale Dimension der Simplizes aus $\cal C$ k ist, also das
(k-1)-Skelett.\idx{simplizialer Komplex!Randkomplex}\\
{\bf Unterkomplexe} sind allgemein Teilmengen des Komplexes, die
selbst wieder Komplexe sind.\idx{simplizialer Komplex!Unterkomplex}

Ist $\cal C$ ein simplizialer Komplex und C $\subset\cal C$, so ist der
{\bf Stern}\idx{simplizialer Komplex!Stern} st(C;$\cal C$) von C der kleinste
Unterkomplex von Elementen aus $\cal C$, der C enthält. Der {\bf Antistern}
\idx{simplizialer Komplex!Antistern} ast(C;$\cal C$) von C ist der
Unterkomplex von $\cal C$ aller Elemente, deren Schnitt mit C leer ist. Damit
kann der {\bf Link}\idx{simplizialer Komplex!Link}
link(C;$\cal C$) als der Durchschnitt von Stern und Antistern von C definiert
werden. In gleicher Weise läßt sich dies auch für die Abschlüsse der Zellen
in CW-Komplexen definieren.

\subsection{Abbildungen von CW-Komplexen}

Eine stetige Abbildung f zwischen zwei simplizialen Komplexen $\cal C$
und ${\cal D}$ heißt {\bf simplizial},\idx{simpliziale Abbildung} wenn
Simplizes aus $\cal C$ affin auf Simplizes aus ${\cal D}$ abgebildet werden.
Analog kann eine Abbildung $f:(X,{\cal E})\to (Y,{\cal F})$
zwischen CW-Komplexen als {\bf zellulär}\idx{zellul\äre Abbildung} bezeichnet
werden, wenn sie surjektiv ist und Zellen auf Zellen abbildet.\\
Eine simpliziale oder zelluläre Abbildung f heißt {\bf nicht-degenerierend},
\idx{nicht-degenerierend} falls f die Dimension einer jeden Zelle erhält,
wenn also die Dimension von S und f(S) für jede Zelle S aus $\cal C$
übereinstimmt.\\
Eine simpliziale Abbildung $f:{\cal C} \to {\cal C}$ eines simplizialen
Komplexes $\cal C$ auf sich, heißt {\bf Symmetrie},\idx{Symmetrie!simplizialer
Komplexe}\label{symm} falls sie bijektiv ist und die Dimension aller Simplizes
erhält. Somit sind die Automorphismen von $\cal C$ gerade dessen Symmetrien.
Analog ist eine Symmetrie eines CW-Komplexes wieder eine bijektive zelluläre
Abbildung eines CW-Komplexes auf sich, die die Dimension aller Zellen erhält.

Um sich den Trägerraum eines Komplexes genauer anschauen zu können und
diesen zu beschreiben, ist es sinnvoll hier den Begriff der
{\bf Mannigfaltigkeit}\idx{Mannigfaltigkeit} der Dimension d einzuführen.
Dabei handelt es sich um einen Hausdorff-Raum mit abzählbarer Basis
(eine Basis der Topologie bestehe aus abzählbar vielen offenen Mengen),
in dem jeder Punkt eine offene Umgebung besitzt, die zu einer offenen Teilmenge
des $\R^d$ homöomorph ist. Insbesondere bezeichnet man einen Hausdorff-Raum,
in dem jeder Punkt eine zu einer Kreisscheibe homöomorphe Umgebung besitzt
als {\bf Fläche}\idx{Fl\äche}. In bezug auf simpliziale Komplexe ist eine
{\bf 2-Pseudo-Mannigfaltigkeit}\idx{Pseudo-Mannigfaltigkeit} ein endlicher
geschlossener simplizialer 2-Komplex, das heißt ein Komplex, in dem jede Kante
(jedes 1-Simplex) an genau zwei Dreiecken (2-Simplizes) beteiligt ist.

{\scsi
Das Bild eines simplizialen 2-Komplexes unter einer stetigen Abbildung
$f:{\cal C}\to \R^3$ muß nicht durchdringungsfrei sein. Im Falle von
Triangulierungen der Kleinschen Flasche etwa (vgl. die Dissertation von Cervone,
\cite{Ce:93}) ist die beschriebene (nicht orientierbare) Mannigfaltigkeit nur
mit Selbstdurchdringung im $\R^3$ darstellbar, was aus dem 2-Komplex
nicht ohne weiteres ersichtlich ist. Bijektive Abbildungen f, deren Bild
durchdringungsfrei ist, heißen Einbettung von $\cal C$ in den $\R^3$.
Ist f nur lokal bijektiv, also in einer Umgebung jedes Punktes, so heißt f
Immersion.
}

Da wir uns im dritten Kapitel mit regulären Karten beschäftigen wollen,
sei hier noch die Definition dieser angegeben. Zerlegt man eine geschlossene
reelle 2-Man\-nig\-fal\-tig\-keit in $f_2$ einfach zusammenhängende
(wegzusammenhängend und jede Schleife nullhomotop, das heißt zu einem Punkt
zusammenziehbar), nicht-überlappende Gebiete (Seiten), deren Durchschnitte
$f_1$ Kanten bilden, die sich in $f_0$ Ecken schneiden, so nennt man diese
Unterteilung eine {\bf Karte} der 2-Mannigfaltigkeit. {\bf Regulär} heißt die
Karte, wenn ihre Automorphismengruppe Flaggen-transitiv ist, aber dazu später
mehr (vgl. Symmetriebegriffe Seite \pageref{flag}, eine kurze Beschreibung
findet sich in \cite{BoWi:87}).

Wie oben erwähnt, sind die Zellen von CW-Komplexen und die Simplizes
simplizialer Komplexe selbst topologische Räume mit der durch den Trägerraum
induzierten Topologie. Speziell kennt man aber umgekehrt bei einem
simplizialen Komplex auch dessen Trägerraum schon dann bis auf Homöomorphie,
wenn die Anzahl der wesentlichen Simplizes (solche, die nicht schon als
Seiten größerer Simplizes vorkommen), sowie deren Inzidenzen, bekannt ist.
Dazu beachte man, daß der Trägerraum X eines simplizialen n-Komplexes als
disjunkte Vereinigung der d-Simplizes $S_d$ des Kom\-plex\-es dargestellt werden
kann.
$$X=(S_0+\ldots+S_0)+\ldots+(S_n+\ldots+S_n)$$
Indiziert man die 0-Simplizes mit positiven ganzen Zahlen, den sogenannte
Simplexzahlen,\idx{Simplexzahl} so ist damit eine Äquivalenzrelation $\sim$ :=
"`zu identifizierende 0-Simplizes"' gegeben, die mit den Simplizes den
Quotientenraum X$/_\sim$ beschreibt und so einem Homöomorphismus in X
bereitstellt. In Abbildung \ref{inzidenz} entsteht so aus acht disjunkten
Teilräumen des $\E^2$, den Dreiecken, mittels Eckenidentifikation der
Quotientenraum "`Rand eines Oktaeder"' und damit als Trägerraum eine
2-Mannigfaltigkeit. Wieder gilt eine Analogie zu zweidimensionalen CW-Komplexen,
deren 2-Zellen durch Polygonzüge (Ecken und Kanten) begrenzt sind.

\begin{figure}[htb]
$$
\beginpicture
\unitlength1cm
\setlinear
\setcoordinatesystem units <0.6cm,0.6cm>
\setplotarea x from -2.5 to 2.5, y from -2.5 to 2.5
\put{ \beginpicture
\setsolid
\plot -0.909 0.188 -1.083 -0.157 /
\plot -1.083 -0.157 0 1.5 /
\plot 0 1.5 -0.909 0.188 /
\plot -1.083 -0.157 0.909 -0.188 /
\plot 0.909 -0.188 0 1.5 /
\plot 0.909 -0.188 1.083 0.157 /
\plot 1.083 0.157 0 1.5 /
\plot 0 -1.5 -1.083 -0.157 /
\plot 0 -1.5 0.909 -0.188 /
\setdashes <1mm>
\plot 1.083 0.157 -0.909 0.188 /
\plot -0.909 0.188 0 -1.5 /
\plot 0 -1.5 1.083 0.157 /
\put {\scsi 1} [Br] at -1.2 0.2
\put {\scsi 2} [Br] at -1.4 -0.2
\put {\scsi 3} [Bl] at 1.2 -0.2
\put {\scsi 4} [Bl] at 1.3 0.2
\put {\scsi 5} [bl] at 0 1.6
\put {\scsi 6} [tl] at 0 -1.7
\endpicture } at 3 0
\put{ \beginpicture
\setsolid
\plot -1 0 -1.5 -0.866 -1 -1.732 -0.5 -0.866 0.5 -0.866 1 0 2 0 1.5 0.866 0.5 0.866 0 0 -1 0 /
\plot -1 0 -0.5 -0.866 /
\plot 0 0 0.5 -0.866 /
\plot 0.5 0.866 1 0 /
\plot 1.5 0.866 1 0 /
\plot -1.5 -0.866 -0.5 -0.866 0 0 1 0 /
\put {\scsi 5} [br] at -1 0
\put {\scsi 1} [Br] at -1.5 -0.866
\put {\scsi 6} [tl] at -1 -1.732
\put {\scsi 2} [tl] at -0.5 -0.9
\put {\scsi 6} [tl] at 0.5 -0.866
\put {\scsi 4} [tl] at 1 -0.05
\put {\scsi 6} [tl] at 2 0
\put {\scsi 1} [bl] at 1.5 0.9
\put {\scsi 5} [br] at 0.5 0.9
\put {\scsi 3} [br] at 0 0.05
\endpicture } at -3 0
\endpicture
$$
\caption{Der Quotientenraum "`Oktaeder"'}
\label{inzidenz}
\end{figure}

Unter Ausnutzung der Nummerierung der 0-Simplizes eines simplizialen Komplexes
(bei Betrachtung des Quotientenraumes) kann jede Symmetrie
\idx{Symmetrie!eines simplizialen Komplexes} als Element der Permutationsgruppe
$S_n$ dargestellt werden, wenn n die Anzahl der verschiedenen 0-Simplizes
beziehungsweise Eckpunkte bezeichnet.

\subsection{Inzidenzen und Anheftungen}

Die Betrachtung von allgemeinen CW-Komplexen hat gegenüber simplizialen
Komplexen den Vorteil eine größere Vielfalt bei der Beschreibung von
Zellenberandungen zu haben. Auch besteht etwa ein zu \S$^2$ homöomorpher
simplizialer Komplex aus mindestens 14 Simplizes (zum Aufbau des Randes eines
3-Simplex braucht man vier 0-Simplizes, sechs 1-Simplizes und vier 2-Simplizes),
während zu einem homöomorphen allgemeinen CW-Komplex ganze zwei Zellen
benötigt werden (eine 2-Zelle und eine 0-Zelle). Zudem sind CW-Komplexe als
ein Axiomensystem erfüllende Hausdorff-Räume nicht von vornherein auf
Vektorräume beschränkt, wie dies bei den simplizialen Komplexen der Fall ist.
Gerade in bezug auf die orientierten Matroide kann für diese so eine
natürliche Topologie angegeben werden, die unabhängig von simplizialen
Eigenschaften ist.\\
Natürlich haben CW-Komplexe nicht nur Vorteile. Aufgrund ihrer größeren
Allgemeinheit ist eine algebraische Beschreibung nicht so elegant
möglich, wie es bei den simplizialen Komplexen mittels der Simplexzahlen
geschehen kann. Was hier bei den simplizialen Komplexen die Inzidenzangaben
sind, sind in gewisser Hinsicht die {\bf Anheftungsabbildungen} bei den
CW-Komplexen (X,$\cal E$). Dabei handelt es sich um
\idx{CW-Komplex!Anheftungsabbildung} stetige Abbildungen
$\varphi:\S^{n-1}\to X^{n-1}$ der (n-1)-Sphäre in das (n-1)-Skelett von X, so
daß der Quotientenraum $X\cup_{\varphi} B^n$, nach
dem x und $\varphi$(x) für äquivalent erklärt werden, wieder ein CW-Komplex,
jetzt mit einer n-Zelle mehr ist. Der Zellenrand der neuen Zelle ist
$\varphi(\S^{n-1})\subset X^{n-1}$, ein stetiges Bild der (n-1)-Sphäre.
(Das Anheften kann man sich bildlich so vorstellen, daß zwei Luftballons
zusammengeklebt werden, wobei man die Berührungspunkte der beiden Ballons
identifiziert.) Mittels Anheftungsabbildungen können so alle Gerüste
von CW-Komplexen erzeugt werden. Genau dies ist eine der Schwierigkeiten beim
Umgang mit CW-Komplexen, denn eine Algebraisierung der Anheftungsabbildungen
(im Gegensatz zu den genauen Inzidenzvorschriften der simplizialen Komplexe) ist
erst durch Einsatz der Homologietheorie zu erreichen. Deren Darstellung würde
an dieser Stelle allerdings etwas zu weit führen. Als einführende Lektüre sei
hier stellvertretend das Topologiebuch von Ossa (vgl.\cite{Os:92}, Seite 159ff.)
genannt.

Als eine der Wurzeln der Homologietheorie und als fundamentales Ergebnis bei
der Untersuchung von Polyedern darf allerdings der Begriff der
Euler-Charakteristik\idx{Euler-Charakteristik} und der Eulersche Polyedersatz
\idx{Eulerscher Polyedersatz} nicht fehlen.\\
Mit $f_k(\cal C)$ für $0\leq k\leq d$ seien dabei die Anzahlen der
k-dimensionalen Elemente eines d-Komplexes $\cal C$ (entweder eines simplizialen
oder CW-Komplexes) bezeichnet. Die Folge $f({\cal C})=(f_0,f_1,\ldots,f_d)$
wird {\bf f-Vektor}\idx{f-Vektor} von $\cal C$ genannt.\\
Über den f-Vektor läßt sich die {\bf Euler-Charakteristik} $\chi(\cal C)$
eines d-Komplexes als
$$ \chi({\cal C}) = \sum\limits_{k=0}^{d-1} (-1)^k f_k({\cal C})=2-2g, $$
definieren. Der {\bf Eulersche Polyedersatz} besagt dazu
\begin{satz}
Die Euler-Charakteristik ist eine Homöomorphieinvariante.
\end{satz}
Dies bedeutet, daß homöomorphe d-Komplexe die gleiche Eulercharakteristik
besitzen. Die Konstante g beschreibt dabei das {\bf Geschlecht} des
\idx{Geschlecht} topologischen Gebildes, das heißt die Anzahl der Henkel
beziehungsweise Löcher, die das Objekt hat. (Hier tiefer einzusteigen, würde
in die Homotopietheorie, die Theorie der stetigen Verformbarkeit, führen, zu
der wieder der Verweis auf \cite{Os:92}, Seite 70ff., gegeben sei.)
Für Polyeder P ohne Henkel im $\R^3$, das heißt Polyeder vom Geschlecht
Null, gilt nach oben also
$\chi(P)=\#\mbox{Ecken}-\#\mbox{Kanten}+\#\mbox{Flächen}=2$.
Für endliche CW-Komplexe kann die Euler-Charakteristik als
Wechselsumme ihrer Zellenanzahlen in den einzelnen Dimensionen leicht
berechnet werden. So ist etwa $\chi(\S^n) = 1 + (-1)^n$ und
$\chi(\S^1\times\S^1) = 1 - 2 + 1 = 0$.

\section{Orientierte Matroide}

Die Autoren Björner, Las Vergnas, Sturmfels, White und Ziegler beginnen ihr
Buch über orientierte Matroide (vgl. \cite{Bj:93}) mit dem Satz
\begin{quote}{\sf
"`{\it Oriented matroids} can be thought of as a combinatorial abstraction
of point configurations over the reals, of real hyperplane arrangements, of
convex polytopes, and of directed graphs."'
}\end{quote}
Eine Verbindung zwischen CW-Komplexen und orientierten Matroiden zu suchen
erscheint nach diesem Satz nicht schwierig, da mit reellen Hyperebenen schon
etwas in der Art der benötigten n-Zellen gegeben ist und mit den spezielleren
simplizialen Komplexen sowohl konvexe Polytope, als auch Punktkonfigurationen
erfaßt werden können.

Nun fußt die Theorie der orientierten Matroide in einer Vielzahl
unterschiedlicher Axiomensysteme, die untereinander äquivalent (Die Beweise
der Äquivalenz sind nach \cite{Bj:93} alles andere als trivial) doch, wie
in dem einführenden Satz angedeutet, die unterschiedlichsten Ausgangsebenen
beschreiben. In \cite{Bj:93} werden vier fundamentale Axiomensysteme
\idx{orientiertes Matroid!Axiomensysteme} hervorgehoben:

\btab{ll}
(1) & Kreisaxiome aus der Motivation gerichteter Graphen,\\
(2) & Orthogonalitätsaxiome orthogonaler Paare reeller Vektorunterräume,\\
(3) & Chirotope von Punktkonfigurationen und konvexen Polytopen, sowie\\
(4) & Vektoraxiome reeller Hyperebenenarrangements.
\etab

{\scsi
Der Begriff Chirotop ist eine Abwandlung des Begriffs Chiralität nach
Dreiding und Dress. In der organischen Chemie bezeichnet Chiralität
(Händigkeit) eine Dissymmetrie im räumlichen Aufbau chemischer
Verbindungen und stellt eine notwendige und hinreichende Voraussetzung für das
Auftreten optischer Aktivität dar. (vgl. Literatur zur organischen Chemie,
etwa Flörke/Wolff, Kursthemen Chemie, Organische Chemie und Biochemie, Dümmler
Verlag, Bonn 1984)
}

In \cite{Bj:93} werden orientierte Matroide über alle vier Axiomensysteme
studiert und mit Beispielen untermauert. Da hier auch mittels orientierter
Matroide argumentiert werden soll, sei zunächst eine Einführung gegeben,
was orientierte Matroide überhaupt darstellen und wie sie, für diese Arbeit
nützlich, eingesetzt werden können.

Wie auch in anderen Publikationen, so möchte ich auch hier zunächst anhand
der Chiro\-top\-axiome orientierte Matroide einführen, um dann mit den
"`Vektoraxiomen für reelle Hyperebenenarrangements"', die schon in Richtung
dessen gehen, was für allgemeine CW-Komplexe benötigt wird, das
Einsatzgebiet für diese Arbeit abzustecken.

\subsection{Orientierte Matroide von Punktkonfigurationen}

Orientierte Matroide beziehen sich immer auf eine endliche Menge E,
die der Einfachheit halber als eine geordnete Indexmenge $\{1,2,\ldots,n\}$
beschrieben sein soll. Die Elemente von E können als Indizes von Punkten im
reellen euklidischen Raum, von Hyperebenen oder auch abstrakt, ohne direkten
Bezug zu etwas "`Realisiertem"' aufgefaßt werden.

Zunächst seien die Elemente von E als Indizes von n Punkten
$p_1,\ldots,p_n$ im $\R^{r-1}$ in allgemeiner Lage aufzufassen oder einfacher
als die Punkte selbst. Verwendet man homogene Koordinaten {(jeder affine Punkt
$p_i$ im $\R^{r-1}$ mit den Koordinaten $(p_i^1,p_i^2,\ldots,p_i^{r-1})$ wird
anschaulich als Vektor $(1,p_i)$ im $\R^r$ aufgefaßt)}, so definieren die n
Punkte eine (n$\times$r)-Matrix über $\R$.
$$\left(\begin{array}{cccc}
        1 & p_1^1 & \ldots & p_1^{r-1} \\
        1 & p_2^1 & \ldots & p_2^{r-1} \\
        \vdots & \vdots & \ddots & \vdots \\
        1 & p_n^1 & \ldots & p_n^{r-1} \end{array}\right)$$

{\scsi
Motivation des Einsatzes homogener Koordinaten ist die, daß Punkte des
affinen euklidischen (r-1)-Raumes durch Einbettung in den projektiven r-Raum
$\P^r$ bezüglich ihrer Lage besser beschrieben werden können. Stichwort
ist hierbei die Kompaktifizierung des euklidischen Raumes, in dem Sinne,
daß nun unendlich ferne Punkte ebenfalls erfaßt werden können. Man
vergleiche dies mit der stereographischen Projektion der 2-Sphäre ohne den
Nordpol N auf $\R^2$, die mittels des Zusatzes $N \mapsto \{\infty\}$ einen
Homöomorphismus zwischen dem (nun kompakten) $\R^2\cup\{\infty\}$ und der
kompakten $S^2$ darstellt.
}

Eine Auswahl $(\lambda_1,\ldots,\lambda_r)$, mit $\lambda_i\in E$, von r
paarweise verschiedenen Punkten aus E bildet ein r-Simplex, dessen
Orientierung mittels des Vorzeichens der (r$\times$r)-Unterdeterminante
sign(det$(\lambda_1,\ldots,\lambda_r)$) oben definierter Matrix aus den
$\lambda_i$-ten Zeilen, analog dem Umlaufsinn eines Dreiecks (vgl.
Abb.\ref{orient}), beschrieben werden kann (vgl. \cite{BoEg:91}).
Genauer beschreibt die Determinante
$\det(\lambda_1,\ldots,\lambda_k,\ldots,\lambda_r)$ die Seite der
orientierten Hyperebene
$\mbox{aff}\{\lambda_1,\ldots,\lambda_{k-1},\lambda_{k+1},\ldots,\lambda_r\}$,
auf der der Punkt $\lambda_k$, für alle k aus $\{1,\ldots,r\}$, liegt.

\begin{figure}[hbt]
$$
\beginpicture
\unitlength1cm
\setlinear
\setcoordinatesystem units <0.6cm,0.6cm>
\setplotarea x from 0 to 5, y from -1 to 3
\put{ \beginpicture
\setsolid
\plot 0 0 2 0 0 2 0 0 /
\put {\circle*{0.1}} [Bl] at 0 0
\put {\circle*{0.1}} [Bl] at 0 2
\put {\circle*{0.1}} [Bl] at 2 0
\put {\circle{0.5}} [Bl] at 0.6 0.6
\put {\vector(0,1){0.15}} [Bl] at 1 0.5
\put {\scsi (0,0)} [tr] at 0 0
\put {\scsi (1,0)} [tl] at 2 0
\put {\scsi (0,1)} [br] at 0 2
\put {$\left|\begin{array}{ccc} 1 & 0 & 0 \\
                                1 & 1 & 0 \\
                                1 & 0 & 1
             \end{array}\right|= +1$} [Bl] at 3 1
\endpicture } at -5.5 0
\put{ \beginpicture
\setsolid
\plot 0 0 2 0 0 2 0 0 /
\put {\circle*{0.1}} [Bl] at 0 0
\put {\circle*{0.1}} [Bl] at 0 2
\put {\circle*{0.1}} [Bl] at 2 0
\put {\circle{0.5}} [Bl] at 0.6 0.6
\put {\vector(0,-1){0.15}} [Bl] at 1 0.7
\put {\scsi (0,0)} [tr] at 0 0
\put {\scsi (1,0)} [tl] at 2 0
\put {\scsi (0,1)} [br] at 0 2
\put {$\left|\begin{array}{ccc} 1 & 0 & 0 \\
                                1 & 0 & 1 \\
                                1 & 1 & 0
             \end{array}\right|= -1$} [Bl] at 3 1
\endpicture } at 5.5 0
\endpicture
$$
\caption{Orientierung eines Dreiecks}
\label{orient}
\end{figure}

Als {\bf orientiertes Matroid zur Punktmenge E} wird nun die Information
bezeichnet, die sich aus den Vorzeichen der Determinanten zu allen
r-elementigen Untermengen $\{\lambda_1,\ldots,\lambda_r\}$ von E ergibt.
Bezeichnet $\Lambda (n,r)$ die Menge aller geordneten r-Tupel
\idx{geordnete r-Tupel} von n Elementen, das heißt
$$\Lambda (n,r) := \left\{(\lambda_1,\ldots,\lambda_r)~|~1\leq\lambda_1
<\ldots<\lambda_r \leq n,\lambda_i\in\{1,\ldots,n\},1\leq i\leq r \right\},$$
so kann folgende Definition gegeben werden:
\bcent
\fbox{\parbox{14.2cm}{
  Eine Abbildung $\chi:\Lambda (n,r)\to\{-1,0,+1\}$ oder deren eindeutige
  alternierende Erweiterung $\chi:\{1,\ldots,n\}^r\to\{-1,0,+1\}$ heißt
  {\bf orientiertes Matroid} vom Rang r mit n Punkten, wenn für alle
  $\lambda\in\Lambda (n,r+1)$ und für alle $\mu\in\lambda (n,r-1)$ die Menge
  $$\left\{(-1)^i\cdot\chi (\lambda_1,\ldots,\lambda_{i-1},\lambda_{i+1},
  \ldots,\lambda_{r+1})\cdot\chi (\mu_1,\ldots,\mu_{r-1},\lambda_i)~|~i\in\{1,
  \ldots,r+1\}\right\}$$
  entweder $\{-1,+1\}$ enthält oder gleich $\{0\}$ ist.
}}\ecent

Dabei heißt $\chi:E^r\to\{-1,0,+1\}$ {\bf alternierend},\idx{alternierend} wenn
$$\chi(x_{\sigma_1},x_{\sigma_2},\ldots,x_{\sigma_r}) =
\mbox{sign}(\sigma)\chi(x_1,x_2,\ldots,x_r)$$
für alle $x_i~(1\leq i\leq r)$ aus E und jede Permutation $\sigma$ aus der
Menge S$_E$ aller Permutationen der Elemente aus E gilt. Das entstehende
orientierte Matroid heißt {\bf simplizial},
\idx{orientiertes Matroid!simpliziales $\sim$} wenn die Abbildung $\chi$
E$^r$ in $\{-1,+1\}$ abbildet, das heißt, wenn $\chi(\lambda )\neq 0$ für
alle $\lambda\in\Lambda (n,r)$ ist (vgl. \cite{BoEg:91}). Es heißt
{\bf affin} oder {\bf azyklisch},\idx{orientiertes Matroid!affines $\sim$}
\idx{orientiertes Matroid!azyklisches $\sim$} wenn für alle
$\lambda\in\Lambda(n,r+1)$ die Menge
$$\left\{(-1)^i\cdot\chi (\lambda_1,\ldots,\lambda_{i-1},\lambda_{i+1},
\ldots,\lambda_{r+1})~|~i\in\{1,\ldots,r+1\}\right\}$$
entweder $\{-1,+1\}$ enthält oder gleich $\{0\}$ ist. Abbildung \ref{cube}
zeigt ein orientiertes Matroid zum dreisimensionalen Würfel. Die Schreibweise
von $\chi(\Lambda)$ ist so zu verstehen, daß zeilenweise alle Vorzeichen zu
den kanonisch geordneten (elementweise "`$\leq$"') Tupeln aus $\Lambda$
aufgeführt sind.

\begin{figure}[htb]
$$
\beginpicture
\unitlength1cm
\setlinear
\setcoordinatesystem units <0.6cm,0.6cm>
\setplotarea x from -3 to 5, y from -2 to 2
\setsolid \thicklines
\put {\beginpicture
\setsolid
\plot 0.597 -1.378 -1.281 -1.144 /
\plot 0.597 -1.378 0.597 0.501 /
\plot 0.597 0.501 -1.281 0.735 /
\plot -1.281 0.735 -1.281 -1.144 /
\plot 1.281 -0.735 0.597 -1.378 /
\plot 1.281 -0.735 1.281 1.144 /
\plot 1.281 1.144 0.597 0.501 /
\plot -0.597 1.378 1.281 1.144 /
\plot -1.281 0.735 -0.597 1.378 /
\setdashes <1mm>
\plot -0.597 -0.501 1.281 -0.735 /
\plot -1.281 -1.144 -0.597 -0.501 /
\plot -0.597 -0.501 -0.597 1.378 /
\endpicture} at -2.5 0
\put {\scsi$\left(\begin{array}{cccc}
             1 & 0 & 0 & 0 \\
             1 & 1 & 0 & 0 \\
             1 & 0 & 1 & 0 \\
             1 & 0 & 0 & 1 \\
             1 & 1 & 1 & 0 \\
             1 & 1 & 0 & 1 \\
             1 & 0 & 1 & 1 \\
             1 & 1 & 1 & 1
      \end{array}\right)$} [Bl] at 0 0
\put {\scsi$\chi(\Lambda)=\left.\begin{array}{cccccccccccccccc}
          \{&+&0&+&+&+&-&0&-&-&+&+&+&-&-&\\
            &0&+&+&0&+&-&-&-&-&0&+&-&+&0&\\
            &+&+&-&-&-&+&-&+&+&+&+&-&-&-&\\
            &0&+&+&-&0&-&+&+&-&-&0&+&-&0&\\
            &+&+&+&+&-&-&-&0&-&-&-&+&0&-&\}
            \end{array}\right.$} [Bl] at 4.8 0
\endpicture
$$
\caption{Ein Würfel mit seinem Chirotop}
\label{cube}
\end{figure}

Der {\bf Satz von Radon} besagt, daß für jede endliche Punktmenge X im $\E^r$
mit einer Punkteanzahl $\geq r+2$ eine Zerlegung von X in disjunkte Teilmengen
X$_1$ und X$_2$ existiert, für die
$\mbox{conv X}_1~\cap\mbox{conv X}_2~\neq~\emptyset$ gilt. Eine solche
Zerlegung heißt {\bf Radonpartition}.\idx{Radonpartition}
Die Radonpartitionen oben definierter Punktmenge E liefern die sogenannten
{\bf Kreise} (eine Bezeichnung, die von der Motivation über gerichtete
Graphen her stammt) des orientierten Matroids. Diese sind für
$\mu\in\Lambda(n,r+1)$ und $1\leq i\leq n$ gegeben durch
\idx{orientiertes Matroid!Kreis eines $\sim$}
$$ C_\mu(i) := \left\{\begin{array}{ll}
   (-1)^j\chi(\mu_1,\ldots,\mu_{j-1},\mu_{j+1},\ldots,\mu_{r+1}) &
   \mbox{für } i=\mu_j\\
   0 & \mbox{sonst} \end{array} \right.$$
Die Menge ${\cal K}(\chi):=\{\pm C_\mu|\mu\in \Lambda (n,r+1)\}$ beschreibt
damit alle Kreise des orientierten Matroids $\chi$.

Ist analog $\lambda\in\Lambda (n,d-1)$, so heißt
$C^*_\lambda(i) := \chi(\lambda_1,\ldots,\lambda_{r-1},i)$ für
$1\leq i\leq n$ {\bf Kokreis}\idx{orientiertes Matroid!Kokreis eines $\sim$}
von $\chi$\label{kokreis} und
${\cal K}^*(\chi) := \{\pm C^*_\lambda|\lambda\in\Lambda (n,r-1)\}$
ist die Menge aller Kokreise.

Ist $\chi$ durch eine konkrete Punktkonfiguration gegeben, so entsprechen
die Kokreise den Hyperebenen, die durch die Punkte mit $C^*_\lambda(i)=0$
gegeben sind. Am Beispiel des Chirotops zum Würfel (Abb.\ref{cube}) ergibt sich
etwa ein Kreis $C_\mu$ mit $\mu=(2,4,5,7,8)$ als $(0-0++0-0)$.\\
Der Kokreis $C^*_\lambda$ mit $\lambda=(1,3,7)$ hat die Darstellung
$(0+00++0+)$, woraus sich ablesen läßt, daß es sich um eine Stützhyperebene
handelt, da alle Punkte auf einer Seite der durch die Punkte 1,3 und 7
induzierten Hyperebene H (angezeigt durch vier +, sowie den Punkt 4 auf H) liegen.

Setzt man Mittel der linearen Algebra ein und untersucht die Eigenschaften
der verwendeten Determinanten genauer, so kann obige Situation auch
unabhängig von einer konkreten Punktkonfiguration beziehungsweise einer
konkreten Matrix als Definition für abstrakt aufzufassende {\bf Chirotope}
\idx{Chirotop} vom Rang r auf E dienen. Dies führt(e) zu den folgenden
"`Chirotopaxiomen"'\idx{Chirotopaxiome}

\bcent
\fbox{\parbox{13cm}{
\btab{ll}
(B0) & $\chi$ ist nicht identisch mit der Nullabbildung\\
(B1) & $\chi$ ist alternierend\\
(B2) & Für alle $x_1,x_2,\ldots,x_r,y_1,y_2,\ldots,y_r,$ aus E mit\\
     & $\chi(y_i,x_2,x_3,\ldots,x_r)\cdot\chi(y_1,y_2,\ldots,y_{i-1},x_i,
       y_{i+1},y_{i+2},\ldots,y_r) \geq 0$\\
     & bei $i=1,2,\ldots,r$ gilt
       $\chi(x_1,x_2,\ldots,x_r)\cdot\chi(y_1,y_2,\ldots,y_r) \geq 0$
\etab}}
\ecent

Die Äquivalenz von Chirotopen und orientierten Matroiden wurde 1982 durch
Lawrence mit folgendem Satz gesichert.

\begin{satz}
Sei $r\in\N$ und E sei eine Menge. Eine Abbildung
$\chi:E^r\to\{-1,0,+1\}$ ist genau dann äquivalent zu einem orientierten
Matroid vom Rang r auf E, wenn sie ein Chirotop ist.
\end{satz}

Der Beweis ist nachzulesen in \cite{Bj:93}, Seite 128ff. Wie oben angedeutet,
fordert Axiom B2 die Erfüllung abstrakter Graßmann-Plücker-Relationen,
\idx{Gra\3\-mann-Pl\ücker-Relation} die in ihrer expliziten Form mit
Determinanten für alle $x_1,x_2,\ldots,x_r,y_1,\ldots,y_r \in \R^r$ die
Identität
$$\det(x_1,x_2,\ldots,x_r) \cdot \det(y_1,y_2,\ldots,y_r)$$
$$= \sum\limits_{i=1}^r (-1)^{i-1} \det(y_i,x_2,\ldots,x_r) \cdot
    \det(x_1,y_1,\ldots,y_{i-1},y_{i+1},\ldots,y_r)$$
liefert (vgl.\cite{Na:72}, Stichwort "`Graßmannsche Mannigfaltigkeit"').

Mittels formaler Brackets (formaler Determinanten) lassen sich die allgemeinen
{\bf k-summandigen Graßmann-Plücker-Relationen} für 3$\leq$k$\leq$r+1, mit
\idx{Gra\3\-mann-Pl\ücker-Relation!k-summandige}
Mengen paarweise verschiedener Elemente $A=\{a_1,\ldots,a_{d-k+1}\}$,
$B=\{b_1,\ldots,b_{k-2}\}$ und C=$\{c_1,\ldots,c_k\}$ aus E schreiben als

{\small
$$\{A|B|C\}=$$
$$\sum\limits_{i=1}^k(-1)^{i+1}\cdot
[a_1,\ldots,a_{d-k+1},c_1,\ldots,c_{i-1},c_{i+1},\ldots,c_k]\cdot
[a_1,\ldots,a_{d-k+1},b_1,\ldots,b_{k-2},c_i]=0$$}

Unabhängig von gegebenen Punkten kann so ein orientiertes Matroid mittels
einer Vorzeichenliste definiert werden, die zu den
formalen Brackets $[\lambda_1,\ldots,\lambda_r]$ gehört und mit der die
abstrakten Graß\-mann-Plücker-Relationen B2 gelten.

Die oben angeführten Graßmann-Plücker-Relationen liefern also eine weitere
Charakterisierung für orientierte Matroide. Bemerkenswert ist, daß der
folgende Satz (siehe \cite{Bj:93}, Seite 138) über dreisummandige
Graßmann-Plücker-Relationen
\idx{Gra\3\-mann-Pl\ücker-Relation!dreisummandige $\sim$en}
gilt.

Bevor wir allerdings zur Formulierung des Satzes kommen, seien an dieser Stelle
zunächst die Definitionen der verwendeten Begriffe Matroid und Basen eines
Matroids eingefügt (vgl. dazu auch Aigner, Kombinatorik Bd.II (\cite{Aig:76}),
Seite 17ff., sowie \cite{Schu:92}, Seite 30).\label{matroid}

Der {\bf Steinitzsche Austauschsatz}\idx{Steinitz, Austauschsatz von} besagt,
daß wenn in einem endlichdimensionalen Vektorraum V ein Vektor v nicht linear
abhängig von einer unabhängigen Menge von Vektoren $\{u_1,\ldots,u_n\}$, aber
abhängig von $\{u_1,\ldots,u_n,w\}$ ist, so ist der Vektor w linear abhängig
von $\{u_1,\ldots,u_n,v\}$.\\
Als Verallgemeinerung hiervon wird mit E=$\{1,\ldots,n\}$ und einer Teilmenge
$\B$ der Potenzmenge von E das geordnete Paar (E,$\B$) als ein {\bf Matroid}
\idx{Matroid} bezeichnet, wenn die leere Menge \O\ in $\B$ liegt, mit jeder
Menge $B\in\B$ auch deren Teilmengen in $\B$ liegen, sowie für alle $B_1,
B_2\in\B$ und $x\in B_1\backslash B_2$ ein y aus $B_2\backslash B_1$ existiert,
so daß $(B_1\backslash\{x\})\cup\{y\}$ Element von $\B$ ist. Die Mengen aus
$\B$ heißen {\bf unabhängig}, maximale Mengen in $\B$ werden aufgrund ihrer
gemeinsamen Mächtigkeit {\bf Basen des Matroids} genannt. Diese Mächtigkeit
wird auch als {\bf Rang des Matroids} bezeichnet. Ist E zusammen mit einer
(wie oben definierten) Abbildung $\chi$ ein orientiertes Matroid, so kann man
alle (formalen) Brackets betrachten, die unter $\chi$ ungleich Null sind. Die
Menge der zugehörigen r-Tupel aus $\Lambda(n,r)$ all dieser Brackets nennt man
supp($\chi$). Dieser {\bf Träger} supp($\chi$) bildet die Menge der Basen eines
Matroids, das dem {\bf orientierten Matroid} (E,$\chi$) {\bf zugrundeliegende
Matroid}. In bezug auf die zu erfüllenden Graßmann-Plücker-Relationen ist ein
Matroid die Einschränkung eines über dem Körper GF(3) definierten
orientierten Matroids auf GF(2), was sich auch aus dem folgenden ableiten
läßt.
\begin{satz}\label{dgpr}
Eine Abbildung $\chi:E^r\to\{-1,0,+1\}$ (in der bisherigen Notation) ist genau
dann ein Chirotop, wenn folgende zwei Bedingungen erfüllt sind:

\btab{ll}
(B1$'$) & $\chi$ ist alternierend, und die Menge der r-Untermengen
          $\{x_1,\ldots,x_r\}$ aus E\\
        & mit $\chi(x_1,\ldots,x_r)\neq 0$ ist die Menge der Basen eines
          Matroids vom \\
        & Rang r auf E.\\
(B2$''$) & Für alle $x_1,\ldots,x_r,y_1,y_2\in E$ gilt, falls\\
         & $\chi(y_1,x_2,\ldots,x_r)\cdot\chi(x_1,y_2,x_3,\ldots,x_r)\geq 0$
           und\\
         & $\chi(y_2,x_2,\ldots,x_r)\cdot\chi(y_1,x_1,x_3,\ldots,x_r)\geq 0$
           erfüllt sind, auch\\
         & $\chi(x_1,x_2,\ldots,x_r)\cdot\chi(y_1,y_2,x_3,\ldots,x_r)\geq 0$
\etab
\end{satz}

Die dreisummandigen Graßmann-Plücker-Relationen können im Rang r für
disjunkte Teilmengen $A=\{a_1,\ldots,a_{d-2}\}$ und $B=\{b_1,\ldots,b_4\}$
von E mittels (formaler) Brackets geschrieben werden als
$$\{a_1,\ldots,a_{d-2}|b_1,\ldots,b_4\}:=
  \begin{array}{l}
    +[a_1,\ldots,a_{d-2},b_1,b_2]\cdot [a_1,\ldots,a_{d-2},b_3,b_4]\\
    -[a_1,\ldots,a_{d-2},b_1,b_3]\cdot [a_1,\ldots,a_{d-2},b_2,b_4]\\
    +[a_1,\ldots,a_{d-2},b_1,b_4]\cdot [a_1,\ldots,a_{d-2},b_2,b_3]
  \end{array}=0$$
Im Falle von $\chi:E^d\to\{-1,+1\}$ läßt sich so in einem Programm testen,
ob bei Vorgabe von $\chi$ ein Chirotop und damit ein orientiertes Matroid
vorliegt. Allgemein muß hier zusätzlich überprüft werden, ob durch eine
vorgelegte Vorzeichenliste die nach B1$'$ geforderten Basen eines Matroids
gegeben sind --- als Alternative kann man hier auch die k-summandigen
Graßmann-Plücker-Relationen von B2 nachrechnen.

Wie man sich überlegen kann, ist aus einer Punktkonfiguration über die
zugehörige Punktmatrix für n $\geq$ d Punkte immer ein orientiertes Matroid
durch ein Chirotop gegeben, da hierbei die k-summandigen
Graßmann-Plücker-Relationen immer erfüllt sind (nachweisbar ist dies mittels
des Laplaceschen Entwicklungssatzes für Determinanten). Umgekehrt kann
zu diesem auch wieder eine (zumindest isomorphe) Punktkonfiguration angeben
werden, da man ja weiß, daß eine solche existiert. Im allgemeinen Fall kann
allerdings von einem Chirotop nicht auf eine zugehörige Punktkonfiguration
geschlossen werden. Dieses Problem, zu einem orientierten Matroid eine
entsprechende Punktmatrix zu finden, bezeichnet man als Suche nach einer
Realisierung.\label{real}

Als eine {\bf Realisierung}\idx{Realisierung} eines orientierten Matroids
${\cal M}=(E,\chi)$ vom Rang r über einer totalgeordneten n-elementigen Menge
E bezeichnet man eine Abbildung $\Phi$, die E in den $\R^r$ abbildet, so daß
$\chi(e_1,e_2,\ldots,e_r)=\mbox{sign det}(\Phi(e_1),\Phi(e_2),\ldots,\Phi(e_r))$
für alle $e_i\in E$ gilt, wenn $\chi:E^r\to\{-1,0,+1\}$ das zugehörige
Chirotop bezeichnet. Existiert solch ein $\Phi$, so heißt $\cal M$
realisierbar.

{\scsi
Eine wichtige Eigenschaft im Zusammenhang mit der Realisierung von
orientierten Matroiden ist, daß diese immer (zumindest) lokal realisierbar
sind (in diesem Sinne sind sie lokal in den Euklidischen Raum einbettbar). Dazu
wird in \cite{Bj:93} (Korollar 3.6.3, Seite 140) gezeigt, daß jedes azyklische
Rang r orientierte Matroid als abstrakte (r-1)-dimensionale affine
Punktkonfiguration angesehen werden kann, von der jede (r+2)-punktige
Unterkonfiguration koordinatisierbar ist. Bokowski und Richter-Gebert haben
zu diesem Themenbereich interessante Arbeiten beigesteuert.
}

Die Suche nach Realisierungen beliebiger orientierter Matroide ist als eine
wichtige Fragestellung in \cite{Bj:93} angesprochen und Gegenstand aktueller
Arbeiten, wie zum Beispiel auch \cite{Schu:92} und \cite{Dau:89}, in denen von
simplizialen orientierten Matroiden ausgegangen wird. Hier soll nun das Problem
der Realisierbarkeit von allgemeinen CW-Komplexen, im Sinne durchdringungsfreier
Darstellung im euklidischen Raum, auf die Problematik bei orientierten Matroiden
übertragen werden, wobei nun auch der nichtsimpliziale Fall zuzulassen ist.
Gibt es nämlich zu einem beliebigen CW-Komplex $\cal C$ ein realisierbares
orientiertes Matroid $(E,\chi)$, das gewisse Eigenschaften von $\cal C$
bewahrt, so sind durch $\Phi$ Punkte im $\R^r$ gegeben, mit denen $\cal C$ ohne
Selbstdurchdringungen dargestellt werden kann. Solch ein $(E,\chi)$ heißt eine
{\bf Matroideinbettung} von $\cal C$.\idx{Matroideinbettung} Zunächst muß
hierzu aber eine Verbindung zwischen CW-Komplexen und orientierten Matroiden
angegeben werden, die gerade diese "`gewissen Eigenschaften"' beschreibt.\\
Da wir es bei CW-Komplexen mit zum $\R^n$ homöomorphen Hausdorff-Räumen,
zutun haben, soll nun eine weitere Repräsentationsmöglichkeit orientierter
Matroide angegeben werden, die (Pseudo-)Hyperebenen nutzt, welche dazu dienen
sollen, oben geforderte Verbindung zwischen den beiden betrachteten
Objektklassen zu schaffen.

\subsection{Orientierte Matroide von Hy\-per\-ebe\-nen- und
            Pseu\-do\-hy\-per\-ebe\-nen\-arrange\-ments}

Zur Motivation betrachte man wieder eine endliche Menge $E=\{1,2,\ldots,n\}$
und ein zentrales (den Nullpunkt enthaltendes) Arrangement orientierter
Hyperebenen im $\R^r$ (Beispiel Abb.\ref{hyper}), das gegeben sei durch
$${\cal A}=(H_e=\{x\in \R^r:<x,a_e>=0\})_{e\in E}$$ mit
einer Familie $(a_e)_{e\in E}$ von Normalenvektoren ungleich dem Nullvektor
und dem üblichen Skalarprodukt $<x,y>:=\sum_{i=1}^rx_i\cdot y_i$ für x und
y aus $\R^r$ (beziehungsweise einem Vektorraum über $\K$).
Um Orientierungen unterscheiden zu können, seien die durch die Hyperebenen
gegebenen Halbräume in Richtung der Normalen als positiv ausgezeichnet.
Zu allen $x\in\R^r$ sei $\sigma(x)=(\sigma_e(x)=\mbox{sign}<x,a_e>)_{e\in E}$
ein Vorzeichenvektor mit $|E|=n$ Komponenten der Gestalt $+$, $-$ oder $0$,
als Abkürzung für $+1$, $-1$ und $0$ entsprechend der Signumfunktion,
der die Lage von x bezüglich jeder Hyperebene $H_e$ beschreibt.
Die Abbildung $\sigma$ bildet dabei den $\R^r$ in $\{+,-,0\}^E$, die Menge
aller Vorzeichenvektoren mit $|E|$ Komponenten, ab.
Jene Punkte des $\R^r$, die gleiche Vorzeichenvektoren besitzen, bilden
Zellen der Zerlegung, die durch $\cal A$ auf $\R^r$ induziert wird. Diese Zellen
sind konvexe offene Teilmengen linearer Teilräume des $\R^r$, und heißen
{\bf Topes}\idx{Tope} beziehungsweise {\bf Regionen}.\label{cell}
Mit $\sigma$ ist so eine Äquivalenzrelation gegeben, denn es gilt für
Punkte $x,y\in\R^r$ die Gleichheit $\sigma(x)=\sigma(y)$ genau dann, wenn
x und y aus der gleichen Zelle der Zerlegung des $\R^r$ durch $\cal A$ stammen.
Anstelle aller $x\in \R^r$ brauchen so nur deren Äquivalenzklassen
bezüglich $\sigma$ betrachtet werden, denn die Vorzeichenvektoren im Bild des
$\R^r$ unter $\sigma$ induzieren genau die Zellen, was eine kombinatorische
Beschreibung von $\cal A$ durch $\sigma$ darstellt.

\begin{figure}[htb]
$$
\beginpicture
\unitlength0.6cm
\setlinear
\setcoordinatesystem units <0.6cm,0.6cm>
\setplotarea x from 0 to 14, y from 0 to 9.5
\setsolid
\plot 5 8 5 5 4 4 4 7 6 9 6 8.35 /
\plot 4 4.65 3.5 4.5 3.5 7.5 6.5 8.5 6.5 7.5 /
\plot 3.5 7.5 3.5 8.5 6.5 7.5 6.5 4.5 6 4.65 /
\plot 4 8.35 4 9 6 7 6 4.2 5 5 /
\plot 6.5 6.85 7 7 8.5 5 2.5 3 1 5 3.5 5.8 /
\plot 3.5 3.35 3.5 2 4 2.2 /
\plot 4 3.5 4 1.5 5 2.5 5 3.85 /
\plot 5 2.5 6 1.5 6 4.15 /
\plot 6 2.2 6.5 2 6.5 4.3 /
\put {\scsi H$_1$} [Bl] at 2 4.5
\put {\scsi H$_2$} [Bl] at 4.3 5.7
\put {\scsi H$_3$} [Bl] at 2.8 7
\put {\scsi H$_4$} [Bl] at 2.8 8
\put {\scsi H$_5$} [Bl] at 3.5 9.2
\put {\scsi $+$} [Bl] at 2.5 3.5 \put {\scsi $-$} [Bl] at 2.5 2.7
\put {\scsi $+$} [Bl] at 4.5 7
\put {\scsi $-$} [Bl] at 5.25 7
\put {\scsi $+$} [Bl] at 3.75 7.25
\put {\scsi $+$} [Bl] at 3.75 8
\put {\scsi Regionen} [Bl] at 9 9
\put {\scsi $++++-$, $+-++-$,} [Bl] at 9 8
\put {\scsi $+--+-$, $+----$,} [Bl] at 9 7.5
\put {\scsi $+---+$, $++--+$,} [Bl] at 9 7
\put {\scsi $+++-+$, $+++++$,} [Bl] at 9 6.5
\put {\scsi $-+++-$, $--++-$,} [Bl] at 9 6
\put {\scsi $---+-$, $-----$,} [Bl] at 9 5.5
\put {\scsi $----+$, $-+--+$,} [Bl] at 9 5
\put {\scsi $-++-+$, $-++++$} [Bl] at 9 4.5
\endpicture
$$
\caption{Fünf Hyperebenen im $\R^3$}
\label{hyper}
\end{figure}

Nun betrachte man zu einer endlichen Menge E allgemeiner zunächst beliebige,
das heißt nicht über eine Abbildung $\sigma$ an Hyperebenen gekoppelte,
Vorzeichenvektoren\idx{Vorzeichenvektor} $X,Y\in\{+,-,0\}^E$.
Der {\bf Träger}\idx{Vorzeichenvektor!Tr\äger} eines solchen Vektors X ist
$\ul{X}=\{e\in E:X_e\neq 0\}$, seine {\bf Nullmenge}
\idx{Vorzeichenvektor!Nullmenge} $z(X)=X^0=E\backslash\ul{X}=\{e\in E:X_e=0\}$,
die Menge seiner positiven Elemente $X^+=\{e\in E:X_e=+\}$ und analog die Menge
seiner negativen Elemente $X^-=\{e\in E:X_e=-\}$. Der Vektor, der für alle
$e\in E$ aus 0 besteht, heiße analog dem $\R^d$ Nullvektor und sei mit 0
bezeichnet. Zu $X$ sei $-X$ gegeben durch die Vorzeichenumkehrung von X, also
$(-X_e)=-$ für $X_e=+$, $(-X_e)=+$ für $X_e=-$ und $(-X_e)=0$ für $X_e=0$.
Weiter sei eine {\bf Zusammensetzung}\idx{Vorzeichenvektor!Zusammensetzung}
$X\circ Y$ von X und Y dadurch gegeben, daß $(X\circ Y)_e=X_e$ für
$X_e\neq 0$ und $Y_e$ sonst gelte. Die {\bf Trennungsmenge}
\idx{Vorzeichenvektor!Trennungsmenge} von X und Y sei
$S(X,Y)=\{e\in E:X_e=-Y_e\neq 0\}$.

Mit diesen Vorgaben heißt eine Menge ${\cal L}\subseteq\{+,-,0\}^E$ {\bf Menge
der Kovektoren eines orientierten Matroids}, wenn folgende Axiome erfüllt
sind, die dementsprechend als {\bf Kovektoraxiome}\idx{Kovektoraxiome}
bezeichnet werden.

\bcent
\fbox{\parbox{14cm}{
\btab{ll}
(L0) & ${\cal L}$ enthält den Nullvektor 0.\\
(L1) & Mit $X\in\cal L$ ist auch $-X$ in ${\cal L}$ enthalten.\\
(L2) & Sind X und Y aus $\cal L$, so auch deren Zusammensetzung $X\circ Y$\\
(L3) & Zu X und Y aus $\cal L$ und e aus S(X,Y) existiert ein Z in
       ${\cal L}$ mit Z$_e=0$\\
     & und Z$_f=(X\circ Y)_f=(Y\circ X)_f$ für alle f $\notin$ S(X,Y).
\etab}}
\ecent

Um die Begriffswahl der Vorzeichenvektoren als "`\ul{Ko}vektoren eines
orientierten Matroids"' verstehen zu können, müssen an dieser Stelle die
Begriffe Polarität und Dualität eingeführt werden, wie sie im gegenwärtigen
Zusammenhang zu verstehen sein sollen. Dazu ist ein kurzer Abstecher in die
Welt der topologischen Vektorräume nötig, der uns die Definition der
gewünschten Begriffe liefert.

Allgemein werden zwei topologische Vektorräume F und G über einem gemeinsamen
Körper $\K$ als ein {\bf duales Paar}\idx{duales Paar} (F,G) bezeichnet, wenn
es zwischen ihnen eine Bilinearform $f:F\times G\to\K$ gibt, für die zum einem
für jedes feste $0\neq x_0\in F$ ein $y\in G$ existiert, so daß
$f(x_0,y)\neq 0$ gilt, und zum anderen analog zu jedem festen $0\neq y_0\in G$
ein $x\in F$ gewählt werden kann, mit $f(x,y_0)\neq 0$. Ist nun (F,G) solch
ein duales Paar, so ist die {\bf Polare}\idx{Polare} $A^*$ einer Menge
$A\subseteq F$ definiert als
$$ A^*=\{y\in G~:~|f(x,y)|\leq 1,~\forall x\in A\}.$$

Nach Grünbaum (vgl. \cite{Gr:67}, Seite 46ff.) heißen zwei d-dimensionale
Polytope $P$ und $P^*$ dual\idx{dual} wenn zwischen ihnen eine bijektive
Abbildung $\Psi$ der r-Seiten ($1\leq r\leq d$) von $P$ auf die Seiten von
$P^*$ existiert, die Inklusionen umkehrt. Dies bedeutet, daß für zwei Seiten
$F_1$ und $F_2$ von $P$ mit $F_1\subset F_2$ unter $\Psi$ gilt, daß
$\Psi(F_1)\supset\Psi(F_2)$ ist. In der Theorie der orientierten Matroide
wird dieser Sachverhalt als {\bf Polarität}\idx{Polarit\ät} bezeichnet.\\
Da hier Vektorräume über $\R$ betrachtet werden, ist eine Bilinearform f
durch das übliche Skalarprodukt auf $\R^d$ gegeben. Damit ist die zu einem
Polytop $P$ gehörige polare Menge $P^*$ gerade
$\{y\in\R^d~|~<x,y>\leq 1,~\forall x\in P\}$.
Bildet man die konvexe Hülle der Extrempunkte von $P^*$, so ist diese ebenfalls
ein Polytop, das als das zu P polare bezeichnet wird und entsprechend seiner
Herkunft ebenfalls die Bezeichnung $"`P^*"'$ erhält. Nach obiger Definition ist
$P^*$ auch dual zu $P$, da sich eine inklusionsumkehrende Abbildung der
Seiten von $P$ auf die Seiten von $P^*$ angeben läßt. Anschaulicher ist dieser
Sachverhalt darzustellen, wenn man die Situation betrachtet, daß der einen
Punkt P im $\R^d$ beschreibende Ortsvektor $x_P$ zum einen den Punkt als solchen
bezeichnet, aber auch als Normale einer zu $x_P$ orthogonalen (Hyper-)Ebene mit
der Hesseschen Normalform $<x_P,x>=1$ aufgefaßt werden kann (vgl. obige
Definition der Hyperebenenarrangements). Betrachtet man nun einerseits die
konvexe Hülle einer endlichen Anzahl von Punkten $x_P$ und andererseits die
durch die (Hyper-)Ebenen mit den Normalen $x_P$ eingeschlossene Zelle
(gerade der Durchschnitt der Halbräume, die durch $<x_P,x>\leq 1$ gegeben sind,
so ergibt sich gerade die oben eingeführte Polarität zwischen einer
Punktkonfiguration und der entsprechenden polaren (Hyper-)Ebenenkonfiguration.
\label{polar} Eindrucksvolle Beispiele sind die Polaritäten der platonischen
Körper im $\R^3$ (das selbstpolare Tetraeder, die polaren Paare Würfel und
Oktaeder, sowie Dodekaeder und Ikosaeder).
Speziell entsprechen unter Polarität im $\R^3$ also Punkte Flächen, Kanten
Kanten und Flächen Punkten. Verallgemeinert auf höhere Dimensionen
bedeutet dies gerade, daß im $\R^d$ (d-i)-Seiten von $P$ im polaren Fall zu
i-Seiten von $P^*$ werden (was gleichzeitig Dualität induziert).\\
Bei orientierten Matroiden sei mit Dualität\idx{Dualit\ät} die Situation
bezeichnet, wenn man, wie Bokowski in seinem Beitrag zum "`Handbook of Convex
Geometry"' (\cite{Bo:93}) angibt, etwa die Kreise eines orientierten Matroids
$\cal M$ als Kokreise eines anderen orientierten Matroids $\cal M^*$ ansieht,
welches dann das zu $\cal M$ Duale darstellt (vgl. Kreis-Definitionen von
Seite \pageref{kokreis}).\\
Wenn also von Vektoren und Kovektoren orientierter Matroide die Rede ist, so
spiegelt dies gerade die dualen Versionen wider. (Da die Dualität orientierter
Matroide in dieser Arbeit nicht weiter vertieft wird, möchte ich auf
\cite{Bj:93},S.115ff.,S.157ff. und \cite{Schu:92}, Seite 40ff. verweisen.)

Der Einfachheit halber sei im folgenden eine Menge ${\cal L}$, die die
Kovektoraxiome erfüllt beziehungsweise ein Paar $(E,{\cal L})$ als
orientiertes Matroid bezeichnet. Dieses habe den Rang r, entsprechend der
Dimension des Raumes, in dem die Hyperebenen liegen und n$=|E|$ Elemente.

Eine Teilordnung "`$\leq$"' auf $\{+,-,0\}$ mit $0<+$, $0<-$ und bezüglich
der $+$ und $-$ nicht vergleichbar sind, induziert auf $\{+,-,0\}^E$ über
den komponentenweisen Vergleich der Vorzeichenvektoren eine Produktteilordnung.
(Es gilt $y\leq x$, wenn für alle $e\in E$ die Komponente $y_e\in\{0,x_e\}$
ist.) Ist ${\cal L}$ ein orientiertes Matroid nach obiger Definition, so wird
dieses, wenn man es mit einem abstrakten \^1-Element vereinigt und mit
eben definierter Teilordnung betrachtet zu einem Verband, dem (großen)
Seitenverband $\hat{\cal L}={\cal L}\cup \{\hat{1}\}$\idx{Seitenverband} von
$\cal L$. Bezüglich "`$\leq$"' maximale Elemente in $\cal L$ heißen
{\bf Topes} oder auch Regionen\idx{Topes!eines orientierten Matroids} von
$\cal L$ und entsprechen gerade den im realisierbaren Fall von Hyperebenen
"`ausgeschnittenen"' Teilräumen. (vgl. Abb.\ref{hyper})

Bei der Untersuchung orientierter Matroide sind auch Teilstrukturen dieser
interessant, die aus Operationen auf einem orientierten Matroid entstehen. So
ist eine {\bf Restriktion}\idx{Restriktion} eines Vorzeichenvektors
$X\in\{+,-,0\}^E$ auf eine Teilmenge F$\subset$E der Vorzeichenvektor
$X|_F\in\{+,-,0\}^F$, definiert durch $(X|_F)_e=X_e$ für alle e$\in$F. Ist
${\cal L}\subseteq\{+,-,0\}^E$
die Menge der Kovektoren eines orientierten Matroids $\cal M$ auf E und ist
A$\subseteq$E, so ist die Menge der Kovektoren der {\bf Deletion}\idx{Deletion}
${\cal M}\backslash A$ die Menge ${\cal L}\backslash A
:=\{X|_{E\backslash A}:X\in{\cal L}\}\subseteq\{+,-,0\}^{E\backslash A}$.
Die Menge der Kovektoren der {\bf Kontraktion}\idx{Kontraktion} ${\cal M}/A$
ist die Menge ${\cal L}/ A:=\{X|_{E\backslash A}:X\in{\cal L}\mbox{ und }
A\subseteq X^0\}\subseteq\{+,-,0\}^{E\backslash A}$.
Unter einer {\bf Reorientierung}\idx{Reorientierung} $_{-A}{\cal M}$ versteht
man die Menge $_{-A}{\cal L}:=\{_{-A}X:X\in{\cal L}\}$, wobei $_{-A}X$ definiert
ist durch $(_{-A}X)^+=(X^+\backslash A)\cup (X^-\cap A), (_{-A}X)^0=X^0$ und
$(_{-A}X)^-=(X^-\backslash A)\cup (X^+\cap A)$, also einer lokalen
Vorzeichenumkehrung.

Gehen wir nochmals zurück zum Ausgangspunkt eines zentralen
Hyperebenenarrangements $\cal A$. Da Mengen von Hyperebenen
H$_e=\{x\in\R^r:<x,a_e>=0\}$ betrachtet werden, kann man äquivalent statt der
H$_e$ auch deren Einschränkung auf die Einheitssphäre $\S^{r-1}\subset\R^r$
heranziehen (Großkreise auf der $\S^2$ oder im allgemeinen Fall (r-2)-Sphären
auf der $\S^{r-1}$), ohne daß an der Repräsentation der Schnitteigenschaften
etwas verändert wird. (Dies ist möglich, da durch die "`zentralen"'
Hyperebenen schon der lineare respektive projektive Fall betrachtet wird,
eine Kompaktifizierung der Hyperebenen also nur formal vollzogen werden muß,
eben durch den Übergang zu einem Sphärensystem.)
Die durch die $\{x\in\R^r:<x,a_e>=0,\|x\|=1\}$ induzierte Zerlegung der
$\S^{r-1}$ läßt sich nun analog den Hyperebenen orientieren, indem man jene
Hemisphären als positiv annimmt, die in Richtung der Normalen $(a_e)_{e\in E}$
der H$_e$ liegen. Werden die Schnitteigenschaften der (r-2)-Sphären
untereinander nicht verändert, so können diese sogar mittels stetiger
Abbildungen (in \cite{Bj:93} werden diese als "`zahm"' bezeichnet, vgl. S. 225)
deformiert werden, wobei die Information des zugehörigen orientierten Matroids
erhalten bleibt. So gelangt man zur Definition von Pseudosphären. Dabei wird
eine Teilmenge $S\subset\S^{r-1}$ als {\bf Pseudosphäre}\idx{Pseudosph\äre}
bezeichnet, wenn ein Homöomorphismus $h:\S^{r-1}\to\S^{r-1}$ existiert, so
daß $S=h(S^{r-2})$ gilt. Hierbei ist $S^{r-2}=\{x\in\S^{r-1}:x_r=0\}$ der
Äquator der $\S^{r-1}$ (vgl. Abb.\ref{pseudo}). Ein {\bf
Pseudosphärenarrangement} in $\S^r$ ist dann als ein endliches Mengensystem
${\cal A}=(S_e)_{e\in E}$ definiert, wenn (vgl. \cite{Bj:93}, S.227) zum einen
$S_A=\Cap_{e\in A} S_e$ für alle $A\subseteq E$ eine Sphäre darstellt und zum
anderen für $S_A\not\subseteq S_e$ mit $A\subseteq E,e\in E$, sowie $S^+_e$ und
$S^-_e$ den beiden Hemisphären von $S_e$, $S_A\cap S_e$ eine Pseudosphäre in
$S_A$ mit den Seiten $S_A\cap S^+_e$ und $S_A\cap S^-_e$ ist.

Ein Ergebnis von Folkman und Lawrence aus dem Jahre 1978 liefert nun den
Zusammenhang zwischen Pseudosphärenarrangements und orientierten Matroiden.
\begin{satz}
Sei ${\cal A}=(S_e)_{e\in E}$ ein orientiertes Pseudosphärenarrangement in
der $\S^r$ und $\sigma:\S^r\to\{+,-,0\}^E$ die Abbildung, die definiert
sei über
$$\sigma(x)_e=\left\{\begin{array}{ll}
                        +, & \mbox{ für } x\in S^+_e\\
                        -, & \mbox{ für } x\in S^-_e\\
                        0, & \mbox{ für } x\in S_e\end{array}\right.$$
Dann ist ${\cal L(A)} := \{\sigma(x):x\in\S^r\}\cup\{0\}\subseteq\{+,-,0\}^E$
die Menge der Kovektoren eines orientierten Matroids. Ist dim($S_e$)=k, so
ist der Rang von $\cal L(A)$ gleich r-k. Ist $\cal A$ essentiell, das heißt
$S_E=\Cap_{e\in E} S_e$ ist leer, so ist der Rang von $\cal L(A)$ gleich r+1.
\end{satz}

Mit dem Beweis der Umkehrung schließt sich hier der Topologische
Repräsentationssatz nach Folkman und Lawrence für orientierte Matroide an
(vgl \cite{Bj:93}, Seite 233). Dieser besagt
\idx{Topologischer Repr\äsentationssatz}
\begin{satz}
{\bf Topologischer Repräsentationssatz:} Ist ${\cal L}\subseteq\{+,-,0\}^E$, so
ist $\cal L$ genau dann die Menge der Kovektoren eines schleifenfreien
orientierten Matroids vom Rang r+1, wenn ein orientiertes
Pseudosphärenarrangement $\cal A$ auf der $\S^{r+1+k}$ existiert, für das
$k=\mbox{dim}(\Cap_{e\in E}\S_e)$ gilt und bezüglich dem $\cal L$ gleich
$\cal L(A)$ ist.
\end{satz}

Schleifenfrei bedeutet, daß die Vorzeichenstruktur jedes Topes keine Null
enthält. Als Folgerung hieraus läßt sich zeigen, daß schleifenfreie
orientierte Matroide vom Rang r+1 (bis auf Reorientierung und Isomorphie)
bijektiv zu essentiellen orientierten Arrangements von Pseudosphären auf
$\S^r$ (bis auf topologische Äquivalenz) korrespondieren.

Dieses herausragende Ergebnis liefert also die Äquivalenz von
Pseudosphärenarrangements und orientierten Matroiden respektive der gewählten
Axiomatik. Hier ist es wichtig, daß von \ul{Pseu\-do}\-sphären die Rede ist,
da ein topologisches Arrangement (im Sinne informationserhaltender stetiger
Verformung) nicht homöomorph zu einem Sphärensystem sein muß.
Ist dies aber der Fall, so handelt es sich um die Darstellung eines
realisierbaren orientierten Matroids, analog der Definition der Realisierung,
wie sie im Abschnitt über die Chirotope angegeben wurde (Seite \pageref{real}).

Zu bemerken ist, daß der reelle projektive Raum $\P^d$ der Dimension d aus
der $\S^d$ durch Identifikation aller antipodischen (diametral
gegenüberliegenden) Punkte hervorgeht. Ist $\pi(x)=\{x,-x\}:\S^d\to \P^d$,
so werden in natürlicher Weise die Nullpunkt-symmetrischen Teilmengen
der d-Sphäre mit allgemeinen Teilmengen des projektiven Raumes
identifiziert. Hierüber können oben eingeführte Pseudosphären auch als
Pseudohyperebenen in $\P^d$ aufgefaßt werden, womit auch eine Klassifikation
orientierter Matroide im projektiven Raum mittels Pseudohyperebenen gegeben ist.

\begin{figure}[htb]
$$
\beginpicture
\unitlength0.6cm
\setlinear
\setcoordinatesystem units <0.6cm,0.6cm>
\setplotarea x from -3 to 3, y from -3 to 3
\put {\beginpicture
  \setsolid
  \circulararc 180 degrees from 2 0 center at 0 0
  \ellipticalarc axes ratio 4:1 180 degrees from -2 0 center at 0 0
  \setdashes<1.5mm>
  \ellipticalarc axes ratio 4:1 -180 degrees from -2 0 center at 0 0
  \setsolid
  \plot -1.73 1 -3 1 -4 -1 3 -1 4 1 1.73 1 /
  \circulararc 120 degrees from -1.73 -1 center at 0 0
  \put {\vector(0,1){1.5}} [Bl] at 0 0
  \put {\vector(1,0){0.2}} [Bl] at 0 -0.5
  \put {\scsi $\S^{d-1}$} [Bl] at 1.5 2
  \put {\scsi $\S^{d-2}_a$} [Bl] at 1.5 -0.8
  \put {\scsi a} [Bl] at 0.3 1.5
  \put {\scsi $H_a$} [Bl] at 2.5 0.5
  \setquadratic
  \plot -2 0 -1.3 0.3 -0.7 0.1 0.2 -0.1 1.2 0.15 1.6 0.2 2 0 /
  \setdashes<1mm>
  \plot 2 0 1.3 -0.3 0.7 -0.1 -0.2 0.1 -1.2 -0.15 -1.6 -0.2 -2 0 /
\endpicture} at -4 1
\put {\beginpicture
  \setsolid \setlinear
  \circulararc 360 degrees from 2 0 center at 0 0
  \setquadratic
  \plot -1.732 -1 -0.5 -1 0.5 -0.2 1 0.5 1.732 1 /
  \plot 0 -2 0.35 -1 0 0 -0.5 1 0 2 /
  \plot 1.732 -1 1 0 0.3 0.3 -1 0.5 -1.732 1 /
  \setdashes<1mm>
  \plot 1.732 1 0.5 1 -0.5 0.2 -1 -0.5 -1.732 -1 /
  \plot 0 2 -0.35 1 0 0 0.5 -1 0 -2 /
  \plot -1.732 1 -1 0 -0.3 -0.3 1 -0.5 1.732 -1 /
  \setdots<1mm> \setlinear
  \plot 0.5 1.95 0.5 3 -0.5 2 -0.5 -3 0.5 -2 0.5 -1.95 /
  \plot -2 0.4 -3 1 -2 2 3 -1 2 -2 1.35 -1.55 /
  \plot -2 -0.4 -3 -1 -2 -2 3 1 2 2 1.35 1.55 /
\endpicture} at 4 0
\endpicture
$$
\caption{Von einer Hyperebene zu einer Pseudosphäre}
\label{pseudo}
\end{figure}

Zusammenfassend kann man sagen, daß sich orientierte Matroide als
Verallgemeinerung aus den verschiedensten Bereichen der Mathematik motivieren
lassen. Dabei führt in verschiedenen Fällen die allgemeinste Struktur
bezüglich einer zu erfüllenden Eigenschaft unabhängig von der gewählten
Ausgangssituation zu diesem Konzept. Der Name "`Matroid"' stammt dabei von
Whitney aus dem Jahre 1935 und leitet sich von einer Klasse fundamentaler
Beispiele solcher Objekte ab, die aus Matrizen hervorgehen. (Als
Übersichtsartikel sei hierzu \cite{Bo:93} genannt.)

\section{Symmetriebegriffe}

Ist ein allgemeiner CW-Komplex realisierbar, so ist es erstrebenswert,
möglichst viele Symmetrieeigenschaften von diesem in die Realisierung
"`hinüberzuretten"'. Was man hierunter versteht, soll im folgenden
beschrieben werden.

Wenn man im $\E^d$ von {\bf Symmetrien}\idx{Symmetrie} spricht, so meint man
damit die Automorphismen von $\E^d$, die Abstände und Orthogonalität
invariant lassen. Dies sind gerade die orthogonalen Transformationen.
Die Menge aller orthogonalen Transformationen ist bezüglich der
Hintereinanderausführung eine Gruppe, die {\bf orthogonale Gruppe}
${\cal O}(d)$.\idx{orthogonale Gruppe} Elemente aus ${\cal O}(d)$ lassen sich
als orthogonale (d$\times$d)-Matrizen über $\R$ darstellen und stellen damit
eine Untergruppe der allgemeinen linearen Gruppe $\mbox{GL}_d(\R)$, der Gruppe
aller (d$\times$d)-Matrizen über $\R$, dar.
$${\cal O}(d) = \{ A\in\mbox{GL}_d(\R)~|~A^T A = I_d\}$$
{\scsi
Eine Matrix heißt orthogonal, wenn sie die kanonischen Einheitsvektoren auf
eine Orthonormalbasis abbildet, das heißt, wenn ihre Spaltenvektoren
auf Länge Eins normiert sind, sowie das Skalarprodukt von je zwei
verschiedenen Spaltenvektoren Null ergibt. Als Folgerung hieraus ist die
Determinante einer orthogonalen Matrix immer $\pm 1$. Wie mit linearer Algebra
gezeigt werden kann, haben orthogonale Matrizen die Eigenschaft, daß ihre
Inversen gerade ihre Transponierten sind, was zur Charakterisierung
der orthogonalen Gruppe dient.\\
Zur Erinnerung ist eine Gruppe eine Menge G zusammen mit einer Abbildung
$*:G\times G\to G$, der Gruppenoperation, die folgendes Axiomensystem erfüllt.
So ist G bezüglich $*$ abgeschlossen, besitzt ein neutrales Element e
($g*e=e*g$ für alle $g\in G$) und zu jedem $g\in G$ existiert ein inverses
Element $g^{-1}$ ($g*g^{-1}=e=g^{-1}*g$). Als Literatur zur Linearen Algebra
sei Herrn Artmanns Buch \cite{Art:89}, sowie zur Einführung in die
Algebra Serge Langs Buch \cite{La:79} empfohlen.
}\\
Damit wird die Matrizenmultiplikation zur Gruppenoperation.
Matrizen aus ${\cal O}(d)$ mit Determinante -1 heißen {\bf Spiegelungen} oder
auch {\bf Reflektionen},\idx{Reflektion}\idx{Spiegelung} Matrizen mit
Determinante +1 werden als {\bf Drehungen}\idx{Drehung} oder {\bf Rotationen}
\idx{Rotation} bezeichnet. Die Drehungen bilden dabei eine Untergruppe der
${\cal O}(d)$, die als spezielle orthogonale Gruppe ${\cal SO}(d)$ bezeichnet
wird.\idx{spezielle orthogonale Gruppe}\\
{\scsi
Die Untergruppe ${\cal SO}(d)$ ist gesondert ausgezeichnet, da es sich bei
ihr um die Komponente des Eins-Elements (das heißt Neutralelement) von
${\cal O}(d)$ handelt, wenn diese in ihre zwei Zusammenhangskomponenten
zerlegt wird. (vgl. \cite{Os:92}, Seite 130ff.)}

Einen geometrischen d-Komplex $\cal C$, das heißt die metrische Realisierung
eines kombinatorisch vorgelegten allgemeinen CW- oder simplizialen d-Komplexes,
bezeichnet man als {\bf symmetrisch},\idx{kombinatorischer Komplex}
\idx{geometrischer Komplex}\idx{geometrischer Komplex!symmetrischer $\sim$}
wenn eine Untergruppe G der ${\cal O}(d)$ existiert, deren sämtliche
Elemente $g\in G$ Automorphismen auf $\cal C$ sind, das heißt für die
sämtlich $g({\cal C}) = {\cal C}$ gilt. Die $g\in G$ heißen Symmetrien
von $\cal C$, G selbst heißt, gemäß der Gruppenstruktur, die Symmetriegruppe
\idx{Symmetriegruppe} von $\cal C$.\idx{Symmetriegruppe!$\sim$ geometrischer
Komplexe}
Die Anzahl der Elemente einer Symmetriegruppe bezeichnet man als deren
(Symmetrie-){\bf Ordnung}. Dies ist nicht mit der Ordnung einer Symmetrie, das
heißt eines Elements einer Symmetriegruppe, zu verwechseln, die angibt, wie oft
die betreffende Symmetrie angewendet werden muß, um wieder die Identität zu
erreichen. Mit "`hoher Symmetrie"' bezeichnet man dabei die Situation, daß
die Anzahl der Elemente der betreffenden Symmetriegruppe, gemessen an
der Anzahl der permutierten Ecken eines Komplexes, groß ist
\idx{Symmetrieordnung} (etwa auch die Situation, daß die Symmetriegruppe des
geometrischen eine "`große"' Untergruppe der Symmetriegruppe des kombinatorische
Komplexes ist).

Betrachtet man, wie auf Seite \pageref{symm}, die Inzidenzstrukturen
simplizialer Komplexe, die durch die Simplexzahlen impliziert werden, so
heißt die sich durch Umnummerierung (Permutation) der Ecken (Menge der
0-Simplizes) ergebende Automorphismengruppe des Komplexes, deren Elemente
nicht-degenerierend auf den Simplizes des Komplexes wirken, Symmetriegruppe
des simplizialen Komplexes.\idx{Symmetriegruppe!$\sim$ simplizialer Komplexe}
Mittels der Simplexzahlen kann jede solche Symmetrie, wie schon erwähnt,
als Element der Permutationsgruppe $S_n$, mit der Anzahl n der auftretenden
verschiedenen 0-Simplizes, beschrieben werden, also als Permutation auf der
Eckenmenge $E=\{1,\ldots,n\}$. Auch hier heißt die Gesamtheit der Symmetrien
des Komplexes entsprechend ihrer Gruppenstruktur Symmetriegruppe des
simplizialen Komplexes.

Da simpliziale Komplexe einen Spezialfall von CW-Komplexen darstellen, kann
man in einfacher Weise den Symmetriebegriff für simpliziale Komplexe
auf die CW-Komplexe erweitern. Wie im einen Fall bei nicht-degenerierenden,
bijektiven simplizialen Abbildungen von Symmetrien gesprochen wird, so können
dimensionserhaltende bijektive Selbstabbildungen von CW-Komplexen auch als
Symmetrien bezeichnet werden, wie dies schon im Abschnitt über CW-Komplexe
beschrieben wurde (in jedem Fall betrachtet man die Automorphismen des
Komplexes).

Für (zunächst) simpliziale d-Komplexe kann man den Begriff der Flagge
\idx{Flagge} für das $d+1$-Tupel (0-Simplex, 1-Simplex, $\ldots$, d-Simplex)
einführen, dessen Elemente paarweise inzidieren (Im Falle von 2-Kom\-plexen
ist dies ein Tripel (Ecke, Kante, Seite)). Automorphismen kombinatorischer
Komplexe\label{flag} erhalten solche Flaggen. Eine Gruppe G kombinatorischer
Automorphismen heißt nun {\bf Flaggen-transitiv},\idx{Flaggen-transitiv}
falls es zu jedem Paar von Flaggen des Komplexes einen Automorphismus aus G
gibt, der die erste auf die zweite Flagge abbildet. Ist die Symmetriegruppe
eines kombinatorischen Komplexes flaggentransitiv, so bezeichnet man den
betreffenden Komplex als {\bf regulär}.
\idx{kombinatorischer Komplex!$\sim$regul\ärer}
Bekannt ist diese Eigenschaft vor allem von den (regulären) Platonischen
Körpern, zu denen Analoga gesucht werden, indem man Möglichkeiten der
Realisierung regulärer (CW-)Komplexe untersucht, wie etwa die Dycksche Karte
(vgl. Kap.3).

Jede Symmetrie eines geometrischen Komplexes (eine als Matrix "`darstellbare"'
Symmetrie) induziert nun eine kombinatorische Symmetrie auf dem
zugrundeliegenden kombinatorischen Komplex. Folglich ist die Gruppe der
geometrischen Symmetrien eine Untergruppe der Gruppe der kombinatorischen
Symmetrien eines Komplexes.\\
In der Tat handelt es sich im allgemeinen Fall "`nur"' um eine Untergruppe, es
treten nämlich zumeist mehr kombinatorische als metrische Symmetrien auf (vgl.
die Betrachtung der Dyckschen Karte im dritten Kapitel). Die "`überzähligen"'
kombinatorischen Symmetrien bezeichnet man als versteckte Symmetrien
\idx{versteckte Symmetrien} des realisierbaren kombinatorischen Komplexes mit
dieser Eigenschaft. Im Falle konvexer 3-Polytope im $\E^3$ sind die metrische
und kombinatorische Symmetriegruppe stets gleich (isomorph), doch schon im
nicht-konvexen Fall können versteckte Symmetrien auftreten (vgl. \cite{Bo:91},
Seite 4). Als Beispiel für eine Symmetrieuntersuchung möge ein Torus dienen,
wie er in Kapitel 2, Seite \pageref{torus1} beschrieben ist.

Analog der bisherigen Definitionen werden Permutationen $\sigma\in S_n$ auf
der Grundmenge $E=\{1,\ldots,n\}$ eines orientierten Matroids $\chi$ als
Symmetrien von diesem bezeichnet, wenn es sich bei $\sigma$ um einen
Automorphismus handelt, der $\chi$ auf $\chi$ selbst (Drehung) oder
auf $-\chi$ (Spiegelung, Reflektion) abbildet. Die Gesamtheit aller
solchen Automorphismen eines orientierten Matroids $\chi$ heißt wiederum,
entsprechend ihrer Gruppenstruktur, Symmetriegruppe von $\chi$.
\idx{Symmetriegruppe!$\sim$ orientierter Matroide} Dieser Symmetriebegriff
läßt sich anschaulich an den Symmetrien der durch Pseudosphärensysteme
induzierten Komplexe illustrieren, aber auch an Vorzeichenwechseln des
Vorzeichenvektors eines orientierten Matroids festmachen.

Bei der Untersuchung der möglichen Symmetrien im Raum (hier ist der
dreidimensionale Euklidische Raum gemeint), die im letzten Jahrhundert
erfolgte und deren zentrales Ergebnis zum Ende des 19. Jahrhunderts die
Klassifizierung aller Raumsymmetrien lieferte, stellte sich heraus, daß bei
Raumsymmetrien von Gitterstrukturen, das heißt von periodischen Strukturen,
die den gesamten $\R^3$ ausfüllen, Symmetrieelemente nur mit den Ordnungen
1, 2, 3, 4 und 6 auftreten können. (vgl. etwa die Darstellung der
geschichtlichen Entwicklung der Symmetriekonzepte der Kristallographie von E.
Scholz \cite{Scho:89}). Dieses Ergebnis wurde in der Diplomarbeit von Ronald
Dauster (\cite{Dau:89}, Seite 30ff.) aufgegriffen und auf Realisierungen
orientierter Matroide übertragen. Danach sind affine Realisierungen
simplizialer orientierter Matroide nur mit oben angegebenen Ordnungen möglich,
was zumindest die Verifikation der Symmetrien bei vermeindlichen Realisierungen
unterstützt. Affine Realisierung bedeutet hier gerade, daß das orientierte
Matroid von einer Matrix stammt deren Zeilen explizite Punkte in homogenen
Koordinaten beschreiben, also alle in der selben Hyperebene des $\R^d$ liegen,
also zu $(1,p_i)$ normiert werden können und so einer affinen Punktmenge
im $\R^{d-1}$ entsprechen.

Da wir uns in dieser Arbeit aber mit allgemeinen CW-Komplexen beschäftigen
wollen, infolge dessen gerade eine Erweiterung auf nichtsimpliziale orientierte
Matroide anstreben, haben wir diese Einschränkung der Ordnungen der
Symmetrieelemente leider nicht zur Verfügung. Statt dessen stellt sich die
interessante Frage, welche Symmetrien bei der Realisierung beliebiger orientierter
Matroide auftreten können, ob Einschränkungen existieren oder ob es sogar eine
maximale auftretende Elementordnung gibt. Ein solches Ergebnis würde zu einem
Analogon der Klassifizierung der Raumsymmetrien führen und sicher einer
Bereicherung der Darstellung der Theorie der orientierten Matroide dienen.

\chapter{Von CW-Komplexen zu orientierten Matroiden}

Nach der Bereitstellung der wesentlichen Fundamentalia soll es in den beiden
nun folgenden Kapiteln um die Möglichkeiten gehen, zu einem allgemeinen
d-dimensionalen CW-Komplex orientierte Matroide zu finden, mittels derer
eventuell zugehörige Koordinaten gefunden werden können. Diese sollen dann
das Bild des vorgelegten CW-Komplexes unter einer ho\-möo\-mor\-phen
Abbildung als durchdringungsfreien CW-Komplex (je zwei offene Zellen haben
leeren Schnitt) im Euklidischen Raum beschreiben.

In diesem Kapitel soll dabei die grundlegende Vorgehensweise dargestellt
werden, wie zunächst solche bezüglich Symmetrieeigenschaften des Komplexes
"`verträglichen"' orientierten Matroide bestimmt werden können. Dazu gilt
nämlich folgende Aussage.
\begin{quote}
{\sf
Existiert zu einem vorgelegten CW-Komplex $\C$ kein bezüglich vorgegebener
Symmetrieeigenschaften verträgliches orientiertes Matroid, so ist $\C$ nicht
mit diesen Symmetrien in den Euklidischen Raum einbettbar.
}
\end{quote}
Würde nämlich doch eine solche Einbettung existieren, so gehört zu dieser
eine Punktkonfiguration, die eine Matrix und somit ein orientiertes Matroid
induzieren würde, welches zu $\C$ "`verträglich"' wäre.

Im abschließenden dritten Kapitel sollen die bereitgestellten Mittel dazu
dienen, einige Beispiele zu untersuchen, wobei es als Abschluß um symmetrische
orientierte Matroide zur Dyckschen Karte gehen soll. Auch sollen weitere
Fragestellungen angegeben werden, die sich in diesem Zusammenhang stellen
(vgl. etwa \cite{Bo:86} und \cite{Bo:91}).

\section{Erste Definition der Verträglichkeit}

Zunächst soll allerdings eine Verträglichkeitsdefinition angegeben werden,
die eine allgemeine Brücke zwischen CW-Komplexen und orientierten Matroiden
schlägt.\\
Im Grundlagenteil wurde dazu als wichtiges Ergebnis aus der Theorie der
orientierten Matroide aufgeführt, wie sich unter Anwendung des topologischen
Repräsentationssatzes (Pseudo-)Sphärenarrangements und orientierte Matroide
verbinden lassen. Ausgehend von diesem Satz soll es nun gelingen, dies auch auf
allgemeine CW-Komplexen zu beziehen.\\
Betrachtet man als Ausgangspunkt ein beliebiges (Pseudo-)Sphärensystem auf
der $\S^d$, so ist dort durch das Arrangement eine Zerlegung in Zellen gegeben,
die wegen der Homöomorphieeigenschaften von Sphäre und $\R^d$ auch als
CW-Komplex interpretiert werden kann (vgl. Seite \pageref{cell}). Die maximalen
Zellen dieses Komplexes sind jene d-dimensionalen Gebiete auf der $\S^d$, deren
Rand durch Schnitte und dadurch induzierte Segmente der
(d$-$1)-(Pseudo-)Sphären, die gerade den niederdimensionalen Zellen
entsprechen, gegeben ist. Aufgrund seiner Entstehung sei dieser Komplex als
durch ein (Pseudo-)Sphärensystem induzierter CW-Komplex \idx{induzierter
CW-Komplex} bezeichnet. Handelt es sich bei einem Pseudosphärensystem um die
Darstellung eines orientierten Matroids und ist dieses sogar realisierbar (wenn
das Pseudosphärenarrangement zu einem Sphärenarrangement mit gleicher
Schnittstruktur homöomorph ist), so entspricht das zugehörige Sphärensystem
gerade der Realisierung des durch das Pseudosphärensystem induzierten
CW-Komplexes.\\
Betrachtet man das dem entstandenen Sphärensystem entsprechende
Hyperebenenarrangement, so ist mit dessen zugehörigen Normalenvektoren eine
Koordinatenmatrix gegeben, die ein Chirotop induziert, das ein zum
Sphärensystem äquivalentes orientiertes Matroid beschreibt. Entsprechend der
Definition der Polarität von Seite \pageref{polar} hat man mit diesen Normalen
eine Punktkonfiguration eines zum Ausgangssphärensystem polaren Komplexes. Das
Zusammenspiel der Realisierung eines durch ein Sphärensystem induzierten
Komplexes einerseits und des durch die extrahierte Matrix gegebenen Chirotops
andererseits ist über die Äquivalenz der Darstellungen orientierter Matroide
zu deuten. Aufgrund der verschiedenen Zugangsweisen soll klar unterschieden
werden, ob bei der Untersuchung eines CW-Komplexes von einem
Pseudosphärenarrangement oder einem Chirotop ausgegangen wird. Zunächst wollen
wir so nur die Darstellung eines durch ein Sphärensystem induzierten
CW-Komplexes betrachten. Wie das zugehörige Chirotop in Zusammenhang mit der
Struktur des Komplexes steht, kann dabei als gesonderte Fragestellung
aufgefaßt werden.

Aus obigen Grundüberlegungen ergibt sich nun die entscheidende erste
Definition, die die Verbindung zwischen CW-Komplexen und orientierten Matroiden
charakterisiert:
\begin{quote}
{\bf Ein orientiertes Matroid $\cal M$ soll verträglich zu einem beliebigen
CW-Komplex $\cal C$ heißen, wenn $\cal C$ als Unterkomplex in einem durch ein
$\cal M$ darstellendes Pseudosphärenarrangement induzierten CW-Komplex
enthalten ist.}
\end{quote}
Ist dieses Pseudosphärenarrangement zu einem Sphärensystem homöomorph, so ist
auch eine Realisierung des CW-Komplexes als entsprechende Teilstruktur des
Sphärensystems gegeben.\\
{\scsi
Hier könnte man das Pseudo- und das Sphärensystem auch als homotop bezeichnen,
wenn sich die beiden stetig unter Beibehaltung aller Schnitteigenschaften
ineinander überführen lassen, was man auch als Streckung des
Pseudosphärensystems bezeichnet.
}

\begin{figure}[htb]
$$
\beginpicture
\unitlength0.6cm
\setlinear
\setcoordinatesystem units <0.6cm,0.6cm>
\setplotarea x from -6 to 6, y from -3 to 3
\put {\circle*{0.15}} [Bl] at -5.5 -1.5
\put {\circle*{0.15}} [Bl] at -5.5 0
\put {\circle*{0.15}} [Bl] at -5 -1
\put {\circle*{0.15}} [Bl] at -4.5 0.5
\put {\circle*{0.15}} [Bl] at -4 -0.5
\put {\circle*{0.15}} [Bl] at -4 -1.5
\put {\circle*{0.15}} [Bl] at -3.5 0
\put {\circle*{0.15}} [Bl] at -3 -1
\plot -5.5 -1.5 -5.5 0 -4.5 0.5 -3.5 0 -3 -1 -4 -1.5 -5.5 -1.5 /
\plot -5.5 -1.5 -5 -1 -4 -0.5 -3.5 0 /
\plot -5.5 0 -5 -1 -4 -1.5 /
\plot -4.5 0.5 -4 -0.5 -3 -1 /
\put {\small CW-Komplex $\mapsto$ orientiertes Matroid
      $\mapsto$ Matrix} [Bl] at -6 -2.5
\circulararc 360 degrees from 1 0 center at 0 0
\ellipticalarc axes ratio 3:1 180 degrees from 1 0 center at 0 0
\startrotation by 0.5 0.866
\ellipticalarc axes ratio 3:1 180 degrees from 1 0 center at 0 0
\startrotation by -0.5 0.866
\ellipticalarc axes ratio 3:1 180 degrees from 1 0 center at 0 0
\stoprotation
\put {$\left(\begin{array}{cc}
             \cdot & \cdot \\
             \cdot & \cdot \\
             \cdot & \cdot
             \end{array}\right)$} [Bl] at 4 -0.5
\endpicture
$$
\caption{Der Weg von einem CW-Komplex zu einer möglichen Realisierung}
\label{quest}
\end{figure}

Um von beliebig vorgelegten CW-Komplexen entscheiden zu können, ob diese in
durch Sphä\-rensysteme induzierten CW-Komplexen als Unterkomplexe vorkommen,
muß zu\-nächst untersucht werden, wie solche Unterkomplexe aussehen können.
Dazu sei ein kleines Beispiel betrachtet, welches zeigen soll, wie sich diese
Unterkomplex\-eigenschaft äußert und welche Schwierigkeiten bei der
Betrachtung solcher Pseudosphärensysteme auftreten können.

Als erstes Beispiel betrachten wir dazu einen 1-Komplex, der aus sechs 0-Zellen
und sechs 1-Zellen bestehe, die ein topologisches Sechseck beschreiben (siehe
dazu Abbildung \ref{hexagon}). Der 1-Komplex entspricht dem \ul{Rand}komplex
einer 2-Zelle, so wie auch im allgemeinen Fall d-CW-Komplexe als (Teil des)
Randkomplex(es) einer (d+1)-Zelle gedeutet werden sollen. Dies geschieht in
Verallgemeinerung der Betrachtung von Randstrukturen von Mannigfaltigkeiten
und impliziert damit, in welcher Dimension ein zugehöriges beziehungsweise
verträgliches Sphärensystem zu suchen sein soll.\\
Da wir zum einen Durchdringungsfreiheit garantieren wollen, respektive die
Eigenschaft zu erhalten ist, daß alle offenen Zellen nach Abbildung in den
euklidischen Raum disjunkt sind, sowie zum anderen eine affin lineare
Realisierung angestrebt wird, ist im allgemeinen Fall die kleinste Dimension,
in der mit einer Realisierung zu rechnen ist, um eins höher als die maximale
Dimension der auftretenden Zellen des vorgelegten Komplexes.
Betrachtet man als einfachen Fall etwa ein (d+1)-Simplex, so stellt dessen Rand
gerade einen d-Komplex dar, der eine (d+1)-Zelle umschreibt.
Zusammenfassend soll also ein d-Komplex immer als (Teil der) Berandung
\ul{einer} (d+1)-Zelle aufgefaßt werden, womit als Ausgangssituation die
Realisierbarkeit eines d-dimensionalen CW-Komplexes immer im $\R^{d+1}$ zu
untersuchen ist.\\
Übertragen auf die Sphären bedeutet dies, daß der vorgelegte d-Komplex
die Berandung einer (d+1)-Zelle beschreibt, die auf einer $\S^{d+1}$ liege,
im Rand der (d+2)-Kugel. Die Pseudosphären, die den vorgelegten Komplex
implizieren sollen sind stetige Verformungen der $\S^d$, womit sich als
allgemeine Situation die Betrachtung orientierter Matroide im Rang d+2 ergibt,
entsprechend $\S^{d+1}=\partial B^{d+2}\subset\R^{d+2}$.\\
Nun aber zu unserem Sechseckbeispiel.

\begin{figure}[htb]
$$
\beginpicture
\unitlength0.6cm
\setlinear
\setcoordinatesystem units <0.6cm,0.6cm>
\setplotarea x from -6 to 6, y from -3 to 3.5
\put{ \beginpicture
   \setsolid \thicklines
   \setquadratic
   \plot 0.5 0.5 1.5 0.8 2.5 0.0 /
   \plot 2.5 0.0 2.5 1.0 3.5 1.5 /
   \plot 3.5 1.5 2.8 2.0 3.0 3.0 /
   \plot 3.0 3.0 2.3 2.7 1.5 3.5 /
   \plot 1.5 3.5 1.2 2.6 0.5 2.5 /
   \plot 0.5 2.5 1.1 1.5 0.5 0.5 /
   \setlinear \thinlines
   \put {\circle*{0.15}} [Bl] at 0.5 0.5
   \put {\circle*{0.15}} [Bl] at 2.5 0.0
   \put {\circle*{0.15}} [Bl] at 3.5 1.5
   \put {\circle*{0.15}} [Bl] at 3.0 3.0
   \put {\circle*{0.15}} [Bl] at 1.5 3.5
   \put {\circle*{0.15}} [Bl] at 0.5 2.5
   \endpicture } at 0 0
\put{ \beginpicture
      \setsolid \setlinear
        \ellipticalarc axes ratio 3:1 360 degrees from 2.5 0 center at 0 0
        \startrotation by 0 -1 about 0 0
        \ellipticalarc axes ratio 3:1 360 degrees from 2.5 0 center at 0 0
        \stoprotation
        \plot 0 2.5 0 -2.5 /
        \setdashes<1mm>
        \circulararc 360 degrees from 2.5 0 center at 0 0
        \put {\circle*{0.15}} [Bl] at 0 0.85
        \put {\circle*{0.15}} [Bl] at 0 -0.85
        \put {\circle*{0.15}} [Bl] at 0.8 0.8
        \put {\circle*{0.15}} [Bl] at -0.8 0.8
        \put {\circle*{0.15}} [Bl] at 0.8 -0.8
        \put {\circle*{0.15}} [Bl] at -0.8 -0.8
      \endpicture } at 5 0
\put{ \beginpicture
        \setsolid \setlinear
        \ellipticalarc axes ratio 3:1 360 degrees from 2.5 0 center at 0 0
        \startrotation by 0.5 0.866 about 0 0
        \ellipticalarc axes ratio 3:1 360 degrees from 2.5 0 center at 0 0
        \stoprotation
        \startrotation by -0.5 0.866 about 0 0
        \ellipticalarc axes ratio 3:1 360 degrees from 2.5 0 center at 0 0
        \stoprotation
        \setdashes<1mm>
        \circulararc 360 degrees from 2.5 0 center at 0 0
        \put {\circle*{0.15}} [Bl] at 0.95 0
        \put {\circle*{0.15}} [Bl] at -0.95 0
        \put {\circle*{0.15}} [Bl] at 0.475 0.823
        \put {\circle*{0.15}} [Bl] at -0.475 0.823
        \put {\circle*{0.15}} [Bl] at 0.475 -0.823
        \put {\circle*{0.15}} [Bl] at -0.475 -0.823
      \endpicture } at -5 0
\endpicture
$$
\caption{Einbettung eines topologischen Sechsecks}
\label{hexagon}
\end{figure}

Wie schon Abbildung \ref{hexagon} einer Darstellung des topologischen Sechsecks
zeigt, gibt es neben der anschaulichen Realisierung als ebenes Sechseck (die
"`übliche"') in einem Sphärensystem noch eine weitere Möglichkeit, ein
Sechseck als Unterkomplex zu finden. Dieser zweite Fall zeichnet sich sogar
dadurch aus, daß die Anzahl der auftretenden Sphären minimal ist (fünf statt
der sechs in der "`üblichen"' Darstellung). Zur Charakterisierung dieser
beiden Möglichkeiten kann zum einen ein beliebiger d-dimensionaler CW-Komplexes
als Randkomplex eines einzigen Topes (Gebietes auf der $\S^{d+1}$) auftreten
oder aber auch der Randstruktur einer Vereinigung von (benachbarten) Topes
entsprechen. Im Falle mehrerer beteiligter Topes ist die durch den vorgelegten
Komplex umschriebene (d+1)-Zelle das Innere der Vereinigung der Abschlüsse der
auftretenden Regionen.

Wird von einer weiteren Realisierungsmöglichkeit des Sechsecks (einer
nichtkonvexen) zu einem zugehörigen Sphärenarrangement übergegangen, so
zeigt sich, daß die beiden zuerst aufgeführten Fälle die im Sinne der
Unterkomplexeigenschaft einzigen Möglichkeiten sind, den Komplex in
einem Sphärensystem darzustellen. Läßt man nämlich zu, daß Segmente einer
(Pseudo-)Sphäre auch mehrmals zur Beschreibung vorgelegter Zellen Verwendung
finden, das heißt wenn nicht benachbarte Teile einer (Pseudo-)Sphäre
verschiedenen Zellen des gegebenen CW-Komplexes entsprechen sollen (vgl.
Abb.\ref{linehex}), kann hier die Unterkomplexeigenschaft verletzt werden.

\begin{figure}[htb]
$$
\beginpicture
\unitlength0.6cm
\setlinear
\setcoordinatesystem units <0.6cm,0.6cm>
\setplotarea x from -6 to 6, y from -3 to 3
\put{ \beginpicture
      \setsolid \setlinear
      \plot 0 1 -2.5 0 -1 0 0 -1 1 0 2.5 0 0 1 /
      \put {\circle*{0.15}} [Bl] at 0 1
      \put {\circle*{0.15}} [Bl] at -2.5 0
      \put {\circle*{0.15}} [Bl] at -1 0
      \put {\circle*{0.15}} [Bl] at 0 -1
      \put {\circle*{0.15}} [Bl] at 1 0
      \put {\circle*{0.15}} [Bl] at 2.5 0
      \endpicture } at -3 0
\put{ \beginpicture
        \setsolid \setlinear
        \ellipticalarc axes ratio 3:1 -180 degrees from 2.5 0 center at 0 0
        \startrotation by 0.819 0.574 about 0 0
        \ellipticalarc axes ratio 3:1 -180 degrees from 2.5 0 center at 0 0
        \stoprotation
        \startrotation by 0.766 -0.643 about 0 0
        \ellipticalarc axes ratio 3:1 -180 degrees from 2.5 0 center at 0 0
        \stoprotation
        \startrotation by 0.643 0.766 about 0 0
        \ellipticalarc axes ratio 3:1 180 degrees from 2.5 0 center at 0 0
        \stoprotation
        \startrotation by 0.5 -0.866 about 0 0
        \ellipticalarc axes ratio 3:1 --180 degrees from 2.5 0 center at 0 0
        \stoprotation
        \setdashes<1mm>
        \circulararc 360 degrees from 2.5 0 center at 0 0
        \put {\circle*{0.15}} [Bl] at 0.11 1.29
        \put {\circle*{0.15}} [Bl] at 1.24 -0.71
        \put {\circle*{0.15}} [Bl] at -1.45 -0.68
        \put {\circle*{0.15}} [Bl] at -0.3 -0.82
        \put {\circle*{0.15}} [Bl] at 0.27 -0.83
        \put {\circle*{0.15}} [Bl] at -0.05 -1
        \put {\circle{0.25}} [Bl] at -1.12 -0.09
        \put {\circle{0.25}} [Bl] at  1.05 -0.23
      \endpicture } at 3 0
\endpicture
$$
\caption{Ein "`verbotenes"' Sphärensystem}
\label{linehex}
\end{figure}

Was bei der polygonalen Darstellung des Sechsecks plausibel erscheint,
widerspricht der Definition der induzierten Zellen im Sphärenarrangement, nach
der jeder Schnitt von (Pseudo-)Sphären immer einer niederdimensionalen Zelle
entspricht. Dies ist in diesem Fall nicht gewährleistet. Vielmehr würde es
sich hier um die Realisierung eines topologischen Achtecks im Rand der
Vereinigung von vier Topes handeln. Im Sechseck sind die beiden zusätzlichen
0-Zellen (nicht ausgefüllte Kreise in Abb. \ref{linehex}) in der Realisierung
nicht vorhanden. Solch eine Situation soll grundsätzlich verboten sein, denn
sind $S_1$ und $S_2$ zwei benachbarte Zellen des induzierten Komplexes
($\ol{S_1}\cap\ol{S_2}\neq\emptyset$), so ist zwar int($\ol{S_1}\cup\ol{S_2}$)
nach Definition eine Zelle, sie ist aber kein Element des durch das
Sphärensystem induzierten Komplexes und verletzt somit die
Unterkomplexeigenschaft. Der betrachtete Unterkomplex des induzierten
CW-Komplexes hätte zwar den selben zugrundeliegenden Raum, aber mehr
Zellen als die Realisierung des vorgelegten CW-Komplexes, was nicht erlaubt ist.

Mit diesem ersten Beispiel läßt sich die Definition von oben nun weiter
präzisieren.
\begin{quote}
{\bf Ein orientiertes Matroid $\cal M$ soll verträglich zu einem beliebigen
CW-Komplex $\cal C$ heißen, wenn $\cal C$ dem Randkomplex des Abschlusses
einer Vereinigung von Topes eines $\cal M$ darstellenden
Pseudosphärenarrangements entspricht.}
\end{quote}

Zur Vertiefung des gerade Beschriebenen und als Ausgangsbasis für die
Untersuchung allgemeiner zweidimensionaler CW-Komplexe, zu der im dritten
Kapitel weiter Beispiele behandelt werden sollen, möge als zweites
Beispiel ein 2-Komplex betrachtet werden, der den Rand eines Torus beschreibt
(vgl. dazu auch \cite{Dau:89} und \cite{BoWi:87}). Der vorgelegte CW-Komplex
sei dazu nach Abbildung \ref{torus1} gegeben.

\begin{figure}[htb]
$$
\input torus
\settorus
$$
\caption{Torus}
\label{torus2}
\end{figure}

Nach der Beschreibung aller Eigenschaften des Torus, die sich mit den bisher
eingeführten Begriffen angeben lassen, soll ein Algorithmus entwickelt werden,
der die Vorgehensweise im generellen Fall kombinatorischer Komplexe
unterstützen soll.

\section{Von einem CW-Komplex zu einem Algorithmus}

Es soll nun ein Torus im $\R^3$ betrachtet werden, dessen kombinatorische
Struktur als zweidimensionaler CW-Komplex nach Abbildung \ref{torus1}
durch Kantenidentifikation gegeben sei. Entgegen der Betrachtung der
Einbettung triangulierter Tori (vgl. etwa \cite{Dau:89} und \cite{BoEg:91}),
sei hier gerade der Nichtsimplizialität allgemeiner CW-Komplexe Rechnung
getragen und eine Zerlegung in Vierecke betrachtet.

\begin{figure}[htb]
$$
\beginpicture
\unitlength0.6cm
\setlinear
\setcoordinatesystem units <0.6cm,0.6cm>
\setplotarea x from -3.5 to 3.5, y from -2 to 2
\plot -3 1.5 3 1.5 3 -1.5 -3 -1.5 -3 1.5 /
\plot -1 1.5 -1 -1.5 /
\plot  1 1.5  1 -1.5 /
\plot -3 0.5 3 0.5 /
\plot -3 -0.5 3 -0.5 /
\put {1} [br] at -3.1 1.6  \put {1} [bl] at  3.1 1.6  \put {1} [tl] at  3.1 -1.6
\put {1} [tr] at -3.1 -1.6 \put {2} [br] at  -1 1.6   \put {2} [tr] at  -1 -1.6
\put {3} [bl] at  1 1.6    \put {3} [tl] at  1 -1.6   \put {4} [br] at -3.1 0.5
\put {4} [bl] at  3.1 0.5  \put {5} [br] at -1.1 0.6  \put {6} [bl] at  1.1 0.6
\put {7} [tr] at -3.1 -0.5 \put {7} [tl] at  3.1 -0.5 \put {8} [tr] at -1.1 -0.6
\put {9} [tl] at  1.1 -0.6
\endpicture
$$
\caption{CW-Komplex "`Rand eines Torus"'}
\label{torus1}
\end{figure}

Der vorgelegte Komplex besitzt neun 0-Zellen, 18 1-Zellen und neun 2-Zellen,
nach der Eulerformel also Geschlecht Eins. Da von der Wahl der Zellen her schon
auf Realisierungen geschlossen werden kann (so wie eine mögliche in Abbildung
\ref{torus3} zu sehen ist), wäre es nun interessant herauszufinden, wie sich
dieser Komplex als Unterkomplex in ein Sphärensystem einfügt.

Was allerdings bei der Betrachtung etwa einer Pyramide oder eines anderen
konvexen 3-Polytops mit einer geringen Zellenanzahl noch anschaulich
verständlich erscheint (indem man sich 2-Sphären so "`ineinandergesteckt"'
denkt, daß pro 2-Zelle des Komplexes eine Sphäre diese durch einen Auschnitt
repräsentiert), gestaltet sich schon hier, insbesondere bei noch größerer
Zellenanzahl und einem Nicht-Konvexität induzierenden topologischen Geschlecht,
schon schwieriger. Mangels Übersichtlichkeit in der möglichen Darstellung als
Sphärensystem soll deshalb eine Alternative zur Suche nach orientierten
Matroiden angewandt werden, die unter Berücksichtigung der Symmetriegruppe des
vorgelegten CW-Komplexes zugehörige Chirotope bestimmen helfen soll, was eine
zweite Verträglichkeitsdefinition liefert.

\begin{figure}[htb]
$$
\input tritor
\settritor
$$
\caption{Eine mögliche Realisierung des Torus}
\label{torus3}
\end{figure}

Da Chirotope durch (formale) Matrizen bestimmt werden und dadurch implizit eine
eventuelle Punktkonfigurationen betrachtet wird, ist es wenig sinnvoll, eine
Verträglichkeitsdefinition von Chirotopen zu beliebigen CW-Komplexen anzugeben.
Vielmehr soll der Fall simplizialer Komplexe dahingehend erweitert werden, daß
nun mehr als (r+1) 0-Zellen/Punkte pro r-Seite zugelassen seien, in der Sprache
der (n$\times$r)-Matrizen also auch singuläre (r$\times$r)-Untermatrizen
auftreten können. Im Fall von 2-Komplexen fordert dies von der Gestalt der
vorgelegten CW-Komplexe, daß alle maximalen Zellen gleicher Dimension seien und
der Rand des Abschlusses einer jeden 2-Zelle einem Polygonzug,
im höherdimensionalen Fall einem Polytop, entspreche, der die 0- und 1-Zellen,
sowie deren Inzidenzen induziert. In Aigners Buch zur
Kombinatorik \cite{Aig:76} heißen solche 2-dimensionalen CW-Komplexe
Landkarten, was mit der Definition der Karten aus dem Abschnitt über
Symmetriebegriffe gleichzusetzen ist. Ausgehend von den 0-Zellen ergibt sich
mittels deren Sterne wieder der Komplex, was das Vorhandensein der nötigen n
Punkte für Chirotope, zusammen mit dem Rang r über die Dimension der maximalen
Zellen + 2 sicherstellt.\\
Nun ist aber ein Chirotop noch nicht verträglich mit einem CW-Komplex nur
aufgrund der Tatsache, daß die Anzahl der Punkte und der Rang übereinstimmen.
Als weitere Kopplung dient die Eigenschaft in gewissem Sinne gleiche
Symmetrieeigenschaften zu haben. Dazu sei daran erinnert, daß die Realisierung
eines kombinatorischen Komplexes eine Untergruppe der Symmetriegruppe von diesem
als eigene Symmetriegruppe besitzt. Die Realisierung weist hierbei den 0-Zellen
aber gerade jene Koordinaten zu, die in eine Matrix geschrieben, ein
zugehöriges Chirotop liefern. Damit kommt man nun zu folgender zweiter
Verträglichkeitsdefinition von CW-Komplexen und orientierten Matroiden.

\begin{quote}
{\bf Ein Chirotop $\chi$ soll verträglich zu einem polyedrisch begrenzten
CW-Komplex $\cal C$ heißen, wenn Punktanzahl und Rang übereinstimmen und
die Automorphismengruppe ${\cal A}(\chi)$ eine Untergruppe der Symmetriegruppe
${\cal A}({\cal C})$ des CW-Komplexes ist.}
\end{quote}

Nun soll für das Torusbeispiel zunächst die Symmetriegruppe des vorgelegten
Komplexes bestimmt werden. Dies kann mittels eines kurzen Programmes geschehen,
dessen Aufbau sich wie folgt gliedert.

\subsection{Bestimmung der Symmetriegruppe}

Vorgelegt sei eine Liste, die angibt, welche 0-Zellen im Rand des Abschlusses
jeder 2-Zelle eines vorgelegten 2-CW-Komplexes liegen. Die 2-Zellen seien dabei
durch geschlossene Polygonzüge begrenzt, die durch die 0-Zellen und sie
verbindende 1-Zellen in den Rändern der Abschlüsse der 2-Zellen gegeben
sind. Dies geschieht entsprechend obiger Definition und ist gerade im Hinblick
auf die Untersuchung von Karten von Mannigfaltigkeiten sehr zweckmäßig.\\
Zusätzlich sei auf den Rändern der 2-Zellen willkürlich eine Orientierung in
Form eines Durchlaufsinnes eingeführt. Diese soll dazu dienen, auf die
tatsächlich vorhandenen 1-Zellen, also Kanten, zurückschliessen zu können.
{\scsi
Entgegen simplizialer Komplexe ist eine 0-Zelle/Ecke im Rand einer 2-Zelle/Seite
im allgemeinen nicht mit allen anderen Ecken der Seite durch eine 1-Zelle/Kante
verbunden, was bedeutet, daß die Valenz einer Ecke im allgemeinen kleiner oder
gleich der Gesamt\-eckenzahl $-1$ ist, was gesondert berücksichtigt werden
muß.
}

Die entsprechende Zellenliste sei als Datei folgender Form vorgelegt:
\begin{verbatim}
torus.gon

4
1254 2365 3146 4587 5698 6479 7821 8932 9713*
\end{verbatim}
{\scsi
Die Datei {\it torus.gon} gliedert sich wie folgt: {\bf 4} besagt, daß pro
2-Zelle vier 0-Zellen vorkommen, also Vierecke; die folgende Liste enthält die
0-Zellen im Rand der abgeschlossenen 2-Zellen entsprechend ihres Durchlaufsinns.
}

Beim Einlesen der Liste wird die maximale Anzahl der vorhandenen 0-Zellen
bestimmt und als {\it npkt} gespeichert. In einer
({\it npkt}$\times${\it npkt})-Binärmatrix (Adjazenzmatrix) wird dabei auch
gespeichert, welche 0-Zellen durch 1-Zellen "`verbunden"' sind, also welche
"`Kanten"' vorliegen. Anschließend kann die Bestimmung der Symmetriegruppe
beginnen. Dazu werden in einer rekursiven Routine alle {\it npkt}! Permutationen
der 0-Zellen durchlaufen und als Abbildungsvorschrift $\pi$ an die
Symmetrieprüfroutine übergeben (der Algorithmus zur Erzeugung der
Permutationen stammt aus dem Algorithmenbuch von R. Sedgewick \cite{Sed:91}).
In dieser wird getestet, ob $\pi$ jede 2-Zelle auf eine vorhandene
2-Zelle abbildet und ob in diesem Fall deren Rand respektive Kanten der
Orientierung entsprechend durchlaufen werden. Erfüllt $\pi$ dies für alle
vorgelegten 2-Zellen, so wird $\pi$ in die Symmetriegruppe übernommen, sonst
gestrichen und mit der nächsten Permutation fortgefahren.\\
Nachdem alle Permutationen durchlaufen sind, liegen jene Elemente der
Permutationsgruppe $S_{\it npkt}$ vor, die einen Automorphismus auf dem
vorgelegten Komplex beschreiben. Diese werden als Symmetriegruppe in einer
Datei gespeichert. Angemerkt sei, daß durch den Durchlauf aller Permutationen
für eine große Anzahl von 0-Zellen sehr viel Rechenzeit nötig ist.
{\scsi
(Für die acht 0-Zellen eines 3-Würfels etwa eine Sekunde, obiger Torus wird
in etwa 10 Sekunden abgearbeitet, die zwölf 0-Zellen der Dyckschen Karte
beanspruchen dann schon etwas über einer Stunde -- gemessen auf einem
66 MHz-i486-PC mit einem C-Programm unter Linux.)} Für obigen Torus ergeben
sich so folgende Symmetrien (in Zykelschreibweise):
\begin{verbatim}
No.1 : identity                     No.37 : (162435)(798)
No.2 : (47)(58)(69)                 No.38 : (195)(276)(384)
No.3 : (23)(56)(89)                 No.39 : (1675)(2398)
No.4 : (23)(47)(59)(68)             No.40 : (195)(236478)
No.5 : (24)(37)(68)                 No.41 : (1576)(2893)
No.6 : (2734)(5896)                 No.42 : (186)(254793)
No.7 : (2437)(5698)                 No.43 : (153426)(789)
No.8 : (27)(34)(59)                 No.44 : (186)(294)(375)
No.9 : (12)(45)(78)                 No.45 : (16)(25)(34)(79)
No.10 : (12)(48)(57)(69)            No.46 : (194376)(285)
No.11 : (132)(465)(798)             No.47 : (16)(29)(57)
No.12 : (132)(495768)               No.48 : (1926)(4785)
No.13 : (1452)(3768)                No.49 : (125697)(384)
No.14 : (179652)(348)               No.50 : (1287)(3594)
No.15 : (146982)(375)               No.51 : (1397)(2684)
No.16 : (1782)(3495)                No.52 : (136587)(294)
No.17 : (123)(456)(789)             No.53 : (147)(258)(369)
No.18 : (123)(486759)               No.54 : (17)(28)(39)
No.19 : (13)(46)(79)                No.55 : (147)(268359)
No.20 : (13)(49)(58)(67)            No.56 : (17)(29)(38)(56)
No.21 : (145893)(276)               No.57 : (1538)(4697)
No.22 : (1793)(2486)                No.58 : (18)(35)(49)
No.23 : (1463)(2759)                No.59 : (168)(239745)
No.24 : (178563)(249)               No.60 : (1948)(2365)
No.25 : (1254)(3867)                No.61 : (157248)(369)
No.26 : (128964)(357)               No.62 : (18)(27)(39)(45)
No.27 : (14)(25)(36)                No.63 : (168)(249)(357)
No.28 : (174)(285)(396)             No.64 : (192738)(465)
No.29 : (14)(26)(35)(89)            No.65 : (159)(287463)
No.30 : (174)(295386)               No.66 : (1849)(2563)
No.31 : (139854)(267)               No.67 : (1629)(4587)
No.32 : (1364)(2957)                No.68 : (19)(26)(48)
No.33 : (15)(38)(67)                No.69 : (159)(267)(348)
No.34 : (1835)(4796)                No.70 : (183729)(456)
No.35 : (15)(24)(36)(78)            No.71 : (167349)(258)
No.36 : (184275)(396)               No.72 : (19)(28)(37)(46)
\end{verbatim}
Der vorgelegte Torus hat also eine kombinatorische Symmetriegruppe der
Ordnung 72. Nun gilt es Chirotope zu finden, die zumindest eine Untergruppe
dieser Symmetriegruppe als eigene Symmetriegruppe besitzen. Diese können dann
im Falle ihrer Realisierbarkeit dazu eingesetzt werden, zu dem vorgelegten
Komplex Koordinaten zu finden.

Zunächst müssen dazu aus der vorgelegten Symmetriegruppe $\cal A$ Untergruppen
G extrahiert werden, die als Symmetriegruppen zugehöriger Chirotope in
Betracht kommen. Genauer soll die Anzahl der zu suchenden Chirotope auf
die Anzahl derer beschränkt werden, die die gewünschten Symmetrieeigenschaften
besitzen. Die betreffenden Untergruppen können wieder mittels eines Programms
erzeugt werden, welches ausgehend von den Symmetrien $\sigma\in {\cal A}$ mit
der höchsten Elementordnung ord($\sigma$) (es ist eine Realisierung mit
möglichst hoher Symmetrie angestrebt) deren zyklische Gruppen
($\{id,\sigma,\sigma^2,\ldots,\sigma^{ord(\sigma)-1}\}$) bestimmt und alle
erzeugten Elemente aus $\cal A$ streicht, so daß am Ende die verschiedenen
maximalen zyklischen Untergruppen der vorgelegten Symmetriegruppe aufgelistet
werden. Die Wahl solcher zyklischen Untergruppen begründet sich in der leichten
Erzeugbarkeit und dient der Tatsache, daß solche Gruppen wahrscheinlicher als
Symmetriegruppen von Chirotopen angenommen werden, als etwa größere
zusammengesetzte. Für den Torus sind dies (die Zahlen in den eckigen Klammern
geben die Nummern der Elemente der vorgelegten obigen Automorphismengruppe an):
\begin{verbatim}
No.1 : { id, [12], [17], [2], [11], [18] }
No.2 : { id, [14], [38], [47], [69], [49] }
No.3 : { id, [15], [63], [68], [44], [26] }
No.4 : { id, [21], [69], [58], [38], [31] }
No.5 : { id, [24], [44], [33], [63], [52] }
No.6 : { id, [30], [53], [3], [28], [55] }
No.7 : { id, [36], [53], [9], [28], [61] }
No.8 : { id, [37], [17], [27], [11], [43] }
No.9 : { id, [40], [69], [5], [38], [65] }
No.10 : { id, [42], [63], [8], [44], [59] }
No.11 : { id, [46], [53], [19], [28], [71] }
No.12 : { id, [64], [17], [54], [11], [70] }
No.13 : { id, [6], [4], [7] }
No.14 : { id, [13], [35], [25] }
No.15 : { id, [16], [62], [50] }
No.16 : { id, [22], [72], [51] }
No.17 : { id, [23], [45], [32] }
No.18 : { id, [34], [20], [57] }
No.19 : { id, [39], [56], [41] }
No.20 : { id, [48], [10], [67] }
No.21 : { id, [60], [29], [66] }
No.22 : { id }
\end{verbatim}

Mittels dieser Untergruppen kann man sich nun auf die Suche nach etwaig
zugehörigen Chirotopen begeben, was wie folgt geschehen soll.

\subsection{Erzeugung verträglicher Chirotope}

Mit der Anzahl $npkt$ der vorkommenden 0-Zellen und der Dimension $d$ der
maximalen Zellen des vorgelegten CW-Komplexes ist die Gestalt "`möglich
zugehöriger"', respektive verträglicher Chirotope festgelegt.\\
Sie sind darstellbar durch Listen mit insgesamt ${npkt \choose d+2}$
Vorzeichen aus $\{-,0,+\}$, im Fall des Torus also mit ${9 \choose 4}=126$
Elementen, die den Vorzeichen der formalen Brackets $[\lambda]$ mit
$\lambda\in\Lambda(npkt,d+2)$ entsprechen.\\
Da die Anzahl der Möglichkeiten diese ${npkt \choose d+2}$ Vorzeichen
aufzufüllen mit $3^{npkt \choose d+2}$ (im Fall des Torus
$3^{126}\geq 10^{60}$) recht groß ist, soll mittels zusätzlicher Bedingungen
diese Anzahl verringert werden.\\
Dazu kann zuerst einmal die Gestalt der vorgelegten 2-Zellen dienen. Da diese
in der Realisierung als Facetten im Rand der entstehenden 3-Zelle eben sein
sollen, können die 0-Zellen im Rand ihrer Abschlüsse, bei einer Anzahl
größer als drei, entsprechend als linear abhängige Punkte gedeutet werden.
Da die Vorzeichen zu (zunächst) formalen Determinanten gehören, können die
Vorzeichen jener Brackets zu 0 gesetzt werden, die eine Auswahl der Indizes der
0-Zellen des Randes ein und derselben abgeschlossenen 2-Zelle enthalten.\\
Für den Torus sind so die Vorzeichen zu den Brackets $[1245]$, $[1278]$,
$[1346]$, $[1379]$, $[2356]$, $[2389]$, $[4578]$, $[4679]$ und $[5689]$ zu 0
zu setzen, was die Anzahl der Auf\-füll\-mög\-lich\-kei\-ten von $3^{126}$ auf
immerhin schon $3^{117}< 10^{56}$ reduziert.\\
Mit den zu erfüllenden Symmetrien läßt sich diese Zahl noch weiter
verringern. Dies geschieht durch Kopplung der Brackets unter den Elementen
$\sigma$ einer Symmetriegruppe G, die entsteht, wenn man die Orbits der Brackets
$[\lambda]$, die Menge aller Bilder der $[\lambda]$ unter den $\sigma\in G$, als
Äquivalenzklassen betrachtet.\\
Für den vorgelegten Torus sei als zu erfüllende Symmetrieeigenschaft der
Realisierung im folgenden als Beispiel die Untergruppe G =
$\{ id, [12], [17], [2], [11], [18] \}$ vorgelegt. Mit dieser erhält man eine
Kopplung, durch die mit der Vorgabe von 23 Bracketvorzeichen für
Repräsentanten der Orbits alle übrigen Vorzeichen bestimmbar sind. In Tabelle
\ref{tab} sind in den Zeilen die 23 verschiedenen Bracketorbits aufgelistet,
von denen die Brackets der ersten Spalte, die entsprechend lexikographischer
Ordnung minimal sind, als Repäsentanten gewählt werden.
Die Vorzeichen vor den übrigen Brackets sind die jeweils durch die
lexikographische Ordnung nach der Permutation zu berücksichtigen
Vorzeichenwechsel.

\begin{table}[htb]
{\small
$$
\begin{array}{cccccc}
Identität                   & {123456789\choose 312978645} &
{123456789\choose 231564897} & {123456789\choose 123789456} &
{123456789\choose 312645978} & {123456789\choose 231897564}\\
        &         &         &         &         &        \\
+[1234] & +[1239] & +[1235] & +[1237] & +[1236] & +[1238]\\
+[1245] & +[1379] & +[2356] & +[1278] & +[1346] & +[2389]\\
+[1246] & +[1389] & -[2345] & +[1279] & +[1356] & -[2378]\\
+[1247] & +[1369] & +[2358] & -[1247] & -[1369] & -[2358]\\
+[1248] & +[1349] & +[2359] & -[1257] & -[1367] & -[2368]\\
+[1249] & +[1359] & +[2357] & -[1267] & -[1368] & -[2348]\\
+[1256] & -[1378] & -[2346] & +[1289] & -[1345] & -[2379]\\
+[1258] & +[1347] & +[2369] & -[1258] & -[1347] & -[2369]\\
+[1259] & +[1357] & +[2367] & -[1268] & -[1348] & -[2349]\\
+[1269] & +[1358] & +[2347] & -[1269] & -[1358] & -[2347]\\
+[1456] & +[3789] & +[2456] & +[1789] & +[3456] & +[2789]\\
+[1457] & -[3679] & +[2568] & +[1478] & -[3469] & +[2589]\\
+[1458] & -[3479] & +[2569] & +[1578] & -[3467] & +[2689]\\
+[1459] & -[3579] & +[2567] & +[1678] & -[3468] & +[2489]\\
+[1467] & -[3689] & -[2458] & +[1479] & -[3569] & -[2578]\\
+[1468] & -[3489] & -[2459] & +[1579] & -[3567] & -[2678]\\
+[1469] & -[3589] & -[2457] & +[1679] & -[3568] & -[2478]\\
+[1489] & +[3459] & -[2579] & +[1567] & +[3678] & -[2468]\\
+[1568] & +[3478] & -[2469] & +[1589] & +[3457] & -[2679]\\
+[1569] & +[3578] & -[2467] & +[1689] & +[3458] & -[2479]\\
+[4567] & -[6789] & +[4568] & -[4789] & +[4569] & -[5789]\\
+[4578] & +[4679] & +[5689] & +[4578] & +[4679] & +[5689]\\
+[4579] & +[5679] & -[5678] & +[4678] & +[4689] & -[4589]
\end{array}$$}
\caption{Orbits der Brackets unter einer gewählten Symmetrieuntergruppe}
\label{tab}
\end{table}

Da bereits bestimmt wurde, welche Brackets sicherlich zu 0 zu setzen sind,
sind für eine vollständige Vorzeichenliste noch 21 Vorzeichen vorzugeben.
Allerdings reduziert sich die Anzahl der zu untersuchenden Vorzeichenlisten
noch nicht auf $3^{21}$, da Symmetrien $\sigma$ Chirotope $\chi$ sowohl auf
$\chi$ selbst, als auch auf $-\chi$ abbilden können, was von vornherein
nicht abzusehen ist und zu einer Gesamtzahl von möglichen Listen von, im Fall
des Torus, $2^5\cdot 3^{21}< 3.5\cdot 10^{11}$ führt. Der Faktor $2^5$ rührt
daher, daß pro Symmetrie ungleich der Identität $\chi$ oder $-\chi$ vorliegen
kann, also pro Permutation aus G auf $\chi$ und $-\chi$ zu testen ist.
\label{symmtest} Selbst diese Zahl von Vorzeichenlisten ist mit fast 335
Milliarden noch ziemlich hoch, weshalb auch diese Zahl durch weitere
Überlegungen reduziert werden soll, was wie folgt geschieht.

\subsection{Reduktion auf verträgliche Matroide}

Dazu nutzen wir die Tatsache, daß die Vorzeichenlisten letztendlich Chirotopen
entsprechen sollen und daß mit jedem Chirotop ein diesem zugrunde liegendes
Matroid gegeben ist. Von der Matroidseite her betrachtet, muß also zunächst
ein solches vorliegen, um es zu einem Chirotop orientieren zu können.
Betrachtet man die Brackets als Basen eines Matroids, so reduzieren sich nach
der Definition, die das Erfülltsein der Graßmann-Plücker-Relationen über
GF(2) fordert, die Auffüllmöglichkeiten für die "`Bracketvorzeichen"' auf
$2^{21}$, entsprechend der Möglichkeit, für jede Bracket 0 oder 1 zu setzen.
Der nächste Schritt soll dementsprechend der Erzeugung bezüglich der
vorgelegten Symmetrien verträglicher Matroide gewidmet sein. Zur ersten
Vorauswahl werden dazu die unter der Symmetriegruppe in Äquivalenzklassen
aufgeteilten 3-summandigen Graßmann-Plücker-Relationen untersucht.

Zur Erinnerung lassen sich die 3-summandigen Graßmann-Plücker-Relationen
mittels zweier Mengen $A:=\{a_1,\ldots,a_{r-2}\}$ und $B:=\{b_1,\ldots,b_4\}$
paarweise verschiedener Elemente aus $E=\{1,\ldots,npkt\}$ im Rang r
darstellen als
$$\{A|B\}=
  \begin{array}{l}
    +[a_1,\ldots,a_{r-2},b_1,b_2]\cdot [a_1,\ldots,a_{r-2},b_3,b_4]\\
    -[a_1,\ldots,a_{r-2},b_1,b_3]\cdot [a_1,\ldots,a_{r-2},b_2,b_4]\\
    +[a_1,\ldots,a_{r-2},b_1,b_4]\cdot [a_1,\ldots,a_{r-2},b_2,b_3]
  \end{array}=0$$
was zu ${npkt \choose r-2}\cdot{npkt-r+2 \choose 4}$, im Falle des Torus also
${9\choose 2}\cdot{7\choose 4}=1260$ verschiedenen Gleichungen führt.
Betrachtet man diese Gleichungen über dem zweielementigen Körper GF(2), so
müssen äquivalente Gleichungen vom Typ $X + Y + Z + XYZ = 0$ erfüllt sein, in
denen die X, Y und Z den Beträgen der Vorzeichen obiger Produkte entsprechen.
Der Summand $XYZ$ als Produkt über alle in der Gleichung vorkommenden Brackets
muß bei einer Betrachtung über GF(2) hinzugefügt werden, um auch die
erlaubten Fälle mit gleichzeitig $X=1$, $Y=1$ und $Z=1$ abzudecken.

Da die zu entstehenden Matroide bezüglich der vorgelegten Symmetriegruppe G
verträglich sein sollen, braucht man, analog der Brackets, nur Repräsentanten
der Orbits der Graßmann-Plücker-Relationen unter G betrachten, was deren
Anzahl auf 215 reduziert. Ein Teil dieser ist in Tabelle \ref{orbtab}
aufgezeigt.

\begin{table}[htb]
{\small
$$
\begin{array}{ccccc}
\{12|3456\} & \{12|3457\} & \{12|3458\} & \{12|3459\} & \{12|3467\} \\
\{12|3468\} & \{12|3469\} & \{12|3489\} & \{12|3568\} & \{12|3569\} \\
\{12|4567\} & \{12|4568\} & \{12|4569\} & \{12|4578\} & \{12|4579\} \\
\{12|4589\} & \{12|4679\} & \{12|4689\} & \{12|5689\} & \{14|2356\} \\
\ldots      &             &             &             &             \\
\{48|1257\} & \{48|1259\} & \{48|1267\} & \{48|1269\} & \{48|1279\} \\
\{48|1356\} & \{48|1357\} & \{48|1359\} & \{48|1367\} & \{48|1369\} \\
\{48|1379\} & \{48|1567\} & \{48|1569\} & \{48|1579\} & \{48|1679\} \\
\{48|2356\} & \{48|2357\} & \{48|2359\} & \{48|2367\} & \{48|2369\} \\
\{48|2379\} & \{48|2567\} & \{48|2569\} & \{48|2579\} & \{48|2679\} \\
\{48|3567\} & \{48|3569\} & \{48|3579\} & \{48|3679\} & \{48|5679\} \\
\end{array}$$}
\caption{Repräsentanten der Orbits der GPR unter G}
\label{orbtab}
\end{table}

Zu beachten ist hierbei, daß eine Permutation auf den Elementen einer
k-sum\-man\-di\-gen Graßmann-Plücker-Relation das Vorzeichen des
repräsentierten Polynoms ändert. Dies geschieht für Permutationen
$\pi_1:A\to A$, $\pi_2:B\to B$ und $\pi_3:C\to C$ gemäß
$$\{\pi_1(A)|\pi_2(B)|\pi_3(C)\} =
\mbox{sgn}(\pi_1)\cdot\mbox{sgn}(\pi_2)\cdot\mbox{sgn}(\pi_3)\cdot\{A|B|C\}$$
Bei Betrachtungen der Relationen über GF(2) ist dies nicht zu berücksichtigen,
wohl aber im Falle orientierter Matroide über GF(3).

Setzt man in diese 215 Gleichungen wiederum ein, welche Brackets unter der
Symmetriegruppe G bezüglich ihrer Orbits gleich beziehungweise welche Null
sind und streicht alle redundanten Gleichungen (doppelte oder vom Typ
$X + X + 0 = 0$), so erhält man für den Torus letztendlich 204 Gleichungen
(darunter 80 zweisummandige, wenn ein Summand gleich Null war), mit denen nun
mittels Fallunterscheidungen alle zur Symmetriegruppe G gehörigen Matroide
mit 9 Punkten im Rang 4 bestimmt werden können. Tabelle \ref{gltab} zeigt
einen Ausschnitt aus der Liste dieser Gleichungen, mit den Nummern der
Orbitrepräsentanten in runden Klammern.

\begin{table}[htb]
{\small
$$
\begin{array}{lll}
(1)(3)+(1)(4)+(1)(6)+(1)(3)(4)(6) & = & 0 \\
(1)(3)+(1)(5)+(1)(9)+(1)(3)(5)(9) & = & 0 \\
(1)(3)+(1)(6)+(1)(10)+(1)(3)(6)(10) & = & 0 \\
(1)(3)+(1)(7) & = & 0 \\
(1)(4)+(1)(5) & = & 0 \\
... & & \\
(17)(21)+(17)(23) & = & 0 \\
(19)(21)+(19)(23) & = & 0 \\
(19)(21)+(20)(21) & = & 0 \\
(19)(21)+(20)(23) & = & 0 \\
(19)(23)+(20)(21) & = & 0 \\
(19)(23)+(20)(23) & = & 0 \\
(20)(21)+(20)(23) & = & 0 \\
(21)+(23) & = & 0 \\
\end{array}$$}
\caption{Gleichungen, die alle (9,4)-Matroide unter G erfüllen müssen}
\label{gltab}
\end{table}

Die in den Tabellen \ref{orbtab} und \ref{gltab} aufgeführten Beziehungen
der Brackets in den Graßmann-Plücker-Relationen dienen einem C-Programm
{\sc Sym2mat} (siehe dazu den Anhang) als Ausgangsbasis, alle möglichen,
bezüglich einer Symmetriegruppe G und Bracket-Null- beziehungsweise
Bracket-Eins-Setzungen verträglichen Matroide zu erzeugen. Dies geschieht
mittels eines rekursiven Verfahrens, welches vorhandene Abhängigkeiten
ausnutzt und im folgenden beschrieben wird.

Das Programm {\sc Sym2mat} erhält als Eingabe eine Automorphismengruppe G,
die gewünschte Punktanzahl $npkt$ und den Rang $rang$, sowie nach Berechnung
der Repräsentanten der Bracketorbits unter G gezielt zu Null oder Eins
gesetzte Brackets ("`gesetzt"' bezieht sich hier immer auf Untersuchungen
bezüglich GF(2)).\\
Aus diesen Informationen werden nun die (wie in Tabelle \ref{gltab}) unter G
reduzierten dreisummandigen Graßmann-Plücker-Relationen als zu erfüllendes
Gleichungssystem bereitgestellt. Dabei sind aus Gleichungen vom Typ
$$AB + CD + EF + ABCDEF = 0$$
zum Teil Gleichungen mit nur zwei Summanden (ein Bracketorbit gleich Null)
$$AB + CD = 0$$
oder durch Gleichsetzungen Gleichungen mit Quadraten
$$AA + CD + EE + AACDEE = 0$$
oder Gemische aus beiden Typen entstanden, wenn A, B, C, D, E und F
Repräsentanten von Bracketorbits bezeichnen (eine Begründung für das
Erscheinungsbild dieser Gleichungen findet sind in \cite{BoOlRi:91}).\\
Für das Auffüllen mit Nullen und Einsen sind nun zunächst die
zweisummandigen Gleichungen interessant, da aus ihnen unter Eins-Setzung
eines enthaltenen Bracketorbit\-repräsentanten eventuell Gleichheiten anderer
Orbits induziert werden.\\
Unter den vorhandenen Gleichungen werden zu Beginn solche gesucht, die von der
Gestalt $AA + BB = 0$ beziehungsweise $1A + 1B = 0$ und ihrer kommutativen
Äquivalente sind, da hieraus direkt $A = B$ abzuleiten ist. Diese Gleichheit
wird pro Rekursionsstufe in ein "`Bracket-Gleichheitsfeld"' eingetragen und A
als frei zu bestimmende Bracket für folgende Rekursionsschritte gespeichert.\\
Die aus allen vorhandenen Gleichungen resultierenden Gleichheiten werden nun
in das Gleichungssystem übernommen. Ein Sortierungsschritt eliminiert darauf
alle nach Gleichsetzung von Bracketorbitrepräsentanten redundanten (doppelte
oder triviale) Gleichungen, prüft auf direkte Schlüsse der Form
$1 + B = 0 \follows B = 1$ und $1 + 1 + B = 0\follows B = 0$, sowie Erfüllung
der Gleichungen und stellt so immer das aktuelle reduzierte Gleichungssystem
zur Verfügung. Eine Sortierung der Gleichungen zeichnet sich als
programmtechnisch sinnvoll aus und verfährt nach der lexikographischen Ordnung,
nach der für $AB + CD + EF$ immer gelte, daß $A\leq B$, $C\leq D$, $E\leq F$,
sowie $A\leq C$, $A\leq E$, $C\leq E$, wenn wiederum A, B, C, D, E und F
Repräsentanten von Bracketorbits bezeichnen. Zudem seien die Summanden, die
eine Null enthalten in der Gleichung immer zuletzt aufgeführt.\\
Haben sich nach diesem Verfahren keine freien Bracketorbits ergeben, so werden
die verbliebenen ungesetzten als frei angesehen. Im nächsten Schritt
werden in einer Schleife nacheinander die freien Bracketorbits abwechselnd auf
0 und 1 gesetzt und obige Prozedur wiederholt, woraus sich die Rekursion ergibt.

Folge dieser Vorgehensweise ist der Aufbau einer Baumstruktur von
Bracketorbitäquivalenzen, in der, da alle möglichen Abhängigkeiten und
Gleichheiten in den einzelnen Ästen berücksichtigt werden, alle möglichen
bezüglich der Symmetriegruppe G verträglichen Matroide erzeugt werden.

\begin{figure}[p]
\begin{center}
\btab{ll}
{\bf Routine} & {\sf Löse\_Gleichungssystem\_über\_GF(2)}\\
          & \\
Eingabe : & - Anzahl der Gleichungen \\
          & - Das Gleichungssystem \\
          & - Gesetzte Bracketorbits \\
          & - Abhängigkeiten der Bracketorbits untereinander \\
          & \\
Schritt 1 : & - Untersuche alle Gleichungen auf Gleichheiten \\
            & \hspace*{3ex} $\left.\begin{array}{l}
               AA+BB=0\\
               AA+1B=0\\
               AA+B1=0\\
               1A+BB=0\\
               1A+1B=0\\
               1A+B1=0\\
               A1+BB=0\\
               A1+1B=0\\
               A1+B1=0\end{array}\right\} \follows A = B,~A\mbox{ frei}$\\
            & - Minimiere Gleichheiten : $A=B,~B=C\follows A=C$\\
            & - Übertrage Gleichheiten in das Gleichungssystem\\
            & - Sortiere und reduziere Gleichungssystem\\
            & \hspace*{3ex} Überprüfe dabei Schlüsse auf $A=1,~A=0$\\
            & \hspace*{3ex} Überprüfe dabei, ob Gleichungen erfüllt sind\\
            & - Bei Erfüllung aller Gleichungen, schreibe mögliches Matroid\\
Schritt 2 : & - Ermittlung weiterer freier Bracketorbits aus\\
            & \hspace*{3ex} $\left.\begin{array}{l}
              AB+AC=0\\
              BA+CA=0\\
              BA+AC=0\end{array}\right\} \follows A\mbox{ frei}$\\
            & - Bis jetzt keine Freien, so Ungesetzte frei\\
Schritt 3 : & - Setzen der freien Bracketorbits auf 0 und 1\\
            & \hspace*{3ex} Gehe in einer Schleife alle Orbits durch\\
            & \hspace*{4ex} Setze freien Orbit auf 0 bzw. 1\\
            & \hspace*{4ex} {\sf Löse\_Gleichungssystem\_über\_GF(2)}\\
            & \hspace*{4ex} mit neuen Annahmen
\etab
\caption{Routine zur Bestimmung möglicher verträglicher Matroide}
\label{solvegpr}
\end{center}
\end{figure}

Die Ausgabe von {\sc Sym2mat} liefert so die dreisummandigen
Graßmann-Plücker-Relationen erfüllende Elementlisten, die entsprechend der
über GF(2) betrachteten Brackets jedem $[\lambda]$ mit $\lambda\in\Lambda(n,r)$
eine 0 oder 1 zuweist. Da das Erfülltsein der dreisummandigen
Graßmann-Plücker-Relationen nur eine notwendige Bedingung für das Vorliegen
eines Matroids darstellt, müßten nun in einem folgenden Schritt die
Elementlisten ebenfalls mit den übrigen k-summandigen
Graßmann-Plücker-Relationen über GF(2) getestet werden, damit letztendlich
nur Matroide vorliegen, die dann verträglich zur Symmetriegruppe G orientiert
werden sollen.
{\scsi
Hierzu haben Bokowski et al. gezeigt, daß über GF(2) die sogenannten
"`Odd-Polynomials"' zu den Graßmann-Plücker-Polynomen, die aufsummierten
ungeraden elementarsymmetrischen Funktionen der auftretenden Bracketprodukte,
darauf zu überprüfen sind, ob sie mit den vorgegebenen Betragswerten der
Elementeliste Null ergeben. Als Beispiel für die 4-summandigen
Graßmann-Plücker-Relationen sind so Gleichungen vom Typ
$A+B+C+D+ABC+ABD+ACD+BCD=0$ als Summe der ersten und dritten
elementarsymmetrischen Funktion zu testen, wenn A,B,C und D jeweils das
Produkt zweier Brackets darstellen.
}

Für das Torusbeispiel von oben liefert {\sc Sym2mat} 43 verschiedene
Elementlisten, von denen ein Teil in Tabelle \ref{tormat} dargestellt ist.
Dabei kann das triviale Matroid (erste Zeile, nur Nullen) von vornherein als
"`unerwünschtes"' gestrichen werden, da es sicher jede Symmetrie erfüllt,
aber keine Information über den vorgelegten Komplex liefert. Interessant sind
gerade solche Matroide, die lineare Unabhängigkeiten, eine "`Räumlichkeit"'
des Komplexes induzieren, dazu aber gleich mehr.

\begin{table}[htb]
\begin{center}
{\scriptsize\tt
\btab{c}
000000000000000000000000000000000000000000000000000000000000000\\
000000000000000000000000000000000000000000000000000000000000000\\
\hline
000000000000000000000000000000000000000000000000000000000000000\\
000000000000000000000000000000000000000000000000111011101111011\\
\hline
000000000000000000000000000000000000000000000000100000100000000\\
000000000000100010000000000000100000000001000000000000000000000\\
\hline
$\vdots$\\
\hline
111111011111111111011101111111111101111111111110011011011111101\\
111111101111010101111111101110011110111110111111111011101111011\\
\hline
111111011111111111011101111111111101111111111111111111111111101\\
111111101111111111111111111111111111111111111111000000000000000\\
\hline
111111011111111111011101111111111101111111111111111111111111101\\
111111101111111111111111111111111111111111111111111011101111011
\etab
}
\end{center}
\caption{Mögliche Matroid-Elementlisten zum Torusbeispiel}
\label{tormat}
\end{table}

{\sc Sym2mat} liefert ausgehend vom Test der nur dreisummandigen
Graßmann-Plücker-Polynome "`möglicherweise"' Matroide. Um nun sicherzugehen,
daß für die weitere Bearbeitung nur wirkliche Matroide vorliegen, müssen die
ausgegebenen Elementlisten darauf getestet werden, ob sie Matroide darstellen
oder nicht. Dazu können, wie angedeutet, die übrigen k-summandigen
($4\leq k\leq rang+1$) Graßmann-Plücker-Relationen in Form der
Odd-Polynomials getestet werden. Ebenso kann der Matroidbeweis aber auch auf
dem Nachweis basieren, daß die Elementlisten das Basisaxiom der Definition
eines Matroids erfüllen müssen. Nach \cite{Bj:93}, Seite 81, muß
so überprüft werden, ob die Menge
$${\cal B}=\{[\lambda_1,\ldots,\lambda_d]\neq 0~|~(\lambda_1,\ldots,\lambda_d)
\in\Lambda (n,d)\}$$
die Eigenschaft besitzt, daß für je zwei "`Basen"' $B_1$ und $B_2$ aus
$\cal B$ und alle $b_1\in B_1\backslash B_2$ ein $b'_i\in B_2\backslash B_1$
so existiert, daß $((B_1\backslash b_1)\cup b'_i)$ aus $\cal B$ stammt, was
nach Abbildung \ref{testmatroid} geschehen kann.

\begin{figure}[htb]
\begin{center}
\btab{ll}
{\bf Routine} & {\sf Teste\_auf\_Matroid}\\
          & \\
Eingabe : & - Elementliste $\ul{\cal M}$ des vermeindlichen Matroids \\
          & - Für alle $1\leq i\leq {npkt \choose rang}$ \\
          & \hspace*{2ex} Ist $matroid[i]\neq 0$ \\
          & \hspace*{4ex} Sei $B_1$ die Bracket zu $matroid[i]$ \\
          & \hspace*{4ex} Für alle $1\leq (j\neq i)\leq {npkt \choose rang}$ \\
          & \hspace*{6ex} Ist $matroid[j]\neq 0$ \\
          & \hspace*{8ex} Sei $B_2$ die Bracket zu $matroid[j]$ \\
          & \hspace*{8ex} $P = B_1\backslash B_2$ \\
          & \hspace*{8ex} $Q = B_2\backslash B_1$ \\
          & \hspace*{8ex} Für alle $p\in P$\\
          & \hspace*{10ex} Existiert in Q kein Element q mit \\
          & \hspace*{12ex} $(B_1\backslash p)\cup q\in{\cal B}$\\
          & \hspace*{10ex} so ist $\ul{\cal M}$ kein Matroid\\
Ausgabe : & - Matroid oder nicht
\etab
\caption{Routine zum Test auf die Matroideigenschaft einer Elementliste}
\label{testmatroid}
\end{center}
\end{figure}

Liegt die Elementliste des zu überprüfenden vermeindlichen Matroids als ein
Feld $matroid[i]$ mit $1\leq i\leq {n\choose d}$ und Einträgen 0 oder 1 vor,
etwa wie die Ausgabe von {\sc Sym2mat}, so läßt sich mit einer Routine
entsprechend Abbildung \ref{testmatroid} auf die Matroideigenschaft der Liste
testen. Dazu sind im schlechtesten Fall, wenn alle Basen überprüft werden
müßten, ${npkt\choose rang}\cdot\left({npkt\choose rang}-1\right)\cdot rang^2$
Operationen nötig. Dies ist zwar auf den ersten Blick wesentlich mehr, als die
Überprüfung der Graßmann-Plücker-Relationen ergeben würde, bedenkt man
aber, daß dazu alle k-summandigen ($4\leq k\leq rang+1$) rekursiv erzeugt und
in Bracketgleichungen (die Odd-Poly\-no\-mi\-als) umgewandelt werden
müssen, so ist der (Rechen-)\-Aufwand von der gleichen Größenordnung, so
daß der viel einfacher zu implementierende Basistest auch seine Berechtigung
besitzt.

Führt man diese Überprüfung durch, so ergeben sich aus den 43 vorgelegten
Listen für den Torus 25 nichttriviale Matroide, von denen ein Teil in Tabelle
\ref{torusmat} aufgelistet ist. Diese könnten nun als Eingabe für ein
Matroid-Orientierungsprogramm dienen, welches mit seiner Beschreibung das Ziel
dieser Arbeit darstellt und eine Möglichkeit liefert, zu einem CW-Komplex
über dessen Symmetrien verträgliche orientierte Matroide zu erzeugen.

\begin{table}[htb]
\begin{center}
{\scriptsize\tt
\btab{c}
000000000000000000000000000000000000000000000000000000000000000\\
000000000000000000000000000000000000000000000000111011101111011\\
\hline
000000000000000000000000000000000000011111111111111111100000000\\
000000000111111111111111111001111111111111111110000000000000000\\
\hline
000000000000000000000000000000000000011111111111111111100000000\\
000000000111111111111111111001111111111111111110111011101111011\\
\hline
$\vdots$\\
\hline
111111011111111111011101111111111101111111111110011011011111101\\
111111101111010101111111101110011110111110111111111011101111011\\
\hline
111111011111111111011101111111111101111111111111111111111111101\\
111111101111111111111111111111111111111111111111000000000000000\\
\hline
111111011111111111011101111111111101111111111111111111111111101\\
111111101111111111111111111111111111111111111111111011101111011\\
\etab
}
\end{center}
\caption{Die bezüglich G verträglichen Matroide zum Torus}
\label{torusmat}
\end{table}

Betrachtet man zuvor die Struktur der erzeugten Matroide, so stellt man fest,
daß unter diesen noch viele "`uninteressante"' zu finden sind. Dies sind jene,
die "`zuviele"' Nullen enthalten, entsprechend der Determinantendeutung der
Basen also eine "`flache"' Punktkonfiguration repräsentieren.\\
Um solche von vorn herein ausschließen zu können, kann man sich des
vorgelegten Komplexes bedienen und fordern, welche Brackets sicher nicht Null,
also Basen des Matroids sein sollen. Solche Basen ergeben sich aus den Brackets,
die sich aus den Indizes von Punkten benachbarter Zellen, die gefordert nicht
in einer Ebene liegen sollen, zusammensetzen lassen. Für den Torus ergibt diese
Forderung bei abgeschlossenen benachbarten Zellen $\mbox{conv}\{a,b,c,d\}$ und
$\mbox{conv}\{a,b,e,f\}$ mit den 0-Zellen-Indizes a bis f und der gemeinsamen
Kante $\ol{ab}$, folgende Brackets als Basen eines Matroids:
\begin{center}
{\small
[abce], [abcf], [abde], [abdf], [acde], [acdf],
[acef], [adef], [bcde], [bcdf], [bcef], [bdef]
}
\end{center}
Von den ${6\choose 4}$ Möglichkeiten, aus obigen Punkten eine Bracket zu
bilden also jene, die nicht die Zellen selbst und zwei gegenüberliegende
"`Außenkanten"' (hier etwa $\ol{cd}$ und $\ol{ef}$) beschreiben.

\begin{figure}[htb]
$$
\beginpicture
\unitlength0.6cm
\setlinear
\setcoordinatesystem units <0.6cm,0.6cm>
\setplotarea x from -2 to 2, y from -2 to 2
\plot -1.5 -1.5 -1.5 1.5 1.5 1.5 1.5 -1.5 -1.5 -1.5 /
\plot -1.5 0 1.5 0 /
\put {a} [Br] at -1.55 0
\put {b} [Bl] at  1.55 0
\put {c} [tr] at -1.55 -1.55
\put {d} [tl] at  1.55 -1.55
\put {e} [br] at -1.55  1.55
\put {f} [bl] at  1.55  1.55
\endpicture
$$
\caption{Benachbarte 2-Zellen}
\end{figure}

Eine gezielte Bestimmung der Nichtbasen und "`sicherlich"' Basen ergibt so eine
Grundstruktur für die verträglichen Matroide, die letztendlich über
{\sc Sym2mat} zu echten Rang (d+2) Matroiden führt. Ein Algorithmus für eine
solche Bestimmung läßt sich Abbildung \ref{setbases} entnehmen.

\begin{figure}[htb]
\begin{center}
\btab{ll}
{\bf Routine} & {\sf Setze\_sichere\_Basen}\\
          & \\
Eingabe : & CW-Komplex in Form der Zellenberandungen mit Rang r\\
Prozedur :& Für jedes Paar A, B von (r-2)-Zellen \\
          & \hspace*{2ex} erzeuge A$\cap$B \\
          & \hspace*{2ex} Ist $|A\cap B|\geq r-2$, also zumindest ein\\
          & \hspace*{2ex} (r-3)-Simplex im Rand, so erzeuge folgende Brackets \\
          & \hspace*{4ex} - für alle Auswahlen von (r-1) 0-Zellen aus A \\
          & \hspace*{6ex} die Brackets durch Ergänzung um einen Punkt aus B \\
          & \hspace*{4ex} - für alle Auswahlen von (r-1) 0-Zellen aus B \\
          & \hspace*{6ex} die Brackets durch Ergänzung um einen Punkt aus A \\
          & \hspace*{4ex} Sind die erzeugten Brackets nicht durch Auswahl von \\
          & \hspace*{4ex} Punkten einer (r-2)-Zelle entstanden und sind keine \\
          & \hspace*{4ex} Punkte doppelt vorhanden, so setze diese Brackets als \\
          & \hspace*{4ex} Basen in der Matroidliste auf den Wert 1 \\
          & Setze die Brackets zu Auswahlen von r Punkten einer (r-2)-Zelle \\
          & als Nichtbasis in der Matroidliste auf den Wert 0 \\
Ausgabe : & Liste, in der die Basen und Nichtbasen, sowie freien Brackets \\
          & verzeichnet sind.
\etab
\caption{Routine zur Bestimmung einer Grundstruktur für die Elementlisten}
\label{setbases}
\end{center}
\end{figure}

Damit ergibt sich als Grundstruktur für den Torus folgendes Bild.

{\small\tt
\begin{center}
1111110111?111????011101?11???1?1101111?1?111????1???1?111???0?\\
11?11110111????1???111?1??111???1?1?1??11??1?111111011101111011\\
\end{center}
}

womit von obigen 25 verträglichen Matroiden noch sechs übrigbleiben, die eine
Chance besitzen, zu verträglichen Rang 4 orientierten Matroiden des Torus
zu werden.

\begin{table}[htb]
\begin{center}
{\scriptsize\tt
\btab{c}
111111011101110001011101011010101101111010111110010001011110000\\
110111101110010100011111001110011010100110011111111011101111011\\
\hline
111111011111111111011101111111111101111111111110011011011111101\\
111111101111010101111111101110011110111110111111111011101111011\\
\hline
111111011111111110011101111101111101111111111001111111111101101\\
111111101111101111111101111111101111111111110111111011101111011\\
\hline
111111011111111110011101111101111101111111111111111111111101101\\
111111101111111111111111111111111111111111111111111011101111011\\
\hline
111111011111111111011101111111111101111111111001111111111111101\\
111111101111101111111101111111101111111111110111111011101111011\\
\hline
111111011111111111011101111111111101111111111111111111111111101\\
111111101111111111111111111111111111111111111111111011101111011\\
\etab
}
\end{center}
\caption{Die bezüglich G verträglichen Rang 4 Matroide zum Torus}
\label{torus4mat}
\end{table}

\clearpage
\subsection{Orientierung der erzeugten Matroide}

Im letzten Schritt sollen die erzeugten, bezüglich 0-Zellenanzahl
und Rang eines CW-Komplexes $\C$, sowie einer vorgelegten Symmetriegruppe G
verträglichen Matroide so orientiert werden, daß weiterhin die
Verträglichkeitseigenschaft bezüglich G gewährleistet ist.\\
Dazu werden wiederum zur Minimierung des Rechenaufwandes, die unter den
Symmetrien aus G reduzierten, zu erfüllenden dreisummandigen
Graßmann-Plücker-Relationen eingesetzt. Daß diese erfüllt sind, stellt für
die zu entstehenden orientierten Matroide nun eine notwendige und hinreichende
Bedingung dar, was sich aus dem Satz über die dreisummandigen Relationen von
Seite \pageref{dgpr} ergibt. Ziel ist die rekursive Erzeugung aller zu einem
vorgelegten Matroid bezüglich G verträglichen Orientierungen, sofern solche
überhaupt existieren.

Auf Seite \pageref{symmtest} war davon die Rede, daß die Anzahl der
Möglichkeiten ein orientiertes Matroid zu erzeugen davon abhängt, wie die
Elemente g$\in$G das entsprechende Chirotop $\chi$ abbilden, ob als Rotation
auf sich selbst oder als Reflexion auf sein negatives. Gerade bei der
Bestimmung der Orbits der Brackets unter G ist dies eine wichtige Information.
Erzwänge man nämlich eine bestimmte Variante (etwa alle id$\neq$g$\in$G seien
Rotationen), so kann es passieren, daß kein solches verträgliches orientiertes
Matroid existiert, da sich schon bei der Aufstellung der Bracketbahnen
Widersprüche ergeben.

Für den dreidimensionalen Würfel etwa (acht Punkte im Rang vier mit den
2-Zellen-Berandungen 1264, 2586, 5378, 3147, 4687 und 1253) bei Vorlage der
Symmetriegruppe $\{id, (23)(67), (24)(57), (34)(56), (234)(576), (243)(567)\}$
und dem zugrunde liegenden Matroid

\begin{center}{\small
$1011110111111101101111101110111111111111110111011111011011111110111101$
}\end{center}

ergibt sich für die Basis-Bracket [1234] folgender Orbit:
$$
\begin{array}{|ccc|c|c|c|c|}
\hline
id & \mapsto & (23)(67) & (24)(57) & (34)(56) & (234)(576) & (243)(567) \\
\hline
+[1234] & \mapsto & -[1234] & -[1234] & -[1234] & +[1234] & +[1234] \\
\hline
\end{array}
$$
Egal wie man sich entscheiden würde (alle id$\neq$g$\in$G Reflexionen
beziehungsweise Rotationen), der Orbit würde immer induzieren, daß
$[1234]=-[1234]$, also $[1234]=0$ gelten muß, was einen Widerspruch zum
zugrunde liegenden Matroid darstellt.\\
Hier ließe sich aus obigem Orbit allerdings direkt ablesen, daß es sich bei
den Permutationen (23)(67), (24)(57) und (34)(56) um Reflexionen, sowie bei
(234)(576) und (243)(567) um Rotationen handeln muß. Betrachtet man dazu
die Symmetriegruppe des Würfelchirotops (die identisch zur kombinatorischen
Symmetriegruppe ist), daß durch geeignete Koordinatenwahl (vgl. Abb.
\ref{cube}) entstanden ist, so entspricht dies auch der "`Wirklichkeit"'.
\vskip4mm

\centerline{\small\tt
+0+++-0--+++--0++0+----0+-+0++---+-++++---0++-0-++--0+-0++++---0---+0-
}\vskip4mm

Aus dieser Überlegung läßt sich nun folgender kleiner Hilfssatz ableiten, der
besagt:
\begin{quote}
Wann immer ein Element g einer für ein zu erzeugendes Chirotop geforderten
Symmetriegruppe G eine Bracket $[\lambda]\neq 0$ bis auf eine lexikographische
Sortierung $\sigma$ auf sich abbildet, so bildet in diesem Fall g das
Chirotop $\chi$ auf $\mbox{sgn}(\sigma)\chi$ ab.
\end{quote}
Der Beweis hierzu ist einfach in dem Sinne, daß g als geforderte Symmetrie
$\chi$ sicher auf $\chi$ selbst oder $-\chi$ abbildet. Wird eine Basis-Bracket
$[a_1\ldots a_r]$ des zugrunde liegenden Matroids nun, unter Vorschaltung einer
Sortierungspermutation (eine Permutation, die die $a_1$ bis $a_r$ der Größe
nach sortiert, also auf $\{1,\ldots,r\}$ wirkt), auf sich abgebildet, so ist
hierdurch, aufgrund der Symmetrieforderung, eindeutig festgelegt, ob nach g
$\chi$ oder $-\chi$ vorliegt, da sich die Symmetrie auf alle Basis-Brackets
bezieht.$\Box$

Für die übrigen Symmetrien, die keine Bracket bis auf ihr Vorzeichen
invariant lassen, bleibt nichts anderes übrig, als zunächst beide
Möglichkeiten in Betracht zu ziehen, wie sie auf ein $\chi$ wirken könnten
oder eine Fixierung vorzugeben. Schaltet man vor die Orientierungsversuche zu
einem Matroid einen Test nach obigem Hilfssatz vor, so kann sich hier der
Aufwand aber schon erheblich reduzieren. In bezug auf das Torusbeispiel ergibt
sich so, daß die Symmetrie $(47)(58)(69)$ die Bracket $[1247]$ auf $-[1247]$
abbildet und somit eine Reflexion der Chirotope $\chi$ zum Torus darstellt.
Damit bleiben $2^4$ Möglichkeiten für die anderen Elemente aus G, wie die
Bracketorbits bezüglich ihrer Gleichheiten zu deuten sind.

Ein erster Eindruck, wie ein Matroidorientierungsprogramm aussehen könnte,
ergibt sich nun wie folgt.
\begin{enumerate}
\item Lege eine Automorphismengruppe G und eine verträgliche Matroidliste
      für n Punkte im Rang r vor.
\item Bestimme die Bracketorbits unter G und prüfe auf Selbstabbildungen in
      obigem Sinne. Liegen solche vor, so merke den Symmetrietyp (Rot/Ref).
\item Bestimme die Orbits der dreisummandigen Graßmann-Plücker-Polynome
      unter G als Ausgangsebene für ein System zur Bestimmung der zugehörigen
      Chirotopvorzeichen.
\item Setze nacheinander alle Kombinationen für die unbestimmten Symmetrietypen
      in die Bracketorbits ein, so daß sich nun die gewünschten
      Bracketgleichheiten ergeben und
\begin{itemize}
\item Stelle das zugehörige "`Gleichungs"'system durch Einsetzen der
      Bracket\-orbitrepräsentanten (Vorzeichenwechsel berücksichtigen)
      in die GPP auf.
\item Ermittle die zugehörigen Chirotope durch Auffüllen und Erschließen von
      Vorzeichen für die Brackets in den GPP.
\end{itemize}
\item Gib alle ermittelten Vorzeichenlisten der Chirotope aus.
\end{enumerate}

Das Einsetzen und Erschließen von Vorzeichen in den GPP begründet sich
in der Deutung der Brackets als Determinanten von (r$\times$r)-Teilmatrizen
einer (n$\times$r)-Matrix, für die die Graßmann-Plücker-Relationen erfüllt
sind.\\
Stellt man das zu untersuchende Gleichungssystems auf, so zeigt sich noch eine
weitere Möglichkeit, auftretende Symmetrietypen als unsinnig zu erkennen.
Treten nämlich unter Berücksichtigung aller Sortierungsvorzeichen und
Nullsetzungen von Bracketorbits Gleichungen vom Typ
$(+[A])(+[B]) + (+[A])(+[B]) = 0$ auf, so kann die Wahl der Art der
Abbildungsvorschrift von $\chi\mapsto\pm\chi$ als nicht richtig angesehen
werden, da nichtsinguläre Matrizen A und B diese Gleichung nicht erfüllen
können.\\
Genauer ergibt sich für die dreisummandigen Gleichungen so, daß die
Vorzeichen der Determinanten derart gegeben sind, daß entweder genau ein
Summand (ein Bracketprodukt) positiv oder genau einer negativ ist. Ist ein
Summand 0, so sind die anderen beiden von entgegengesetztem Vorzeichen, denn nur
so ist zu erreichen, daß die Gleichung zu Null werden kann, wenn alle
Determinanten nichtnull sind. Für die Gleichungen
$$ A + (-B) + C = 0$$
ergeben sich so die Vorzeichen der Bracketprodukte A, $-$B und C zu
$$(+,-,+),~(+,-,-),~(+,+,-),~(-,-,+),~(-,+,+),~(-,+,-)$$
Solche Konstellationen gilt es nun im vorgelegten Gleichungssystem zu erzeugen.
Dazu bedienen wir uns zunächst wieder der Gleichungen, die nach dem Ersetzen
der Brackets durch ihre Repräsentanten unter G nur noch zwei Summanden
enthalten, die nichtnull sind, während der dritte durch eine Nichtbasis des
zugrunde liegenden Matroids verschwunden ist. Wieder kann bei Gleichungen
vom Typ AB + AC auf das Verhalten von B und C geschlossen werden, wenn A
gesetzt wurde.

Grundsätzlich ergeben sich aus den zweisummandigen Gleichungen folgende zu
be\-rücksichtigenden Möglichkeiten:
{\small
$$
\begin{array}{|l|l|l|l|}
\multicolumn{4}{l}{A=+ \mbox{ für }} \\
\hline
-(A+)~+~(BB) & -(+A)~+~(BB) & (A-)~+~(BB) & (-A)~+~(BB) \\
\hline
\multicolumn{4}{l}{B=+ \mbox{ für }} \\
\hline
-(AA)~+~(+B) & -(AA)~+~(B+) & (AA)~+~(-B) & (AA)~+~(B-) \\
\hline
\multicolumn{4}{l}{A=- \mbox{ für }} \\
\hline
-(A-)~+~(BB) & (A+)~+~(BB) & -(-A)~+~(BB) & (+A)~+~(BB) \\
\hline
\multicolumn{4}{l}{B=- \mbox{ für }} \\
\hline
-(AA)~+~(-B) & (AA)~+~(+B) & -(AA)~+~(B-) & (AA)~+~(B+) \\
\hline
\multicolumn{4}{l}{A=B \mbox{ für }} \\
\hline
 (+A)~+~(-B) &  (+A)~+~(B-) &  (-A)~+~(+B) &  (-A)~+~(B+) \\
 (A+)~+~(-B) &  (A+)~+~(B-) &  (A-)~+~(+B) &  (A-)~+~(B+) \\
-(+A)~+~(+B) & -(+A)~+~(B+) & -(-A)~+~(-B) & -(-A)~+~(B-) \\
-(A+)~+~(+B) & -(A+)~+~(B+) & -(A-)~+~(-B) & -(A-)~+~(B-) \\
\hline
\multicolumn{4}{l}{A=-B \mbox{ für }} \\
\hline
-(A+)~+~(B-) &  (A+)~+~(+B) & -(A-)~+~(B+) &  (A-)~+~(-B) \\
 (+A)~+~(+B) & -(+A)~+~(B-) & -(-A)~+~(B+) &  (-A)~+~(-B) \\
 (A+)~+~(B+) & -(A+)~+~(-B) & -(A-)~+~(+B) &  (A-)~+~(B-) \\
 (+A)~+~(B+) & -(+A)~+~(-B) & -(-A)~+~(+B) &  (-A)~+~(B-) \\
\hline
\multicolumn{4}{l}{A=B \mbox{ oder } A=-B \mbox{ für }} \\
\hline
-(AA)~+~(BB) & & & \\
\hline
\end{array}
$$}

Weiterhin lassen sich Vorzeichen aus dreisummandigen Gleichungen erschließen,
wenn genau ein Bracketvorzeichen unbestimmt ist. Gilt dann, daß das Vorzeichen
der beiden anderen Summanden gleich ist, so muß mit dem ungesetzten das
Vorzeichen des noch unbestimmten Summanden negativ dem der anderen sein.
Etwa gilt für $(++)~+~(--)~+~(-A)\fol (+,+,?)$, daß $A=+$ sein muß, damit
der letzte Summand negatives Vorzeichen erhält. Frei ist die Wahl von A, wenn
die bestimmten Summanden entgegengesetztes Vorzeichen besitzen.

Ist in einer Gleichung mehr als ein Vorzeichen unbestimmt, so können die
jeweils ungesetzen Bracketvorzeichen als Freiheitsgrade für die Vorzeichenwahl
angesehen werden. Ein Setzen dieser, sowie ein Erschließen weiterer Vorzeichen
nach obiger Auswahl, liefert bei rekursiver Bearbeitung des Gleichungssystems
wie bei der Matroiderzeugung eine Baumstruktur, in der alle verträglichen
orientierten Matroide mit der gewünschten Symmetrieeigenschaft erzeugt werden.

Mittels eines so aufgebauten Such- und Ersetzprogramms können nun die
zuvor bereitgestellten Matroide orientiert werden, was für das Torusbeispiel
für die vier orientierbaren Matroide zu den Vorzeichenlisten aus den Tabellen
\ref{torusom1} bis \ref{torusom6} führt. Hierbei ergibt sich zunächst, das
bei Vorgabe der Symmetriegruppe G mit dem erzeugenden Element
$\sigma = (132)(495768)$ folgende Symmetrietypen zuläßig sind:
$$\begin{array}{c|c|c|c|c|c}
\mbox{id} & \sigma & \sigma^2 & \sigma^3 & \sigma^4 & \sigma^5 \\
\hline
+ & + & + & - & - & - \\
+ & + & - & - & - & + \\
+ & - & + & - & + & - \\
+ & - & - & - & + & + \\
\end{array}$$
Hierbei bedeutet $+$ die Wirkung von $\sigma^i$ als Rotation und $-$ als
Reflektion auf die zu entstehenden $\chi$. Besonders wichtig ist, daß man
den gewählten Symmetrietyp beim Ermitteln der vollständigen Vorzeichenliste
wieder berücksichtigt, wenn man aus den Bracketorbitrepräsentanten die
übrigen Brackets rekonstruiert. Berücksichtigt man noch, daß der Symmetrietyp
von Quadraten von Permutationen eindeutig als Rotation festgelegt ist
($\mbox{typ}(\sigma)^{2k}=1$), so bleibt hier als einzig zuläßiger
Symmetrietyp
$$\begin{array}{c|c|c|c|c|c}
\mbox{id} & \sigma & \sigma^2 & \sigma^3 & \sigma^4 & \sigma^5 \\
\hline
+ & - & + & - & + & -
\end{array}$$
übrig. Was bei kleinen vorgelegten Gruppen, noch leicht von Hand zu lösen
ist, ist bei größeren Gruppen, deren Elemente zusammengesetzt sind, zwar
auch, aber mit ungleich höherem Aufwand zu bewerkstelligen, was so zur Zeit
noch nicht implementiert ist.

\begin{table}[htb]
Gewähltes Matroid:
\begin{center}
{\scriptsize\tt
\begin{tabular}{c}
111111011101110001011101011010101101111010111110010001011110000\\
110111101110010100011111001110011010100110011111111011101111011
\end{tabular}}\end{center}\vskip2mm

Vorgelegte Automorphismengruppe:
{\small
$$\{\mbox{id},(132)(495768),(123)(456)(789),(47)(58)(69),(132)(465)(798),
   (123)(486759)\}$$}\vskip2mm

Erzeugte Chirotope:
\begin{center}
{\scriptsize\tt\begin{tabular}{c}
+++---0+--0+--000+0---0+0++0-0+0++0-+--0+0++--+00+000-0---+0000\\
--0--++0+--00-0+000--+++00+-+00++0+0-00--00+--+----0++-0+---0--\\
\hline
+++---0-++0-++000-0+++0-0--0+0-0--0++--0+0++--+00+000-0-++-0000\\
++0++--0+--00-0+000--+++00+-+00++0+0-00--00+--+-+++0--+0-+++0++\\
\hline
---+++0+--0+--000+0---0+0++0-0+0++0--++0-0--++-00-000+0+--+0000\\
--0--++0-++00+0-000++---00-+-00--0-0+00++00-++-+---0++-0+---0--\\
\hline
---+++0-++0-++000-0+++0-0--0+0-0--0+-++0-0--++-00-000+0+++-0000\\
++0++--0-++00+0-000++---00-+-00--0-0+00++00-++-++++0--+0-+++0++\\
\end{tabular}}
\end{center}
\caption{\label{torusom1} Chirotope zum ersten Torusmatroid}
\end{table}

\begin{table}[htb]
\begin{center}
{\scriptsize\tt
\begin{tabular}{c}
111111011111111110011101111101111101111111111111111111111101101\\
111111101111111111111111111111111111111111111111111011101111011\\
\hline\hline
+++---0+---+-----00---0+++++0+++++0-+---++++--++++--+-----0--0-\\
-----++0+------+++---++++++-+++++++--+-----+--+----0++-0+---0--\\
\hline
+++---0-+++-+++++00+++0-----0-----0++---++++--++++--+---++0++0+\\
+++++--0+------+++---++++++-+++++++--+-----+--+-+++0--+0-+++0++\\
\hline
---+++0+---+-----00---0+++++0+++++0--+++----++----++-+++--0--0-\\
-----++0-++++++---+++------+-------++-+++++-++-+---0++-0+---0--\\
\hline
---+++0-+++-+++++00+++0-----0-----0+-+++----++----++-+++++0++0+\\
+++++--0-++++++---+++------+-------++-+++++-++-++++0--+0-+++0++\\
\end{tabular}}
\end{center}
\caption{\label{torusom4} Chirotope zum vierten Torusmatroid}
\end{table}

\begin{table}[htb]
\begin{center}
{\scriptsize\tt
\begin{tabular}{c}
111111011111111111011101111111111101111111111001111111111111101\\
111111101111101111111101111111101111111111110111111011101111011\\
\hline\hline
+++---0+--++--++++0---0+-++---+-++0-+--++-++-00--+++--+---+++0+\\
--+--++0+--++0++--+--+0+--+-+--0+-++--+--+++0-+-+++0--+0-+++0++\\
\hline
+++---0-++--++----0+++0-+--+++-+--0++--++-++-00--+++--+-++---0-\\
++-++--0+--++0++--+--+0+--+-+--0+-++--+--+++0-+----0++-0+---0--\\
\hline
---+++0+--++--++++0---0+-++---+-++0--++--+--+00++---++-+--+++0+\\
--+--++0-++--0--++-++-0-++-+-++0-+--++-++---0+-++++0--+0-+++0++\\
\hline
---+++0-++--++----0+++0-+--+++-+--0+-++--+--+00++---++-+++---0-\\
++-++--0-++--0--++-++-0-++-+-++0-+--++-++---0+-+---0++-0+---0--\\
\end{tabular}}
\end{center}
\caption{\label{torusom5} Chirotope zum fünften Torusmatroid}
\end{table}

\begin{table}[htb]
\begin{center}
{\scriptsize\tt
\begin{tabular}{c}
111111011111111111011101111111111101111111111111111111111111101\\
111111101111111111111111111111111111111111111111111011101111011\\
\hline\hline
+++---0+++++++++++0---0---+------+0-++++----++----++-++---+++0+\\
+++++++0+++++++---+++-------+------++-+++++-++--+++0--+0-+++0++\\
\hline
+++---0+---+-----+0---0+++++-+++++0-+---++++--++++--+-----+--0-\\
-----++0+------+++---++++++-+++++++--+-----+--+----0++-0+---0--\\
\hline
+++---0+---+------0---0+++++++++++0-+---++++--++++--+--------0-\\
-----++0+------+++---++++++-+++++++--+-----+--+----0++-0+---0--\\
\hline
+++---0+--++--++++0---0+-++---+-++0-+--++-++-+---+++--+---+++0+\\
--+--++0+--+++++--+--+-+--+-+---+-++--+--++++-+-+++0--+0-+++0++\\
\hline
+++---0+--++--++++0---0+-++---+-++0-+--++-++--+--+++--+---+++0+\\
--+--++0+--++-++--+--+++--+-+--++-++--+--+++--+-+++0--+0-+++0++\\
\hline
+++---0+--++--++++0---0+-++---+-++0-+--++-++--+--+++--+---+++0+\\
--+--++0+--++-++--+--+++--+-+--++-++--+--+++--+----0++-0+---0--\\
\hline
+++---0-+++-++++++0+++0-----------0++---++++--++++--+---+++++0+\\
+++++--0+------+++---++++++-+++++++--+-----+--+-+++0--+0-+++0++\\
\hline
+++---0-+++-+++++-0+++0-----+-----0++---++++--++++--+---++-++0+\\
+++++--0+------+++---++++++-+++++++--+-----+--+-+++0--+0-+++0++\\
\hline
+++---0-++--++----0+++0-+--+++-+--0++--++-++-+---+++--+-++---0-\\
++-++--0+--+++++--+--+-+--+-+---+-++--+--++++-+----0++-0+---0--\\
\hline
+++---0-++--++----0+++0-+--+++-+--0++--++-++--+--+++--+-++---0-\\
++-++--0+--++-++--+--+++--+-+--++-++--+--+++--+-+++0--+0-+++0++\\
\hline
+++---0-++--++----0+++0-+--+++-+--0++--++-++--+--+++--+-++---0-\\
++-++--0+--++-++--+--+++--+-+--++-++--+--+++--+----0++-0+---0--\\
\hline
+++---0-----------0+++0+++-++++++-0+++++----++----++-++-++---0-\\
-------0+++++++---+++-------+------++-+++++-++-----0++-0+---0--\\
\hline
---+++0+++++++++++0---0---+------+0-----++++--++++--+--+--+++0+\\
+++++++0-------+++---+++++++-++++++--+-----+--+++++0--+0-+++0++\\
\hline
---+++0+--++--++++0---0+-++---+-++0--++--+--++-++---++-+--+++0+\\
--+--++0-++--+--++-++---++-+-++--+--++-++---++-++++0--+0-+++0++\\
\hline
---+++0+--++--++++0---0+-++---+-++0--++--+--++-++---++-+--+++0+\\
--+--++0-++--+--++-++---++-+-++--+--++-++---++-+---0++-0+---0--\\
\hline
---+++0+--++--++++0---0+-++---+-++0--++--+--+-+++---++-+--+++0+\\
--+--++0-++-----++-++-+-++-+-+++-+--++-++----+-++++0--+0-+++0++\\
\hline
---+++0+---+-----+0---0+++++-+++++0--+++----++----++-+++--+--0-\\
-----++0-++++++---+++------+-------++-+++++-++-+---0++-0+---0--\\
\hline
---+++0+---+------0---0+++++++++++0--+++----++----++-+++-----0-\\
-----++0-++++++---+++------+-------++-+++++-++-+---0++-0+---0--\\
\hline
---+++0-++--++----0+++0-+--+++-+--0+-++--+--++-++---++-+++---0-\\
++-++--0-++--+--++-++---++-+-++--+--++-++---++-++++0--+0-+++0++\\
\hline
---+++0-++--++----0+++0-+--+++-+--0+-++--+--++-++---++-+++---0-\\
++-++--0-++--+--++-++---++-+-++--+--++-++---++-+---0++-0+---0--\\
\hline
---+++0-++--++----0+++0-+--+++-+--0+-++--+--+-+++---++-+++---0-\\
++-++--0-++-----++-++-+-++-+-+++-+--++-++----+-+---0++-0+---0--\\
\hline
---+++0-+++-++++++0+++0-----------0+-+++----++----++-++++++++0+\\
+++++--0-++++++---+++------+-------++-+++++-++-++++0--+0-+++0++\\
\hline
---+++0-+++-+++++-0+++0-----+-----0+-+++----++----++-+++++-++0+\\
+++++--0-++++++---+++------+-------++-+++++-++-++++0--+0-+++0++\\
\hline
---+++0-----------0+++0+++-++++++-0+----++++--++++--+--+++---0-\\
-------0-------+++---+++++++-++++++--+-----+--++---0++-0+---0--\\
\end{tabular}}
\end{center}
\caption{\label{torusom6} Chirotope zum sechsten Torusmatroid}
\end{table}

\clearpage
\section{Zusammenfassung der Vorgehensweise}

Da nun alle Schritte abgearbeitet wurden, die zu einem vorgelegten CW-Komplex
(angegebenen Aufbaus) über Symmetrie(unter)gruppen verträgliche orientierte
Matroide in Form von Chirotopen erzeugen, ist das Ziel dieser Arbeit erreicht.

Zusammenfassend kann die "`verträgliche Chirotoperzeugung"' nun wie folgt
beschrieben werden:

\begin{itemize}
\item Gegeben sei ein CW-Komplex $\C$, mit (im Fall eines 2-Komplexes)
      polygonal berandeten 2-Zellen.
\item Bestimme zu $\C$ dessen kombinatorische Symmetriegruppe Aut($\C$).
\item Bestimme eine Grundstruktur für die verträglichen Matroide.
\item Wähle aus Aut($\C$) eine "`geeignete"' Untergruppe aus und lege
      diese als feste (Untergruppe einer) Symmetriegruppe G verträglicher
      orientierter Matroide vor.
\item Erzeuge rekursiv bezüglich G verträgliche Matroide $\ul{\cal M}$
      mit der gewünschten Grundstruktur.
\item Versuche jedes so entstandene verträgliche Matroid $\ul{\cal M}$ zu
      orientieren, so daß $\ul{\cal M}$ zu bezüglich G verträglichen
      orientierten Matroiden $\cal M$ in Form von Chirotopen $\chi$ wird.
\item Nach der Bearbeitung liegen nun zu $\C$ im Existenzfall alle
      bezüglich $G\leq \mbox{Aut}(\C)$ verträglichen Chirotope $\chi$ vor,
      deren zugrunde liegendes Matroid die gewünschte Grundstruktur besitzt
      und eine Symmetriegruppe aufweisen, die zumindest G enthält.
\end{itemize}

In dem nun anschließenden letzten Kapitel sollen dazu noch einige Beispiele
bearbeitet werden.

\chapter{Beispiele}

In diesem abschließenden dritten Kapitel sollen nun mittels des Schemas aus
Kapitel 2 einige CW-Komplexe bearbeitet und als Beispiele dargestellt werden.

\section{Die Papposkonfiguration in der Ebene}

Zu Beginn widmen wir uns als einem Rang 3 Beispiel dem Satz von Pappos, der
besagt\idx{Pappos, Satz von}:
\begin{quote}
Auf zwei Geraden $g_1$ und $g_2$ seien je drei Punkte $A_1,~B_1,~C_1$ und
$A_2,~B_2,~C_2$ gegeben, die vom Schnittpunkt $O$ der beiden Geraden
verschieden sind. Dann liegen die Schnittpunkte
$A=(B_1C_2\cap B_2C_1)$, $B=(A_1C_2\cap A_2C_1)$ und $C=(A_1B_2\cap A_2B_1)$
der kreuzweisen Verbindungen auf einer Geraden $g$.
\end{quote}
Die entsprechende Abbildung hierzu (Abb.\ref{pappos}) liefert nun einen
eindimensionalen CW-Komplex, dessen 0-Zellen mit den Ziffern 1 bis 9 indiziert
seien. 

\begin{figure}[htb]
$$
\beginpicture
\unitlength1cm
\setlinear
\setcoordinatesystem units <1cm,1cm>
\setplotarea x from 0 to 12, y from 0 to 6
\plot 1.5 1.5 6 4 /
\plot 1.5 1.5 10.5 4.5 /
\plot 1.5 1.5 8.5 0.5 /
\plot 1.5 3.5 10.5 4.5 /
\plot 1.5 3.5 8.5 0.5 /
\plot 1.5 3.5 5 1 /
\plot 5 1 10.5 4.5 /
\plot 6 4 8.5 0.5 /
\plot 3.1 2.38 7.15 2.38 /
\put {\circle*{0.1}} [Bl] at 1.5 1.5
\put {\circle*{0.1}} [Bl] at 5 1
\put {\circle*{0.1}} [Bl] at 8.5 0.5
\put {\circle*{0.1}} [Bl] at 1.5 3.5
\put {\circle*{0.1}} [Bl] at 6 4
\put {\circle*{0.1}} [Bl] at 10.5 4.5
\put {\circle*{0.1}} [Bl] at 3.075 2.38
\put {\circle*{0.1}} [Bl] at 4.125 2.38
\put {\circle*{0.1}} [Bl] at 7.175 2.38
\put {$A_1$} [br] at 1.5 3.55
\put {$B_1$} [br] at 6 4.05
\put {$C_1$} [br] at 10.5 4.55
\put {$A_2$} [tr] at 1.5 1.45
\put {$B_2$} [tr] at 5 0.95
\put {$C_2$} [bl] at 8.5 0.55
\put {$A$} [Bl] at 2.2 2.3
\put {$B$} [Bl] at 4.1 1.9
\put {$C$} [Bl] at 7.5 2.2
\endpicture
$$
\caption{CW-Komplex zum Satz von Pappos}
\label{pappos}
\end{figure}

Aus der Abbildung kann man diese Indizierung etwa nach $A_1\hat{=}1$,
$B_1\hat{=}2$, $C_1\hat{=}3$, $A_2\hat{=}4$, $B_2\hat{=}5$, $C_2\hat{=}6$,
$A\hat{=}7$, $B\hat{=}8$ und $C\hat{=}9$ anbringen. Zur Bestimmung der
Symmetriegruppe überlegt man sich, daß in diesem Fall jedes
0-Zellen-Tripel einer Geraden auf ein anderes 0-Zellen-Tripel, welches
ebenfalls auf einer Geraden liegen soll, abzubilden ist.\\
Es sind also jene Permutationen aus der ${\cal S}_9$ gesucht, die
die Menge
$$\left\{\{1,2,3\},\{1,4,8\},\{1,5,9\},\{2,4,7\},\{2,6,9\},\{3,5,7\},
\{3,6,8\},\{4,5,6\},\{7,8,9\}\right\}$$
auf sich abbilden.

Testet man nun mit einem Programm alle 9! Permutationen der 0-Zellen, so
erhält man 108 Automorphismen, die die Automorphismengruppe des Papposkomplexes
darstellen. Diese sind in Tabelle \ref{papposaut} aufgezeigt.

\begin{table}%[htb]
{\footnotesize
$$
\begin{array}{lllll}
identity&(49)(58)(67)&(23)(45)(89)&(23)(48)(59)(67)&(24)(38)(59)\\
(298354)(67)&(285)(349)&(25)(39)(67)&(28)(34)(67)&(258)(394)\\
(245389)(67)&(29)(35)(48)&(12)(56)(78)&(12)(49)(57)(68)&(132)(456)(798)\\
(132)(486957)&(142)(387)(569)&(198642)(357)&(187562)(349)&(157862)(39)\\
(186572)(34)&(156872)(394)&(145792)(386)&(192)(356)(487)&(123)(465)(789)\\
(123)(475968)&(13)(46)(79)&(13)(47)(58)(69)&(146973)(285)&(196473)(25)\\
(183)(246)(597)&(154783)(296)&(189653)(247)&(153)(297)(468)&(147963)(28)\\
(197463)(258)&(124)(378)(596)&(129754)(368)&(14)(37)(69)&(197364)(58)\\
(14)(25)(36)(79)&(196374)(285)&(153624)(789)&(184)(296)(375)&(154)(237)(689)\\
(184)(236597)&(137964)(258)&(136974)(28)&(127865)(349)&(126875)(39)\\
(145)(273)(698)&(195)(264873)&(142635)(798)&(195)(274)(386)&(15)(26)(34)(78)\\
(186275)(394)&(15)(27)(68)&(187265)(49)&(135)(279)(486)&(138745)(269)\\
(176)(349)&(16)(39)(58)&(176)(245983)&(16)(295483)&(16)(24)(35)(89)\\
(176)(295384)&(16)(25)(34)&(176)(285)(394)&(176)(258)&(16)(28)(49)\\
(176)(235489)&(16)(238459)&(17)(34)(58)&(167)(394)&(17)(248953)\\
(167)(298453)&(167)(238954)&(17)(235984)&(167)(285)&(17)(25)(49)\\
(167)(258)(349)&(17)(28)(39)&(167)(248359)&(17)(29)(38)(45)&(127568)(34)\\
(126578)(394)&(148)(279563)&(198)(263)(457)&(138)(264)(579)&(135698)(274)\\
(18)(26)(57)&(156278)(49)&(157268)(349)&(18)(27)(39)(56)&(148)(269)(357)\\
(192738)(456)&(124689)(375)&(129)(365)(478)&(146379)(58)&(19)(36)(47)\\
(136479)(285)&(137469)(25)&(189)(236)(475)&(159)(237846)&(159)(247)(368)\\
(183729)(465)&(147369)(258)&(19)(28)(37)(46)&&
\end{array}$$}
\caption{Die Automorphismengruppe des Papposkomplexes}
\label{papposaut}
\end{table}

Die sich ergebenden verschiedenen zyklischen Untergruppen dieser
Automorphismengruppe sind, wenn obige Zykel der Reihe nach durchnummeriert
werden, aus Tabelle \ref{papposgrp} abzulesen.

\begin{table}%[htb]
{\footnotesize $$\begin{array}{ll}
\{id,[6],[7],[4],[10],[11]\} & \{id,[16],[25],[2],[15],[26]\}\\
\{id,[18],[44],[70],[95],[97]\} & \{id,[19],[68],[57],[81],[86]\}\\
\{id,[20],[69],[94],[79],[50]\} & \{id,[21],[79],[55],[69],[85]\}\\
\{id,[22],[81],[91],[68],[49]\} & \{id,[23],[105],[80],[54],[38]\}\\
\{id,[29],[81],[100],[68],[47]\} & \{id,[30],[74],[41],[61],[102]\}\\
\{id,[32],[95],[73],[44],[60]\} & \{id,[33],[54],[62],[105],[90]\}\\
\{id,[35],[61],[108],[74],[48]\} & \{id,[36],[68],[39],[81],[101]\}\\
\{id,[40],[61],[28],[74],[99]\} & \{id,[42],[81],[27],[68],[107]\}\\
\{id,[43],[15],[67],[25],[53]\} & \{id,[46],[95],[8],[44],[87]\}\\
\{id,[52],[105],[9],[54],[104]\} & \{id,[56],[81],[13],[68],[93]\}\\
\{id,[58],[69],[14],[79],[92]\} & \{id,[63],[81],[12],[68],[77]\}\\
\{id,[64],[10],[65],[7],[72]\} & \{id,[66],[81],[3],[68],[83]\}\\
\{id,[71],[81],[5],[68],[76]\} & \{id,[75],[7],[84],[10],[78]\}\\
\{id,[96],[25],[82],[15],[106]\} & \{id,[17],[37]\}\\
\{id,[24],[98]\} & \{id,[31],[89]\}\\
\{id,[34],[59]\} & \{id,[45],[51]\}\\
\{id,[88],[103]\} & \{id\}
\end{array}$$}
\caption{Die zyklischen Untergruppen zum Papposkomplex}
\label{papposgrp}
\end{table}

Nach der Bestimmung dieser Gruppen zum Papposkomplex kann nun begonnen werden
zugehörige Matroide zu suchen. Dazu betrachten wir zuerst die Gestalt des
vorgelegten CW-Komplexes, nach dem im Rang 3 die drei Punkte auf einer Geraden
sicher eine Nichtbasis des Matroids liefern. Damit stehen folgende Brackets
sicher für die Determinanten singulärer Matrizen und können als Nichtbasen
sicher zu Null gesetzt werden:
$$[123],[148],[159],[247],[269],[357],[368],[456]\mbox{ und }[789]$$
Aus den "`benachbarten 1-Zellen"' folgt nun eine Wahl für Basen, die
ein verträgliches Matroid sicherlich besitzt, womit sich eine
Grundstruktur zu
{\tt $$
\begin{array}{c}
0111111111111111011110?111111111111101111?\\
11101111111?101110111101111111111111111110
\end{array}
$$}

ergibt. Mittels {\sc Sym2mat} erhält man hierzu bei Vorgabe der vollen
Symmetriegruppe Aut($\cal C$) die beiden Matroide
{\tt $$
\begin{array}{c}
011111111111111101111001111111111111011110\\
111011111110101110111101111111111111111110
\end{array}
$$}

und
{\tt $$
\begin{array}{c}
011111111111111101111011111111111111011111\\
111011111111101110111101111111111111111110
\end{array}
$$}

von denen das erste eine Symmetriegruppe der Ordnung 432 und das zweite eine
Symmetriegruppe der Ordnung 108 besitzt, die jeweils die komplette
Automorphismengruppe des Papposkomplexes enthalten beziehungsweise zu dieser
identisch sind.

Zur Symmetrieuntergruppe mit dem erzeugenden Element $\sigma=(298354)(67)$
erhält man zwei weitere, nämlich
{\tt $$
\begin{array}{c}
011111111111111101111001111111111111011111\\
111011111111101110111101111111111111111110
\end{array}
$$}
sowie
{\tt $$
\begin{array}{c}
011111111111111101111011111111111111011110\\
111011111110101110111101111111111111111110
\end{array}
$$}

Bezüglich der Symmetriegruppe $G=\{<(298354)(67)>\}$ sind diese Matroide
allerdings mittels des vorgestellten Algorithmus nicht orientierbar.

\clearpage
\section{Der dreidimensionale Würfel}

Ein vielverwendetes Beispiel (vgl. \cite{Bj:93}) stellt der dreidimensionale
Würfel dar, der auch hier behandelt werden soll. Gegeben als ein aus
Vierecken aufgebauter CW-Komplex, läßt er sich durch die Liste der Polygonzüge
$$\left\{\{1264\},\{2586\},\{5378\},\{3147\},\{4687\},\{1253\}\right\}$$
beschreiben.

\begin{figure}[htb]
$$
\beginpicture
\unitlength1cm
\setlinear
\setcoordinatesystem units <1cm,1cm>
\setplotarea x from 0 to 7, y from 0 to 6
\plot 0.5 2 0.5 3.5 2 3.5 2 5 3.5 5 3.5 3.5 5 3.5 6.5 3.5 6.5 2 5 2 3.5 2
      3.5 0.5 2 0.5 2 2 0.5 2 /
\plot 2 2 2 3.5 3.5 3.5 3.5 2 2 2 /
\plot 5 3.5 5 2 /
\put {1} [tl] at 2.05 4.95 \put {3} [tr] at 3.45 4.95
\put {1} [tl] at 0.55 3.45 \put {2} [tr] at 1.95 3.45
\put {5} [tr] at 3.45 3.45 \put {3} [tr] at 4.95 3.45
\put {1} [tr] at 6.45 3.45 \put {4} [bl] at 0.55 2.05
\put {6} [br] at 1.95 2.05 \put {8} [br] at 3.45 2.05
\put {7} [br] at 4.95 2.05 \put {4} [br] at 6.45 2.05
\put {4} [bl] at 2.05 0.55 \put {7} [br] at 3.45 0.55
\endpicture
$$
\caption{CW-Komplex zum 3-Würfel}
\label{wuerfel}
\end{figure}

Seine Symmetriegruppe hat die Ordnung 48 und setzt sich aus den
Permutationen aus Tabelle \ref{cubeaut} zusammen. Betrachet man nun
noch den Komplex zur Erzeugung einer Grundstruktur für die verträglichen
Matroide, so ist diese wie folgt gegeben:

{\small\tt
$$10111101111111?11011111?111?111?111111?111?111?11111011?11111110111101$$
}

Bezüglich der Symmetriegruppe $\{<(147852)(36)>\}$ erhält man daraus folgende
verträglichen Matroide
{\small\tt
$$\begin{array}{c}
1011110111111101101111101110111011111101110111011111011011111110111101\\[1mm]
1011110111111101101111101110111111111111110111011111011011111110111101\\[1mm]
1011110111111101101111111110111011111101111111011111011111111110111101\\[1mm]
1011110111111101101111111110111111111111111111011111011111111110111101\\[1mm]
1011110111111111101111101111111011111101110111111111011011111110111101\\[1mm]
1011110111111111101111101111111111111111110111111111011011111110111101\\[1mm]
1011110111111111101111111111111011111101111111111111011111111110111101\\[1mm]
1011110111111111101111111111111111111111111111111111011111111110111101
\end{array}
$$
}

\begin{table}%[htb]
{\footnotesize
$$
\begin{array}{lllll}
\mbox{identity}&(34)(56)&(23)(67)&(243)(567)&(234)(576)\\
(24)(57)&(12)(35)(46)(78)&(12)(36)(45)(78)&(1352)(4786)&(147852)(36)\\
(137862)(45)&(1462)(3785)&(1253)(4687)&(126873)(45)&(13)(25)(47)(68)\\
(1473)(2685)&(13)(27)(45)(68)&(146853)(27)&(125874)(36)&(1264)(3587)\\
(1374)(2586)&(14)(26)(37)(58)&(135864)(27)&(14)(27)(36)(58)&(15)(48)\\
(165)(348)&(15)(23)(48)(67)&(1675)(2483)&(1765)(2348)&(175)(248)\\
(156)(384)&(16)(38)&(1576)(2384)&(16)(24)(38)(57)&(176)(238)\\
(1756)(2438)&(1567)(2843)&(167)(283)&(157)(284)&(1657)(2834)\\
(17)(28)&(17)(28)(34)(56)&(18)(25)(36)(47)&(18)(264735)&(18)(253746)\\
(18)(26)(37)(45)&(18)(27)(35)(46)&(18)(27)(36)(45)
\end{array}$$}
\caption{Die Automorphismengruppe des 3-Würfels}
\label{cubeaut}
\end{table}

Legt man wiederum $\{<(147852)(36)>\}$ als Symmetriegruppe für verträgliche
orientierte Matroide vor, so erhält man nur für eins der oben bestimmten Matroide
Vorzeichenlisten:

{\small\tt
$$\begin{tabular}{c}
+0+-+-0-++-+-+0-+0++-++0---0+--+---+-+--+-0-++0+-++-0++0-++--+-0--++0-\\[1mm]
-0-+-+0+--+-+-0+-0--+--0+++0-++-+++-+-++-+0+--0-+--+0--0+--++-+0++--0+\\[1mm]
-0++--0-+--++-0++0-++-+0++-0-+--+-+---+--+0---0++-++0+-0+-+-++-0-+-+0+\\[1mm]
+0--++0+-++--+0--0+--+-0--+0+-++-+-+++-++-0+++0--+--0-+0-+-+--+0+-+-0-
\end{tabular}
$$
}

Diese orientierten Matroiden besitzen die Grundstruktur des "`echten"' Würfelmatroids
(siehe Abbildung \ref{cube}). Dabei ist die zweite Vorzeichenliste die negative der ersten,
sowie die vierte die negative der dritten, es liegen also nur zwei verschiedene
Orientierungen zum CW-Komplex "`Würfel"' mit der gegebenen Symmetriegruppe vor. Interessant
ist hier zu erwähnen, daß die erste Orientierungsvariante eine Symmetriegruppe
der Ordnung 12 besitzt, während die zweite Variante die volle Symmetriegruppe des
Würfelkomplexes der Ordnung 48 aufweist.

\clearpage
\section{Ein weiterer Torus}

Als nächstes Beispiel sei ein weiterer Torus in Gestalt des CW-Komplexes
in Abbildung \ref{tor2} vorgelegt.

\begin{figure}[htb]
$$
\beginpicture
\unitlength0.75cm
\setlinear
\setcoordinatesystem units <0.75cm,0.75cm>
\setplotarea x from 0 to 7, y from 0 to 7
\setsolid
\plot 0.5 0.5 0.5 6.5 6.5 6.5 6.5 0.5 0.5 0.5 /
\plot 0.5 2.5 6.5 2.5 /
\plot 0.5 4.5 6.5 4.5 /
\plot 2.5 0.5 2.5 6.5 /
\plot 4.5 0.5 4.5 6.5 /
\plot 0.5 2.5 2.5 0.5 /
\plot 0.5 4.5 4.5 0.5 /
\plot 0.5 6.5 6.5 0.5 /
\plot 2.5 6.5 6.5 2.5 /
\plot 4.5 6.5 6.5 4.5 /
\put {1} [br] at 0.45 6.55 \put {2} [br] at 2.5 6.55
\put {3} [bl] at 4.5 6.55  \put {1} [bl] at 6.55 6.55
\put {4} [br] at 0.45 4.5  \put {5} [tr] at 2.45 4.45
\put {6} [tr] at 4.45 4.45 \put {4} [bl] at 6.55 4.5
\put {8} [tr] at 0.45 2.5  \put {9} [bl] at 2.55 2.55
\put {7} [bl] at 4.55 2.55 \put {8} [tl] at 6.55 2.5
\put {1} [tr] at 0.45 0.45 \put {2} [tr] at 2.5 0.45
\put {3} [tl] at 4.5 0.45  \put {1} [tl] at 6.55 0.45
\endpicture
$$
\caption{Ein weiterer Torus}
\label{tor2}
\end{figure}

Dieser CW-Komplex, der auch einen simplizialen Komplex darstellt, besitzt
eine Symmetriegruppe der Ordnung 108. Aus dieser legen wir die Untergruppe der
Ordnung 6 mit den Erzeugenden (123)(456)(789) und (16)(25)(34)(78)(9) als
Symmetriegruppe vor, bezüglich der verträgliche orientierte Matroide erzeugt
werden sollen.

Aus den Zellenberandungen läßt sich für verträgliche Matroide zunächst
folgende Grundstruktur angeben,

\begin{center}
{\footnotesize\tt\begin{tabular}{c}
??????1????1?1????1?1111?????????11????1????????????1????1???1?\\
????1?11?????????11??????????????1????????1?????????111?1?11???
\end{tabular}}
\end{center}

mittels der sich mit {\sc Sym2mat} insgesamt 52 verträgliche Matroide
erzeugen lassen. Diese Matroide sind in den Tabellen \ref{tor6matA}
und \ref{tor6matB} aufgezeigt.

Zu diesen 52 Matroiden lassen sich nun mittels eines Orientierungsprogramms,
welches in der in Kapitel 2 vorgestellten Vorgehensweise fußt, zu insgesamt
17 Matroiden Vorzeichenlisten erzeugen. Von diesen 17 sind 7 unzuläßig,
da sie zu mehr als einen Symmetrietyp Vorzeichenlisten liefern, also nicht
eindeutig bezüglich einer Symmetriegruppe orientierbar sind.
Die 10 verbliebenen orientierbaren Matroide sind in Tabelle \ref{mattor}
aufgezeigt.

Stellvertretend für alle erzeugten Orientierungen der verträglichen Matroide
sind in Tabelle \ref{orimattor} die Orientierungen des fünften Matroids
von Tabelle \ref{mattor} aufgezeigt. Hierbei wurden die erzeugten negativen
Listen zu bereits vorkommenden weggelassen.

Die Auswahl gerade dieses Matroids ist nicht willkürlich, sondern entlehnt
sich der Kenntnis, daß eine nichtsimpliziale Realisierung des Torus von
Abbildung \ref{tor2} existiert, die gerade ein solches zugrunde liegendes
Matroid besitzt. Mit dem Programm {\sc Chiman} zur Manipulation von Chirotopen
von Peter Schuchert läßt sich zu den erzeugten orientierten Matroiden der
Aufbau der induzierten konvexen Hülle einer entsprechenden Punktkonfiguration
ermitteln. Dadurch ist bestimmt, welche formalen Punkte die konvexe Hülle einer
entsprechenden Punktkonfiguration zum vorliegenden orientierten Matroid
aufspannen. Für die orientierten Matroide aus Tabelle \ref{orimattor} sind
dies für das erste, zweite, vierte und fünfte alle 9 Punkte, für das dritte
und sechste nur die Punkte 1 bis 6. Für die übrigen orientierten Matroide
liefert {\sc Chiman} keine Ergebnisse.

Abbildung \ref{tor2real} zeigt das äußere Erscheinungsbild der Realisierung
des Torus in Form eines Oktaeder, in dem zwei gegenüberliegende Dreiecke
die Öffnungen des Henkels bilden. Im Innern befinden sich die restlichen
drei Punkte, die auch die Spiegelsymmetrie induzieren.

\begin{figure}[htb]
$$
\beginpicture
\unitlength1cm
\setlinear
\setcoordinatesystem units <1cm,1cm>
\setplotarea x from 0 to 6, y from 0 to 6
\plot 3 5.5 1.5 4 0.5 2 3 0.5 4.5 2 5.5 4 3 5.5 /
\plot 3 5.5 0.5 2 4.5 2 3 5.5 /
\setdashes<1mm>
\plot 1.5 4 3 0.5 5.5 4 1.5 4 /
\setlinear
\setshadegrid span <1.5pt>
\vshade 0.5 2 2 <,z,,> 1.5 1.4 4 3 0.5 0.5 /
\vshade 3 5.5 5.5 <,z,,> 4.5 2 4.6 5.5 4 4 /
\setdashes<5mm>
\plot 0.5 1.4 6 4.8 /
\put {\circle*{0.1}} [Bl] at 3 5.5
\put {\circle*{0.1}} [Bl] at 1.5 4
\put {\circle*{0.1}} [Bl] at 5.5 4
\put {\circle*{0.1}} [Bl] at 0.5 2
\put {\circle*{0.1}} [Bl] at 4.5 2
\put {\circle*{0.1}} [Bl] at 3 0.5
\endpicture
$$
\caption{Realisierung des zweiten Torus}
\label{tor2real}
\end{figure}

\begin{table}[htb]
\begin{center}
{\scriptsize\tt
\begin{tabular}{c}
000000111011110011111111011110011111010110100010111111101110111\\
100111110110110111110000111000110111110111110000000111111011100\\[1mm]
000000111011110011111111011110011111010110101010111111111110111\\
100111110110110111110100111100110111110111110011000111111111111\\[1mm]
000000111101011101111111101011101111001101111101100011101111010\\
111011110101101111101111000001101100001101111110000111111011100\\[1mm]
000000111101011101111111101011101111001101111101100111111111010\\
111011110101101111101111010101101101001101111111000111111111111\\[1mm]
000000111111111111111111111111111111011111100011111111101111111\\
111111110111111111111000111001111111111111110000000111111011100\\[1mm]
000000111111111111111111111111111111011111110111111011111111111\\
111111110111111111111011101101111110111111111101000111111111111\\[1mm]
000000111111111111111111111111111111011111110111111111111111111\\
111111110111111111111011111101111111111111111101000111111111111\\[1mm]
000000111111111111111111111111111111011111111111100011101111111\\
111111110111111111111111000001111110001111111110000111111011100\\[1mm]
000000111111111111111111111111111111011111111111111011111111111\\
111111110111111111111111101101111110111111111111000111111111111\\[1mm]
000000111111111111111111111111111111011111111111111111101111111\\
111111110111111111111111111001111111111111111110000111111011100\\[1mm]
000000111111111111111111111111111111011111111111111111111111111\\
111111110111111111111111111101111111111111111111000111111111111\\[1mm]
111111101011010101101111000011011110100110101001100111110111011\\
100010111110101111100100010111100101000111110011111011101111011\\[1mm]
111111101011011101111111100011011111101110111101100011100111011\\
101011111110101111101111000011101100000111111110111111111011100\\[1mm]
111111101011011101111111100011011111101110111101100111110111011\\
101011111110101111101111010111101101000111111111111111111111111\\[1mm]
111111101011110101111111010011011111110110100001111111100111011\\
100111111110101111110000111011100111110111110000111111111011100\\[1mm]
111111101011110101111111010011011111110110101001111111110111011\\
100111111110101111110100111111100111110111110011111111111111111\\[1mm]
111111101011111101101111110011011110111110111101111111110111011\\
101110111110101111111111111111101111110111111111111011101111011\\[1mm]
111111101011111101111111110011011111111110110101111011110111011\\
101111111110101111111011101111101110110111111101111111111111111\\[1mm]
111111101011111101111111110011011111111110110101111111110111011\\
101111111110101111111011111111101111110111111101111111111111111\\[1mm]
111111101011111101111111110011011111111110111101111011110111011\\
101111111110101111111111101111101110110111111111111111111111111\\[1mm]
111111101011111101111111110011011111111110111101111111100111011\\
101111111110101111111111111011101111110111111110111111111011100\\[1mm]
111111101011111101111111110011011111111110111101111111110111011\\
101111111110101111111111111111101111110111111111111111111111111\\[1mm]
111111100101111010101111110100100110111101010110011010010100110\\
011110111001010001111011101110011110111000101101111011101111011\\[1mm]
111111101111111111101111110111111110111111111111111111110111111\\
111110111111111111111111111111111111111111111111111011101111011\\[1mm]
111111101111111111111111110111111111111111100011111111100111111\\
111111111111111111111000111011111111111111110000111111111011100\\[1mm]
111111101111111111111111110111111111111111110111111011110111111\\
111111111111111111111011101111111110111111111101111111111111111
\end{tabular}}
\end{center}
\caption{\label{tor6matA} Matroide zum zweiten Torusbeispiel Teil A}
\end{table}

\begin{table}[htb]
\begin{center}
{\scriptsize\tt
\begin{tabular}{c}
111111101111111111111111110111111111111111110111111111110111111\\
111111111111111111111011111111111111111111111101111111111111111\\[1mm]
111111101111111111111111110111111111111111111111100011100111111\\
111111111111111111111111000011111110001111111110111111111011100\\[1mm]
111111101111111111111111110111111111111111111111111011110111111\\
111111111111111111111111101111111110111111111111111111111111111\\[1mm]
111111101111111111111111110111111111111111111111111111100111111\\
111111111111111111111111111011111111111111111110111111111011100\\[1mm]
111111101111111111111111110111111111111111111111111111110111111\\
111111111111111111111111111111111111111111111111111111111111111\\[1mm]
111111110111111110111111111101110111111111011111011110011101111\\
011111111011011001111111111111011111111010101111111111111111111\\[1mm]
111111111011110111111111011111011111110110100011111111101111111\\
100111111110111111110000111011110111110111110000111111111011100\\[1mm]
111111111011110111111111011111011111110110101011111111111111111\\
100111111110111111110100111111110111110111110011111111111111111\\[1mm]
111111111101111111111111111111101111111101111111100011101111110\\
111111111101111111111111000011111110001101111110111111111011100\\[1mm]
111111111101111111111111111111101111111101111111111011111111110\\
111111111101111111111111101111111110111101111111111111111111111\\[1mm]
111111111101111111111111111111101111111101111111111111101111110\\
111111111101111111111111111011111111111101111110111111111011100\\[1mm]
111111111101111111111111111111101111111101111111111111111111110\\
111111111101111111111111111111111111111101111111111111111111111\\[1mm]
111111111111011101111111101011111111101111111101100011101111011\\
111011111111101111101111000011101100001111111110111111111011100\\[1mm]
111111111111011101111111101011111111101111111101100111111111011\\
111011111111101111101111010111101101001111111111111111111111111\\[1mm]
111111111111111011111111111110111111111111100010111111101110111\\
111111111111110111111000111010111111111111110000111111111011100\\[1mm]
111111111111111011111111111110111111111111110110111111111110111\\
111111111111110111111011111110111111111111111101111111111111111\\[1mm]
111111111111111011111111111110111111111111111110111111101110111\\
111111111111110111111111111010111111111111111110111111111011100\\[1mm]
111111111111111011111111111110111111111111111110111111111110111\\
111111111111110111111111111110111111111111111111111111111111111\\[1mm]
111111111111111111101111111111111110111111111111111111111111111\\
111110111111111111111111111111111111111111111111111011101111011\\[1mm]
111111111111111111111111111111111111111111100011111111101111111\\
111111111111111111111000111011111111111111110000111111111011100\\[1mm]
111111111111111111111111111111111111111111110111111011111111111\\
111111111111111111111011101111111110111111111101111111111111111\\[1mm]
111111111111111111111111111111111111111111110111111111111111111\\
111111111111111111111011111111111111111111111101111111111111111\\[1mm]
111111111111111111111111111111111111111111111111100011101111111\\
111111111111111111111111000011111110001111111110111111111011100\\[1mm]
111111111111111111111111111111111111111111111111111011111111111\\
111111111111111111111111101111111110111111111111111111111111111\\[1mm]
111111111111111111111111111111111111111111111111111111101111111\\
111111111111111111111111111011111111111111111110111111111011100\\[1mm]
111111111111111111111111111111111111111111111111111111111111111\\
111111111111111111111111111111111111111111111111111111111111111
\end{tabular}}
\end{center}
\caption{\label{tor6matB} Matroide zum zweiten Torusbeispiel Teil B}
\end{table}

\begin{table}%[htb]
\begin{center}
{\scriptsize\tt\begin{tabular}{c}
111111101011011101111111100011011111101110111101100111110111011\\
101011111110101111101111010111101101000111111111111111111111111\\[1mm]
111111101011110101111111010011011111110110101001111111110111011\\
100111111110101111110100111111100111110111110011111111111111111\\[1mm]
111111101011111101111111110011011111111110110101111111110111011\\
101111111110101111111011111111101111110111111101111111111111111\\[1mm]
111111101011111101111111110011011111111110111101111011110111011\\
101111111110101111111111101111101110110111111111111111111111111\\[1mm]
111111101011111101111111110011011111111110111101111111110111011\\
101111111110101111111111111111101111110111111111111111111111111\\[1mm]
111111101111111111101111110111111110111111111111111111110111111\\
111110111111111111111111111111111111111111111111111011101111011\\[1mm]
111111101111111111111111110111111111111111111111111111110111111\\
111111111111111111111111111111111111111111111111111111111111111\\[1mm]
111111111101111111111111111111101111111101111111111111111111110\\
111111111101111111111111111111111111111101111111111111111111111\\[1mm]
111111111111111011111111111110111111111111111110111111111110111\\
111111111111110111111111111110111111111111111111111111111111111\\[1mm]
111111111111111111111111111111111111111111111111111111111111111\\
111111111111111111111111111111111111111111111111111111111111111
\end{tabular}}
\end{center}
\caption{\label{mattor}Die orientierbaren Matroide zum zweiten Torus}
\end{table}

\begin{table}%[htb]
\begin{center}
{\scriptsize\tt\begin{tabular}{c}
+++++++0+0--+--+0----++++-00+-0+-+--+-++-0++++0-++-++-++0+-+0+-\\
+0-+-++-+-+0-0++-++-+--++-+++-+0--++-+0-++-++--+---++-+++--++--\\[1mm]
+++++++0+0--+--+0----++++-00+-0+-+--+-++-0++-+0-++-++-++0+-+0+-\\
+0-+-++-+-+0-0++-++-++-++-+++-+0--++-+0-++-++-++---++-+++--++--\\[1mm]
+++++++0+0-----+0----+++++00+-0+-+--++++-0++++0-+-+++-++0+-+0+-\\
+0---++-+-+0-0++-++++--+---++-+0---++-0-++-++--+---++-+++--++--\\[1mm]
+++++++0-0+---+-0+++++++-+00-+0-+-++++-++0-+-+0+-+-++-++0++-0++\\
-0+----+++-0+0-+-+++-+-++-++++-0+--+-+0+-+-++-++-----+----+----\\[1mm]
+++++++0-0+---+-0+++++++-+00-+0-+-++++-++0-+-+0+-+--+-++0++-0++\\
-0+----+++-0+0-+-+++-+-+++++++-0+----+0+-+-++-++-----+----+----\\[1mm]
+++++++0-0+-----0+++++++++00-+0-+-+++++++0----0+-+--+-++0++-0++\\
-0-----+++-0+0-+-++++++-++++++-0-----+0+-+-+-+++-----+----+----\\[1mm]
+++---+0+0--++++0-+++++---00+-0+--+++----0+---0-++--+-+-0+-+0+-\\
+0+++--++-+0-0++-+---++-+++-+-+0+++--+0-++-+-++-+++--+---++--++\\[1mm]
+++---+0+0--++-+0-+++++-+-00+-0+--+++-+--0++-+0-++-++-+-0+-+0+-\\
+0-++--++-+0-0++-+--++-++-+-+-+0-+++-+0-++-++-+-+++--+---++--++\\[1mm]
+++---+0+0--++-+0-+++++-+-00+-0+--+++-+--0++-+0-++--+-+-0+-+0+-\\
+0-++--++-+0-0++-+--++-++++-+-+0-++--+0-++-++-+-+++--+---++--++\\[1mm]
+++---+0-0+-+++-0+---++---00-+0-++--+---+0-+++0+--+++-+-0++-0++\\
-0+++++-++-0+0-+-+-----+----++-0+++++-0+-+-++---+++++-++++-++++\\[1mm]
+++---+0-0+--++-0+---++--+00-+0-++--++--+0-+++0+-+-++-+-0++-0++\\
-0+-+++-++-0+0-+-+-+---++-+-++-0++-+-+0+-+-++---+++++-++++-++++\\[1mm]
+++---+0-0+--++-0+---++--+00-+0-++--++--+0-+-+0+-+-++-+-0++-0++\\
-0+-+++-++-0+0-+-+-+-+-++-+-++-0++-+-+0+-+-++-+-+++++-++++-++++
\end{tabular}}
\end{center}
\caption{\label{orimattor}Orientierungen für ein Matroid zum zweiten Torus}
\end{table}

\clearpage
\section{Das Dodekaeder}

Als weiteren Vertreter für die nichtsimplizialen Platonischen Körper sei
nun ein Dodekaeder als Beispiel-Komplex gewählt. Mit seinen 20 Punkten stellt
dieser Komplex schon eine Herausforderung an die Rechenzeit der implementierten
Programme dar, was gerade bei der Bestimmung der Symmetriegruppe und der
Erzeugung von Orientierungen von Matroiden als Behinderung ins Auge fällt.

\begin{figure}[htb]
$$
\beginpicture
\unitlength0.75cm
\setlinear
\setcoordinatesystem units <0.75cm,0.75cm>
\setplotarea x from -3 to 3, y from -3 to 3
\setsolid
\plot -0.173 2.407 1.312 1.267 /
\plot 1.312 1.267 2.578 0.905 /
\plot 2.578 0.905 1.874 1.821 /
\plot 1.874 1.821 0.173 2.749 /
\plot 0.173 2.749 -0.173 2.407 /
\plot -1.312 1.918 0.173 2.749 /
\plot -1.312 1.918 -2.578 1.063 /
\plot -2.578 1.063 -1.874 1.365 /
\plot -1.874 1.365 -0.173 2.407 /
\plot -1.874 1.365 -1.439 -0.417 /
\plot -1.439 -0.417 0.53 -0.477 /
\plot 0.53 -0.477 1.312 1.267 /
\plot -0.173 -2.749 -1.874 -1.821 /
\plot 1.312 -1.918 -0.173 -2.749 /
\plot 1.312 -1.918 0.53 -0.477 /
\plot -1.439 -0.417 -1.874 -1.821 /
\plot 2.578 -1.063 1.312 -1.918 /
\plot 2.578 -1.063 2.578 0.905 /
\plot -2.578 -0.905 -2.578 1.063 /
\plot -1.874 -1.821 -2.578 -0.905 /
\setdashes <1mm>
\plot 1.874 1.821 1.439 0.417 /
\plot 1.439 0.417 -0.53 0.477 /
\plot -0.53 0.477 -1.312 1.918 /
\plot -0.173 -2.749 0.173 -2.407 /
\plot 0.173 -2.407 1.874 -1.365 /
\plot 1.874 -1.365 2.578 -1.063 /
\plot 0.173 -2.407 -1.312 -1.267 /
\plot -1.312 -1.267 -0.53 0.477 /
\plot 1.439 0.417 1.874 -1.365 /
\plot -2.578 -0.905 -1.312 -1.267 /
\endpicture
$$
\caption{Ein Dodekaeder}
\label{dodeka}
\end{figure}

Von Vorteil ist allerdings, daß die Symmetriegruppe als die Duale der
Ikosaedergruppe der Ordnung 120 bekannt ist. Trotzdem ist bei ${20\choose 4}
=4845$ Elementen für die Matroide mit erheblichem Aufwand zu rechnen.
Deshalb soll zunächst die Grundstruktur der Matroide nach der Erscheinung des
vorgelegten Komplexes angegeben werden. Diese ist in Abbildung \ref{dodegrs}
zu sehen.

\begin{figure}[htb]
\begin{center}{\scriptsize\tt\begin{tabular}{l}
001111111?111????01111111?111????1111111?111????????????????????????????\\
???1111????????111????????00111???110111???11111???11???????1???????????\\
?????????01111111?111????1111111?111????????????11?????????????????11???\\
???????1??????????11????????1?????????11????????????1?????1?????????????\\
?1111111?111????11??????11????1????????????11??????????1??????????11????\\
????1?????????????????????????????1??????????????1111??????????111??????\\
????0011?????111011?????11111?????1111??????????????????????????????????\\
??????????1????????????11??????????1????????????????????????????????????\\
??????????????1??????????????11???????11?1???????11?????????????????????\\
????????????????????????????????1??011?????11111?????1111???????????????\\
??????????????????????????1??11?????1111??????11????????????????????????\\
?????????1?10111???11111???11???????????????????????????1111???111?????1\\
?????1??????????????11?????11????1????????????????????1?????????????????\\
?????????????????????????????????01111111?111????1111111?111?????????1?1\\
11????????????????????????????????????????11?11?????1?11?????10011??111?\\
????011??111??1??????????1111111?111????11??????11????1?????????????????\\
??????????????????11????????1?????????11????????????1?????1?????????????\\
?11????????????1????????????11??????????1??????????11????????1?????????1\\
1????????????1????????????????????1?????????????????????????????????????\\
??????????????????11????????????1?????1?????????????????????????????????\\
???????????????????????????????????????????????????????????1??????????11\\
????????1????????????????????????????????????????????11??????111??????11\\
???????????????????????????????????10111???11111???11???????1???????????\\
????????1111???111?????11????1??????????????111????1011??111??1?????????\\
?111??111???1?????????11??11?????????1???????????????????1111111?111????\\
11???1110011??1??????11??????????11???????????????????????????????????11\\
1?????11????11????011??11????????11??????11????1????????????11??????????\\
1??????????????????????????????11????????????1?????1??????????????1?????\\
?1111????????11???????????????????????????????????111?????11????11????01\\
1??11??1????????????????????????????????????????????????????????????????\\
????1????11??1?????1??????????????????????????????????????????????????1?\\
????????????????????????????????????????????????????????????????????1???\\
??????11????????????1????????????????????111????11?????11????1??????????\\
?????11????1011??111??1??????????111??111???1?????????11??11?????????11?\\
???????1?????????0011????11111?011????11111?11????11111?1???????????????\\
???????????????????????????????????1??????????????011????11111?11????111\\
11?????????????????????????????????11?????11????11????011??11??1?????11?\\
???11111?????????11?????????????????????????????????????????1????11??1??\\
1??1???????11?????????????????????????????????????????1???????????1?????\\
?????????????????????????????????????????????????1??1???????????????????\\
????????????????????????????????????????????????????????????????????1???\\
??1??????????????111???11???1?????????11???1?????????11????????1????????\\
?011????11111?11????11111?1?????????????????????????????????????????????\\
?????1????11??1?????11????11111?1???????11??????????????????????????????\\
????????????????11??1??1??011?????11111?????111?????????????????????????\\
?????????????????1??11?????111???????11?????????????????????????????????\\
1?11??????11???????????????????????????????????1????????????????????????\\
???????????1???????1????????????????????????????????????????????????????\\
??????????????1?????????11????11111???????1111??????????????????????????\\
????111???111??111??001101111?1???????11????????????????????????????????\\
?????????1????11??1??1???????????????????????????????????????????????11?\\
?1??1???????????????????????????????????????????????????????????????????\\
???????????????????????1?????1??????????????111???11???11??11111111???11\\
??1?????11??1?????01111?111111?1??11001???????11???????11?????????????11\\
????????????111?11?0111??????111??????111????????????11????????????111?1\\
1?011???????11?????????????????????????????????1?1??????????????????????\\
???????????1?1????????????????????????????111???11???11??11111111???11??\\
1?????11??1?????01111?111111?????111??????111????????????11?????????????\\
11?11?0111??????11?????????????????????????????????1?1??????????????????\\
???????????????1?1???????????????????????????????????????????111111?????\\
????????????????????1??1??1111111??1100??????11?????11????11??????????11\\
11011????111????11????11??????????111101?????1?????1??????????????1?????\\
1?????????111111??????????????1??????????1??1??1111111???111????11????11\\
??????????111101?????11????1??????????????1?????1?????????111111????????\\
??????1????????????????11110111110111110????1???1??1?1111111011??11??111\\
111111??11??1??1?111???????????????111111111011??11??111111111??11??1??1\\
?111???????????????1111001110111111111011111111111111100000111111???111?\\
??1111111???111111110               
\end{tabular}}\end{center}
\caption{\label{dodegrs} Die Grundstruktur zu Dodekaeder-Matroiden}
\end{figure}

Betrachtet man sich diese Grundstruktur, so erscheint es wenig sinnvoll,
dieses Beispiel testen zu wollen, da hierzu derzeitige Computer eine
Rechenzeit benötigen würden, die den zeitlichen Rahmen dieser Diplomarbeit
sprengen würde.

\clearpage
\section{Die Dycksche Karte}

In \cite{Bo:91} stellte Bokowski eine Realisierung der Dyckschen Karte
vor, deren zugehöriger CW-Komplex in Abbildung \ref{dyck} zu sehen ist.

\begin{figure}[htb]
$$
\beginpicture
\unitlength1cm
\setlinear
\setcoordinatesystem units <1cm,1cm>
\setplotarea x from -6 to 6, y from -6 to 6
\plot 0 0 3 0 3.939 0.695 4.078 1.902 4.096 2.868 /
\plot 3 0 2.121 2.121 3.939 0.695 /
\plot 2.121 2.121 4.078 1.902 /
\plot 2.121 2.121 4.096 2.868 /
\plot 0 0 2.121 2.121 2.294 3.277 1.539 4.229 0.868 4.924 /
\plot 2.121 2.121 0 3 2.294 3.277 /
\plot 0 3 1.539 4.229 /
\plot 0 3 0.868 4.924 /
\plot 0 0 0 3 -0.695 3.939 -1.902 4.078 -2.868 4.096 /
\plot 0 3 -2.121 2.121 -0.695 3.939 /
\plot -2.121 2.121 -1.902 4.078 /
\plot -2.121 2.121 -2.868 4.096 /
\plot 0 0 -2.121 2.121 -3.277 2.294 -4.229 1.539 -4.924 0.868 /
\plot -2.121 2.121 -3 0 -3.277 2.294 /
\plot -3 0 -4.229 1.539 /
\plot -3 0 -4.924 0.868 /
\plot 0 0 -3 0 -3.939 -0.695 -4.078 -1.902 -4.096 -2.868 /
\plot -3 0 -2.121 -2.121 -3.939 -0.695 /
\plot -2.121 -2.121 -4.078 -1.902 /
\plot -2.121 -2.121 -4.096 -2.868 /
\plot 0 0 -2.121 -2.121 -2.294 -3.277 -1.539 -4.229 -0.868 -4.924 /
\plot -2.121 -2.121 0 -3 -2.294 -3.277 /
\plot 0 -3 -1.539 -4.229 /
\plot 0 -3 -0.868 -4.924 /
\plot 0 0 0 -3 0.695 -3.939 1.902 -4.078 2.868 -4.096 /
\plot 0 -3 2.121 -2.121 0.695 -3.939 /
\plot 2.121 -2.121 1.902 -4.078 /
\plot 2.121 -2.121 2.868 -4.096 /
\plot 0 0 2.121 -2.121 3.277 -2.294 4.229 -1.539 4.924 -0.868 /
\plot 2.121 -2.121 3 0 3.277 -2.294 /
\plot 3 0 4.229 -1.539 /
\plot 3 0 4.924 -0.868 /
\put {\scsi 12} [bl] at 0.3 0.1
\put {\scsi 1} [tl] at 2 2
\put {\scsi 2} [tr] at 2.7 -0.1
\put {\scsi 3} [bl] at 2 -2
\put {\scsi 4} [bl] at 0.1 -2.7
\put {\scsi 5} [br] at -2 -2
\put {\scsi 6} [bl] at -2.7 0.1
\put {\scsi 7} [tr] at -2 2
\put {\scsi 8} [tr] at -0.1 2.7
\put {\scsi 9} [bl] at 4 0.7
\put {\scsi 4} [bl] at 4.1 2
\put {\scsi 11} [bl] at 4.1 2.9
\put {\scsi 10} [bl] at 2.3 3.3
\put {\scsi 3} [bl] at 1.6 4.3
\put {\scsi 11} [bl] at 0.9 5
\put {\scsi 9} [br] at -0.7 4
\put {\scsi 2} [br] at -2 4.1
\put {\scsi 11} [br] at -2.9 4.1
\put {\scsi 10} [br] at -3.3 2.3
\put {\scsi 1} [br] at -4.3 1.6
\put {\scsi 11} [br] at -5 0.9
\put {\scsi 9} [tr] at -4 -0.7
\put {\scsi 8} [tr] at -4.1 -2
\put {\scsi 11} [tr] at -4.1 -2.9
\put {\scsi 10} [tr] at -2.3 -3.3
\put {\scsi 7} [tr] at -1.6 -4.3
\put {\scsi 11} [tr] at -0.9 -5
\put {\scsi 9} [tl] at 0.7 -4
\put {\scsi 6} [tl] at 2 -4.1
\put {\scsi 11} [tl] at 2.9 -4.1
\put {\scsi 10} [tl] at 3.3 -2.3
\put {\scsi 5} [tl] at 4.3 -1.6
\put {\scsi 11} [tl] at 5 -0.9
\endpicture
$$
\caption{Die Dycksche Karte}
\label{dyck}
\end{figure}

Als Analogon zu den Platonischen Körpern besitzt die Dycksche reguläre Karte,
die in der üblichen Notation für kombinatorische 2-Mannigfaltigkeiten mit
$\{3,8\}_6$ bezeichnet wird, eine flaggentransitive Automorphismengruppe.
Ihr Aufbau ist bestimmt durch zwölf Ecken, die mit der Valenz 8 über 48 Kanten
32 Seiten in Form von Dreiecken bilden. Über die Eulerformel läßt sich hier
als Geschlecht 3 bestimmen.\\
Bei Vorlage einer 2-Mannigfaltigkeit als kombinatorischem Komplex, ist es
interessant zu erfahren, ob eine Einbettung dieser in den dreidimensionalen
euklidischen Raum existiert, die aus einer endlichen Menge ebener Polygone
besteht, deren Vereinigung frei von Selbstdurchdringungen ist und zu der
kombinatorischen Mannigfaltigkeit korrespondiert.\\
Im Jahre 1986 fand nun Bokowski eine Einbettung oben beschriebener
Karte (beschrieben in \cite{Bo:86} und \cite{HoWi:91}), die allerdings keine
Symmetrien aufweist. Mit Hilfe des vorgestellten Schemas soll nun nach
orientierten Matroiden gesucht werden, die eine mögliche symmetrische
Einbettung der Dyckschen Karte induzieren könnten.

Über den Komplex läßt sich zunächst aussagen, daß aufgrund des Aufbaus
aus Dreiecken keine offensichtlichen Nichtbasen für verträgliche Matroide
existieren, aber anhand der benachbarten Zellen wohl eine (wenn auch nicht
allzu stark besetzte) Grundstruktur induziert werden kann, die in der
folgenden Form gegeben ist:

{\tt $$
\begin{array}{c}
????????1????1?????????????????1??????1??1???????1?????\\
?????????1???????1???????????????????1????1??????1?????\\
??????????????????????1???1?????1?????1??????????1?????\\
???????1????1??????????????1??????1?????1???????????1??\\
??????????????????1???1????1?????????????????1????1????\\
????????????????1??1??????????????1?????1??????????1???\\
???????????1??1?????????????????1???????????1??1???????\\
???1???1???????????????????1????????????????11?????1???\\
1??????????1???????????1????1?????????????1????????????
\end{array}
$$}

Als Symmetriegruppe für den Komplex erhält man eine Gruppe der Ordnung 192,
deren Elemente in Tabelle \ref{dyckaut} aufgeführt sind.

\begin{table}[p]
{\scsi $$
\begin{array}{lll}
\mbox{identity} & (37)(48)(9C)(AB) & (2864)(9CAB)\\
(24)(37)(68)(BC) & (26)(48)(9A)(BC) & (26)(37)(9B)(AC)\\
(2468)(9BAC) & (28)(37)(46)(9A) & (29)(4C)(6A)(8B)\\
(2C8A6B49)(37) & (2A)(4B)(69)(8C) & (2B896C4A)(37)\\
(2B)(49)(6C)(8A) & (2A4C698B)(37) & (2C)(4A)(6B)(89)\\
(294B6A8C)(37) & (12)(34)(56)(78)(9C)(AB) & (12)(38)(47)(56)\\
(18765432)(9A) & (14765832)(9BAC) & (1652)(3874)(9B)(AC)\\
(1652)(3478)(9A)(BC) & (14365872)(BC) & (18365472)(9CAB)\\
(1C8792)(3A65B4) & (192)(3B8)(47C)(5A6) & (1B87A2)(3965C4)\\
(1A2)(3C8)(47B)(596) & (1A43B2)(5987C6) & (1B2)(3A8)(479)(5C6)\\
(1943C2)(5A87B6) & (1C2)(398)(47A)(5B6) & (13)(48)(57)(9A)\\
(1753)(9BAC) & (13)(24)(57)(68)(9C)(AB) & (1753)(2864)\\
(13)(26)(57)(BC) & (1753)(26)(48)(9CAB) & (13)(28)(46)(57)(9B)(AC)\\
(1753)(2468)(9A)(BC) & (13)(2A69)(4C8B)(57) & (1753)(2B8A6C49)\\
(13)(296A)(4B8C)(57) & (1753)(2C896B4A) & (13)(2B)(4A)(57)(6C)(89)\\
(1753)(2A8C694B) & (13)(2C)(49)(57)(6B)(8A) & (1753)(298B6A4C)\\
(12385674)(9CAB) & (12785634)(BC) & (14)(23)(58)(67)\\
(1854)(2763)(9C)(AB) & (16785234)(9BAC) & (16385274)(9A)\\
(1854)(2367)(9A)(BC) & (14)(27)(36)(58)(9B)(AC) & (1C4)(239)(5B8)(67A)\\
(1927B4)(3C85A6) & (1B4)(23A)(5C8)(679) & (1A27C4)(3B8596)\\
(194)(23B)(5A8)(67C) & (1C63A4)(27985B) & (1A4)(23C)(598)(67B)\\
(1B6394)(27A85C) & (15)(48)(9B)(AC) & (15)(37)(9A)(BC)\\
(15)(24)(68)(9A) & (15)(2864)(37)(9BAC) & (15)(26)(9C)(AB)\\
(15)(26)(37)(48) & (15)(28)(46)(BC) & (15)(2468)(37)(9CAB)\\
(15)(2B4A6C89) & (15)(2A69)(37)(4B8C) & (15)(2C496B8A)\\
(15)(296A)(37)(4C8B) & (15)(298C6A4B) & (15)(2C6B)(37)(4A89)\\
(15)(2A8B694C) & (15)(2B6C)(37)(498A) & (1256)(3874)(9A)(BC)\\
(1256)(3478)(9B)(AC) & (14325876)(9CAB) & (18325476)(BC)\\
(16)(25)(34)(78) & (16)(25)(38)(47)(9C)(AB) & (18725436)(9BAC)\\
(14725836)(9A) & (1A6)(259)(3C4)(7B8) & (1B47A6)(25C839)\\
(196)(25A)(3B4)(7C8) & (1C4796)(25B83A) & (1C6)(25B)(394)(7A8)\\
(1983C6)(25A47B) & (1B6)(25C)(3A4)(798) & (1A83B6)(25947C)\\
(1357)(9CAB) & (17)(35)(48)(BC) & (1357)(2864)(9A)(BC)\\
(17)(24)(35)(68)(9B)(AC) & (1357)(26)(48)(9BAC) & (17)(26)(35)(9A)\\
(1357)(2468) & (17)(28)(35)(46)(9C)(AB) & (1357)(2C4A6B89)\\
(17)(29)(35)(4B)(6A)(8C) & (1357)(2B496C8A) & (17)(2A)(35)(4C)(69)(8B)\\
(1357)(294C6A8B) & (17)(2C6B)(35)(498A) & (1357)(2A4B698C)\\
(17)(2B6C)(35)(4A89) & (12345678)(9A) & (12745638)(9BAC)\\
(18)(23)(45)(67)(9B)(AC) & (1458)(2763)(9A)(BC) & (16745238)(BC)\\
(16345278)(9CAB) & (1458)(2367)(9C)(AB) & (18)(27)(36)(45)\\
(1A67C8)(23B459) & (1B8)(279)(3A6)(45C) & (1967B8)(23C45A)\\
(1C8)(27A)(396)(45B) & (1B2398)(45C67A) & (1A8)(27B)(3C6)(459)\\
(1C23A8)(45B679) & (198)(27C)(3B6)(45A) & (129)(38B)(4C7)(56A)\\
(12C349)(56B78A) & (19)(3B)(5A)(7C) & (1C3A5B79)(48)\\
(149)(2B3)(58A)(6C7) & (18B769)(2A54C3) & (1C7A5B39)(2864)\\
(19)(24)(3C)(5A)(68)(7B) & (169)(2A5)(34B)(78C) & (16C389)(2B74A5)\\
(1A59)(26)(3C7B)(48) & (1B7A5C39)(26) & (189)(2C7)(36B)(4A5)\\
(14B729)(36A58C) & (1B3A5C79)(2468) & (1A59)(28)(3B7C)(46)\\
(12B34A)(56C789) & (12A)(38C)(4B7)(569) & (1B395C7A)(48)\\
(1A)(3C)(59)(7B) & (18C76A)(2954B3) & (14A)(2C3)(589)(6B7)\\
(1A)(24)(3B)(59)(68)(7C) & (1B795C3A)(2864) & (16B38A)(2C7495)\\
(16A)(295)(34C)(78B) & (1C795B3A)(26) & (195A)(26)(3B7C)(48)\\
(14C72A)(36958B) & (18A)(2B7)(36C)(495) & (195A)(28)(3C7B)(46)\\
(1C395B7A)(2468) & (12A78B)(34C569) & (12B)(38A)(497)(56C)\\
(1A7C593B)(48) & (1B)(3A)(5C)(79) & (18932B)(4A76C5)\\
(14B)(2A3)(58C)(697) & (1C5B)(24)(3A79)(68) & (197C5A3B)(2864)\\
(16A74B)(2938C5) & (16B)(2C5)(34A)(789) & (193C5A7B)(26)\\
(1C5B)(26)(397A)(48) & (14936B)(2C58A7) & (18B)(297)(36A)(4C5)\\
(1B)(28)(39)(46)(5C)(7A) & (1A3C597B)(2468) & (12C)(389)(4A7)(56B)\\
(12978C)(34B56A) & (1C)(39)(5B)(7A) & (197B5A3C)(48)\\
(14C)(293)(58B)(6A7) & (18A32C)(4976B5) & (1A7B593C)(2864)\\
(1B5C)(24)(397A)(68) & (16C)(2B5)(349)(78A) & (16974C)(2A38B5)\\
(1B5C)(26)(3A79)(48) & (1A3B597C)(26) & (18C)(2A7)(369)(4B5)\\
(14A36C)(2B5897) & (193B5A7C)(2468) & (1C)(28)(3A)(46)(5B)(79)
\end{array}$$}
\caption{Die Symmetriegruppe der Dyckschen Karte}
\label{dyckaut}
\end{table}

Zur Erzeugung verträglicher Matroide sei hieraus eine zyklische
Untergruppe der Ordnung sechs, etwa $\{<(1C8792)(3A65B4)>\}$
als Symmetriegruppe ausgewählt. Zu dieser wurden nach 10 Stunden Rechenzeit
auf einem i486-66 bereits 1332 verschiedene verträgliche Matroide mit oben
angegebener Grundstruktur erzeugt, wobei noch nicht alle zugehörigen
Matroide erzeugt wurden. Vier willkürlich ausgewählte von diesen sind in
Abbildung \ref{dyckmat} als Beispiele angegeben.

\begin{figure}%[htb]
{\tt $$\begin{array}{c}
0000011110000111100000000000000111111110111100000111100\\
0000000111100000111100000000000000011110111100000111100\\
0000000000000000000001111111101111011110000000111100000\\
0000111100011110000000000011110111100001111000000011110\\
0000000000000000111111110111100000000000000001111011110\\
0000000000000111100111100000000001111000111100000111100\\
0000000001111011110000000000001111000000011110111100000\\
1111001111000000000000000111100000000000011111111011110\\
1111000000011110000011110111100000000000011110000000000\\
\end{array}$$
$$\begin{array}{c}
0101111111011111101111100000001110111110111111011111111\\
1101011111110111111111110111101101111110101110111101111\\
1111011110110011010011101111101111110111111111111101111\\
1011111111111111111111001111000111101111111111111011111\\
1110010111111101111100110111111110111110111111111011111\\
0111100011111111111111100111011111111111111110101111110\\
1101101101111110111011100011111111011110011110111111111\\
1011101111111111111100001111101111100111011111111111111\\
1111101111011110111111111101111110001110111110111111001\\
\end{array}$$
$$\begin{array}{c}
0101111111011111111101111111111111111111111101011111111\\
1101011111111111111111110111111111111111111111101111111\\
1111111111111111110111111111111111111111111111111111111\\
1011111111111111111111111111111111111111111111111111111\\
1110010111111111111111111111111011111111111111111111111\\
1111110111111111111111111111111111111111111110101111110\\
1101111111111111111111010111111111111110011110111111111\\
1111101111111111111111111111111111110111011111111111111\\
1111101111111111111111111111111110001110011111111111101\\
\end{array}$$
$$\begin{array}{c}
0111011111110111111111111111101111111110111111110111111\\
1111111011111010111111110111111110111111111111111111111\\
1111110101111010111111111110111111111111011011111111111\\
1110111111111111111111011111111111111111111111111110111\\
1111001100011110111111111111010111111111111111111111111\\
1110111101111111111111111111111111111111111111111111111\\
0111111111111111111111111110110111111111111111111011111\\
1111111111111111111101111111111111111101011111111111111\\
1111101111111111111111111111110101011011111110111110111\\
\end{array}$$}
\caption{Verträgliche Matroide zur Dyckschen Karte}
\label{dyckmat}
\end{figure}

Versucht man nun der Reihe nach alle erzeugten Matroide zu orientieren,
so erhält man schon für die ersten beiden mit {\sc Sym2mat} erzeugten
Matroide verträgliche Vorzeichenlisten, die in den Tabellen \ref{dyckori1}
bis \ref{dyckori2B} zu sehen sind.
Hier wurden wiederum nur die "'positiven"' Listen aufgeführt, so daß sich
im ersten Fall zwei verträgliche Chirotope und im zweiten Fall acht
verschiedene Vorzeichenlisten ergeben.

\begin{table}%[htb]
{\tt
\begin{center}
\begin{tabular}{c}
0000011110000111100000000000000111111110111100000111100\\
0000000111100000111100000000000000011110111100000111100\\
0000000000000000000001111111101111011110000000111100000\\
0000111100011110000000000011110111100001111000000011110\\
0000000000000000111111110111100000000000000001111011110\\
0000000000000111100111100000000001111000111100000111100\\
0000000001111011110000000000001111000000011110111100000\\
1111001111000000000000000111100000000000011111111011110\\
1111000000011110000011110111100000000000011110000000000\\[2mm]
00000++++0000----00000000000000++++----0----00000----00\\
0000000----00000----000000000000000++++0++++00000----00\\
000000000000000000000++++----0----0----0000000----00000\\
0000++++000++++00000000000++++0++++0000----0000000----0\\
0000000000000000----++++0++++0000000000000000++++0++++0\\
0000000000000----00++++0000000000----000++++00000++++00\\
000000000----0----000000000000----0000000----0----00000\\
++++00++++000000000000000++++000000000000++++----0----0\\
----0000000----00000++++0++++000000000000----0000000000\\[2mm]
00000-++-0000+--+00000000000000+--+-++-0+--+00000-++-00\\
0000000-++-00000+--+000000000000000+--+0-++-00000-++-00\\
000000000000000000000-++-+--+0-++-0-++-0000000+--+00000\\
0000+--+000+--+00000000000-++-0+--+0000-++-0000000+--+0\\
0000000000000000+--+-++-0+--+0000000000000000+--+0-++-0\\
0000000000000+--+00-++-0000000000+--+000-++-00000+--+00\\
000000000-++-0+--+000000000000-++-0000000+--+0-++-00000\\
-++-00+--+000000000000000+--+000000000000+--+-++-0+--+0\\
+--+0000000-++-00000-++-0+--+000000000000+--+0000000000\\
\end{tabular}
\end{center}
}
\caption{\label{dyckori1}Das erste orientierbare Matroid zur Dyckschen Karte}
\end{table}

\begin{table}%[htb]
{\footnotesize\tt
\begin{center}\begin{tabular}{c}
00000++++0000----000++++00----0++++-------+--0000----00\\
0000000----00000----000000000++++00----0++++-------+--0\\
0----00000----0000000++++-------+------000000---+--0000\\
0000++++000----00++++0----+++++++-++000----0000000----0\\
00000000000++++0----+++++++-++0----0000000000+++++++-++\\
0000000000000++++00----0++++-------+--00----00000----00\\
00000++++-------+------000000---+--000000++++0----+++++\\
++-++0----0000000000+++++++-++00000000000++++-------+--\\
----000000---+--0000+++++++-++0000000000---+--000000000\\[2mm]
00000++++0000----000++++00----0++++----------0000----00\\
0000000----00000----000000000++++00++++0++++----------0\\
0----00000----0000000++++--------------000000------0000\\
0000++++000++++00++++0----++++++++++000----0000000----0\\
00000000000++++0----++++++++++0----0000000000++++++++++\\
0000000000000----00++++0++++----------00++++00000++++00\\
00000++++--------------000000------000000----0----+++++\\
+++++0++++0000000000++++++++++00000000000++++----------\\
----000000------0000++++++++++0000000000------000000000\\[2mm]
00000++++0000----000----00++++0++++----+----+0000----00\\
0000000----00000----000000000----00++++0+++++++++----+0\\
0++++00000++++0000000++++----+----+----000000+----+0000\\
0000++++000++++00----0++++++++-++++-000----0000000----0\\
00000000000----0----++++-++++-0++++0000000000++++-++++-\\
0000000000000----00++++0----+++++----+00++++00000++++00\\
00000--------+----+++++000000+----+000000----0---------\\
++++-0++++0000000000-----++++-00000000000++++----+----+\\
----000000+----+0000++++-++++-0000000000+----+000000000\\[2mm]
00000+--+0000-++-000-++-00+--+0+--+-++-+--+-+0000+--+00\\
0000000+--+00000-++-000000000+--+00-++-0-++-+--+-++-+-0\\
0-++-00000+--+0000000-++--++--++-+--++-000000+--+-+0000\\
0000-++-000-++-00+--+0+--+-++-+--+-+000+--+0000000-++-0\\
00000000000+--+0-++--++-+--+-+0+--+0000000000+--+-++-+-\\
0000000000000-++-00+--+0+--+-++-+--+-+00+--+00000-++-00\\
00000+--++--++--+-++--+000000-++-+-000000-++-0+--++--+-\\
++-+-0-++-0000000000-++-+--+-+00000000000-++-+--++--+-+\\
-++-000000-++-+-0000+--++--+-+0000000000+--+-+000000000
\end{tabular}
\end{center}
}
\caption{\label{dyckori2A}Orientierungen eines Matroids zur Dyckschen Karte
         Teil A}
\end{table}

\begin{table}%[htb]
{\footnotesize\tt
\begin{center}\begin{tabular}{c}
00000-++-0000+--+000+--+00-++-0+--+-++--+--+-0000-++-00\\
0000000-++-00000+--+000000000-++-00+--+0-++--++-+-++-+0\\
0+--+00000-++-0000000-++-+--++-++-+-++-000000-+--+-0000\\
0000+--+000+--+00-++-0-++--++--+--+-000-++-0000000+--+0\\
00000000000-++-0+--+-++--+--+-0-++-0000000000+--++-++-+\\
0000000000000+--+00-++-0-++-+--+-+--+-00-++-00000+--+00\\
00000-++--++--+--+--++-000000+-++-+000000+--+0-++--++-+\\
-++-+0+--+0000000000+--+-+--+-00000000000+--+-++--+--+-\\
+--+000000+-++-+0000-++--+--+-0000000000-+--+-000000000\\[2mm]
00000-++-0000+--+000-++-00+--+0+--+-++-++-+++0000-++-00\\
0000000-++-00000+--+000000000+--+00-++-0-++-+--+--+---0\\
0-++-00000+--+0000000-++-+--+--+----++-000000++-+++0000\\
0000+--+000-++-00+--+0+--+-++-++-+++000-++-0000000+--+0\\
00000000000+--+0+--+-++-++-+++0+--+0000000000+--+--+---\\
0000000000000-++-00+--+0+--+-++-++-+++00+--+00000-++-00\\
00000+--+-++-++-++++--+000000--+---000000-++-0-++-+--+-\\
-+---0-++-0000000000-++-++-+++00000000000+--+-++-++-+++\\
+--+000000--+---0000-++-++-+++0000000000++-+++000000000\\[2mm]
00000-++-0000+--+000-++-00+--+0+--+-++-++--++0000-++-00\\
0000000-++-00000+--+000000000+--+00+--+0-++-+--+--++--0\\
0-++-00000+--+0000000-++-+--+--++---++-000000++--++0000\\
0000+--+000+--+00+--+0+--+-++-++--++000-++-0000000+--+0\\
00000000000+--+0+--+-++-++--++0+--+0000000000+--+--++--\\
0000000000000+--+00-++-0+--+-++-++--++00-++-00000+--+00\\
00000+--+-++-++--+++--+000000--++--000000+--+0-++-+--+-\\
-++--0+--+0000000000-++-++--++00000000000+--+-++-++--++\\
+--+000000--++--0000-++-++--++0000000000++--++000000000\\[2mm]
00000----0000++++000++++00----0++++-----+-++-0000++++00\\
0000000++++00000++++000000000++++00----0++++-----+-++-0\\
0----00000----0000000++++++++-+-++-----000000-+-++-0000\\
0000----000----00++++0----+++++-+--+000++++0000000++++0\\
00000000000++++0+++++++++-+--+0----0000000000+++++-+--+\\
0000000000000++++00----0++++-----+-++-00----00000----00\\
00000++++++++-+-++-----000000-+-++-000000++++0+++++++++\\
-+--+0----0000000000+++++-+--+00000000000----++++-+-++-\\
++++000000-+-++-0000----+-+--+0000000000-+-++-000000000
\end{tabular}
\end{center}
}
\caption{\label{dyckori2B}Weitere Orientierungen eines Matroids zur Dyckschen
         Karte Teil B}
\end{table}

Es gilt abzuwarten, wieviele orientierte Matroide zur Dyckschen Karte
insgesamt existieren, die bezüglich der vorgelegten Symmetriegruppe
verträglich sind. Schon aufgrund des Ergebnisses für die ersten beiden
erzeugten Matroide kann man optimistisch sein, daß eine symmetrische
Einbettung zumindest nicht ausgeschlossen ist. (U. Brehm gibt eine symmetrische
Einbettung der Dyckschen Karte mit einer Symmetrie der Ordnung sechs an, die
sicher auch hier zu finden ist.)

\clearpage
\section{Schlußgedanken}

Bei der Bearbeitung der Dyckschen Karte, oder beim Herangehen an einen
Komplex der Größenordnung eines Dodekaeder, zeigt sich, daß sich unter
Berücksichtigung von Symmetrieeigenschaften die aufzuwendende Rechenzeit
zur Bearbeitung eines solchen Komplexes um Potenzen verringern läßt.
Gerade bei der Bereitstellung einer Grundstruktur für verträgliche Matroide
läßt sich hier mittels der Berücksichtigung von Symmetrien mehr erreichen,
als zur Zeit implementiert ist. Nutzt man weitere mögliche Vereinfachungen und
Reduktionen aus, so kann hier weiter optimiert werden, gerade wenn es um
größere Beispiele geht.
Kleine Beispiele lassen sich aber ohne weiteres mittels des vorgestellten
Vorgehens in vertretbarer Zeit interessanten Ergebnissen zuführen.

Hat man verträgliche orientierte Matroide zu einem Komplex erzeugt,
stellt sich nun die Frage, wie man die erzeugten orientierten Matroide
realisiert, um letztendlich Einbettungen zu den vorgelegten Komplexen zu
bekommen.\\
In Anbetracht dessen, daß die erforderliche Rechenzeit für Komplexe mit mehr
als zehn 0-Zellen weiter verringert werden kann und der technische
Fortschritt stetig schnellere Prozessoren für "`Rechenmaschinen"' liefert,
wird es in absehbarer Zeit auch möglich sein, etwa die duale
Mannigfaltigkeit zur Dyckschen Karte mit immerhin schon 32 0-Zellen und
Achtecken als 2-Zellenberandungen für die Welt der orientierten Matroide zu
erschließen und über diese die noch offene Frage der Einbettbarkeit zu
klären.

\section{Die verwendeten Programme}

Die in dieser Arbeit vorgestellten Algorithmen sind als ANSI-C Programme
implementiert und sollten mit einem ANSI-C Compiler (etwa GCC) auf jeder
Rechnerplattform lauffähig sein.

Trotz sorgfältiger Tests kann es passiert sein, das die Programme nicht
völlig fehlerfrei sind, gerade da bei der Speicherausnutzung zur höheren
Effizienz auf Pointer und dynamische Allokierung zurückgegriffen wurde.
Daher ist die Nutzung der erstellten Programme -- es handelt sich
schließlich um Mittel zum Zweck -- auf eigene Gefahr und ohne jegliche
Gewähr von Seiten des Autors. Sollten Fehler aufgetreten sein, so ist
der Autor für jeden Hinweis dankbar. Der Quelltext zu den Programmen
ist, da auf keinerlei Urheberrechte geachtet wurde, im Rahmen weiterer
Forschung frei zu verändern und weiterzugeben -- ein Urheberrechtsanspruch
besteht von Seiten des Autors nicht.

Folgende Programme fanden für diese Arbeit Verwendung:

\begin{verbatim}
vrzsym.c      berechnet zu einer Vorzeichenliste (Chirotop) deren
              Symmetrien
              Format der Eingabedatei: Name.chi
              #n Anzahl der Punkte
              #d Rang
              ............... (n ueber d) Vorzeichen +,-,0 *

matsym.c      berechnet zu einer Matroidliste deren Symmetrien
              Format der Eingabedatei: Name.mat wie Name.chi,
              Liste besteht aus Eintraegen 0 und 1

gonsym.c      berechnet zu einem Polygon-Komplex dessen
              Automorphismen
              Format der Eingabedatei: Name.gon
              #n Anzahl der Punkte pro 2-Zelle
              a_1a_2...a_n b_1b_2...b_n ...
              (Liste der Polygone mit Umlaufsinn)
              a_i kann 0..9A..Z sein

surebase.c    berechnet zu einer Datei Name.gon eine Grundstruktur
              fuer zugehoerige Matroide

sym2mat.c     Erzeugt Matroide zu einer Automorphismendatei
              Name.aut : #n Anzahl der Punkte
                         #m Anzahl der folgenden Symmetrien
                         1  2  3  4  5  ...  n  (Identitaet)
                         s1 s2 s3 s4 s5 ...  sn (eine Permutation)
                         usw.

orimat.c      Erzeugt orientierte Matroide zu einer Datei Name.mat
              mit einer Automorphismendatei Name.aut
\end{verbatim}


\nocite{Hof:91}

\bibliographystyle{plain}
\bibliography{\jobname}
\listoffigures
\printindex
\end{document}
