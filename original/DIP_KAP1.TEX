\chapter{Grundlagen}

\section{Grundbegriffe der Topologie}

Um Komplexe allgemein und CW-Komplexe im besonderen definieren zu k"onnen,
ist es n"otig, einige grundlegende Definitionen und Eigenschaften
topologischer R"aume bereitzustellen, die nicht nur hier, sondern in vielen
anderen Bereichen der Mathematik zum "`Grundvokabular"' geh"oren.

\subsection{Topologische R"aume}

Ist X eine beliebige Menge, so kann man auf dieser die Potenzmenge
${\cal P}(X)$, die Menge aller Teilmengen von X, betrachten. W"ahlt man aus
${\cal P}(X)$ eine Teilmenge $\T$ aus, deren Elemente die drei Bedingungen

\bcent
\btab{ll}
1. & $\emptyset\in\T$ und $X\in\T$\\
2. & $T_1,T_2\in\T~\fol~T_1\cap T_2\in\T$\\
3. & $T_i\in\T$ f"ur alle i aus einer Indexmenge I
     $\fol~\Cup\limits_{i\in I} T_i~\in~\T$\\
\etab
\ecent

erf"ullen, so nennt man $\T$ eine {\bf Topologie} auf X\idx{Topologie auf X}
und Elemente $T\in\T$ {\bf offene Mengen}\idx{offene Mengen} des {\bf
topologischen Raumes}\idx{topologischer Raum} (X,$\T$), des Paares X zusammen
mit der Topologie $\T$. Die Komplemente offener Mengen nennt man {\bf
abgeschlossen}\idx{abgeschlossen}.\\
{\scsi
Ebenso k"onnte man eine Topologie $\T$ auf X auch mittels abgeschlossener Mengen
definieren. In diesem Fall w"are dann das Enthaltensein endlicher Vereinigungen
und beliebiger Durchschnitte abgeschlossener Mengen aus $\T$ zu fordern,
was der Komplementbildung obiger Bedingungen entspricht.
}\\
Ist Y eine Teilmenge eines topologischen Raumes (X,$\T$), so wird diese
selbst zu einem topologischen Raum, wenn man auf ihr die Topologie von X
induziert, das hei"st als offene Mengen der Topologie $\T|Y$ die Mengen
$\{T\cap Y~|~T\in\T\}$ w"ahlt. $\T|Y$ nennt man dann die von X auf Y
{\bf induzierte Topologie}, die oft auch als {\bf Spurtopologie} auf
Y\idx{Spurtopologie}\idx{induzierte Topologie} bezeichnet wird.

Als Beispiele f"ur Topologien k"onnen etwa die durch Metriken
d metrischer R"aume (X,d) induzierten Topologien angef"uhrt werden.\\
{\scsi
Zur Erinnerung hier die Definition einer Metrik und eines metrischen Raumes.
Sei dazu X eine Menge. Eine Funktion $d:X\times X\to\R$ hei"st Metrik, wenn\\
1. d(x,y) = d(y,x) $\geq$ 0\\
2. d(x,y) = 0 $\iff$ x = y\\
3. d(x,z) $\leq$ d(x,y) + d(y,z),\\
gilt. Das Paar (X,d) hei"st dann ein metrischer Raum. Betrachtet man etwa
$(\R,\|\cdot\|_2)$, so stellen metrische R"aume oft verwendete Spezialf"alle
topologischer R"aume dar.
}\\
Die offenen Mengen dieser durch Metriken induzierten Topologien sind gegeben
durch
$$\T_d=\{U\subseteq X:(\forall~p\in U) (\exists~\varepsilon > 0)
\mbox{ mit } \{y:d(y,p)<\varepsilon\}\subseteq U\}.$$
Sei $\|\cdot\|:V\to\R$ eine Abbildung eines Vektorraumes V "uber einem K"orper
$\K$ in die reellen Zahlen. $\|\cdot\|$ hei"st eine {\bf Norm}\idx{Norm} auf V,
wenn f"ur alle $v,w\in V$ und $k\in\K$ $\|v\|$ dann und nur dann gleich 0 ist,
wenn $v=0$ gilt, $\|k\cdot v\|$ gleich $|k|\cdot\|v\|$ ist, sowie $\|v+w\|\leq
\|v\|+\|w\|$ gilt. (V,$\|\cdot\|$) hei"st dann ein {\bf normierter Raum}.
\idx{normierter Raum} Normierte R"aume sind metrische R"aume, wenn man f"ur
die d(x,y) die Norm der Differenz $\|x-y\|$ von x und y setzt.

Der $\R^n$ ist bez"uglich der durch die Euklidische Norm $\|x\|_2 :=
\sqrt{\sum_{i=1}^n x_i^2}$ und der so induzierten Metrik gegebenen
offenen Mengen ein topologischer Raum, der auch als {\bf Euklidischer Raum}
$\E^n$ bezeichnet wird. Oft spricht man hier auch von der "`"ublichen"'
Topologie auf $\R^n$. (Im folgenden sei $\R$ und $\E$ synonym verwendet, da immer
der Euklidische Raum gemeint sein wird.)

Spielt es keine Rolle, wie die Topologie eines topologischen Raumes
geartet ist, werden Aussagen "uber beliebige topologische R"aume gemacht
oder ist es klar, welche Topologie (die "`"ubliche"') gemeint ist, so spricht
man oft der Einfachheit halber von einem topologischen Raum X, ohne
zus"atzliche Angabe von $\T$.

Eine Abbildung $f:X\to Y$ zwischen topologischen R"aumen X und Y hei"st
{\bf stetig},\idx{stetig} wenn die Urbilder offener Mengen in Y offen in X sind.
Dies bedeutet, da"s es zu jeder offenen Menge V in Y eine offene Menge U in X
so gibt, da"s f(U) = V gilt.
Ist $f:X\to Y$ eine bijektive Abbildung zwischen X und Y, die selbst
und deren Umkehrabbildung $f^{-1}:Y\to X$ stetig sind, so hei"st f ein
{\bf Hom"oomorphismus}\idx{Hom\"oomorphismus}. Gibt es zwischen zwei
topologischen R"aume X und Y solch einen Hom"oomorphismus, so hei"sen X und Y
{\bf hom"oomorph}\idx{hom\"oomorph}.
Als wichtiges Beispiel sei angef"uhrt, da"s die n-dimensionale offene Vollkugel
(in einem normierten Raum) $\stackrel{\circ}{B^n} := \{x~:~\|x\|< 1\}$
hom"oomorph dem $\R^n$ ist (als Abbildung w"ahle man etwa
$\phi:\stackrel{\circ}{B^n}\to\R^n$ mit $\phi(x)=\frac{1}{1-\|x\|}x$). Weiter
ist die (n$-$1)-Sph"are $\S^{n-1}:=\{x~:~\|x\|=1\}$, der Rand der Vollkugel
$B^n$, von der man einen Punkt wegl"a"st, sie "`punktiert"', hom"oomorph dem
$\R^{n-1}$. (Man nutze die stereographische Projektion
$\phi:\S^{n-1}-\{e_n\}\to\R^{n-1}$ mit
$y=(\frac{x_1}{1-x_n},\ldots,\frac{x_{n-1}}{1-x_n})$).

Bestimmte Teilmengen topologischer R"aume k"onnen nun durch folgende Begriffe
charakterisiert werden. Ist M eine beliebige Teilmenge eines
topologischen Raumes (X,$\T$), so hei"st eine weitere Teilmenge $U\subset X$
eine {\bf Umgebung}\idx{Umgebung} von M, falls es eine offene Menge T in $\T$
gibt, so da"s $M\subset T\subset U$ gilt.
Ein Punkt p aus M ist {\bf innerer Punkt}\idx{innerer Punkt} von M, falls M
Umgebung von  $\{$p$\}$ ist, p ist {\bf Ber"uhrungspunkt}\idx{Ber\"uhrungspunkt}
von M, falls jede Umgebung von p nichtleeren Schnitt mit M hat und p ist ein
{\bf H"aufungspunkt}\idx{H\"aufungspunkt}, falls p Ber"uhrungspunkt von
M$-\{$p$\}$ (M ohne den Punkt p) ist. Ein Punkt p hei"st {\bf Randpunkt}
\idx{Randpunkt} von M, wenn er Ber"uhrungspunkt von M und X$-$M ist.
Aus der Definition einer Topologie mittels abgeschlossener Mengen
folgt, da"s jede offene Menge $T\in\T$ in einer kleinsten abgeschlossenen
Menge, dem {\bf Abschlu"s} $\ol{T}$\idx{Abschlu\3} von T, enthalten ist. Dieser
Abschlu"s oder auch diese {\bf abgeschlossene H"ulle} von T ist der Durchschnitt
aller abgeschlossenen Mengen, die T enthalten.

Eine Teilmenge eines topologischen Raumes soll {\bf kompakt} hei"sen, wenn sie
das "Uberdeckungskriterium von Heine-Borel erf"ullt. Dieses besagt, da"s
eine Menge K in einem topologischen Raum X genau dann kompakt\idx{kompakt}
ist, wenn jede offene "Uberdeckung (jede Vereinigung offener Mengen aus X, die
eine Obermenge von K ist) die Auswahl einer endlichen Teil"uberdeckung
(also eine "Uberdeckung mit nur endlich vielen offenen Mengen) zul"a"st.
{\bf Konvex} hei"st eine Teilmenge K eines topologischen Vektorraumes (eines
Vektorraumes mit Topologie, bez"uglich der Addition und Multiplikation mit
Skalaren stetig sind)\idx{konvex}, wenn f"ur je zwei Punkte a und b aus K
die Verbindungsstrecke $\{x=\lambda a +(1-\lambda) b:\lambda\in [0,1]\}$
der beiden auch in K liegt. Eine nichtleere Teilmenge S einer Teilmenge K
eines Vektorraumes hei"st {\bf Extremmenge} von K\idx{Extremmenge}
falls f"ur alle $x,y\in K,~0<t<1$, und $(1-t)x+ty\in S$ die Punkte x und y aus
S stammen. Die {\bf Extrempunkte}\idx{Extrempunkt} von K sind die einpunktigen
Extremmengen von K (vgl.\cite{Ru:91}, Seite 74).
Eine kompakte, konvexe Teilmenge K des $\R^d$ hei"st nun {\bf Polytop} oder
Polyedermenge, falls die Menge ihrer Extrempunkte endlich ist.
(vgl.\cite{Gr:67}, S.31)\idx{Polytop}

\subsection{Produkttopologie}

Betrachtet man eine Mengenfamilie $\{X_j:j\in J\}$, eine Menge, die zu jedem
Element j einer Indexmenge J eine Menge X$_j$ enth"alt, so ist das {\bf
kartesische Produkt}\idx{kartesisches Produkt} der X$_j$ die Menge aller
Auswahlfunktionen $f:J\to\Cup\limits_{j\in J} X_j$ mit $f(j)\in X_j$,
geschrieben als
$$
\prod\limits_{j\in J}X_j:=
\left\{f~|~f:J\to\Cup\limits_{j\in J} X_j,f(j)\in X_j\right\}.
$$
F"ur $\{f(j):j\in J\}$ schreibt man kurz $(x_j)_{j\in J}$, was dem Sachverhalt
der Auswahl eines Elements aus allen $X_j$ wohl gerechter wird.
Da"s ein unendliches kartesisches Produkt nichtleer ist, ist nicht ohne weiteres
einsichtig und wird durch das Auswahlaxiom sichergestellt, welches besagt:
\begin{quote}
{\bf
F"ur jede Mengenfamilie $\{X_j:j\in J\}$ nichtleerer Mengen ist das Produkt
$\prod\limits_{j\in J}X_j$ nichtleer.
}
\end{quote}
Bei Betrachtung endlicher Produkte kommt dieses Axiom nicht zum tragen, ist
aber bei der "Ubertragung auf den allgemeinen Fall unendlicher Mengenfamilien
unerl"a"slich, weswegen es hier bei der Darstellung allgemeiner topologischer
Grundlagen auch nicht fehlen soll.

W"ahlt man die $X_j$ als topologische R"aume, so ist mit den Topologien
$\T_j$ der $X_j$ auch auf dem Produkt eine Topologie, die
{\bf Produkttopologie}, \idx{Produkttopologie} gegeben. Setzt man
$$\B:=\left\{\prod\limits_{j\in J}U_j~:~U_j\in\T_j, U_j=X_j,\mbox{ bis auf
endlich viele Ausnahmen } j\in J\right\},$$
so ist die Produkttopologie gerade $\T := \left\{\Cup B~:~B\in\B\right\}.$
$\B$ hei"st eine {\bf Basis}\idx{Basis einer Topologie} der Topologie, da jede
offene Menge als Vereinigung von Mengen aus $\B$ dargestellt werden kann. Die
Formulierung "`bis auf endlich viele Ausnahmen"' ist dabei eine Folge des
Auswahlaxioms.

\subsection{Teilweise geordnete Mengen und Verb"ande}

Da gerade vom Auswahlaxiom die Rede war, soll an dieser Stelle die Terminologie
geordneter Mengen dargestellt werden, da auf Komplexen eine nat"urliche Ordnung
existiert, diese Mengen aber auch sonst eine hilfreiche Erg"anzung bieten.

Eine Relation "`$\leq$"' auf einer Menge X hei"st {\bf Partialordnung} oder
einfach nur Ordnung, wenn sie transitiv ($x\leq y, y\leq z\fol x\leq z~\forall
x,y,z\in X$), reflexiv ($x\leq x~\forall x\in X$) und antisymmetrisch ($x\leq y,
y\leq x\fol x = y~\forall x,y\in X$) ist. Die Menge X hei"st zusammen mit
$\leq$ eine {\bf teilweise geordnete Menge} oder {\bf Poset} (partial ordered
set) (X,$\leq$).\idx{Ordnung}\idx{Poset}\idx{teilweise geordnete Menge}
(X,$\leq$) hei"st {\bf totalgeordnet},\idx{totalgeordnet} wenn f"ur alle
x,y$\in$X allemal x$\leq$y oder y$\leq$x gilt. Jede totalgeordnete Teilmenge
einer teilweise geordneten Menge hei"st {\bf Kette} oder Turm.\idx{Kette}
\idx{Turm} Ist jede Kette in (X,$\leq$) nach oben beschr"ankt, so hei"st
(X,$\leq$) auch induktiv geordnet.
Hier kann nun ein zum Auswahlaxiom "aquivalentes Lemma, das Lemma von Zorn
angef"uhrt werden, das den gegenw"artigen Bezug zu den topologischen R"aumen
darstellt.\idx{Lemma von Zorn}
\begin{quote}
{\bf Jede induktiv geordnete Menge besitzt maximale Elemente.}
\end{quote}
Besitzt eine teilweise geordnete Menge (X,$\leq$) ein eindeutiges minimales
Element, so sei dieses mit \^0 bezeichnet, analog ein eindeutiges maximales
Element mit \^1. Gibt es in einer teilweise geordneten Menge (X,$\leq$) diese
Elemente, so hei"st (X,$\leq$) {\bf beschr"ankt}. Gilt f"ur zwei Elemente x
und y aus einer beschr"ankten teilweise geordneten Menge X, da"s x$<$y ist, und
existiert kein weiteres Element z in X, welches zwischen x und y liegt
(x$<$z$<$y), so bezeichnet man y als {\bf Decke}\idx{Decke} von x
beziehungsweise x als {\bf Kodecke} von y. Die Decken von \^0 hei"sen dann
{\bf Atome}\idx{Atome} und die Kodecken von \^1 {\bf Koatome} von X.
Besitzen alle maximalen Ketten $x_0<x_1<\ldots<x_l$ die gleiche L"ange $\ell$,
so hei"st (X,$\leq$) rein und $\ell$ ist die L"ange
von X. In diesem Fall ist der Rang $\rho(x)$ eines x$\in$X die L"ange der
geordneten Teilmenge $X_{\leq x}:=\{y\in X:y\leq x\}$. Wie diese Teilmengen
teilweise geordneter Mengen, so k"onnen auch {\bf Ordnungsintervalle}
\idx{Ordnungsintervalle} $[x,y]\subseteq X$ mit $x,y\in X$ als Mengen
$\{z\in X:x\leq z\leq y\}$ definiert werden, analog offene Ordnungsintervalle
$(x,y) := \{z\in X:x<z<y\}$ im "ublichen Sinne.

Eine teilweise geordnete Menge (X,$\leq$) hei"st {\bf Verband},\idx{Verband}
wenn zu allen Paaren x,y$\in$ X kleinste obere Schranken $x\vee y$ (Joins) und
gr"o"ste untere Schranken $x\wedge y$ (Meets) existieren. In einem endlichen
Verband L existieren Joins und Meets f"ur beliebige Teilmengen von L, deshalb
ist L beschr"ankt und es ist \^0 = $\wedge$L und \^1 = $\vee$L.
K"urzer kann man sagen, da"s eine teilweise geordnete Menge L genau dann ein
Verband ist, wenn \^0 existiert und wenn f"ur alle Paare x,y$\in$L die Joins
$x\vee y$ existieren. 

\subsection{Quotientenr"aume}

Nach diesem Abstecher nun zur"uck zu den topologischen R"aumen. Hier ist
ein weiterer wichtiger Begriff, gerade im Zusammenhang mit Komplexen und
darauf induzierten Topologien, der des Quotientenraumes, welcher im folgenden
eingef"uhrt werden soll.

Sind X und Y Mengen, so kann mittels eines Tricks die Summe oder {\bf disjunkte
Vereinigung}\idx{Summe von Mengen}\idx{disjunkte Vereinigung} der beiden
Mengen definiert werden, wobei X und Y als Teilmengen erhalten bleiben
(Bei der "`normalen"' Vereinigung ist ja $X\cup X = X$).
Diese Summe $X+Y$ wird so als Vereinigung $X\times \{0\} \cup Y\times \{1\}$
erkl"art. Analog ist die {\bf Summe topologischer R"aume} zu bilden,
wenn die offenen Mengen der Topologie der Summe zu
$\{U+V|U\in\cS,V\in\T\}$ gew"ahlt werden, wobei $\cS$ die Topologie auf X und
$\T$ die Topologie auf Y bezeichnet.

Sei $\sim$ eine "Aquivalenzrelation (reflexiv (x $\sim$ x), symmetrisch
(x $\sim$ y $\iff$ y $\sim$ x) und transitiv (x $\sim$ y, y $\sim$ z
$\fol$ x $\sim$ z)) auf einem topologischen Raum $(X,\T)$,
$\tilde{x}:=\{y\in X:x\sim y\}$ eine und $\tilde{X}:=X/_\sim =
\{\tilde{x}|x\in X\}$ die Menge aller $\sim-$"Aquivalenzklassen. Die Abbildung
$\pi:X\to\tilde{X}$, die jedes Element x aus X auf seine "Aquivalenzklasse
abbildet hei"st {\bf Quotientenabbildung}\idx{Quotientenabbildung} des
topologischen Raumes (X,$\T$) in den {\bf Quotientenraum}
$(\tilde{X},\tilde{\T})$ \idx{Quotientenraum} von X modulo $\sim$ mit der
{\bf Quotiententopologie} \idx{Quotiententopologie} $\tilde{\T} :=
\{\tilde{\T}\subset\tilde{X}:\pi^{-1}(\tilde{\T})\in\T\}$.
Die Quotientenabbildung $\pi$ ist stetig und $\tilde{\T}$ ist die feinste
Topologie bez"uglich der $\pi$ stetig ist.\\
{\scsi
Seien $\cS$ und $\T$ zwei Topologien auf X. $\cS$ hei"st feiner als $\T$,
wenn sie mehr offene Mengen enth"alt, also $\cS\supseteq\T$ gilt. Analog
hei"st $\T$ gr"ober als $\cS$.
}\\
Ein Beispiel hierf"ur ist im Abschnitt zu den Abbildungen von CW-Komplexen
angegeben. (siehe Abb.\ref{inzidenz})

Da allein mit Angabe einer Topologie zu einer Menge X noch nicht viel
"uber Eigenschaften der Punkte in X ausgesagt werden kann, fordert man
in Form von Axiomen zus"atzliche "`Trennungseigenschaften"'.

In dieser Arbeit wird folgendes Axiom, das zweite Trennungsaxiom oder
auch {\bf Hausdorffbedingung}, v"ollig gen"ugen, weshalb auf die Auflistung
der weiteren Axiome verzichtet werden soll (vgl. dazu etwa \cite{Os:92},
Seite 50).\idx{Hausdorff-Axiom}
\begin{quote}
(T$_2$) Zu je zwei Punkten x und y in X gibt es disjunkte Umgebungen.
\end{quote}
Ein topologischer Raum, der dieses Axiom erf"ullt wird auch als
{\bf Hausdorff-Raum}\idx{Hausdorff-Raum} bezeichnet.

Nun aber zu den eigentlichen Objekten, um die es in dieser Arbeit gehen soll.

\section{CW-Komplexe}

Nach der doch recht abstrakten Einf"uhrung topologischer R"aume, von
Produkttopologien und Quotientenr"aumen wollen wir uns nun einer konkreten
Klasse solcher R"aume zuwenden, die veranschaulichen, was man mit oben
definierten Begriffen "uberhaupt anfangen kann. Um allerdings die CW-Komplexe
definieren zu k"onnen, werden noch einige zus"atzliche Eigenschaften ben"otigt.

So versteht man unter einer {\bf Zerlegung}\idx{Zerlegung} eines topologischen
Raumes X eine Menge paarweise disjunkter offener Teilmengen von X (versehen mit
der Spurtopologie selbst topologische R"aume), deren Vereinigung ganz X ergibt.
Zu jedem x$\in$X kann man also einen eindeutigen Teilraum in der Zerlegung
angeben. Eine {\bf n-Zelle}\idx{Zelle}\idx{n-Zelle} ist ein topologischer Raum,
der hom"oomorph dem $\R^n$ ist. Damit kann eine {\bf Zellenzerlegung}
\idx{Zellenzerlegung} als eine Zerlegung eines topologischen Raumes in
Teilr"aume definiert werden, die Zellen sind. "Uber die Hom"oomorphie der
d-Zellen zu $\R^d$ kann die {\bf Dimension eines zellenzerlegten Raumes} als die
maximale Dimension n der auftretenden Zellen definiert werden.

Die Vereinigung aller Zellen der Dimension $\leq$ d eines zellenzerlegten
Raumes hei"st dabei {\bf d-Ger"ust} oder {\bf d-Skelett} des betreffenden Raumes.
\idx{Ger\"ust}\idx{Skelett}\\
Weiterhin sollen zwei Zellen {\bf benachbart} hei"sen,\idx{benachbart} wenn der
Durchschnitt ihrer Abschl"usse nichtleer ist.

Im Jahre 1949 f"uhrte J.H.C. Whitehead den Begriff des CW-Komplexes (C f"ur
closure finite und W f"ur weak topology) (vgl. \cite{Ja:90}, Seite 109ff.)
ein, der eine sehr flexible Struktur innerhalb der Topologie darstellt.
Ein {\bf CW-Komplex}\idx{CW-Komplex} ($X,\cal E$) ist ein Hausdorff-Raum X
zusammen mit einer Zellenzerlegung $\cal E$ von X, der folgende drei Axiome
erf"ullt:
\bn
\item (Charakteristische Abbildungen): Zu jeder n-Zelle $e\in\cal E$ existiert
      eine stetige Abbildung $\Phi_e:B^n\to X$ der n-Kugel in X,
      die einen Hom"oomorphismus zwischen der offenen n-Kugel
      (dem $\R^n$) und e induziert und die die (n-1)-Sph"are $\S^{n-1}$
      in das (n-1)-Skelett von $\cal E$ abbildet.
\item (H"ullenendlichkeit): Jeder Punkt der abgeschlossenen H"ulle $\ol{e}$
      einer jeden Zelle $e\in\cal E$ besitzt einen Umgebung, die nur endlich
      viele andere Zellen trifft.
\item (Schwache Topologie): Teilmengen $A\subset X$ sind genau dann
      abgeschlossen, wenn f"ur alle $e\in\cal E$ auch die Mengen $A\cap\ol{e}$
      abgeschlossen sind.
\en

Da es in dieser Arbeit nur um endliche CW-Komplexe gehen soll, ist das
zweite Axiom trivialerweise immer erf"ullt.

\begin{figure}[htb]
$$
\beginpicture
\unitlength1cm
\setlinear
\setcoordinatesystem units <1cm,1cm>
\setplotarea x from -1 to 6, y from -0.5 to 3.5
\setsolid \thicklines
\setquadratic
\plot 1.0 1.5 1.5 2.5 2.5 2.5 3.5 2.5 3.5 1.5 2.0 1.0 1.0 1.5 /
\plot 3.5 1.5 4.5 1.5 5.5 1.0 /
\plot 1.0 1.5 1.0 0.5 2.0 1.0 /
\plot 2.5 2.5 3.0 3.0 2.8 3.5 /
\setlinear \thinlines
\put {\circle*{0.15}} [Bl] at 1.0 1.5
\put {\circle*{0.15}} [Bl] at 2.0 1.0
\put {\circle*{0.15}} [Bl] at 3.5 1.5
\put {\circle*{0.15}} [Bl] at 5.5 1.0
\put {\circle*{0.15}} [Bl] at 2.5 2.5
\put {\circle*{0.15}} [Bl] at 2.8 3.5
\put {\scsi sechs 0-Zellen} [bl] at 4.0 2.5
\put {\scsi sieben 1-Zellen} [tl] at 2.0 0.0
\put {\scsi zwei 2-Zellen} [bl] at -1.0 2.5
\endpicture
$$
\caption{Beispiel eines CW-Komplexes im $\E^2$}
\label{CW-Komplex}
\end{figure}

Ein {\bf Unterkomplex} $(X',{\cal E}')$ eines CW-Komplexes (X,$\cal E$)
\idx{CW-Komplex!Unterkomplex eines $\sim$} ist nun die Vereinigung X$'$ aller
Zellen aus einer Teilmenge $\cal E'$ von Zellen aus $\cal E$ zusammen mit
$\cal E'$, wenn eine der folgenden drei "aquivalenten Bedingungen erf"ullt ist:

\btab{ll}
(1) & $(X',{\cal E}')$ ist ein CW-Komplex.\\
(2) & X$'$ ist abgeschlossen in X.\\
(3) & Der Abschlu"s $\ol{e}$ jeder Zelle e aus $\cal E'$ liegt in X$'$.
\etab

Zum Beweis von $(1)\follows (2)$ ist zu zeigen, da"s f"ur alle
$e\in{\cal E}$ die Mengen $\ol{e}\cap X'$ abgeschlossen in X sind.
Da $(X,{\cal E})$ h"ullenendlich ist, bedeutet dies, die Abgeschlossenheit
von $\ol{e}\cap (\Cup_{e'\in {\cal E}'}e')=
\ol{e}\cap (e'_1\cup\ldots\cup e'_l)$ f"ur $e'_i\in{\cal E}',~1\leq i\leq l$
und alle $e\in\cal E$ zu kl"aren. Hilfreich dazu ist folgende Bemerkung
(vgl.\cite{Ja:90},S. 110).\\
Ist $\cal E$ eine Zerlegung eines Hausdorff-Raumes X, die das erste Axiom f"ur
CW-Komplexe erf"ullt, so ist $\ol{e}=\Phi_e(B^n)$ f"ur jedes $e\in\cal E$
kompakt und der Zellenrand $\ol{e}\backslash e=\Phi_e(\S^{n-1})$ liegt im
(n-1)-Ger"ust von X.\\
Zum Beweis dieser Bemerkung nutzt man, da"s f"ur stetige Abbildungen f und
Mengen M die Inklusion $f(\ol{M})\subset \ol{f(M)}$ gilt. Damit erh"alt man hier
$e\subset\Phi_e(B^n)\subseteq\ol{\Phi_e(\stackrel{\circ}{B^n}})=\ol{e}$.
$\Phi_e(B^n)$ ist als stetiges Bild eines Kompaktums in X abgeschlossen und nach
obiger Inklusion gleich $\ol{e}$. Nun aber zur"uck zum "Aquivalenzbeweis. Da die
$\Phi_e$ von $(X',{\cal E}')$ auch charakteristisch bez"uglich $(X,{\cal E})$
sind, liefert die Bemerkung simultan den Nachweis von $(1)\follows (3)$, da
$\ol{e}$ in $X$ H"ulle von e in $X$, damit aber auch in $X'$ ist. Nutzt man
dies, so ist auch $\ol{e}\cap (e'_1\cup\ldots\cup e'_l)=\ol{e}\cap
(\ol{e}'_1\cup\ldots\cup \ol{e}'_l)$ abgeschlossen.
$(3)\follows (2)$ ist wegen der Betrachtung abgeschlossener Zellen in $X'$
unmittelbar einsichtig. Zum Nachweis von $(2),(3)\follows (1)$ sind die drei
CW-Komplex-Axiome zu "uberpr"ufen. Die $\Phi_e$ von $(X,{\cal E})$ sind wegen
${\cal E'}\subset {\cal E}$ auch f"ur $(X',{\cal E'})$ verwendbar. Ebenso
l"a"st sich f"ur die H"ullenendlichkeit argumentieren.
Da alle in $X'$ abgeschlossenen Mengen dies auch in $X$ sind, m"ussen f"ur
den Nachweis des dritten Axioms nur $e\in{\cal E}\backslash \cal E'$ betrachtet
werden. F"ur ein $A\subset X'$ und ein $e\in{\cal E}\backslash{\cal E'}$ ist
dazu $A\cap\ol{e}$ zu betrachten. Zellen $e\in{\cal E}\backslash{\cal E'}$
tragen nichts zum Schnitt mit A bei. Wegen der H"ullenendlichkeit von X gibt
es aber eine endliche Anzahl von $e'_i\in{\cal E'}$, mit denen
$A\cap\ol{e}=A\cap (\Cup e'_i)\cap\ol{e}$ gilt. Nach oben ist
$A\cap\ol{e}=A\cap (\Cup \ol{e'_i})\cap\ol{e}$.
$A\cap\ol{e}=A\cap (\Cup \ol{e'_i})$ ist nach Voraussetzung abgeschlossen,
somit auch $A\cap\ol{e}$.\hfill $\Box$

Betrachtet man zweidimensionale CW-Komplexe, so kann man, wie gewohnt, die
auftretenden 0-Zellen als "`{\bf Ecken}"', die 1-Zellen als "`{\bf Kanten}"'
und die 2-Zellen als "`{\bf Fl"achen}"' bezeichnen -- als Beispiel sei etwa
der Randkomplex eines W"urfels angef"uhrt, der aus sechs "`Fl"achen"', zw"olf
"`Kanten"' und acht "`Ecken"' besteht.

\subsection{Simpliziale Komplexe}

Als eine spezielle Art von CW-Komplexen k"onnen die simplizialen Komplexe
aufgefa"st werden, die aufgrund ihrer einfacher zu beschreibenden Struktur
in der Vergangenheit oftmals eher eingesetzt wurden, als die
allgemeinen (nichtsimplizialen) CW-Komplexe. Diese seien hier, obwohl sie nicht
direkter Gegenstand dieser Arbeit sein sollen, der Vollst"andigkeit halber,
sowie als zus"atzliche Beispielklasse aufgef"uhrt.

Um Simplizes definieren zu k"onnen, mu"s gesagt werden, wann ein Punkt
x aus einer Teilmenge X des $\E^d$ {\bf linear abh"angig} hei"sen soll, n"amlich
dann, wenn Punkte $x_1,\ldots,x_r\in X$ und Skalare $\lambda_1,\ldots,\lambda_r$
so existieren, da"s $x=\lambda_1x_1+\ldots+\lambda_rx_r$ gilt. Ist
$\lambda_1+\ldots+\lambda_r=1$, so hei"st x {\bf affin abh"angig}, sind zudem
alle $\lambda_i\geq 0$, so {\bf konvex abh"angig}. Eine Teilmenge
$X\subseteq\E^d$ hei"st {\bf affin abh"angig}, falls f"ur $x_1,\ldots,x_r\in X$
und Skalare $\lambda_1,\ldots,\lambda_r$ mit mindestens einem $\lambda_i\neq 0$
eine Relation von der Form $\lambda_1x_1+\ldots+\lambda_rx_r=0$ und
$\lambda_1+\ldots+\lambda_r=0$ existiert. Ansonsten hei"st X affin unabh"angig.
(Alle obigen Ausf"uhrungen gelten auch f"ur allgemeine Vektorr"aume, wie
so oft wird aber der Anschaulichkeit wegen der $\R^d$ verwendet.)

Ein {\bf m-dimensionales Simplex}\idx{m-Simplex}\idx{Simplex} (kurz m-Sim\-plex)
im $\R^n$ ist die {\bf konvexe H"ulle} von m+1 affin unabh"angigen Punkten, m+1
Punkten in allgemeiner Lage, also f"ur $p_0,p_1,\ldots,p_m\in\R^n$, $p_i$
affin unabh"angig mit $1\leq i\leq m$
$$
   \Delta(p_0,p_1,\ldots,p_m) =
   \left\{ p\in\R^n~:~p = \sum\limits_{i=0}^m \lambda_i p_i, \lambda_i\geq 0,
   \sum\limits_{i=0}^m \lambda_i = 1\right\}.
$$
Die konvexe H"ulle von m+1 Einheitsvektoren im $\R^n$ (vgl. Abb.\ref{simplex})
hei"st {\bf Standard-m-Simplex},\idx{Standard-m-Simplex} die $p_i$ hei"sen
{\bf Eckpunkte} (vertices) des Simplex.

\begin{figure}[htb]
$$
\beginpicture
\unitlength1cm
\setlinear
\setcoordinatesystem units <0.6cm,0.6cm>
\setplotarea x from -2 to 2, y from -2 to 2
\put{ \beginpicture
\setsolid
\put {\circle*{0.1}} [Bl] at 0 0
\put {$p_0$} [tr] at 0 0
\endpicture } at -6 0
\put{ \beginpicture
\setsolid
\plot -0.75 -0.75 0.75 0.75 /
\put {\circle*{0.1}} [Bl] at -0.75 -0.75
\put {$p_0$} [tr] at -0.75 -0.75
\put {\circle*{0.1}} [Bl] at 0.75 0.75
\put {$p_1$} [bl] at 0.75 0.75
\endpicture } at -2 0
\put{ \beginpicture
\setsolid
\plot -1 -1 1 -1 0 0.732 -1 -1 /
\put {\circle*{0.1}} [Bl] at -1 -1
\put {$p_0$} [tr] at -1 -1
\put {\circle*{0.1}} [Bl] at 1 -1
\put {$p_1$} [tl] at 1 -1
\put {\circle*{0.1}} [Bl] at 0 0.732
\put {$p_2$} [bl] at 0 0.732
\endpicture } at 2 0
\put{ \beginpicture
\setsolid
\plot -0.884 -0.257 1.085 -0.376 /
\plot 0 1.023 1.085 -0.376 /
\plot -0.884 -0.257 0 1.023 /
\plot -0.2 -0.9 -0.884 -0.257 /
\plot 1.085 -0.376 -0.2 -0.9 /
\setdashes <1mm>
\plot -0.2 -0.9 0 1.023 /
\put {\circle*{0.1}} [Bl] at -0.884 -0.257
\put {$p_0$} [tr] at -0.884 -0.257
\put {\circle*{0.1}} [Bl] at 1.085 -0.376
\put {$p_1$} [bl] at 1.085 -0.376
\put {\circle*{0.1}} [Bl] at 0 1.023
\put {$p_2$} [bl] at 0 1.023
\put {\circle*{0.1}} [Bl] at -0.2 -0.9
\put {$p_3$} [tl] at -0.2 -0.9
\endpicture } at 6 0
\endpicture
$$
\caption{Einige Simplizes}
\label{simplex}
\end{figure}

Die konvexe H"ulle einer Auswahl von r paarweise verschiedenen Eckpunkten
eines m-Simplex S hei"st {\bf r-dimensionales Teilsimplex}\idx{Teilsimplex} oder
{\bf r-Seite}\idx{Simplex!Seite eines $\sim$} von S. Der {\bf Rand} $\partial S$
eines m-Simplex S ist die Vereinigung aller (m-1)-dimensionalen Teilsimplizes
von S. Das {\bf offene Simplex}\idx{offenes Simplex} $\stackrel{\circ}{S}$ ist
dann S ohne dessen Rand, also $S\backslash\partial S$.

Es sei angemerkt, da"s jedes m-Simplex S hom"oomorph zur m-Vollkugel $B^m$
und der Rand von S hom"oomorph zur (m-1)-Sph"are ist. Da die offene Vollkugel
$\stackrel{\circ}{B^n}$ hom"oomorph zu $\R^n$ ist, ist das offene n-Simplex 
eine n-Zelle nach obiger Definition.

Ein {\bf simplizialer Komplex}\idx{simplizialer Komplex} $\cal C$ im $\R^n$ ist
nun eine Menge von Simplizes, die folgende Bedingungen erf"ullt:

\btab{ll}
1. & $\cal C$ enth"alt mit jedem Simplex auch dessen s"amtliche Teilsimplizes.\\
2. & Der Durchschnitt zweier Simplizes aus $\cal C$ ist leer oder\\
   & gemeinsames Teilsimplex der beiden.\\
3. & Enth"alt $\cal C$ unendlich viele Simplizes, so ist $\cal C$ lokal\\
   & endlich, das hei"st jeder Punkt in einem Simplex aus $\cal C$ besitzt\\
   & eine Umgebung, die nur endlich viele Simplizes aus $\cal C$ schneidet.
\etab

(Wiederum steht hier der $\R^n$ beziehungsweise $\E^n$ als Synonym f"ur einen
beliebigen Vektorraum "uber einem K"orper $\K$, in dem o.a. Konzepte "ahnlich
definierbar sind.)

Betrachtet man die offenen Simplizes eines simplizialen Komplexes, so
erf"ullen diese auch die Axiome eines CW-Komplexes. Folglich ist der {\bf einem
simplizialen Komplex} $\cal C$ {\bf zugrundeliegende topologische Raum}
\idx{simplizialer Komplex!zugrundeliegender Raum} oder {\bf Tr"agerraum}
\idx{simplizialer Komplex!Tr\"agerraum} gegeben durch
$$|{\cal C}| := \bigcup\limits_{S\in{\cal C}} S \subset \R^n,$$
analog der Zellenzerlegung des einem CW-Komplex zugrundeliegenden
Hausdorff-Raumes.

Verschiedene Komplexe k"onnen so den gleichen zugrundeliegenden Teilraum des
$\R^n$ besitzen. Simpliziale Komplexe werden auch als "`Polyeder"', gem"a"s
ihrer Darstellung im Anschauungsraum, bezeichnet.\\
Der {\bf Randkomplex} eines simplizialen k-Komplexes $\cal C$ ist der Komplex
der aus allen Simplizes besteht, deren Dimension k-1 nicht "ubersteigt, wenn
die maximale Dimension der Simplizes aus $\cal C$ k ist, also das
(k-1)-Skelett.\idx{simplizialer Komplex!Randkomplex}\\
{\bf Unterkomplexe} sind allgemein Teilmengen des Komplexes, die
selbst wieder Komplexe sind.\idx{simplizialer Komplex!Unterkomplex}

Ist $\cal C$ ein simplizialer Komplex und C $\subset\cal C$, so ist der
{\bf Stern}\idx{simplizialer Komplex!Stern} st(C;$\cal C$) von C der kleinste
Unterkomplex von Elementen aus $\cal C$, der C enth"alt. Der {\bf Antistern}
\idx{simplizialer Komplex!Antistern} ast(C;$\cal C$) von C ist der
Unterkomplex von $\cal C$ aller Elemente, deren Schnitt mit C leer ist. Damit
kann der {\bf Link}\idx{simplizialer Komplex!Link}
link(C;$\cal C$) als der Durchschnitt von Stern und Antistern von C definiert
werden. In gleicher Weise l"a"st sich dies auch f"ur die Abschl"usse der Zellen
in CW-Komplexen definieren.

\subsection{Abbildungen von CW-Komplexen}

Eine stetige Abbildung f zwischen zwei simplizialen Komplexen $\cal C$
und ${\cal D}$ hei"st {\bf simplizial},\idx{simpliziale Abbildung} wenn
Simplizes aus $\cal C$ affin auf Simplizes aus ${\cal D}$ abgebildet werden.
Analog kann eine Abbildung $f:(X,{\cal E})\to (Y,{\cal F})$
zwischen CW-Komplexen als {\bf zellul"ar}\idx{zellul\"are Abbildung} bezeichnet
werden, wenn sie surjektiv ist und Zellen auf Zellen abbildet.\\
Eine simpliziale oder zellul"are Abbildung f hei"st {\bf nicht-degenerierend},
\idx{nicht-degenerierend} falls f die Dimension einer jeden Zelle erh"alt,
wenn also die Dimension von S und f(S) f"ur jede Zelle S aus $\cal C$
"ubereinstimmt.\\
Eine simpliziale Abbildung $f:{\cal C} \to {\cal C}$ eines simplizialen
Komplexes $\cal C$ auf sich, hei"st {\bf Symmetrie},\idx{Symmetrie!simplizialer
Komplexe}\label{symm} falls sie bijektiv ist und die Dimension aller Simplizes
erh"alt. Somit sind die Automorphismen von $\cal C$ gerade dessen Symmetrien.
Analog ist eine Symmetrie eines CW-Komplexes wieder eine bijektive zellul"are
Abbildung eines CW-Komplexes auf sich, die die Dimension aller Zellen erh"alt.

Um sich den Tr"agerraum eines Komplexes genauer anschauen zu k"onnen und
diesen zu beschreiben, ist es sinnvoll hier den Begriff der
{\bf Mannigfaltigkeit}\idx{Mannigfaltigkeit} der Dimension d einzuf"uhren.
Dabei handelt es sich um einen Hausdorff-Raum mit abz"ahlbarer Basis
(eine Basis der Topologie bestehe aus abz"ahlbar vielen offenen Mengen),
in dem jeder Punkt eine offene Umgebung besitzt, die zu einer offenen Teilmenge
des $\R^d$ hom"oomorph ist. Insbesondere bezeichnet man einen Hausdorff-Raum,
in dem jeder Punkt eine zu einer Kreisscheibe hom"oomorphe Umgebung besitzt
als {\bf Fl"ache}\idx{Fl\"ache}. In bezug auf simpliziale Komplexe ist eine
{\bf 2-Pseudo-Mannigfaltigkeit}\idx{Pseudo-Mannigfaltigkeit} ein endlicher
geschlossener simplizialer 2-Komplex, das hei"st ein Komplex, in dem jede Kante
(jedes 1-Simplex) an genau zwei Dreiecken (2-Simplizes) beteiligt ist.

{\scsi
Das Bild eines simplizialen 2-Komplexes unter einer stetigen Abbildung
$f:{\cal C}\to \R^3$ mu"s nicht durchdringungsfrei sein. Im Falle von
Triangulierungen der Kleinschen Flasche etwa (vgl. die Dissertation von Cervone,
\cite{Ce:93}) ist die beschriebene (nicht orientierbare) Mannigfaltigkeit nur
mit Selbstdurchdringung im $\R^3$ darstellbar, was aus dem 2-Komplex
nicht ohne weiteres ersichtlich ist. Bijektive Abbildungen f, deren Bild
durchdringungsfrei ist, hei"sen Einbettung von $\cal C$ in den $\R^3$.
Ist f nur lokal bijektiv, also in einer Umgebung jedes Punktes, so hei"st f
Immersion.
}

Da wir uns im dritten Kapitel mit regul"aren Karten besch"aftigen wollen,
sei hier noch die Definition dieser angegeben. Zerlegt man eine geschlossene
reelle 2-Man\-nig\-fal\-tig\-keit in $f_2$ einfach zusammenh"angende
(wegzusammenh"angend und jede Schleife nullhomotop, das hei"st zu einem Punkt
zusammenziehbar), nicht-"uberlappende Gebiete (Seiten), deren Durchschnitte
$f_1$ Kanten bilden, die sich in $f_0$ Ecken schneiden, so nennt man diese
Unterteilung eine {\bf Karte} der 2-Mannigfaltigkeit. {\bf Regul"ar} hei"st die
Karte, wenn ihre Automorphismengruppe Flaggen-transitiv ist, aber dazu sp"ater
mehr (vgl. Symmetriebegriffe Seite \pageref{flag}, eine kurze Beschreibung
findet sich in \cite{BoWi:87}).

Wie oben erw"ahnt, sind die Zellen von CW-Komplexen und die Simplizes
simplizialer Komplexe selbst topologische R"aume mit der durch den Tr"agerraum
induzierten Topologie. Speziell kennt man aber umgekehrt bei einem
simplizialen Komplex auch dessen Tr"agerraum schon dann bis auf Hom"oomorphie,
wenn die Anzahl der wesentlichen Simplizes (solche, die nicht schon als
Seiten gr"o"serer Simplizes vorkommen), sowie deren Inzidenzen, bekannt ist.
Dazu beachte man, da"s der Tr"agerraum X eines simplizialen n-Komplexes als
disjunkte Vereinigung der d-Simplizes $S_d$ des Kom\-plex\-es dargestellt werden
kann.
$$X=(S_0+\ldots+S_0)+\ldots+(S_n+\ldots+S_n)$$
Indiziert man die 0-Simplizes mit positiven ganzen Zahlen, den sogenannte
Simplexzahlen,\idx{Simplexzahl} so ist damit eine "Aquivalenzrelation $\sim$ :=
"`zu identifizierende 0-Simplizes"' gegeben, die mit den Simplizes den
Quotientenraum X$/_\sim$ beschreibt und so einem Hom"oomorphismus in X
bereitstellt. In Abbildung \ref{inzidenz} entsteht so aus acht disjunkten
Teilr"aumen des $\E^2$, den Dreiecken, mittels Eckenidentifikation der
Quotientenraum "`Rand eines Oktaeder"' und damit als Tr"agerraum eine
2-Mannigfaltigkeit. Wieder gilt eine Analogie zu zweidimensionalen CW-Komplexen,
deren 2-Zellen durch Polygonz"uge (Ecken und Kanten) begrenzt sind.

\begin{figure}[htb]
$$
\beginpicture
\unitlength1cm
\setlinear
\setcoordinatesystem units <0.6cm,0.6cm>
\setplotarea x from -2.5 to 2.5, y from -2.5 to 2.5
\put{ \beginpicture
\setsolid
\plot -0.909 0.188 -1.083 -0.157 /
\plot -1.083 -0.157 0 1.5 /
\plot 0 1.5 -0.909 0.188 /
\plot -1.083 -0.157 0.909 -0.188 /
\plot 0.909 -0.188 0 1.5 /
\plot 0.909 -0.188 1.083 0.157 /
\plot 1.083 0.157 0 1.5 /
\plot 0 -1.5 -1.083 -0.157 /
\plot 0 -1.5 0.909 -0.188 /
\setdashes <1mm>
\plot 1.083 0.157 -0.909 0.188 /
\plot -0.909 0.188 0 -1.5 /
\plot 0 -1.5 1.083 0.157 /
\put {\scsi 1} [Br] at -1.2 0.2
\put {\scsi 2} [Br] at -1.4 -0.2
\put {\scsi 3} [Bl] at 1.2 -0.2
\put {\scsi 4} [Bl] at 1.3 0.2
\put {\scsi 5} [bl] at 0 1.6
\put {\scsi 6} [tl] at 0 -1.7
\endpicture } at 3 0
\put{ \beginpicture
\setsolid
\plot -1 0 -1.5 -0.866 -1 -1.732 -0.5 -0.866 0.5 -0.866 1 0 2 0 1.5 0.866 0.5 0.866 0 0 -1 0 /
\plot -1 0 -0.5 -0.866 /
\plot 0 0 0.5 -0.866 /
\plot 0.5 0.866 1 0 /
\plot 1.5 0.866 1 0 /
\plot -1.5 -0.866 -0.5 -0.866 0 0 1 0 /
\put {\scsi 5} [br] at -1 0
\put {\scsi 1} [Br] at -1.5 -0.866
\put {\scsi 6} [tl] at -1 -1.732
\put {\scsi 2} [tl] at -0.5 -0.9
\put {\scsi 6} [tl] at 0.5 -0.866
\put {\scsi 4} [tl] at 1 -0.05
\put {\scsi 6} [tl] at 2 0
\put {\scsi 1} [bl] at 1.5 0.9
\put {\scsi 5} [br] at 0.5 0.9
\put {\scsi 3} [br] at 0 0.05
\endpicture } at -3 0
\endpicture
$$
\caption{Der Quotientenraum "`Oktaeder"'}
\label{inzidenz}
\end{figure}

Unter Ausnutzung der Nummerierung der 0-Simplizes eines simplizialen Komplexes
(bei Betrachtung des Quotientenraumes) kann jede Symmetrie
\idx{Symmetrie!eines simplizialen Komplexes} als Element der Permutationsgruppe
$S_n$ dargestellt werden, wenn n die Anzahl der verschiedenen 0-Simplizes
beziehungsweise Eckpunkte bezeichnet.

\subsection{Inzidenzen und Anheftungen}

Die Betrachtung von allgemeinen CW-Komplexen hat gegen"uber simplizialen
Komplexen den Vorteil eine gr"o"sere Vielfalt bei der Beschreibung von
Zellenberandungen zu haben. Auch besteht etwa ein zu \S$^2$ hom"oomorpher
simplizialer Komplex aus mindestens 14 Simplizes (zum Aufbau des Randes eines
3-Simplex braucht man vier 0-Simplizes, sechs 1-Simplizes und vier 2-Simplizes),
w"ahrend zu einem hom"oomorphen allgemeinen CW-Komplex ganze zwei Zellen
ben"otigt werden (eine 2-Zelle und eine 0-Zelle). Zudem sind CW-Komplexe als
ein Axiomensystem erf"ullende Hausdorff-R"aume nicht von vornherein auf
Vektorr"aume beschr"ankt, wie dies bei den simplizialen Komplexen der Fall ist.
Gerade in bezug auf die orientierten Matroide kann f"ur diese so eine
nat"urliche Topologie angegeben werden, die unabh"angig von simplizialen
Eigenschaften ist.\\
Nat"urlich haben CW-Komplexe nicht nur Vorteile. Aufgrund ihrer gr"o"seren
Allgemeinheit ist eine algebraische Beschreibung nicht so elegant
m"oglich, wie es bei den simplizialen Komplexen mittels der Simplexzahlen
geschehen kann. Was hier bei den simplizialen Komplexen die Inzidenzangaben
sind, sind in gewisser Hinsicht die {\bf Anheftungsabbildungen} bei den
CW-Komplexen (X,$\cal E$). Dabei handelt es sich um
\idx{CW-Komplex!Anheftungsabbildung} stetige Abbildungen
$\varphi:\S^{n-1}\to X^{n-1}$ der (n-1)-Sph"are in das (n-1)-Skelett von X, so
da"s der Quotientenraum $X\cup_{\varphi} B^n$, nach
dem x und $\varphi$(x) f"ur "aquivalent erkl"art werden, wieder ein CW-Komplex,
jetzt mit einer n-Zelle mehr ist. Der Zellenrand der neuen Zelle ist
$\varphi(\S^{n-1})\subset X^{n-1}$, ein stetiges Bild der (n-1)-Sph"are.
(Das Anheften kann man sich bildlich so vorstellen, da"s zwei Luftballons
zusammengeklebt werden, wobei man die Ber"uhrungspunkte der beiden Ballons
identifiziert.) Mittels Anheftungsabbildungen k"onnen so alle Ger"uste
von CW-Komplexen erzeugt werden. Genau dies ist eine der Schwierigkeiten beim
Umgang mit CW-Komplexen, denn eine Algebraisierung der Anheftungsabbildungen
(im Gegensatz zu den genauen Inzidenzvorschriften der simplizialen Komplexe) ist
erst durch Einsatz der Homologietheorie zu erreichen. Deren Darstellung w"urde
an dieser Stelle allerdings etwas zu weit f"uhren. Als einf"uhrende Lekt"ure sei
hier stellvertretend das Topologiebuch von Ossa (vgl.\cite{Os:92}, Seite 159ff.)
genannt.

Als eine der Wurzeln der Homologietheorie und als fundamentales Ergebnis bei
der Untersuchung von Polyedern darf allerdings der Begriff der
Euler-Charakteristik\idx{Euler-Charakteristik} und der Eulersche Polyedersatz
\idx{Eulerscher Polyedersatz} nicht fehlen.\\
Mit $f_k(\cal C)$ f"ur $0\leq k\leq d$ seien dabei die Anzahlen der
k-dimensionalen Elemente eines d-Komplexes $\cal C$ (entweder eines simplizialen
oder CW-Komplexes) bezeichnet. Die Folge $f({\cal C})=(f_0,f_1,\ldots,f_d)$
wird {\bf f-Vektor}\idx{f-Vektor} von $\cal C$ genannt.\\
"Uber den f-Vektor l"a"st sich die {\bf Euler-Charakteristik} $\chi(\cal C)$
eines d-Komplexes als
$$ \chi({\cal C}) = \sum\limits_{k=0}^{d-1} (-1)^k f_k({\cal C})=2-2g, $$
definieren. Der {\bf Eulersche Polyedersatz} besagt dazu
\begin{satz}
Die Euler-Charakteristik ist eine Hom"oomorphieinvariante.
\end{satz}
Dies bedeutet, da"s hom"oomorphe d-Komplexe die gleiche Eulercharakteristik
besitzen. Die Konstante g beschreibt dabei das {\bf Geschlecht} des
\idx{Geschlecht} topologischen Gebildes, das hei"st die Anzahl der Henkel
beziehungsweise L"ocher, die das Objekt hat. (Hier tiefer einzusteigen, w"urde
in die Homotopietheorie, die Theorie der stetigen Verformbarkeit, f"uhren, zu
der wieder der Verweis auf \cite{Os:92}, Seite 70ff., gegeben sei.)
F"ur Polyeder P ohne Henkel im $\R^3$, das hei"st Polyeder vom Geschlecht
Null, gilt nach oben also
$\chi(P)=\#\mbox{Ecken}-\#\mbox{Kanten}+\#\mbox{Fl"achen}=2$.
F"ur endliche CW-Komplexe kann die Euler-Charakteristik als
Wechselsumme ihrer Zellenanzahlen in den einzelnen Dimensionen leicht
berechnet werden. So ist etwa $\chi(\S^n) = 1 + (-1)^n$ und
$\chi(\S^1\times\S^1) = 1 - 2 + 1 = 0$.

\section{Orientierte Matroide}

Die Autoren Bj"orner, Las Vergnas, Sturmfels, White und Ziegler beginnen ihr
Buch "uber orientierte Matroide (vgl. \cite{Bj:93}) mit dem Satz
\begin{quote}{\sf
"`{\it Oriented matroids} can be thought of as a combinatorical abstraction
of point configurations over the reals, of real hyperplane arrangements, of
convex polytopes, and of directed graphs."'
}\end{quote}
Eine Verbindung zwischen CW-Komplexen und orientierten Matroiden zu suchen
erscheint nach diesem Satz nicht schwierig, da mit reellen Hyperebenen schon
etwas in der Art der ben"otigten n-Zellen gegeben ist und mit den spezielleren
simplizialen Komplexen sowohl konvexe Polytope, als auch Punktkonfigurationen
erfa"st werden k"onnen.

Nun fu"st die Theorie der orientierten Matroide in einer Vielzahl
unterschiedlicher Axiomensysteme, die untereinander "aquivalent (Die Beweise
der "Aquivalenz sind nach \cite{Bj:93} alles andere als trivial) doch, wie
in dem einf"uhrenden Satz angedeutet, die unterschiedlichsten Ausgangsebenen
beschreiben. In \cite{Bj:93} werden vier fundamentale Axiomensysteme
\idx{orientiertes Matroid!Axiomensysteme} hervorgehoben:

\btab{ll}
(1) & Kreisaxiome aus der Motivation gerichteter Graphen,\\
(2) & Orthogonalit"atsaxiome orthogonaler Paare reeller Vektorunterr"aume,\\
(3) & Chirotope von Punktkonfigurationen und konvexen Polytopen, sowie\\
(4) & Vektoraxiome reeller Hyperebenenarrangements.
\etab

{\scsi
Der Begriff Chirotop ist eine Abwandlung des Begriffs Chiralit"at nach
Dreiding und Dress. In der organischen Chemie bezeichnet Chiralit"at
(H"andigkeit) eine Dissymmetrie im r"aumlichen Aufbau chemischer
Verbindungen und stellt eine notwendige und hinreichende Voraussetzung f"ur das
Auftreten optischer Aktivit"at dar. (vgl. Literatur zur organischen Chemie,
etwa Fl"orke/Wolff, Kursthemen Chemie, Organische Chemie und Biochemie, D"ummler
Verlag, Bonn 1984)
}

In \cite{Bj:93} werden orientierte Matroide "uber alle vier Axiomensysteme
studiert und mit Beispielen untermauert. Da hier auch mittels orientierter
Matroide argumentiert werden soll, sei zun"achst eine Einf"uhrung gegeben,
was orientierte Matroide "uberhaupt darstellen und wie sie, f"ur diese Arbeit
n"utzlich, eingesetzt werden k"onnen.

Wie auch in anderen Publikationen, so m"ochte ich auch hier zun"achst anhand
der Chiro\-top\-axiome orientierte Matroide einf"uhren, um dann mit den
"`Vektoraxiomen f"ur reelle Hyperebenenarrangements"', die schon in Richtung
dessen gehen, was f"ur allgemeine CW-Komplexe ben"otigt wird, das
Einsatzgebiet f"ur diese Arbeit abzustecken.

\subsection{Orientierte Matroide von Punktkonfigurationen}

Orientierte Matroide beziehen sich immer auf eine endliche Menge E,
die der Einfachheit halber als eine geordnete Indexmenge $\{1,2,\ldots,n\}$
beschrieben sein soll. Die Elemente von E k"onnen als Indizes von Punkten im
reellen euklidischen Raum, von Hyperebenen oder auch abstrakt, ohne direkten
Bezug zu etwas "`Realisiertem"' aufgefa"st werden.

Zun"achst seien die Elemente von E als Indizes von n Punkten
$p_1,\ldots,p_n$ im $\R^{r-1}$ in allgemeiner Lage aufzufassen oder einfacher
als die Punkte selbst. Verwendet man homogene Koordinaten {(jeder affine Punkt
$p_i$ im $\R^{r-1}$ mit den Koordinaten $(p_i^1,p_i^2,\ldots,p_i^{r-1})$ wird
anschaulich als Vektor $(1,p_i)$ im $\R^r$ aufgefa"st)}, so definieren die n
Punkte eine (n$\times$r)-Matrix "uber $\R$.
$$\left(\begin{array}{cccc}
        1 & p_1^1 & \ldots & p_1^{r-1} \\
        1 & p_2^1 & \ldots & p_2^{r-1} \\
        \vdots & \vdots & \ddots & \vdots \\
        1 & p_n^1 & \ldots & p_n^{r-1} \end{array}\right)$$

{\scsi
Motivation des Einsatzes homogener Koordinaten ist die, da"s Punkte des
affinen euklidischen (r-1)-Raumes durch Einbettung in den projektiven r-Raum
$\P^r$ bez"uglich ihrer Lage besser beschrieben werden k"onnen. Stichwort
ist hierbei die Kompaktifizierung des euklidischen Raumes, in dem Sinne,
da"s nun unendlich ferne Punkte ebenfalls erfa"st werden k"onnen. Man
vergleiche dies mit der stereographischen Projektion der 2-Sph"are ohne den
Nordpol N auf $\R^2$, die mittels des Zusatzes $N \mapsto \{\infty\}$ einen
Hom"oomorphismus zwischen dem (nun kompakten) $\R^2\cup\{\infty\}$ und der
kompakten $S^2$ darstellt.
}

Eine Auswahl $(\lambda_1,\ldots,\lambda_r)$, mit $\lambda_i\in E$, von r
paarweise verschiedenen Punkten aus E bildet ein r-Simplex, dessen
Orientierung mittels des Vorzeichens der (r$\times$r)-Unterdeterminante
sign(det$(\lambda_1,\ldots,\lambda_r)$) oben definierter Matrix aus den
$\lambda_i$-ten Zeilen, analog dem Umlaufsinn eines Dreiecks (vgl.
Abb.\ref{orient}), beschrieben werden kann (vgl. \cite{BoEg:91}).
Genauer beschreibt die Determinante
$\det(\lambda_1,\ldots,\lambda_k,\ldots,\lambda_r)$ die Seite der
orientierten Hyperebene
$\mbox{aff}\{\lambda_1,\ldots,\lambda_{k-1},\lambda_{k+1},\ldots,\lambda_r\}$,
auf der der Punkt $\lambda_k$, f"ur alle k aus $\{1,\ldots,r\}$, liegt.

\begin{figure}[hbt]
$$
\beginpicture
\unitlength1cm
\setlinear
\setcoordinatesystem units <0.6cm,0.6cm>
\setplotarea x from 0 to 5, y from -1 to 3
\put{ \beginpicture
\setsolid
\plot 0 0 2 0 0 2 0 0 /
\put {\circle*{0.1}} [Bl] at 0 0
\put {\circle*{0.1}} [Bl] at 0 2
\put {\circle*{0.1}} [Bl] at 2 0
\put {\circle{0.5}} [Bl] at 0.6 0.6
\put {\vector(0,1){0.15}} [Bl] at 1 0.5
\put {\scsi (0,0)} [tr] at 0 0
\put {\scsi (1,0)} [tl] at 2 0
\put {\scsi (0,1)} [br] at 0 2
\put {$\left|\begin{array}{ccc} 1 & 0 & 0 \\
                                1 & 1 & 0 \\
                                1 & 0 & 1
             \end{array}\right|= +1$} [Bl] at 3 1
\endpicture } at -5.5 0
\put{ \beginpicture
\setsolid
\plot 0 0 2 0 0 2 0 0 /
\put {\circle*{0.1}} [Bl] at 0 0
\put {\circle*{0.1}} [Bl] at 0 2
\put {\circle*{0.1}} [Bl] at 2 0
\put {\circle{0.5}} [Bl] at 0.6 0.6
\put {\vector(0,-1){0.15}} [Bl] at 1 0.7
\put {\scsi (0,0)} [tr] at 0 0
\put {\scsi (1,0)} [tl] at 2 0
\put {\scsi (0,1)} [br] at 0 2
\put {$\left|\begin{array}{ccc} 1 & 0 & 0 \\
                                1 & 0 & 1 \\
                                1 & 1 & 0
             \end{array}\right|= -1$} [Bl] at 3 1
\endpicture } at 5.5 0
\endpicture
$$
\caption{Orientierung eines Dreiecks}
\label{orient}
\end{figure}

Als {\bf orientiertes Matroid zur Punktmenge E} wird nun die Information
bezeichnet, die sich aus den Vorzeichen der Determinanten zu allen
r-elementigen Untermengen $\{\lambda_1,\ldots,\lambda_r\}$ von E ergibt.
Bezeichnet $\Lambda (n,r)$ die Menge aller geordneten r-Tupel
\idx{geordnete r-Tupel} von n Elementen, das hei"st
$$\Lambda (n,r) := \left\{(\lambda_1,\ldots,\lambda_r)~|~1\leq\lambda_1
<\ldots<\lambda_r \leq n,\lambda_i\in\{1,\ldots,n\},1\leq i\leq r \right\},$$
so kann folgende Definition gegeben werden:
\bcent
\fbox{\parbox{14.2cm}{
  Eine Abbildung $\chi:\Lambda (n,r)\to\{-1,0,+1\}$ oder deren eindeutige
  alternierende Erweiterung $\chi:\{1,\ldots,n\}^r\to\{-1,0,+1\}$ hei"st
  {\bf orientiertes Matroid} vom Rang r mit n Punkten, wenn f"ur alle
  $\lambda\in\Lambda (n,r+1)$ und f"ur alle $\mu\in\lambda (n,r-1)$ die Menge
  $$\left\{(-1)^i\cdot\chi (\lambda_1,\ldots,\lambda_{i-1},\lambda_{i+1},
  \ldots,\lambda_{r+1})\cdot\chi (\mu_1,\ldots,\mu_{r-1},\lambda_i)~|~i\in\{1,
  \ldots,r+1\}\right\}$$
  entweder $\{-1,+1\}$ enth"alt oder gleich $\{0\}$ ist.
}}\ecent

Dabei hei"st $\chi:E^r\to\{-1,0,+1\}$ {\bf alternierend},\idx{alternierend} wenn
$$\chi(x_{\sigma_1},x_{\sigma_2},\ldots,x_{\sigma_r}) =
\mbox{sign}(\sigma)\chi(x_1,x_2,\ldots,x_r)$$
f"ur alle $x_i~(1\leq i\leq r)$ aus E und jede Permutation $\sigma$ aus der
Menge S$_E$ aller Permutationen der Elemente aus E gilt. Das entstehende
orientierte Matroid hei"st {\bf simplizial},
\idx{orientiertes Matroid!simpliziales $\sim$} wenn die Abbildung $\chi$
E$^r$ in $\{-1,+1\}$ abbildet, das hei"st, wenn $\chi(\lambda )\neq 0$ f"ur
alle $\lambda\in\Lambda (n,r)$ ist (vgl. \cite{BoEg:91}). Es hei"st
{\bf affin} oder {\bf azyklisch},\idx{orientiertes Matroid!affines $\sim$}
\idx{orientiertes Matroid!azyklisches $\sim$} wenn f"ur alle
$\lambda\in\Lambda(n,r+1)$ die Menge
$$\left\{(-1)^i\cdot\chi (\lambda_1,\ldots,\lambda_{i-1},\lambda_{i+1},
\ldots,\lambda_{r+1})~|~i\in\{1,\ldots,r+1\}\right\}$$
entweder $\{-1,+1\}$ enth"alt oder gleich $\{0\}$ ist. Abbildung \ref{cube}
zeigt ein orientiertes Matroid zum dreisimensionalen W"urfel. Die Schreibweise
von $\chi(\Lambda)$ ist so zu verstehen, da"s zeilenweise alle Vorzeichen zu
den kanonisch geordneten (elementweise "`$\leq$"') Tupeln aus $\Lambda$
aufgef"uhrt sind.

\begin{figure}[htb]
$$
\beginpicture
\unitlength1cm
\setlinear
\setcoordinatesystem units <0.6cm,0.6cm>
\setplotarea x from -3 to 5, y from -2 to 2
\setsolid \thicklines
\put {\beginpicture
\setsolid
\plot 0.597 -1.378 -1.281 -1.144 /
\plot 0.597 -1.378 0.597 0.501 /
\plot 0.597 0.501 -1.281 0.735 /
\plot -1.281 0.735 -1.281 -1.144 /
\plot 1.281 -0.735 0.597 -1.378 /
\plot 1.281 -0.735 1.281 1.144 /
\plot 1.281 1.144 0.597 0.501 /
\plot -0.597 1.378 1.281 1.144 /
\plot -1.281 0.735 -0.597 1.378 /
\setdashes <1mm>
\plot -0.597 -0.501 1.281 -0.735 /
\plot -1.281 -1.144 -0.597 -0.501 /
\plot -0.597 -0.501 -0.597 1.378 /
\endpicture} at -2.5 0
\put {\scsi$\left(\begin{array}{cccc}
             1 & 0 & 0 & 0 \\
             1 & 1 & 0 & 0 \\
             1 & 0 & 1 & 0 \\
             1 & 0 & 0 & 1 \\
             1 & 1 & 1 & 0 \\
             1 & 1 & 0 & 1 \\
             1 & 0 & 1 & 1 \\
             1 & 1 & 1 & 1
      \end{array}\right)$} [Bl] at 0 0
\put {\scsi$\chi(\Lambda)=\left.\begin{array}{cccccccccccccccc}
          \{&+&0&+&+&+&-&0&-&-&+&+&+&-&-&\\
            &0&+&+&0&+&-&-&-&-&0&+&-&+&0&\\
            &+&+&-&-&-&+&-&+&+&+&+&-&-&-&\\
            &0&+&+&-&0&-&+&+&-&-&0&+&-&0&\\
            &+&+&+&+&-&-&-&0&-&-&-&+&0&-&\}
            \end{array}\right.$} [Bl] at 4.8 0
\endpicture
$$
\caption{Ein W"urfel mit seinem Chirotop}
\label{cube}
\end{figure}

Der {\bf Satz von Radon} besagt, da"s f"ur jede endliche Punktmenge X im $\E^r$
mit einer Punkteanzahl $\geq r+2$ eine Zerlegung von X in disjunkte Teilmengen
X$_1$ und X$_2$ existiert, f"ur die
$\mbox{conv X}_1~\cap\mbox{conv X}_2~\neq~\emptyset$ gilt. Eine solche
Zerlegung hei"st {\bf Radonpartition}.\idx{Radonpartition}
Die Radonpartitionen oben definierter Punktmenge E liefern die sogenannten
{\bf Kreise} (eine Bezeichnung, die von der Motivation "uber gerichtete
Graphen her stammt) des orientierten Matroids. Diese sind f"ur
$\mu\in\Lambda(n,r+1)$ und $1\leq i\leq n$ gegeben durch
\idx{orientiertes Matroid!Kreis eines $\sim$}
$$ C_\mu(i) := \left\{\begin{array}{ll}
   (-1)^j\chi(\mu_1,\ldots,\mu_{j-1},\mu_{j+1},\ldots,\mu_{r+1}) &
   \mbox{f"ur } i=\mu_j\\
   0 & \mbox{sonst} \end{array} \right.$$
Die Menge ${\cal K}(\chi):=\{\pm C_\mu|\mu\in \Lambda (n,r+1)\}$ beschreibt
damit alle Kreise des orientierten Matroids $\chi$.

Ist analog $\lambda\in\Lambda (n,d-1)$, so hei"st
$C^*_\lambda(i) := \chi(\lambda_1,\ldots,\lambda_{r-1},i)$ f"ur
$1\leq i\leq n$ {\bf Kokreis}\idx{orientiertes Matroid!Kokreis eines $\sim$}
von $\chi$\label{kokreis} und
${\cal K}^*(\chi) := \{\pm C^*_\lambda|\lambda\in\Lambda (n,r-1)\}$
ist die Menge aller Kokreise.

Ist $\chi$ durch eine konkrete Punktkonfiguration gegeben, so entsprechen
die Kokreise den Hyperebenen, die durch die Punkte mit $C^*_\lambda(i)=0$
gegeben sind. Am Beispiel des Chirotops zum W"urfel (Abb.\ref{cube}) ergibt sich
etwa ein Kreis $C_\mu$ mit $\mu=(2,4,5,7,8)$ als $(0-0++0-0)$.\\
Der Kokreis $C^*_\lambda$ mit $\lambda=(1,3,7)$ hat die Darstellung
$(0+00++0+)$, woraus sich ablesen l"a"st, da"s es sich um eine St"utzhyperebene
handelt, da alle Punkte auf einer Seite der durch die Punkte 1,3 und 7
induzierten Hyperebene H (angezeigt durch vier +, sowie den Punkt 4 auf H) liegen.

Setzt man Mittel der linearen Algebra ein und untersucht die Eigenschaften
der verwendeten Determinanten genauer, so kann obige Situation auch
unabh"angig von einer konkreten Punktkonfiguration beziehungsweise einer
konkreten Matrix als Definition f"ur abstrakt aufzufassende {\bf Chirotope}
\idx{Chirotop} vom Rang r auf E dienen. Dies f"uhrt(e) zu den folgenden
"`Chirotopaxiomen"'\idx{Chirotopaxiome}

\bcent
\fbox{\parbox{13cm}{
\btab{ll}
(B0) & $\chi$ ist nicht identisch mit der Nullabbildung\\
(B1) & $\chi$ ist alternierend\\
(B2) & F"ur alle $x_1,x_2,\ldots,x_r,y_1,y_2,\ldots,y_r,$ aus E mit\\
     & $\chi(y_i,x_2,x_3,\ldots,x_r)\cdot\chi(y_1,y_2,\ldots,y_{i-1},x_i,
       y_{i+1},y_{i+2},\ldots,y_r) \geq 0$\\
     & bei $i=1,2,\ldots,r$ gilt
       $\chi(x_1,x_2,\ldots,x_r)\cdot\chi(y_1,y_2,\ldots,y_r) \geq 0$
\etab}}
\ecent

Die "Aquivalenz von Chirotopen und orientierten Matroiden wurde 1982 durch
Lawrence mit folgendem Satz gesichert.

\begin{satz}
Sei $r\in\N$ und E sei eine Menge. Eine Abbildung
$\chi:E^r\to\{-1,0,+1\}$ ist genau dann "aquivalent zu einem orientierten
Matroid vom Rang r auf E, wenn sie ein Chirotop ist.
\end{satz}

Der Beweis ist nachzulesen in \cite{Bj:93}, Seite 128ff. Wie oben angedeutet,
fordert Axiom B2 die Erf"ullung abstrakter Gra"smann-Pl"ucker-Relationen,
\idx{Gra\3\-mann-Pl\"ucker-Relation} die in ihrer expliziten Form mit
Determinanten f"ur alle $x_1,x_2,\ldots,x_r,y_1,\ldots,y_r \in \R^r$ die
Identit"at
$$\det(x_1,x_2,\ldots,x_r) \cdot \det(y_1,y_2,\ldots,y_r)$$
$$= \sum\limits_{i=1}^r (-1)^{i-1} \det(y_i,x_2,\ldots,x_r) \cdot
    \det(x_1,y_1,\ldots,y_{i-1},y_{i+1},\ldots,y_r)$$
liefert (vgl.\cite{Na:72}, Stichwort "`Gra"smannsche Mannigfaltigkeit"').

Mittels formaler Brackets (formaler Determinanten) lassen sich die allgemeinen
{\bf k-summandigen Gra"smann-Pl"ucker-Relationen} f"ur 3$\leq$k$\leq$r+1, mit
\idx{Gra\3\-mann-Pl\"ucker-Relation!k-summandige}
Mengen paarweise verschiedener Elemente $A=\{a_1,\ldots,a_{d-k+1}\}$,
$B=\{b_1,\ldots,b_{k-2}\}$ und C=$\{c_1,\ldots,c_k\}$ aus E schreiben als

{\small
$$\{A|B|C\}=$$
$$\sum\limits_{i=1}^k(-1)^{i+1}\cdot
[a_1,\ldots,a_{d-k+1},c_1,\ldots,c_{i-1},c_{i+1},\ldots,c_k]\cdot
[a_1,\ldots,a_{d-k+1},b_1,\ldots,b_{k-2},c_i]=0$$}

Unabh"angig von gegebenen Punkten kann so ein orientiertes Matroid mittels
einer Vorzeichenliste definiert werden, die zu den
formalen Brackets $[\lambda_1,\ldots,\lambda_r]$ geh"ort und mit der die
abstrakten Gra"s\-mann-Pl"ucker-Relationen B2 gelten.

Die oben angef"uhrten Gra"smann-Pl"ucker-Relationen liefern also eine weitere
Charakterisierung f"ur orientierte Matroide. Bemerkenswert ist, da"s der
folgende Satz (siehe \cite{Bj:93}, Seite 138) "uber dreisummandige
Gra"smann-Pl"ucker-Relationen
\idx{Gra\3\-mann-Pl\"ucker-Relation!dreisummandige $\sim$en}
gilt.

Bevor wir allerdings zur Formulierung des Satzes kommen, seien an dieser Stelle
zun"achst die Definitionen der verwendeten Begriffe Matroid und Basen eines
Matroids eingef"ugt (vgl. dazu auch Aigner, Kombinatorik Bd.II (\cite{Aig:76}),
Seite 17ff., sowie \cite{Schu:92}, Seite 30).\label{matroid}

Der {\bf Steinitzsche Austauschsatz}\idx{Steinitz, Austauschsatz von} besagt,
da"s wenn in einem endlichdimensionalen Vektorraum V ein Vektor v nicht linear
abh"angig von einer unabh"angigen Menge von Vektoren $\{u_1,\ldots,u_n\}$, aber
abh"angig von $\{u_1,\ldots,u_n,w\}$ ist, so ist der Vektor w linear abh"angig
von $\{u_1,\ldots,u_n,v\}$.\\
Als Verallgemeinerung hiervon wird mit E=$\{1,\ldots,n\}$ und einer Teilmenge
$\B$ der Potenzmenge von E das geordnete Paar (E,$\B$) als ein {\bf Matroid}
\idx{Matroid} bezeichnet, wenn die leere Menge \O\ in $\B$ liegt, mit jeder
Menge $B\in\B$ auch deren Teilmengen in $\B$ liegen, sowie f"ur alle $B_1,
B_2\in\B$ und $x\in B_1\backslash B_2$ ein y aus $B_2\backslash B_1$ existiert,
so da"s $(B_1\backslash\{x\})\cup\{y\}$ Element von $\B$ ist. Die Mengen aus
$\B$ hei"sen {\bf unabh"angig}, maximale Mengen in $\B$ werden aufgrund ihrer
gemeinsamen M"achtigkeit {\bf Basen des Matroids} genannt. Diese M"achtigkeit
wird auch als {\bf Rang des Matroids} bezeichnet. Ist E zusammen mit einer
(wie oben definierten) Abbildung $\chi$ ein orientiertes Matroid, so kann man
alle (formalen) Brackets betrachten, die unter $\chi$ ungleich Null sind. Die
Menge der zugeh"origen r-Tupel aus $\Lambda(n,r)$ all dieser Brackets nennt man
supp($\chi$). Dieser {\bf Tr"ager} supp($\chi$) bildet die Menge der Basen eines
Matroids, das dem {\bf orientierten Matroid} (E,$\chi$) {\bf zugrundeliegende
Matroid}. In bezug auf die zu erf"ullenden Gra"smann-Pl"ucker-Relationen ist ein
Matroid die Einschr"ankung eines "uber dem K"orper GF(3) definierten
orientierten Matroids auf GF(2), was sich auch aus dem folgenden ableiten
l"a"st.
\begin{satz}\label{dgpr}
Eine Abbildung $\chi:E^r\to\{-1,0,+1\}$ (in der bisherigen Notation) ist genau
dann ein Chirotop, wenn folgende zwei Bedingungen erf"ullt sind:

\btab{ll}
(B1$'$) & $\chi$ ist alternierend, und die Menge der r-Untermengen
          $\{x_1,\ldots,x_r\}$ aus E\\
        & mit $\chi(x_1,\ldots,x_r)\neq 0$ ist die Menge der Basen eines
          Matroids vom \\
        & Rang r auf E.\\
(B2$''$) & F"ur alle $x_1,\ldots,x_r,y_1,y_2\in E$ gilt, falls\\
         & $\chi(y_1,x_2,\ldots,x_r)\cdot\chi(x_1,y_2,x_3,\ldots,x_r)\geq 0$
           und\\
         & $\chi(y_2,x_2,\ldots,x_r)\cdot\chi(y_1,x_1,x_3,\ldots,x_r)\geq 0$
           erf"ullt sind, auch\\
         & $\chi(x_1,x_2,\ldots,x_r)\cdot\chi(y_1,y_2,x_3,\ldots,x_r)\geq 0$
\etab
\end{satz}

Die dreisummandigen Gra"smann-Pl"ucker-Relationen k"onnen im Rang r f"ur
disjunkte Teilmengen $A=\{a_1,\ldots,a_{d-2}\}$ und $B=\{b_1,\ldots,b_4\}$
von E mittels (formaler) Brackets geschrieben werden als
$$\{a_1,\ldots,a_{d-2}|b_1,\ldots,b_4\}:=
  \begin{array}{l}
    +[a_1,\ldots,a_{d-2},b_1,b_2]\cdot [a_1,\ldots,a_{d-2},b_3,b_4]\\
    -[a_1,\ldots,a_{d-2},b_1,b_3]\cdot [a_1,\ldots,a_{d-2},b_2,b_4]\\
    +[a_1,\ldots,a_{d-2},b_1,b_4]\cdot [a_1,\ldots,a_{d-2},b_2,b_3]
  \end{array}=0$$
Im Falle von $\chi:E^d\to\{-1,+1\}$ l"a"st sich so in einem Programm testen,
ob bei Vorgabe von $\chi$ ein Chirotop und damit ein orientiertes Matroid
vorliegt. Allgemein mu"s hier zus"atzlich "uberpr"uft werden, ob durch eine
vorgelegte Vorzeichenliste die nach B1$'$ geforderten Basen eines Matroids
gegeben sind --- als Alternative kann man hier auch die k-summandigen
Gra"smann-Pl"ucker-Relationen von B2 nachrechnen.

Wie man sich "uberlegen kann, ist aus einer Punktkonfiguration "uber die
zugeh"orige Punktmatrix f"ur n $\geq$ d Punkte immer ein orientiertes Matroid
durch ein Chirotop gegeben, da hierbei die k-summandigen
Gra"smann-Pl"ucker-Relationen immer erf"ullt sind (nachweisbar ist dies mittels
des Laplaceschen Entwicklungssatzes f"ur Determinanten). Umgekehrt kann
zu diesem auch wieder eine (zumindest isomorphe) Punktkonfiguration angeben
werden, da man ja wei"s, da"s eine solche existiert. Im allgemeinen Fall kann
allerdings von einem Chirotop nicht auf eine zugeh"orige Punktkonfiguration
geschlossen werden. Dieses Problem, zu einem orientierten Matroid eine
entsprechende Punktmatrix zu finden, bezeichnet man als Suche nach einer
Realisierung.\label{real}

Als eine {\bf Realisierung}\idx{Realisierung} eines orientierten Matroids
${\cal M}=(E,\chi)$ vom Rang r "uber einer totalgeordneten n-elementigen Menge
E bezeichnet man eine Abbildung $\Phi$, die E in den $\R^r$ abbildet, so da"s
$\chi(e_1,e_2,\ldots,e_r)=\mbox{sign det}(\Phi(e_1),\Phi(e_2),\ldots,\Phi(e_r))$
f"ur alle $e_i\in E$ gilt, wenn $\chi:E^r\to\{-1,0,+1\}$ das zugeh"orige
Chirotop bezeichnet. Existiert solch ein $\Phi$, so hei"st $\cal M$
realisierbar.

{\scsi
Eine wichtige Eigenschaft im Zusammenhang mit der Realisierung von
orientierten Matroiden ist, da"s diese immer (zumindest) lokal realisierbar
sind (in diesem Sinne sind sie lokal in den Euklidischen Raum einbettbar). Dazu
wird in \cite{Bj:93} (Korollar 3.6.3, Seite 140) gezeigt, da"s jedes azyklische
Rang r orientierte Matroid als abstrakte (r-1)-dimensionale affine
Punktkonfiguration angesehen werden kann, von der jede (r+2)-punktige
Unterkonfiguration koordinatisierbar ist. Bokowski und Richter-Gebert haben
zu diesem Themenbereich interessante Arbeiten beigesteuert.
}

Die Suche nach Realisierungen beliebiger orientierter Matroide ist als eine
wichtige Fragestellung in \cite{Bj:93} angesprochen und Gegenstand aktueller
Arbeiten, wie zum Beispiel auch \cite{Schu:92} und \cite{Dau:89}, in denen von
simplizialen orientierten Matroiden ausgegangen wird. Hier soll nun das Problem
der Realisierbarkeit von allgemeinen CW-Komplexen, im Sinne durchdringungsfreier
Darstellung im euklidischen Raum, auf die Problematik bei orientierten Matroiden
"ubertragen werden, wobei nun auch der nichtsimpliziale Fall zuzulassen ist.
Gibt es n"amlich zu einem beliebigen CW-Komplex $\cal C$ ein realisierbares
orientiertes Matroid $(E,\chi)$, das gewisse Eigenschaften von $\cal C$
bewahrt, so sind durch $\Phi$ Punkte im $\R^r$ gegeben, mit denen $\cal C$ ohne
Selbstdurchdringungen dargestellt werden kann. Solch ein $(E,\chi)$ hei"st eine
{\bf Matroideinbettung} von $\cal C$.\idx{Matroideinbettung} Zun"achst mu"s
hierzu aber eine Verbindung zwischen CW-Komplexen und orientierten Matroiden
angegeben werden, die gerade diese "`gewissen Eigenschaften"' beschreibt.\\
Da wir es bei CW-Komplexen mit zum $\R^n$ hom"oomorphen Hausdorff-R"aumen,
zutun haben, soll nun eine weitere Repr"asentationsm"oglichkeit orientierter
Matroide angegeben werden, die (Pseudo-)Hyperebenen nutzt, welche dazu dienen
sollen, oben geforderte Verbindung zwischen den beiden betrachteten
Objektklassen zu schaffen.

\subsection{Orientierte Matroide von Hy\-per\-ebe\-nen- und
            Pseu\-do\-hy\-per\-ebe\-nen\-arrange\-ments}

Zur Motivation betrachte man wieder eine endliche Menge $E=\{1,2,\ldots,n\}$
und ein zentrales (den Nullpunkt enthaltendes) Arrangement orientierter
Hyperebenen im $\R^r$ (Beispiel Abb.\ref{hyper}), das gegeben sei durch
$${\cal A}=(H_e=\{x\in \R^r:<x,a_e>=0\})_{e\in E}$$ mit
einer Familie $(a_e)_{e\in E}$ von Normalenvektoren ungleich dem Nullvektor
und dem "ublichen Skalarprodukt $<x,y>:=\sum_{i=1}^rx_i\cdot y_i$ f"ur x und
y aus $\R^r$ (beziehungsweise einem Vektorraum "uber $\K$).
Um Orientierungen unterscheiden zu k"onnen, seien die durch die Hyperebenen
gegebenen Halbr"aume in Richtung der Normalen als positiv ausgezeichnet.
Zu allen $x\in\R^r$ sei $\sigma(x)=(\sigma_e(x)=\mbox{sign}<x,a_e>)_{e\in E}$
ein Vorzeichenvektor mit $|E|=n$ Komponenten der Gestalt $+$, $-$ oder $0$,
als Abk"urzung f"ur $+1$, $-1$ und $0$ entsprechend der Signumfunktion,
der die Lage von x bez"uglich jeder Hyperebene $H_e$ beschreibt.
Die Abbildung $\sigma$ bildet dabei den $\R^r$ in $\{+,-,0\}^E$, die Menge
aller Vorzeichenvektoren mit $|E|$ Komponenten, ab.
Jene Punkte des $\R^r$, die gleiche Vorzeichenvektoren besitzen, bilden
Zellen der Zerlegung, die durch $\cal A$ auf $\R^r$ induziert wird. Diese Zellen
sind konvexe offene Teilmengen linearer Teilr"aume des $\R^r$, und hei"sen
{\bf Topes}\idx{Tope} beziehungsweise {\bf Regionen}.\label{cell}
Mit $\sigma$ ist so eine "Aquivalenzrelation gegeben, denn es gilt f"ur
Punkte $x,y\in\R^r$ die Gleichheit $\sigma(x)=\sigma(y)$ genau dann, wenn
x und y aus der gleichen Zelle der Zerlegung des $\R^r$ durch $\cal A$ stammen.
Anstelle aller $x\in \R^r$ brauchen so nur deren "Aquivalenzklassen
bez"uglich $\sigma$ betrachtet werden, denn die Vorzeichenvektoren im Bild des
$\R^r$ unter $\sigma$ induzieren genau die Zellen, was eine kombinatorische
Beschreibung von $\cal A$ durch $\sigma$ darstellt.

\begin{figure}[htb]
$$
\beginpicture
\unitlength0.6cm
\setlinear
\setcoordinatesystem units <0.6cm,0.6cm>
\setplotarea x from 0 to 14, y from 0 to 9.5
\setsolid
\plot 5 8 5 5 4 4 4 7 6 9 6 8.35 /
\plot 4 4.65 3.5 4.5 3.5 7.5 6.5 8.5 6.5 7.5 /
\plot 3.5 7.5 3.5 8.5 6.5 7.5 6.5 4.5 6 4.65 /
\plot 4 8.35 4 9 6 7 6 4.2 5 5 /
\plot 6.5 6.85 7 7 8.5 5 2.5 3 1 5 3.5 5.8 /
\plot 3.5 3.35 3.5 2 4 2.2 /
\plot 4 3.5 4 1.5 5 2.5 5 3.85 /
\plot 5 2.5 6 1.5 6 4.15 /
\plot 6 2.2 6.5 2 6.5 4.3 /
\put {\scsi H$_1$} [Bl] at 2 4.5
\put {\scsi H$_2$} [Bl] at 4.3 5.7
\put {\scsi H$_3$} [Bl] at 2.8 7
\put {\scsi H$_4$} [Bl] at 2.8 8
\put {\scsi H$_5$} [Bl] at 3.5 9.2
\put {\scsi $+$} [Bl] at 2.5 3.5 \put {\scsi $-$} [Bl] at 2.5 2.7
\put {\scsi $+$} [Bl] at 4.5 7
\put {\scsi $-$} [Bl] at 5.25 7
\put {\scsi $+$} [Bl] at 3.75 7.25
\put {\scsi $+$} [Bl] at 3.75 8
\put {\scsi Regionen} [Bl] at 9 9
\put {\scsi $++++-$, $+-++-$,} [Bl] at 9 8
\put {\scsi $+--+-$, $+----$,} [Bl] at 9 7.5
\put {\scsi $+---+$, $++--+$,} [Bl] at 9 7
\put {\scsi $+++-+$, $+++++$,} [Bl] at 9 6.5
\put {\scsi $-+++-$, $--++-$,} [Bl] at 9 6
\put {\scsi $---+-$, $-----$,} [Bl] at 9 5.5
\put {\scsi $----+$, $-+--+$,} [Bl] at 9 5
\put {\scsi $-++-+$, $-++++$} [Bl] at 9 4.5
\endpicture
$$
\caption{F"unf Hyperebenen im $\R^3$}
\label{hyper}
\end{figure}

Nun betrachte man zu einer endlichen Menge E allgemeiner zun"achst beliebige,
das hei"st nicht "uber eine Abbildung $\sigma$ an Hyperebenen gekoppelte,
Vorzeichenvektoren\idx{Vorzeichenvektor} $X,Y\in\{+,-,0\}^E$.
Der {\bf Tr"ager}\idx{Vorzeichenvektor!Tr\"ager} eines solchen Vektors X ist
$\ul{X}=\{e\in E:X_e\neq 0\}$, seine {\bf Nullmenge}
\idx{Vorzeichenvektor!Nullmenge} $z(X)=X^0=E\backslash\ul{X}=\{e\in E:X_e=0\}$,
die Menge seiner positiven Elemente $X^+=\{e\in E:X_e=+\}$ und analog die Menge
seiner negativen Elemente $X^-=\{e\in E:X_e=-\}$. Der Vektor, der f"ur alle
$e\in E$ aus 0 besteht, hei"se analog dem $\R^d$ Nullvektor und sei mit 0
bezeichnet. Zu $X$ sei $-X$ gegeben durch die Vorzeichenumkehrung von X, also
$(-X_e)=-$ f"ur $X_e=+$, $(-X_e)=+$ f"ur $X_e=-$ und $(-X_e)=0$ f"ur $X_e=0$.
Weiter sei eine {\bf Zusammensetzung}\idx{Vorzeichenvektor!Zusammensetzung}
$X\circ Y$ von X und Y dadurch gegeben, da"s $(X\circ Y)_e=X_e$ f"ur
$X_e\neq 0$ und $Y_e$ sonst gelte. Die {\bf Trennungsmenge}
\idx{Vorzeichenvektor!Trennungsmenge} von X und Y sei
$S(X,Y)=\{e\in E:X_e=-Y_e\neq 0\}$.

Mit diesen Vorgaben hei"st eine Menge ${\cal L}\subseteq\{+,-,0\}^E$ {\bf Menge
der Kovektoren eines orientierten Matroids}, wenn folgende Axiome erf"ullt
sind, die dementsprechend als {\bf Kovektoraxiome}\idx{Kovektoraxiome}
bezeichnet werden.

\bcent
\fbox{\parbox{14cm}{
\btab{ll}
(L0) & ${\cal L}$ enth"alt den Nullvektor 0.\\
(L1) & Mit $X\in\cal L$ ist auch $-X$ in ${\cal L}$ enthalten.\\
(L2) & Sind X und Y aus $\cal L$, so auch deren Zusammensetzung $X\circ Y$\\
(L3) & Zu X und Y aus $\cal L$ und e aus S(X,Y) existiert ein Z in
       ${\cal L}$ mit Z$_e=0$\\
     & und Z$_f=(X\circ Y)_f=(Y\circ X)_f$ f"ur alle f $\notin$ S(X,Y).
\etab}}
\ecent

Um die Begriffswahl der Vorzeichenvektoren als "`\ul{Ko}vektoren eines
orientierten Matroids"' verstehen zu k"onnen, m"ussen an dieser Stelle die
Begriffe Polarit"at und Dualit"at eingef"uhrt werden, wie sie im gegenw"artigen
Zusammenhang zu verstehen sein sollen. Dazu ist ein kurzer Abstecher in die
Welt der topologischen Vektorr"aume n"otig, der uns die Definition der
gew"unschten Begriffe liefert.

Allgemein werden zwei topologische Vektorr"aume F und G "uber einem gemeinsamen
K"orper $\K$ als ein {\bf duales Paar}\idx{duales Paar} (F,G) bezeichnet, wenn
es zwischen ihnen eine Bilinearform $f:F\times G\to\K$ gibt, f"ur die zum einem
f"ur jedes feste $0\neq x_0\in F$ ein $y\in G$ existiert, so da"s
$f(x_0,y)\neq 0$ gilt, und zum anderen analog zu jedem festen $0\neq y_0\in G$
ein $x\in F$ gew"ahlt werden kann, mit $f(x,y_0)\neq 0$. Ist nun (F,G) solch
ein duales Paar, so ist die {\bf Polare}\idx{Polare} $A^*$ einer Menge
$A\subseteq F$ definiert als
$$ A^*=\{y\in G~:~|f(x,y)|\leq 1,~\forall x\in A\}.$$

Nach Gr"unbaum (vgl. \cite{Gr:67}, Seite 46ff.) hei"sen zwei d-dimensionale
Polytope $P$ und $P^*$ dual\idx{dual} wenn zwischen ihnen eine bijektive
Abbildung $\Psi$ der r-Seiten ($1\leq r\leq d$) von $P$ auf die Seiten von
$P^*$ existiert, die Inklusionen umkehrt. Dies bedeutet, da"s f"ur zwei Seiten
$F_1$ und $F_2$ von $P$ mit $F_1\subset F_2$ unter $\Psi$ gilt, da"s
$\Psi(F_1)\supset\Psi(F_2)$ ist. In der Theorie der orientierten Matroide
wird dieser Sachverhalt als {\bf Polarit"at}\idx{Polarit\"at} bezeichnet.\\
Da hier Vektorr"aume "uber $\R$ betrachtet werden, ist eine Bilinearform f
durch das "ubliche Skalarprodukt auf $\R^d$ gegeben. Damit ist die zu einem
Polytop $P$ geh"orige polare Menge $P^*$ gerade
$\{y\in\R^d~|~<x,y>\leq 1,~\forall x\in P\}$.
Bildet man die konvexe H"ulle der Extrempunkte von $P^*$, so ist diese ebenfalls
ein Polytop, das als das zu P polare bezeichnet wird und entsprechend seiner
Herkunft ebenfalls die Bezeichnung $"`P^*"'$ erh"alt. Nach obiger Definition ist
$P^*$ auch dual zu $P$, da sich eine inklusionsumkehrende Abbildung der
Seiten von $P$ auf die Seiten von $P^*$ angeben l"a"st. Anschaulicher ist dieser
Sachverhalt darzustellen, wenn man die Situation betrachtet, da"s der einen
Punkt P im $\R^d$ beschreibende Ortsvektor $x_P$ zum einen den Punkt als solchen
bezeichnet, aber auch als Normale einer zu $x_P$ orthogonalen (Hyper-)Ebene mit
der Hesseschen Normalform $<x_P,x>=1$ aufgefa"st werden kann (vgl. obige
Definition der Hyperebenenarrangements). Betrachtet man nun einerseits die
konvexe H"ulle einer endlichen Anzahl von Punkten $x_P$ und andererseits die
durch die (Hyper-)Ebenen mit den Normalen $x_P$ eingeschlossene Zelle
(gerade der Durchschnitt der Halbr"aume, die durch $<x_P,x>\leq 1$ gegeben sind,
so ergibt sich gerade die oben eingef"uhrte Polarit"at zwischen einer
Punktkonfiguration und der entsprechenden polaren (Hyper-)Ebenenkonfiguration.
\label{polar} Eindrucksvolle Beispiele sind die Polarit"aten der platonischen
K"orper im $\R^3$ (das selbstpolare Tetraeder, die polaren Paare W"urfel und
Oktaeder, sowie Dodekaeder und Ikosaeder).
Speziell entsprechen unter Polarit"at im $\R^3$ also Punkte Fl"achen, Kanten
Kanten und Fl"achen Punkten. Verallgemeinert auf h"ohere Dimensionen
bedeutet dies gerade, da"s im $\R^d$ (d-i)-Seiten von $P$ im polaren Fall zu
i-Seiten von $P^*$ werden (was gleichzeitig Dualit"at induziert).\\
Bei orientierten Matroiden sei mit Dualit"at\idx{Dualit\"at} die Situation
bezeichnet, wenn man, wie Bokowski in seinem Beitrag zum "`Handbook of Convex
Geometry"' (\cite{Bo:93}) angibt, etwa die Kreise eines orientierten Matroids
$\cal M$ als Kokreise eines anderen orientierten Matroids $\cal M^*$ ansieht,
welches dann das zu $\cal M$ Duale darstellt (vgl. Kreis-Definitionen von
Seite \pageref{kokreis}).\\
Wenn also von Vektoren und Kovektoren orientierter Matroide die Rede ist, so
spiegelt dies gerade die dualen Versionen wider. (Da die Dualit"at orientierter
Matroide in dieser Arbeit nicht weiter vertieft wird, m"ochte ich auf
\cite{Bj:93},S.115ff.,S.157ff. und \cite{Schu:92}, Seite 40ff. verweisen.)

Der Einfachheit halber sei im folgenden eine Menge ${\cal L}$, die die
Kovektoraxiome erf"ullt beziehungsweise ein Paar $(E,{\cal L})$ als
orientiertes Matroid bezeichnet. Dieses habe den Rang r, entsprechend der
Dimension des Raumes, in dem die Hyperebenen liegen und n$=|E|$ Elemente.

Eine Teilordnung "`$\leq$"' auf $\{+,-,0\}$ mit $0<+$, $0<-$ und bez"uglich
der $+$ und $-$ nicht vergleichbar sind, induziert auf $\{+,-,0\}^E$ "uber
den komponentenweisen Vergleich der Vorzeichenvektoren eine Produktteilordnung.
(Es gilt $y\leq x$, wenn f"ur alle $e\in E$ die Komponente $y_e\in\{0,x_e\}$
ist.) Ist ${\cal L}$ ein orientiertes Matroid nach obiger Definition, so wird
dieses, wenn man es mit einem abstrakten \^1-Element vereinigt und mit
eben definierter Teilordnung betrachtet zu einem Verband, dem (gro"sen)
Seitenverband $\hat{\cal L}={\cal L}\cup \{\hat{1}\}$\idx{Seitenverband} von
$\cal L$. Bez"uglich "`$\leq$"' maximale Elemente in $\cal L$ hei"sen
{\bf Topes} oder auch Regionen\idx{Topes!eines orientierten Matroids} von
$\cal L$ und entsprechen gerade den im realisierbaren Fall von Hyperebenen
"`ausgeschnittenen"' Teilr"aumen. (vgl. Abb.\ref{hyper})

Bei der Untersuchung orientierter Matroide sind auch Teilstrukturen dieser
interessant, die aus Operationen auf einem orientierten Matroid entstehen. So
ist eine {\bf Restriktion}\idx{Restriktion} eines Vorzeichenvektors
$X\in\{+,-,0\}^E$ auf eine Teilmenge F$\subset$E der Vorzeichenvektor
$X|_F\in\{+,-,0\}^F$, definiert durch $(X|_F)_e=X_e$ f"ur alle e$\in$F. Ist
${\cal L}\subseteq\{+,-,0\}^E$
die Menge der Kovektoren eines orientierten Matroids $\cal M$ auf E und ist
A$\subseteq$E, so ist die Menge der Kovektoren der {\bf Deletion}\idx{Deletion}
${\cal M}\backslash A$ die Menge ${\cal L}\backslash A
:=\{X|_{E\backslash A}:X\in{\cal L}\}\subseteq\{+,-,0\}^{E\backslash A}$.
Die Menge der Kovektoren der {\bf Kontraktion}\idx{Kontraktion} ${\cal M}/A$
ist die Menge ${\cal L}/ A:=\{X|_{E\backslash A}:X\in{\cal L}\mbox{ und }
A\subseteq X^0\}\subseteq\{+,-,0\}^{E\backslash A}$.
Unter einer {\bf Reorientierung}\idx{Reorientierung} $_{-A}{\cal M}$ versteht
man die Menge $_{-A}{\cal L}:=\{_{-A}X:X\in{\cal L}\}$, wobei $_{-A}X$ definiert
ist durch $(_{-A}X)^+=(X^+\backslash A)\cup (X^-\cap A), (_{-A}X)^0=X^0$ und
$(_{-A}X)^-=(X^-\backslash A)\cup (X^+\cap A)$, also einer lokalen
Vorzeichenumkehrung.

Gehen wir nochmals zur"uck zum Ausgangspunkt eines zentralen
Hyperebenenarrangements $\cal A$. Da Mengen von Hyperebenen
H$_e=\{x\in\R^r:<x,a_e>=0\}$ betrachtet werden, kann man "aquivalent statt der
H$_e$ auch deren Einschr"ankung auf die Einheitssph"are $\S^{r-1}\subset\R^r$
heranziehen (Gro"skreise auf der $\S^2$ oder im allgemeinen Fall (r-2)-Sph"aren
auf der $\S^{r-1}$), ohne da"s an der Repr"asentation der Schnitteigenschaften
etwas ver"andert wird. (Dies ist m"oglich, da durch die "`zentralen"'
Hyperebenen schon der lineare respektive projektive Fall betrachtet wird,
eine Kompaktifizierung der Hyperebenen also nur formal vollzogen werden mu"s,
eben durch den "Ubergang zu einem Sph"arensystem.)
Die durch die $\{x\in\R^r:<x,a_e>=0,\|x\|=1\}$ induzierte Zerlegung der
$\S^{r-1}$ l"a"st sich nun analog den Hyperebenen orientieren, indem man jene
Hemisph"aren als positiv annimmt, die in Richtung der Normalen $(a_e)_{e\in E}$
der H$_e$ liegen. Werden die Schnitteigenschaften der (r-2)-Sph"aren
untereinander nicht ver"andert, so k"onnen diese sogar mittels stetiger
Abbildungen (in \cite{Bj:93} werden diese als "`zahm"' bezeichnet, vgl. S. 225)
deformiert werden, wobei die Information des zugeh"origen orientierten Matroids
erhalten bleibt. So gelangt man zur Definition von Pseudosph"aren. Dabei wird
eine Teilmenge $S\subset\S^{r-1}$ als {\bf Pseudosph"are}\idx{Pseudosph\"are}
bezeichnet, wenn ein Hom"oomorphismus $h:\S^{r-1}\to\S^{r-1}$ existiert, so
da"s $S=h(S^{r-2})$ gilt. Hierbei ist $S^{r-2}=\{x\in\S^{r-1}:x_r=0\}$ der
"Aquator der $\S^{r-1}$ (vgl. Abb.\ref{pseudo}). Ein {\bf
Pseudosph"arenarrangement} in $\S^r$ ist dann als ein endliches Mengensystem
${\cal A}=(S_e)_{e\in E}$ definiert, wenn (vgl. \cite{Bj:93}, S.227) zum einen
$S_A=\Cap_{e\in A} S_e$ f"ur alle $A\subseteq E$ eine Sph"are darstellt und zum
anderen f"ur $S_A\not\subseteq S_e$ mit $A\subseteq E,e\in E$, sowie $S^+_e$ und
$S^-_e$ den beiden Hemisph"aren von $S_e$, $S_A\cap S_e$ eine Pseudosph"are in
$S_A$ mit den Seiten $S_A\cap S^+_e$ und $S_A\cap S^-_e$ ist.

Ein Ergebnis von Folkman und Lawrence aus dem Jahre 1978 liefert nun den
Zusammenhang zwischen Pseudosph"arenarrangements und orientierten Matroiden.
\begin{satz}
Sei ${\cal A}=(S_e)_{e\in E}$ ein orientiertes Pseudosph"arenarrangement in
der $\S^r$ und $\sigma:\S^r\to\{+,-,0\}^E$ die Abbildung, die definiert
sei "uber
$$\sigma(x)_e=\left\{\begin{array}{ll}
                        +, & \mbox{ f"ur } x\in S^+_e\\
                        -, & \mbox{ f"ur } x\in S^-_e\\
                        0, & \mbox{ f"ur } x\in S_e\end{array}\right.$$
Dann ist ${\cal L(A)} := \{\sigma(x):x\in\S^r\}\cup\{0\}\subseteq\{+,-,0\}^E$
die Menge der Kovektoren eines orientierten Matroids. Ist dim($S_e$)=k, so
ist der Rang von $\cal L(A)$ gleich r-k. Ist $\cal A$ essentiell, das hei"st
$S_E=\Cap_{e\in E} S_e$ ist leer, so ist der Rang von $\cal L(A)$ gleich r+1.
\end{satz}

Mit dem Beweis der Umkehrung schlie"st sich hier der Topologische
Repr"asentationssatz nach Folkman und Lawrence f"ur orientierte Matroide an
(vgl \cite{Bj:93}, Seite 233). Dieser besagt
\idx{Topologischer Repr\"asentationssatz}
\begin{satz}
{\bf Topologischer Repr"asentationssatz:} Ist ${\cal L}\subseteq\{+,-,0\}^E$, so
ist $\cal L$ genau dann die Menge der Kovektoren eines schleifenfreien
orientierten Matroids vom Rang r+1, wenn ein orientiertes
Pseudosph"arenarrangement $\cal A$ auf der $\S^{r+1+k}$ existiert, f"ur das
$k=\mbox{dim}(\Cap_{e\in E}\S_e)$ gilt und bez"uglich dem $\cal L$ gleich
$\cal L(A)$ ist.
\end{satz}

Schleifenfrei bedeutet, da"s die Vorzeichenstruktur jedes Topes keine Null
enth"alt. Als Folgerung hieraus l"a"st sich zeigen, da"s schleifenfreie
orientierte Matroide vom Rang r+1 (bis auf Reorientierung und Isomorphie)
bijektiv zu essentiellen orientierten Arrangements von Pseudosph"aren auf
$\S^r$ (bis auf topologische "Aquivalenz) korrespondieren.

Dieses herausragende Ergebnis liefert also die "Aquivalenz von
Pseudosph"arenarrangements und orientierten Matroiden respektive der gew"ahlten
Axiomatik. Hier ist es wichtig, da"s von \ul{Pseu\-do}\-sph"aren die Rede ist,
da ein topologisches Arrangement (im Sinne informationserhaltender stetiger
Verformung) nicht hom"oomorph zu einem Sph"arensystem sein mu"s.
Ist dies aber der Fall, so handelt es sich um die Darstellung eines
realisierbaren orientierten Matroids, analog der Definition der Realisierung,
wie sie im Abschnitt "uber die Chirotope angegeben wurde (Seite \pageref{real}).

Zu bemerken ist, da"s der reelle projektive Raum $\P^d$ der Dimension d aus
der $\S^d$ durch Identifikation aller antipodischen (diametral
gegen"uberliegenden) Punkte hervorgeht. Ist $\pi(x)=\{x,-x\}:\S^d\to \P^d$,
so werden in nat"urlicher Weise die Nullpunkt-symmetrischen Teilmengen
der d-Sph"are mit allgemeinen Teilmengen des projektiven Raumes
identifiziert. Hier"uber k"onnen oben eingef"uhrte Pseudosph"aren auch als
Pseudohyperebenen in $\P^d$ aufgefa"st werden, womit auch eine Klassifikation
orientierter Matroide im projektiven Raum mittels Pseudohyperebenen gegeben ist.

\begin{figure}[htb]
$$
\beginpicture
\unitlength0.6cm
\setlinear
\setcoordinatesystem units <0.6cm,0.6cm>
\setplotarea x from -3 to 3, y from -3 to 3
\put {\beginpicture
  \setsolid
  \circulararc 180 degrees from 2 0 center at 0 0
  \ellipticalarc axes ratio 4:1 180 degrees from -2 0 center at 0 0
  \setdashes<1.5mm>
  \ellipticalarc axes ratio 4:1 -180 degrees from -2 0 center at 0 0
  \setsolid
  \plot -1.73 1 -3 1 -4 -1 3 -1 4 1 1.73 1 /
  \circulararc 120 degrees from -1.73 -1 center at 0 0
  \put {\vector(0,1){1.5}} [Bl] at 0 0
  \put {\vector(1,0){0.2}} [Bl] at 0 -0.5
  \put {\scsi $\S^{d-1}$} [Bl] at 1.5 2
  \put {\scsi $\S^{d-2}_a$} [Bl] at 1.5 -0.8
  \put {\scsi a} [Bl] at 0.3 1.5
  \put {\scsi $H_a$} [Bl] at 2.5 0.5
  \setquadratic
  \plot -2 0 -1.3 0.3 -0.7 0.1 0.2 -0.1 1.2 0.15 1.6 0.2 2 0 /
  \setdashes<1mm>
  \plot 2 0 1.3 -0.3 0.7 -0.1 -0.2 0.1 -1.2 -0.15 -1.6 -0.2 -2 0 /
\endpicture} at -4 1
\put {\beginpicture
  \setsolid \setlinear
  \circulararc 360 degrees from 2 0 center at 0 0
  \setquadratic
  \plot -1.732 -1 -0.5 -1 0.5 -0.2 1 0.5 1.732 1 /
  \plot 0 -2 0.35 -1 0 0 -0.5 1 0 2 /
  \plot 1.732 -1 1 0 0.3 0.3 -1 0.5 -1.732 1 /
  \setdashes<1mm>
  \plot 1.732 1 0.5 1 -0.5 0.2 -1 -0.5 -1.732 -1 /
  \plot 0 2 -0.35 1 0 0 0.5 -1 0 -2 /
  \plot -1.732 1 -1 0 -0.3 -0.3 1 -0.5 1.732 -1 /
  \setdots<1mm> \setlinear
  \plot 0.5 1.95 0.5 3 -0.5 2 -0.5 -3 0.5 -2 0.5 -1.95 /
  \plot -2 0.4 -3 1 -2 2 3 -1 2 -2 1.35 -1.55 /
  \plot -2 -0.4 -3 -1 -2 -2 3 1 2 2 1.35 1.55 /
\endpicture} at 4 0
\endpicture
$$
\caption{Von einer Hyperebene zu einer Pseudosph"are}
\label{pseudo}
\end{figure}

Zusammenfassend kann man sagen, da"s sich orientierte Matroide als
Verallgemeinerung aus den verschiedensten Bereichen der Mathematik motivieren
lassen. Dabei f"uhrt in verschiedenen F"allen die allgemeinste Struktur
bez"uglich einer zu erf"ullenden Eigenschaft unabh"angig von der gew"ahlten
Ausgangssituation zu diesem Konzept. Der Name "`Matroid"' stammt dabei von
Whitney aus dem Jahre 1935 und leitet sich von einer Klasse fundamentaler
Beispiele solcher Objekte ab, die aus Matrizen hervorgehen. (Als
"Ubersichtsartikel sei hierzu \cite{Bo:93} genannt.)

\section{Symmetriebegriffe}

Ist ein allgemeiner CW-Komplex realisierbar, so ist es erstrebenswert,
m"oglichst viele Symmetrieeigenschaften von diesem in die Realisierung
"`hin"uberzuretten"'. Was man hierunter versteht, soll im folgenden
beschrieben werden.

Wenn man im $\E^d$ von {\bf Symmetrien}\idx{Symmetrie} spricht, so meint man
damit die Automorphismen von $\E^d$, die Abst"ande und Orthogonalit"at
invariant lassen. Dies sind gerade die orthogonalen Transformationen.
Die Menge aller orthogonalen Transformationen ist bez"uglich der
Hintereinanderausf"uhrung eine Gruppe, die {\bf orthogonale Gruppe}
${\cal O}(d)$.\idx{orthogonale Gruppe} Elemente aus ${\cal O}(d)$ lassen sich
als orthogonale (d$\times$d)-Matrizen "uber $\R$ darstellen und stellen damit
eine Untergruppe der allgemeinen linearen Gruppe $\mbox{GL}_d(\R)$, der Gruppe
aller (d$\times$d)-Matrizen "uber $\R$, dar.
$${\cal O}(d) = \{ A\in\mbox{GL}_d(\R)~|~A^T A = I_d\}$$
{\scsi
Eine Matrix hei"st orthogonal, wenn sie die kanonischen Einheitsvektoren auf
eine Orthonormalbasis abbildet, das hei"st, wenn ihre Spaltenvektoren
auf L"ange Eins normiert sind, sowie das Skalarprodukt von je zwei
verschiedenen Spaltenvektoren Null ergibt. Als Folgerung hieraus ist die
Determinante einer orthogonalen Matrix immer $\pm 1$. Wie mit linearer Algebra
gezeigt werden kann, haben orthogonale Matrizen die Eigenschaft, da"s ihre
Inversen gerade ihre Transponierten sind, was zur Charakterisierung
der orthogonalen Gruppe dient.\\
Zur Erinnerung ist eine Gruppe eine Menge G zusammen mit einer Abbildung
$*:G\times G\to G$, der Gruppenoperation, die folgendes Axiomensystem erf"ullt.
So ist G bez"uglich $*$ abgeschlossen, besitzt ein neutrales Element e
($g*e=e*g$ f"ur alle $g\in G$) und zu jedem $g\in G$ existiert ein inverses
Element $g^{-1}$ ($g*g^{-1}=e=g^{-1}*g$). Als Literatur zur Linearen Algebra
sei Herrn Artmanns Buch \cite{Art:89}, sowie zur Einf"uhrung in die
Algebra Serge Langs Buch \cite{La:79} empfohlen.
}\\
Damit wird die Matrizenmultiplikation zur Gruppenoperation.
Matrizen aus ${\cal O}(d)$ mit Determinante -1 hei"sen {\bf Spiegelungen} oder
auch {\bf Reflektionen},\idx{Reflektion}\idx{Spiegelung} Matrizen mit
Determinante +1 werden als {\bf Drehungen}\idx{Drehung} oder {\bf Rotationen}
\idx{Rotation} bezeichnet. Die Drehungen bilden dabei eine Untergruppe der
${\cal O}(d)$, die als spezielle orthogonale Gruppe ${\cal SO}(d)$ bezeichnet
wird.\idx{spezielle orthogonale Gruppe}\\
{\scsi
Die Untergruppe ${\cal SO}(d)$ ist gesondert ausgezeichnet, da es sich bei
ihr um die Komponente des Eins-Elements (das hei"st Neutralelement) von
${\cal O}(d)$ handelt, wenn diese in ihre zwei Zusammenhangskomponenten
zerlegt wird. (vgl. \cite{Os:92}, Seite 130ff.)}

Einen geometrischen d-Komplex $\cal C$, das hei"st die metrische Realisierung
eines kombinatorisch vorgelegten allgemeinen CW- oder simplizialen d-Komplexes,
bezeichnet man als {\bf symmetrisch},\idx{kombinatorischer Komplex}
\idx{geometrischer Komplex}\idx{geometrischer Komplex!symmetrischer $\sim$}
wenn eine Untergruppe G der ${\cal O}(d)$ existiert, deren s"amtliche
Elemente $g\in G$ Automorphismen auf $\cal C$ sind, das hei"st f"ur die
s"amtlich $g({\cal C}) = {\cal C}$ gilt. Die $g\in G$ hei"sen Symmetrien
von $\cal C$, G selbst hei"st, gem"a"s der Gruppenstruktur, die Symmetriegruppe
\idx{Symmetriegruppe} von $\cal C$.\idx{Symmetriegruppe!$\sim$ geometrischer
Komplexe}
Die Anzahl der Elemente einer Symmetriegruppe bezeichnet man als deren
(Symmetrie-){\bf Ordnung}. Dies ist nicht mit der Ordnung einer Symmetrie, das
hei"st eines Elements einer Symmetriegruppe, zu verwechseln, die angibt, wie oft
die betreffende Symmetrie angewendet werden mu"s, um wieder die Identit"at zu
erreichen. Mit "`hoher Symmetrie"' bezeichnet man dabei die Situation, da"s
die Anzahl der Elemente der betreffenden Symmetriegruppe, gemessen an
der Anzahl der permutierten Ecken eines Komplexes, gro"s ist
\idx{Symmetrieordnung} (etwa auch die Situation, da"s die Symmetriegruppe des
geometrischen eine "`gro"se"' Untergruppe der Symmetriegruppe des kombinatorische
Komplexes ist).

Betrachtet man, wie auf Seite \pageref{symm}, die Inzidenzstrukturen
simplizialer Komplexe, die durch die Simplexzahlen impliziert werden, so
hei"st die sich durch Umnummerierung (Permutation) der Ecken (Menge der
0-Simplizes) ergebende Automorphismengruppe des Komplexes, deren Elemente
nicht-degenerierend auf den Simplizes des Komplexes wirken, Symmetriegruppe
des simplizialen Komplexes.\idx{Symmetriegruppe!$\sim$ simplizialer Komplexe}
Mittels der Simplexzahlen kann jede solche Symmetrie, wie schon erw"ahnt,
als Element der Permutationsgruppe $S_n$, mit der Anzahl n der auftretenden
verschiedenen 0-Simplizes, beschrieben werden, also als Permutation auf der
Eckenmenge $E=\{1,\ldots,n\}$. Auch hier hei"st die Gesamtheit der Symmetrien
des Komplexes entsprechend ihrer Gruppenstruktur Symmetriegruppe des
simplizialen Komplexes.

Da simpliziale Komplexe einen Spezialfall von CW-Komplexen darstellen, kann
man in einfacher Weise den Symmetriebegriff f"ur simpliziale Komplexe
auf die CW-Komplexe erweitern. Wie im einen Fall bei nicht-degenerierenden,
bijektiven simplizialen Abbildungen von Symmetrien gesprochen wird, so k"onnen
dimensionserhaltende bijektive Selbstabbildungen von CW-Komplexen auch als
Symmetrien bezeichnet werden, wie dies schon im Abschnitt "uber CW-Komplexe
beschrieben wurde (in jedem Fall betrachtet man die Automorphismen des
Komplexes).

F"ur (zun"achst) simpliziale d-Komplexe kann man den Begriff der Flagge
\idx{Flagge} f"ur das $d+1$-Tupel (0-Simplex, 1-Simplex, $\ldots$, d-Simplex)
einf"uhren, dessen Elemente paarweise inzidieren (Im Falle von 2-Kom\-plexen
ist dies ein Tripel (Ecke, Kante, Seite)). Automorphismen kombinatorischer
Komplexe\label{flag} erhalten solche Flaggen. Eine Gruppe G kombinatorischer
Automorphismen hei"st nun {\bf Flaggen-transitiv},\idx{Flaggen-transitiv}
falls es zu jedem Paar von Flaggen des Komplexes einen Automorphismus aus G
gibt, der die erste auf die zweite Flagge abbildet. Ist die Symmetriegruppe
eines kombinatorischen Komplexes flaggentransitiv, so bezeichnet man den
betreffenden Komplex als {\bf regul"ar}.
\idx{kombinatorischer Komplex!$\sim$regul\"arer}
Bekannt ist diese Eigenschaft vor allem von den (regul"aren) Platonischen
K"orpern, zu denen Analoga gesucht werden, indem man M"oglichkeiten der
Realisierung regul"arer (CW-)Komplexe untersucht, wie etwa die Dycksche Karte
(vgl. Kap.3).

Jede Symmetrie eines geometrischen Komplexes (eine als Matrix "`darstellbare"'
Symmetrie) induziert nun eine kombinatorische Symmetrie auf dem
zugrundeliegenden kombinatorischen Komplex. Folglich ist die Gruppe der
geometrischen Symmetrien eine Untergruppe der Gruppe der kombinatorischen
Symmetrien eines Komplexes.\\
In der Tat handelt es sich im allgemeinen Fall "`nur"' um eine Untergruppe, es
treten n"amlich zumeist mehr kombinatorische als metrische Symmetrien auf (vgl.
die Betrachtung der Dyckschen Karte im dritten Kapitel). Die "`"uberz"ahligen"'
kombinatorischen Symmetrien bezeichnet man als versteckte Symmetrien
\idx{versteckte Symmetrien} des realisierbaren kombinatorischen Komplexes mit
dieser Eigenschaft. Im Falle konvexer 3-Polytope im $\E^3$ sind die metrische
und kombinatorische Symmetriegruppe stets gleich (isomorph), doch schon im
nicht-konvexen Fall k"onnen versteckte Symmetrien auftreten (vgl. \cite{Bo:91},
Seite 4). Als Beispiel f"ur eine Symmetrieuntersuchung m"oge ein Torus dienen,
wie er in Kapitel 2, Seite \pageref{torus1} beschrieben ist.

Analog der bisherigen Definitionen werden Permutationen $\sigma\in S_n$ auf
der Grundmenge $E=\{1,\ldots,n\}$ eines orientierten Matroids $\chi$ als
Symmetrien von diesem bezeichnet, wenn es sich bei $\sigma$ um einen
Automorphismus handelt, der $\chi$ auf $\chi$ selbst (Drehung) oder
auf $-\chi$ (Spiegelung, Reflektion) abbildet. Die Gesamtheit aller
solchen Automorphismen eines orientierten Matroids $\chi$ hei"st wiederum,
entsprechend ihrer Gruppenstruktur, Symmetriegruppe von $\chi$.
\idx{Symmetriegruppe!$\sim$ orientierter Matroide} Dieser Symmetriebegriff
l"a"st sich anschaulich an den Symmetrien der durch Pseudosph"arensysteme
induzierten Komplexe illustrieren, aber auch an Vorzeichenwechseln des
Vorzeichenvektors eines orientierten Matroids festmachen.

Bei der Untersuchung der m"oglichen Symmetrien im Raum (hier ist der
dreidimensionale Euklidische Raum gemeint), die im letzten Jahrhundert
erfolgte und deren zentrales Ergebnis zum Ende des 19. Jahrhunderts die
Klassifizierung aller Raumsymmetrien lieferte, stellte sich heraus, da"s bei
Raumsymmetrien von Gitterstrukturen, das hei"st von periodischen Strukturen,
die den gesamten $\R^3$ ausf"ullen, Symmetrieelemente nur mit den Ordnungen
1, 2, 3, 4 und 6 auftreten k"onnen. (vgl. etwa die Darstellung der
geschichtlichen Entwicklung der Symmetriekonzepte der Kristallographie von E.
Scholz \cite{Scho:89}). Dieses Ergebnis wurde in der Diplomarbeit von Ronald
Dauster (\cite{Dau:89}, Seite 30ff.) aufgegriffen und auf Realisierungen
orientierter Matroide "ubertragen. Danach sind affine Realisierungen
simplizialer orientierter Matroide nur mit oben angegebenen Ordnungen m"oglich,
was zumindest die Verifikation der Symmetrien bei vermeindlichen Realisierungen
unterst"utzt. Affine Realisierung bedeutet hier gerade, da"s das orientierte
Matroid von einer Matrix stammt deren Zeilen explizite Punkte in homogenen
Koordinaten beschreiben, also alle in der selben Hyperebene des $\R^d$ liegen,
also zu $(1,p_i)$ normiert werden k"onnen und so einer affinen Punktmenge
im $\R^{d-1}$ entsprechen.

Da wir uns in dieser Arbeit aber mit allgemeinen CW-Komplexen besch"aftigen
wollen, infolge dessen gerade eine Erweiterung auf nichtsimpliziale orientierte
Matroide anstreben, haben wir diese Einschr"ankung der Ordnungen der
Symmetrieelemente leider nicht zur Verf"ugung. Statt dessen stellt sich die
interessante Frage, welche Symmetrien bei der Realisierung beliebiger orientierter
Matroide auftreten k"onnen, ob Einschr"ankungen existieren oder ob es sogar eine
maximale auftretende Elementordnung gibt. Ein solches Ergebnis w"urde zu einem
Analogon der Klassifizierung der Raumsymmetrien f"uhren und sicher einer
Bereicherung der Darstellung der Theorie der orientierten Matroide dienen.
